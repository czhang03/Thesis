\chapter{Introduction}
\label{chapter:introduction}
\thispagestyle{myheadings}

\section{An overview on Kleene Algebra}
\label{sec:history}

Kleene algebra, named after the eminent mathematician Stephen Cole Kleene, represents a pivotal development in mathematical logic, computer science, and formal languages. 
Originating in the mid-20th century, Kleene algebra emerged from Kleene's seminal work on regular sets and expressions~\cite{Kleene_1956}, where he introduced algebraic structures to formalize language equalities between regular expression. 
In said work, Kleene left an important open question about Kleene Algebra: whether there exists a \emph{complete} algebraic system for regular language equality, i.e. an algebraic system that is capable of derive \emph{all} the language equalities of regular expression.
There are numerous systems proposed, and they are closely related~\cite{Kozen_1990}, yet the modern Kleene Algebra and its completeness proof are often attributed to Kozen~\cite{Kozen_2001,Kozen_1994}.

Given its close relation to regular languages, the study of Kleene Algebra often makes heavy use of automata theory,
and the automata perspective underpins another important property of Kleene Algebra: decidability.
Specifically, the trace equivalence of two automata (or more generally, behavioral equivalences of coalgebras) can be obtained by bisimulation on these automata~\cite{rutten_UniversalCoalgebraTheory_2000}; and bisimulation is usually decidable for finite automata~\cite{Kozen_2008,Smolka_Foster_Hsu_Kappé_Kozen_Silva_2020,Silva_2010}.
This framework is characterized by the bialgebraic approach using category theory~\cite{jacobs_BialgebraicReviewDeterministic_2006}, although we will not dive deep into coalgebra, many of the techniques used in this thesis are inspired by this framework.

The decidability and completeness of Kleene Algebra has led to a suite of applications in programming languages and verifications; specifically, in the areas of network system~\cite{Anderson_Foster_Guha_Jeannin_Kozen_Schlesinger_Walker_2014,Foster_Kozen_Milano_Silva_Thompson_2015, Smolka_Kumar_Kahn_Foster_Hsu_Kozen_Silva_2019},
concurrent programs~\cite{hoare_ConcurrentKleeneAlgebra_2009,Kappé_Brunet_Silva_Wagemaker_Zanasi_2020,Kappé_Brunet_Silva_Zanasi_2018}, 
probabilistic systems~\cite{mciver_UsingProbabilisticKleene_2006, McIver_Rabehaja_Struth_2011}, 
relational verification~\cite{Antonopoulos_Koskinen_Le_Nagasamudram_Naumann_Ngo_2022},
compiler verification~\cite{Kozen_Patron_2000},
and program schematology~\cite{Angus_Kozen_2001}.

Notice that despite the decidability of KA and KAT, the decision procedure is PSPACE-complete~\cite{Cohen_Kozen_Smith_1999}, making equality checking infeasible for large systems.
However, in real-world testing, network systems based on KAT was able to out-perform state of the art network verifier~\cite{Smolka_Kumar_Kahn_Foster_Hsu_Kozen_Silva_2019}.
This achievement can be attributed to  \emph{Guarded Kleene Algebra with Tests} (GKAT)~\cite{Smolka_Foster_Hsu_Kappé_Kozen_Silva_2020}, an efficient fragment of KAT. Concretely, GKAT enjoys more compact automaton structures, which enabled faster equivalence checking.

\section{Our Contributions}



To delve deeper into the utility of Kleene Algebra with Tests (KAT) and Guarded Kleene Algebra with Tests (GKAT), we have formulated two novel systems grounded in these frameworks.

The first system, named Kleene algebra with tests and top (TopKAT), is used to facilitate domain/reachability reasoning.  
The standard KAT lacks the capability to adequately encode notions of domain and reachability, in particular we proved that KAT failed to encode incorrectness logic.
This inspired our development of TopKAT, which extends KAT by requiring a largest element, allowing TopKAT to encode domain comparisons in its relational model. 

Even though TopKAT is incomplete with respect to its relational models, it is complete with respect to domain comparisons, making TopKAT a compact yet complete system to reason about reachability in regular programs.
To obtain the domain completeness result, a novel perspective on the classical reduction techniques~\cite{Pous_Rot_Wagemaker_2021,Kozen_Smith_1997} arises: we are able to characterize reduction simply as an injective homomorphism. 
This fresh look into reduction leads to more streamlined proof of completeness of TopKAT, and can potentially ease the burden for future completeness proof about other extensions.

The second system, Control-Flow GKAT (CF-GKAT), extends Guarded Kleene Algebra with Tests (GKAT) by incorporating control flow constructs such as indicator variables, \command{break}, \command{return}, and \command{goto}.
Addressing the non-local nature inherent in these control flow commands, we devised continuation semantics and introduced the CF-GKAT automaton.
These concepts complete the classical triangular correspondence~\cite{jacobs_BialgebraicReviewDeterministic_2006,Kozen_2008,Smolka_Foster_Hsu_Kappé_Kozen_Silva_2020} among programs, their semantics, and corresponding automata representations.

To design a decision procedure for trace equivalence among CF-GKAT programs, we formulated a lowering procedure capable of converting any CF-GKAT automaton into a GKAT automaton.
The decision procedure for CF-GKAT programs equivalence then follows from the lowering.
Leveraging the efficiency, soundness, and completeness of the GKAT decision procedure, our algorithm not only preserves soundness and completeness with respect to trace equivalences, but also exhibits nearly-linear time efficiency relative to the input expression sizes.

To summarize, I compiled our contributions as a list, for readability:
\begin{itemize}
    \item Proved that KAT insufficiency for domain reasoning, by proving that KAT cannot encode incorrectness logic.
    \item Developed the system TopKAT, and proven it is sound and complete with respect to its trace/language interpretation and general relational model.
    \item Developed a new notion of reduction, which is based on homomorphism from the free algebra.
    \item Proved TopKAT is complete with respect to domain comparisons.
    \item Designed the syntax and semantic of CF-GKAT, enabling control-flow verification of \command{while}-programs with indicator variable and non-local control.
    \item Designed the decision procedure of CF-GKAT based on GKAT automaton.
    \item Proving the decision procedure is sound and complete, and also enjoys nearly-linear time complexity, assuming the number of primitive tests are fixed.
\end{itemize}

\section{Technical Background}

\subsection{Extensions of Kleene algebra And Their Models}

A \emph{Kleene algebra} is an idempotent semiring with a star operation, written
$p^*$, satisfying the following \emph{unfolding}, \emph{left induction},
and \emph{right induction} rules:
\begin{mathpar}
    p^* = 1 + p p^* = 1 + p^* p, \and
    p r + q  \leq  r  \implies  p^* q  \leq  r, \and
    r p + q  \leq  r  \implies  q p^*  \leq  r;
\end{mathpar}
the ordering here is the conventional ordering in idempotent semirings: \(p  \leq  q  \triangleq  p + q = q.\)
It is known that the right-hand version of unfolding and induction rule 
can be removed while preserving the same equational theory~\cite{Kozen_Silva_2020}.
Yet, we will focus on the standard definition of KA in this thesis.
\begin{lemma}\label{the: well known fact about KA}
    Following are well-known facts in Kleene algebra
    \begin{itemize}
        \item All the Kleene algebra operations preserve order.
        \item The following equations are true for the star operation:
              \[ p^*  \cdot  p^* = p^* \\ (p^*)^* = p^*.\]
    \end{itemize}
\end{lemma}

A Kleene algebra with tests (KAT) is a Kleene algebra with an embedded Boolean algebra,
where the conjunction, disjunction, and identities in the Boolean algebra coincide with 
the addition, multiplication, and the identities of Kleene algebra.  
We refer to elements of this embedded Boolean algebra as \emph{tests}.

Given an algebraic theory, we can construct its \emph{free model} 
over a finite set \( Σ \), 
called the \emph{alphabet}~\cite{burrisCourseUniversalAlgebra1981}.  
The free model consists of all the terms formed by \( Σ \) modulo  
provable equivalences of the algebra. The operations of the free model are obtained 
by lifting the term-level operations to equivalence classes.

The above construction can be extended to the case of KAT, 
suppose that we are given two disjoint finite sets $K$ (the
\emph{action alphabet}) and $B$ (the \emph{test alphabet}).  Elements of $K$ and
$B$ are called \emph{primitive actions and primitive tests}, respectively. 
KAT terms over the alphabet \(K, B\) are defined with the following grammar:
\[e  ≜  b  ∈  B  ∣  p  ∈  K  ∣  1  ∣  0  ∣  e_1 + e_2  ∣  e_1  ⋅  e_2  ∣  e^*  ∣  \overline{e_b},\]
where \(e_b\) does not contain primitive actions.
The \emph{free KAT} over \(K, B\), written $\KAT_{K,B}$, 
consists of terms over \(K, B\) modulo provable KAT equivalences.  
The tests of the free KAT are Boolean terms, i.e. terms formed by primitive tests and Boolean operations modulo Boolean axioms.
We sometimes omit the alphabets \(K\) and \(B\) when they
are irrelevant or can be inferred.

In the thesis, we frequently consider terms modulo provable equalities, i.e. in the
context of its corresponding free model.  For example, given \(e_1, e_2  \in  \KAT\),
we will say \(e_1 = e_2\) when they are provably equal using the theory of KAT.
Although the free model seems trivial, it leads to simpler and more modular
proofs of some properties of algebraic theories, as we will see in~\Cref{sec: properties of TopKAT}.

Other important models are language KATs and relational KATs, which we review here.  
An \emph{atom} (short for ``atomic test'') over a test alphabet \(B = \{b_{1}, b_{2},  … , b_{n}\}\) is a sequence of the form
\[\hat{b_{1}}  \cdot  \hat{b_{2}}  \cdot   \cdots   \cdot \hat{b_{n}} \text{ where } \hat{b_{i}}  \in  \{b_{i}, \bar{b_{i}}\},\] 
where each atom is a record of all the primitive tests that it satisfies, and those it doesn't.
We denote atoms as \( α ,  β ,  γ ,  … \) and the set of all atoms as \(\At\).

A \emph{guarded string} (or \emph{guarded word}) over \(K, B\) is an alternation 
between atoms and primitive actions that starts and ends in atoms: 
\[ \alpha _{0}p_{1} \alpha _{1}  \cdots  p_{n}  \alpha _{n} \text{ where } p_{i}  \in  K,  \alpha _{i}  \in  \At;\] 
where each action is ``guarded'' by an atom.
A guarded string is similar to a program trace. These traces record the initial, intermediate and final machine states (represented by the atoms \(αᵢ ∈ \At\)) observed during execution, as well as the actions that occurred between those states. % chktex 36

\begin{example} 
    Let $B = \{ b₁, b₂ \}$ and $K = \{ p₁, p₂ \}$.
    Now the guarded word \[b₁ \overline{b₂} ⋅ p₁ ⋅ \overline{b₁} b₂ ⋅ p₂ ⋅ \overline{b₁} \overline{b₂}\] represents a program trace that starts out in a machine state where $b₁$ is true (but $b₂$ is not).
    The program then executes the action $p₁$, after which $b₂$ is true (but $b₁$ is not).
    Finally, the program goes on to execute the action $p₂$, and halts in a state where neither $b₁$ nor $b₂$ is true.
\end{example}

We denote the set of all guarded
strings over alphabet \(K, B\) as \(GS_{K, B}\), and we will omit the alphabet
\(K, B\) when it is irrelevant or can be inferred from context.  The notation
\( α  w\) denotes a guarded string starting with atom \( α \) with the rest of the string
\(w\); similarly, \(w  α \) denotes a guarded string that ends with atom \( α \) with
rest of the string being \(w\).

\begin{definition}[Language/trace KAT~\cite{Kozen_Smith_1997}]
  The \emph{language KAT} (also called ``\emph{trace KAT}'') over an alphabet \(K, B\) is
  denoted as \(\mathcal{G}_{K, B}\), or simply \(\mathcal{G}\) if no confusion can arise.

  The elements are sets of guarded strings (called \emph{guarded languages}), 
  and the tests are sets of atoms.
  The additive identity 0 is the empty set, and the multiplicative identity 1 is
  the set of all the atoms \(\At\).  The addition operator is set union, and the
  multiplication operator is defined as follows:
    \[W_{1}  ⋄  W_{2}  ≜  \{w_{1}  α  w_{2}  ∣  w_{1}  α   ∈  W_{1},  α  w_{2}  ∈  W_{2}\}.\]
    The star operation is defined non-deterministically 
    iterating the multiplication operator:
    \[W^*  ≜   ⋃_{i  ∈ ℕ} W^i \text{ where } W^0 = \At, W^{k+1} = W  ⋄  W^k.\]
\end{definition}


Another useful type of KAT are relational ones, where each element is a relation
\(R  \subseteq  X  \times  X\) over a fixed set \(X\).  In applications, the set $X$ typically
represents the set of all possible program states, and each relation $R$
represents a program by relating each possible input to the corresponding
output.

\begin{definition}[Relational KAT]
  A relational KAT is a KAT $\mathcal{R}$ consists of relations over a fixed set \(X\) 
  (though $\mathcal{R}$ need not contain every relation over $X$),
  and it is closed under the following operations. 
  The tests are all the relations that are subsets of the identity relation.  
  The additive identity 0 is the empty set, and
  the multiplicative identity is the identity relation:
  \[1  \triangleq  \{(x, x)  \mid  x  \in  X\}.\] The addition operator is set union, and the
  multiplication operation is relational composition:
  \[R_{1} ; R_{2} = \{(x, z)  \mid   \exists  y  \in  X, (x, y)  \in  R_{1}, (y, z)  \in  R_{2}\}.\] 
  Finally, the star operation is defined as:
  \[R^*  \triangleq   \bigcup _{i  \in  \mathbb{N}} R^i \text{ where } R^0 = 1, R^{k+1} = R ; R^k.\] We denote the
  class of all relational KATs as \(\REL\).
\end{definition}

We are also interested in maps between models:
A \emph{KAT homomorphism} \(h\) is a map between two KATs \(\mathcal{K}\) and \(\mathcal{K}'\)
s.t. it preserves the sorts and operations:
given a test \(b\) in \(\mathcal{K}\) then \(h(b)\) is a test in \(\mathcal{K}'\);
and all the KAT operations (complement, identities, addition, multiplication, and star) are preserved:
\begin{align*}
    h & : 𝒦  →  𝒦'\\
    h(b̄) & = \overline{f(b)} \\  
    h(1) & = 1 \\  
    h(0) & = 0 \\
    h(p + q) & = h(p) + h(q) \\  
    h(p ⋅ q) & = h(p) ⋅ h(q) \\  
    h(p^*) & = h(p)^*.
\end{align*}

\subsection{Interpretation, Completeness, and Injectivity}\label{sec: completeness background}

Consider a KAT equation such as \(p ⋅ b ⋅ b̄ = 0\). To determine its
validity in a particular KAT \(\mathcal{K}\), we need to assign meaning to it by
interpreting each primitive as an element in \(\mathcal{K}\); that is, by defining a map
\(\hat{I}\) of type \(K + B  →  𝒦\).  Such a map \(\hat{I}: K + B  →  𝒦\) induces a
unique KAT homomorphism \(I : \KAT_{K,B}  →  \mathcal{K}\) inductively defined on the term 
as follows:
\begin{equation}
    \begin{aligned}
        I(p)       &  \triangleq  \hat{I}(p)    & \text{ where } p  \in  K + B \\
        I(\overline{e_b}) &  \triangleq  \overline{I(e_b)} 
            & \text{\(e_b\) does not contain primitive actions} \\
        I(e_1 + e_2) &  \triangleq  I(e_1) + I(e_2)                     \\
        I(e_1  \cdot  e_2) &  \triangleq  I(e_1)  \cdot  I(e_2)                     \\
        I(e^*)     &  \triangleq  I(t)^*
    \end{aligned}
\end{equation}
In fact, every KAT homomorphism from a free model arises this way: there is a
bijection between functions of type \(K + B  \to  \mathcal{K}\) and KAT homomorphisms of type
\(\KAT_{K, B}  \to  \mathcal{K}\), for any KAT \(\mathcal{K}\).  
Because the homomorphism \(I\) and the function \(\hat{I}\) are equivalent, 
we will refer to them interchangeably as \emph{KAT interpretations} 
and denote both of them as \(I\).

The above result enables us to define a homomorphism from the free KAT just by
defining its action on the primitives; saving us time to check the equations
that a homomorphism must satisfy.  It also allows us to prove that two
interpretations are equal by arguing that they map the primitives to
equal values.

Given a KAT \(𝒦\), and two terms \(e_1, e_2  ∈  \KAT_{K, B}\) we say that \(𝒦  ⊧  e_1 = e_2\) if
\[ ∀  I : \KAT_{K, B}  →  𝒦, I(e_1) = I(e_2).\] In particular, 
for two terms in the free model \(e_1, e_2  ∈  \KAT_{K, B}\),
\(\KAT_{K, B}  ⊧  e_1 = e_2\) is equivalent to \(e_1 = e_2\).  
For a collection of models \(\mathsf{K}\), 
we say that \(𝖪  ⊧  e_1 = e_2\) if for all \(\mathcal{K}  ∈  𝖪\),
\(𝒦  ⊧  e_1 = e_2\).  For example, \(\REL  ⊧  e_1 = e_2\) means that \(e_1 = e_2\) is
valid in all relational KATs.  

Theories like KAT are designed to model practical
programs, so it is important to know if they can model all the desirable
equations between programs. If the theory of KAT can derive all the equalities
for a particular interpretation \(I\), namely:
\[\KAT_{K, B}  \models  e_1 = e_2  \iff  I(e_1) = I(e_2),\]
we say that the theory of KAT is \emph{complete} with respect to \(I\).
Recall that \(\KAT_{K, B}  \models  e_1 = e_2\) is equivalent to \(e_1 = e_2\);  
thus, by definition, an interpretation \(I\) is complete if and only if it is injective.
One of such interpretation is the guarded string interpretation
\(G: \KAT_{K, B}  →  𝒢_{K, B}\)~\cite{Kozen_Smith_1997},
defined by lifting the following action on the primitives:
\[
    G(b) = \{ α   ∣  \text{\(b\) appears positively in \( α \)}\}, \\
    G(p) = \{ α  p  β   ∣   α ,  β   ∈  \At\}.
\]

In several previous works, the term ``free model'' refers to the range (set of reachable elements) of a complete interpretation.  
Since a complete interpretation is an injective homomorphism, such interpretation induces an isomorphism on its range, thus our definition of free model is equivalent to these definitions.

Many previous proofs can also be explained by seeing complete interpretations as injective homomorphisms: the proof for completeness of relational KATs constructs an injective homomorphism $h$ from a language KAT into a relational
KAT~\cite{Kozen_Smith_1997}.  
Since both \(G\) and \(h\) are injective homomorphisms, \(h  ∘  G\) is also an injective homomorphism, hence a complete interpretation.  Since \(h  ∘  G\) is a relational interpretation:
\[\KAT_{K, B}  ⊧  e_1 = e_2  ⟹  \REL  ⊧  e_1 = e_2  ⟹  h  ∘  G(e_1) = h  ∘  G(e_2);\]
then the completeness of \(h  ∘  G\) implies
\((h  ∘  G)(e_1) = (h  ∘  G)(e_2)  ⟺  \KAT_{K, B}  ⊧  e_1 = e_2\). Hence,
\[\KAT_{K, B}  ⊧  e_1 = e_2  ⟺  \REL  ⊧  e_1 = e_2,\]
i.e. the theory of KAT is complete with respect to relational KAT.

Function/homomorphism composition will be used frequently in this paper,
but sometimes simply writing \(h  ∘  g\) will be confusing without knowing
the domain and codomain of function \(h\) and \(g\).
Therefore, we will use the notation \(X \xrightarrow{g} Y \xrightarrow{h} Z\)
to denote the composition of \(h: Y  →  Z\) and \(g: X  →  Y\) when it is desirable.

Besides using composition of injective homomorphisms, another technique commonly
used to prove injectivity is to construct a left inverse: 
if a KAT homomorphism \(f: 𝒦  →  𝒦'\) has a left inverse homomorphism \(g: 𝒦'  →  𝒦\) 
i.e. \(g  ∘  f = id_{𝒦}\), then \(f\) is injective.  
Notice that \(g\) does not need to be a homomorphism for \(f\) to be injective,
however, in the case where \(f\) is an interpretation, 
\(g\) being a homomorphism makes the equality \(g  ∘  f = id_{𝒦}\) easier to check.
Because both \(g  ∘  f\) and \(id_{𝒦}\) are all interpretations,
they are equal if and only if they have the same action on all the primitives.
% AAA: This is only true because

\section{Guarded Kleene Algebra With Tests}

Guarded Kleene Algebra with Tests (GKAT)~\cite{Smolka_Foster_Hsu_Kappé_Kozen_Silva_2020} is an efficient fragment of Kleene algebra with tests. Specifically, GKAT utilizes tests as ``guards'' of the non-deterministic operations like \(+\) and \(*\), to make it more akin to \command{while}-programs. 
A GKAT expression over an alphabet \(K, B\) is defined as follow:
\begin{align*}
    \BExp ∋ e_b, f_b & ≜ 
        b ∈ B ∣ 1 ∣ 0 ∣ e_{b} ∧ f_{b} ∣ e_{b} ∨ f_{b} ∣ \overline{e_{b}} ; \\
    \GKAT ∋ e, f & ≜ 
        p ∈ K ∣ e_b ∣ e ⋅ f ∣  \comITE{e_b}{e}{f} ∣ \comWhile{e_b}{e} .
\end{align*}
Sometimes, like in~\cref{tab: thompson's construction}, we will use the compact notation \(e +_{e_b} f\) for \(\comITE{e_b}{e}{f}\) and \(e^{(e_b)}\) for \(\comWhile{e_b}{e}\).
In fact, these new \command{if}-statement and \command{while}-loop can be embedded back into KAT in the usual manner~\cite{Kozen_1997}:
\begin{align*}
    \comITE{e_b}{e}{f} & ≜ e_b ⋅ e + \overline{e_b} ⋅ f \\  
    \comWhile{e_b}{e} & ≜ (e_b ⋅ e)^* ⋅ \overline{e_b}
\end{align*}
This embedding into KAT gives Guarded Kleene Algebra with Tests a natural trace semantics: by computing the trace of the underlying KAT term; and a PSPACE decision procedure for trace-equivalence. Because the trace semantics of GKAT is obtained by the embedding into KAT we will overload the notation \(G(e)\) as the trace interpretation of a GKAT expression \(e\). 

The main advantage of GKAT, compare to KAT, is its efficient decision procedure. Such decision procedure is enabled by its deterministic nature, namely all the non-deterministic operations are guarded by tests. Thus, endowing GKAT a different automaton structure.

\begin{definition}[GKAT automaton]\label{def:GKAT-automaton}
    A GKAT automaton \(A ≜ ⟨S, δ, ŝ⟩\) over an alphabet \(K, B\) consists of a state set \(S\), a transition function \(δ: S → 2 + K × S\), and a start state \(ŝ ∈ S\).
\end{definition}
In the above definition \(2\) denotes \(\{\accept, \reject\}\), which represents accept and reject respectively.
Intuitively, the transition function of a GKAT automaton tells us, given a state $s$ and an atom $α$ accounting for the truth value of each primitive test, to either \emph{reject} the input, represented by $δ(s, α) = \reject$; \emph{accept} the input, represented by $δ(s, α) = \accept$; or to \emph{transition} to a new state in $S$ after executing an action from $K$, represented by $δ(s, α) ∈ K × S$.

People familiar with coalgebra~\cite{jacobs_IntroductionCoalgebraMathematics_2016,rutten_UniversalCoalgebraTheory_2000} might realize that this definition is a pointed coalgebra over the following \emph{signature} (or \emph{dynamics}):
\[2 + K × (-).\]

A GKAT automaton induces a guarded language in an intuitive manner.
\begin{definition}
 Given a GKAT automaton $A ≜ ⟨ S, δ, \hat{s} ⟩$, we define $G( - )_A: S → 𝒢$ as the (pointwise) smallest function satisfying the following rules for all $s ∈ S$ and $α ∈ \At$:
 \begin{mathpar}
  \inferrule{%
   δ(s, α) = \accept
  }{%
   α ∈ G( s )_A
  }
  \and
  \inferrule{%
   δ(s, α) = (p, s') \\
   w ∈ G( s' )_A
  }{%
   αpw ∈ G( s )_A
  }
 \end{mathpar}
 Finally, we define the guarded language semantics of $A$ by setting $G( A ) = G( \hat{s} )_A$.
\end{definition}

\citet{Smolka_Foster_Hsu_Kappé_Kozen_Silva_2020} was able to design a efficient trace equivalence decision procedure based on the classical Hopcroft-Karp algorithm~\cite{hopcroft_LinearAlgorithmTesting_1971}. 
We record this result below.

\begin{theorem}[Decidability for GKAT~\cite{Schmid_Kappé_Kozen_Silva_2021}]
    Given two finite GKAT automata $A₀$ and $A₁$, it is decidable whether they represent the same guarded language, i.e., whether $G( A₀ ) = G( A₁ )$.
    The algorithm to do this has a complexity that is nearly-linear\footnote{$𝒪(\hat{α}(n))$, where $\hat{α}$ is the inverse Ackermann function; c.f.~\cite{Tarjan75}.} in the total number of states.
\end{theorem}

When given two expression \(e\) and \(f\), the decision procedure will first convert it into two trace equivalent automaton via the \emph{Thompson's Construction}~\cite{Smolka_Foster_Hsu_Kappé_Kozen_Silva_2020}. Concretely, the Thompson's construction give rise to two GKAT automata \(A_e\) and \(A_f\) s.t.
\[G(A_e) = G(e); \qquad G(A_f) = G(f).\]
Then, apply the efficient decision procedure of GKAT automata to decide the trace-equivalence of \(A_e\) and \(A_f\)



