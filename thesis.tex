% BU ECE template for MS thesis and PhD dissertation.
%
%==========================================================================%
% MAIN PREAMBLE 
%==========================================================================%
\documentclass[12pt,letterpaper]{report}          % Single-sided printing for the library
%\documentclass[12pt,twoside]{report} % Double-sided printing
\usepackage[intlimits]{amsmath}
\usepackage{amsfonts,amssymb}
\usepackage{mathtools}
\DeclareSymbolFontAlphabet{\mathbb}{AMSb}
%\usepackage{natbib}
%\usepackage{apalike}
\usepackage{float}
\usepackage[bf]{caption}       
\usepackage{fancyhdr}
%\usepackage{fancyheadings}
\usepackage{fancybox}
\usepackage{ifthen}
\usepackage{bu_ece_thesis}
\usepackage{url}
\usepackage{lscape,afterpage}
\usepackage{xspace}
\usepackage{epstopdf} 
\usepackage{subfig}


%==========================================================================%
%%% graphicx and pdf creation
\usepackage{graphicx}
\usepackage{appendix}
%\usepackage{psfrag}
%\DeclareGraphicsExtensions{.eps}   % extension for included graphics
%\usepackage{thumbpdf}              % thumbnails for ps2pdf
%\usepackage[ps2pdf,                % hyper-references for ps2pdf
%bookmarks=true,%                   % generate bookmarks ...
%bookmarksnumbered=true,%           % ... with numbers
%hypertexnames=false,%              % needed for correct links to figures !!!
%breaklinks=true,%                  % breaks lines, but links are very small
%linkbordercolor={0 0 1},%          % blue frames around links
%pdfborder={0 0 112.0}]{hyperref}%  % border-width of frames 
%                                   % will be multiplied with 0.009 by ps2pdf
%\hypersetup{
%  pdfauthor   = {Joe Graduate <joe.graduate@bu.edu>},
%  pdftitle    = {dissertation.pdf},
%  pdfsubject  = {doctoral dissertations},
%  pdfkeywords = {mathematics, science, technology},
%  pdfcreator  = {LaTeX with hyperref package},
%  pdfproducer = {dvips + ps2pdf}
%}
%==========================================================================%
% customized commands can be placed here
%\newcommand{\figref}[1]{Figure~\ref{#1}}
%\newcommand{\chapref}[1]{Chapter~\ref{#1}}
%\newcommand{\latex}{\LaTeX\xspace}
%==========================================================================%
\usepackage[dvipsnames]{xcolor}
\usepackage{hyperref} 
\hypersetup{breaklinks=true,colorlinks=true,linkcolor=blue,urlcolor=magenta,citecolor=cyan}
\usepackage{breakurl}
\usepackage{algorithm}

%==========================================================================%
% MY PACKAGES
%==========================================================================%

% biblatex
\usepackage[style=numeric,sorting=ynt]{biblatex}
\addbibresource{./thesis.bib} 
\newcommand{\citet}[1]{\Citeauthor{#1}~\cite{#1}}

% theorems
\usepackage{amsthm}
\newtheorem{corollary}{Corollary}
\newtheorem{theorem}{Theorem}
\newtheorem{lemma}{Lemma}
\newtheorem{definition}{Definition}
\newtheorem*{definition*}{Definition}
\newtheorem{example}{Example}
\newtheorem{remark}{Remark}

% For adding inline comments in the text.
\usepackage[margin=false,inline=true]{fixme}
\FXRegisterAuthor{aaa}{anaaa}{\color{cyan}AAA}
\FXRegisterAuthor{mg}{anmg}{\color{red}MG}
\FXRegisterAuthor{cz}{ancz}{\color{orange}CZ}
% \newcommand{\aaa}[1]{\aaanote{#1}}
% \newcommand{\mg}[1]{\mgnote{#1}}
% \newcommand{\cz}[1]{\cznote{#1}}
\newcommand{\aaa}[1]{}
\newcommand{\mg}[1]{}
\newcommand{\cz}[1]{}

\usepackage{annotate-equations}

% graphs
\usepackage{tikz}
\usetikzlibrary{shapes.geometric, positioning, arrows, matrix}

% commutative diagram
\usepackage{tikz-cd}

% inference rule
\usepackage{mathpartir}

% ref
\usepackage{hyperref}
\usepackage{cleveref}
\crefname{ineq}{inequality}{inequalities}
\creflabelformat{ineq}{#2{\upshape(#1)}#3} 
\crefname{equiv}{equivalence}{equivalences}
\creflabelformat{equiv}{#2{\upshape(#1)}#3} 
\crefname{prog}{program}{programs}
\Crefname{prog}{Program}{Programs}


% item spacing
\usepackage{enumitem}

% unicode math symbols
\usepackage{fontspec}
\usepackage{unicode-math}
% support for hat, overline, underline, vec, and sim combining charactors
\directlua{
  local func = luatexbase.new_luafunction'afteracc'
  token.set_lua('afteracc', func, 'protected')

  local nest = tex.nest
  local noad_id = node.id'noad'
  local accent_id = node.id'accent'
  local math_char_id = node.id'math_char'

  lua.get_functions_table()[func] = function()
    local level = nest.top
    local last = level.tail
    if not (last and last.id == noad_id) then
      tex.error'I can only put accents on simple noads.'
      return
    end
    if last.sub or last.sup then
      tex.error'If you want accents on a superscript or subscript, please use braces.'
      return
    end
    local acc = node.new(accent_id, 1)
    acc.nucleus = last.nucleus
    last.nucleus = nil
    acc.accent = node.new(math_char_id)
    acc.accent.fam, acc.accent.char = 0, token.scan_int()

    level.tail = last.prev
    level.head = node.remove(level.head, last)
    node.flush_node(last)

    node.write(acc)
  end
}
\AtBeginDocument{
\begingroup
  \def\UnicodeMathSymbol#1#2#3#4{%
    \ifx#3\mathaccent
      \def\mytmpmacro{\afteracc#1 }%
      \global\letcharcode#1=\mytmpmacro
      \global\mathcode#1="8000
    \fi
  }
  \input{unicode-math-table}
\endgroup
}
% math font, this is needed to render several math command
% \setmathfont{STIX Two Math}

%==========================================================================%
% Macros
%==========================================================================%

% math
\DeclareMathOperator{\post}{\mathrm{post}}
\DeclareMathOperator{\dom}{\mathrm{dom}}
\DeclareMathOperator{\cod}{\mathrm{cod}}
\DeclareMathOperator{\Img}{\mathbf{Im}}
\DeclareMathOperator{\op}{\mathrm{op}}
\newcommand{\injL}{\mathrm{inj}ₗ}
\newcommand{\injR}{\mathrm{inj}ᵣ}
\newcommand{\At}{\mathbf{At}}
\newcommand{\true}{\mathrm{true}}
\newcommand{\false}{\mathrm{false}}
\newcommand{\accept}{\mathbf{acc}}
\newcommand{\reject}{\mathbf{rej}}

% theories
\newcommand{\BExp}{\mathbf{\mathrm{BExp}}}
\newcommand{\KA}{\ensuremath{\mathsf{KA}}}
\newcommand{\KAT}{\ensuremath{\mathsf{KAT}}}
\newcommand{\GKAT}{\ensuremath{\mathsf{GKAT}}}
\newcommand{\CFGKAT}{\ensuremath{\mathsf{CFKAT}}}
\newcommand{\TopKAT}{\ensuremath{\mathsf{TopKAT}}}
\newcommand{\REL}{\ensuremath{\mathbf{\mathrm{REL}}}}
\newcommand{\TopREL}{\ensuremath{\mathbf{\mathrm{TopREL}}}}
\newcommand{\TopGREL}{\ensuremath{\mathbf{\mathrm{TopGREL}}}}
\newcommand{\FailTopREL}{\ensuremath{\mathsf{FailTopREL}}}
\newcommand{\fail}{\mathtt{fail}}
\newcommand{\okState}{\mathinner{ok}}
\newcommand{\errState}{\mathinner{er}}

% program commands 
\newcommand{\command}[1]{\ensuremath{\mathtt{#1}}}
\newcommand{\comSkip}{\command{skip}}
\newcommand{\comError}{\command{error ()}}
\newcommand{\comAssert}[1]{\command{assert}~#1}
\newcommand{\comITE}[3]{\command{if}~#1~\command{then}~#2~\command{else}~#3}
\newcommand{\comWhile}[2]{\command{while}~#1~\command{do}~#2}
\newcommand{\comAssign}[2]{#1:= #2}
\newcommand{\comIT}[2]{\command{if}~#1~\command{then}~#2}
\newcommand{\comLabel}[1]{\command{label}~#1}
\newcommand{\comBrk}{\command{break}}
\newcommand{\comRet}{\command{return}}
\newcommand{\comGoto}[1]{\command{goto}~#1}

\newcommand{\incorTriple}[3]{[#1]~#2~[#3]}
\newcommand{\hoareTriple}[3]{\{#1\}~#2~\{#3\}}


%==========================================================================%
% BEGIN
%==========================================================================%
\begin{document}

% The preliminary pages
% This file contains all the necessary setup and commands to create
% the preliminary pages according to the buthesis.sty option.

\title{Two Variants of Kleene Algebra And Their Applications}

\author{Cheng Zhang}

% Type of document prepared for this degree:
%   1 = Master of Science thesis,
%   2 = Doctor of Philosophy dissertation.
\degree=2

\prevdegrees{B.A., Wheaton College, 2018}

\department{Department of Computer Science}

% Degree year is the year the diploma is expected, and defense year is
% the year the dissertation is written up and defended. Often, these
% will be the same, except for January graduation, when your defense
% will be in the fall of year X, and your graduation will be in
% January of year X+1
\defenseyear{2024}
\degreeyear{2024}

% For each reader, specify appropriate label {First, Second, Third},
% then name, and title. IMPORTANT: The title should be:
%   "Professor of Electrical and Computer Engineering",
% or similar, but it MUST NOT be:
%   Professor, Department of Electrical and Computer Engineering"
% or you will be asked to reprint and get new signatures.
% Warning: If you have more than five readers you are out of luck,
% because it will overflow to a new page. You may try to put part of
% the title in with the name.
\reader{First}{Marco Gaboardi, PhD}{Associate Professor Of Computer Science}
\reader{Second}{Arthur A. de Amorim, PhD}{Assistant Professor Of Computer Science, RIT}

% The Major Professor is the same as the first reader, but must be
% specified again for the abstract page. Up to 4 Major Professors
% (advisors) can be defined. 
\numadvisors=1
\majorprof{Marco Gaboardi, PhD}{Associate Professor Of Computer Science}
% \majorprofb{First M. Last, PhD}{{Professor of Computer Science}}
%\majorprofc{First M. Last, PhD}{{Professor of Astronomy}}
%\majorprofd{First M. Last, PhD}{{Professor of Biomedical Engineering}}

%%%%%%%%%%%%%%%%%%%%%%%%%%%%%%%%%%%%%%%%%%%%%%%%%%%%%%%%%%%%%%%%  

%                       PRELIMINARY PAGES
% According to the BU guide the preliminary pages consist of:
% title, copyright (optional), approval,  acknowledgments (opt.),
% abstract, preface (opt.), Table of contents, List of tables (if
% any), List of illustrations (if any). The \tableofcontents,
% \listoffigures, and \listoftables commands can be used in the
% appropriate places. For other things like preface, do it manually
% with something like \newpage\section*{Preface}.

% This is an additional page to print a boxed-in title, author name and
% degree statement so that they are visible through the opening in BU
% covers used for reports. This makes a nicely bound copy. Uncomment only
% if you are printing a hardcopy for such covers. Leave commented out
% when producing PDF for library submission.
%\buecethesistitleboxpage

% Make the titlepage based on the above information.  If you need
% something special and can't use the standard form, you can specify
% the exact text of the titlepage yourself.  Put it in a titlepage
% environment and leave blank lines where you want vertical space.
% The spaces will be adjusted to fill the entire page.
\maketitle
\cleardoublepage

% The copyright page is blank except for the notice at the bottom. You
% must provide your name in capitals.
\copyrightpage
\cleardoublepage

% Now include the approval page based on the readers information
% Once the approval page is approved by the Mugar Library staff, please
% comment out the "\approvalpagewithcomment" line and uncomment "\approvalpage"
% \approvalpagewithcomment
\approvalpage
\cleardoublepage

% Here goes your favorite quote. This page is optional.
\newpage
%\thispagestyle{empty}
\phantom{.}
\vspace{4in}

\begin{singlespace}
\begin{quote}
  \textit{Facilis descensus Averni;}\\
  \textit{Noctes atque dies patet atri janua Ditis;}\\*
  \textit{Sed revocare gradum, superasque evadere ad auras,}\\
  \textit{Hoc opus, hic labor est.}\hfill{Virgil (from Don's thesis!)}
\end{quote}
\end{singlespace}

% \vspace{0.7in}
%
% \noindent
% [The descent to Avernus is easy; the gate of Pluto stands open night
% and day; but to retrace one's steps and return to the upper air, that
% is the toil, that the difficulty.]

\cleardoublepage

% The acknowledgment page should go here. Use something like
% \newpage\section*{Acknowledgments} followed by your text.
\newpage
\section*{\centerline{Acknowledgments}}
Here go all your acknowledgments. You know, your advisor, funding agency, lab
mates, etc., and of course your family.

As for me, I would like to thank Jonathan Polimeni for cleaning up old LaTeX
style files and templates so that Engineering students would not have to suffer
typesetting dissertations in MS Word. Also, I would like to thank IDS/ISS
group (ECE) and CV/CNS lab graduates for their contributions and tweaks to this
scheme over the years (after many frustrations when preparing their final
document for BU library). In particular, I would like to thank Limor Martin who
has helped with the transition to PDF-only dissertation format (no more printing
hardcopies -- hooray !!!)

The stylistic and aesthetic conventions implemented in this LaTeX
thesis/dissertation format would not have been possible without the help from
Brendan McDermot of Mugar library and Martha Wellman of CAS.

Finally, credit is due to Stephen Gildea for the MIT style file off which this
current version is based, and Paolo Gaudiano for porting the MIT style to one
compatible with BU requirements.

\vskip 1in

\noindent
Janusz Konrad\\
Professor\\
ECE Department
\cleardoublepage

% The abstractpage environment sets up everything on the page except
% the text itself.  The title and other header material are put at the
% top of the page, and the supervisors are listed at the bottom.  A
% new page is begun both before and after.  Of course, an abstract may
% be more than one page itself.  If you need more control over the
% format of the page, you can use the abstract environment, which puts
% the word "Abstract" at the beginning and single spaces its text.

\begin{abstractpage}
% ABSTRACT

Kleene Algebra (KA) is an equational system celebrated for its decidability 
and completeness with respect to regular language equalities.
Because of the desirable properties of Kleene Algebra, 
numerous extensions were developed to reason about
network system~\cite{Anderson_Foster_Guha_Jeannin_Kozen_Schlesinger_Walker_2014,Foster_Kozen_Milano_Silva_Thompson_2015, Smolka_Kumar_Kahn_Foster_Hsu_Kozen_Silva_2019},
concurrent programs~\cite{hoare_ConcurrentKleeneAlgebra_2009,Kappé_Brunet_Silva_Wagemaker_Zanasi_2020,Kappé_Brunet_Silva_Zanasi_2018}, 
probabilistic systems~\cite{mciver_UsingProbabilisticKleene_2006, McIver_Rabehaja_Struth_2011}, 
relational verification~\cite{Antonopoulos_Koskinen_Le_Nagasamudram_Naumann_Ngo_2022},
and program schematology~\cite{Angus_Kozen_2001}.
In this thesis, we focus on two variants of Kleene Algebra with real-world applications.

The first system, Kleene Algebra with tests and top (TopKAT), 
was developed to perform domain and reachablity reasoning.
We showed the conventional extension of Kleene Algebra with tests, 
despite able to encode Hoare logic, is inadequate for domain reasoning.
This leads to our development of TopKAT, which is complete for domain reasoning.
TopKAT was able to soundly encode both propositional incorrectness and Hoare logic~\cite{OHearn_2020,Hoare69},
offering better complexity bound than alternative frameworks~\cite{Möller_O’Hearn_Hoare_2021, Sedlár_2023}.
The our completeness proof for TopKAT relies heavily on a technique called \emph{reduction},
we showed that the reduction from TopKAT to KAT satisfy nice properties that enable us to generate complete interpretations for TopKAT for free, and also gives us a complete decision procedure with minimal effort.

The second system, control-flow Guarded Kleene Algebra with Tests (CF-GKAT) verifies control-flow transformations.
Guarded Kleene Algebra with Tests~\cite{Smolka_Foster_Hsu_Kappé_Kozen_Silva_2020} provides a robust system that is not only sound and complete with respect to trace equivalence, but also enjoys a efficient decision procedure. 
Yet, GKAT remains insufficient as a system to verify several well-known control-flow algorithms~\cite{erosa-hendren-1994,yakdan_NoMoreGotos_2015,kozen_BohmJacopiniTheorem_2008a}, because it lacks important control-flow structures like indicator variable and non-local control-flow structures like \(\comBrk\), \(\comRet\), and \(\command{goto}\).
To obtain CF-GKAT, we extended the syntax and semantics of GKAT to incorporate these essential features. We have developed a efficient decision procedure for CF-GKAT program utilizing \emph{CF-GKAT automata}, an automata model that closely emulates CF-GKAT programs, and can be efficiently lowered into GKAT automata. 
Furthermore, this decision procedure is sound and complete: the algorithm will output true if and only if the two input programs are trace equivalent.
\end{abstractpage}
\cleardoublepage

% Now you can include a preface. Again, use something like
% \newpage\section*{Preface} followed by your text

% Table of contents comes after preface
\tableofcontents
\cleardoublepage

% If you do not have tables, comment out the following lines
\newpage
\listoftables
\cleardoublepage

% If you have figures, uncomment the following line
\newpage
\listoffigures
\cleardoublepage

% List of Abbrevs is NOT optional (Martha Wellman likes all abbrevs listed)
\chapter*{List of Abbreviations}

\begin{center}
  \begin{tabular}{lll}
    \hspace*{2em} & \hspace*{1in} & \hspace*{4.5in} \\
    CF-GKAT  & \dotfill & Control-Flow GKAT \\
    FailTopKAT & \dotfill & TopKAT with Failure \\
    FailTopREL & \dotfill & Relational TopKAT with Failure \\
    GKAT   & \dotfill & Guarded Kleene Algebra with Tests \\
    HL & \dotfill & Hoare Logic \\
    IL & \dotfill & Incorrectness Logic\\
    KA  & \dotfill & Kleene Algebra \\
    KAT & \dotfill & Kleene Algebra With Tests \\
    REL & \dotfill & Relational KAT \\
    TopGREL & \dotfill & General Relational TopKAT\\
    TopKAT & \dotfill & Kleene Algebra With Top and Tests\\
    TopREL & \dotfill & Relational TopKAT \\
  \end{tabular}
\end{center}
\cleardoublepage

% END OF THE PRELIMINARY PAGES

\newpage
\endofprelim
        
\cleardoublepage

% Bodies
\chapter{Introduction}
\label{chapter:introduction}
\thispagestyle{myheadings}

\section{An overview on Kleene Algebra}
\label{sec:history}

Kleene algebra, named after the eminent mathematician Stephen Cole Kleene, represents a pivotal development in mathematical logic, computer science, and formal languages. 
Originating in the mid-20th century, Kleene algebra emerged from Kleene's seminal work on regular sets and expressions~\cite{Kleene_1956}, where he introduced algebraic structures to formalize fundamental operations on regular languages. 
Kleene left a important open question about Kleene Algebra: whether there exists \emph{complete} algebraic system for regular language equality: an algebraic system that is capable of derive all the language equalities of regular expression.
There are numerous systems proposed and are closely related~\cite{Kozen_1990}, yet the modern Kleene Algebra and its completeness proof is often attributed to Kozen~\cite{Kozen_2001,Kozen_1994}.

Given the close relation to regular languages, it is no surprise that the study of Kleene Algebra often makes heavy use of automata theory.
The automata perspective underpins the other important property of Kleene Algebra, that is, decidability.
Specifically, the behavioral equivalence of two automata (or more generally) coalgebra can be obtained by bisimulation on the automata~\cite{rutten_UniversalCoalgebraTheory_2000}.
And bisimulation usually is decidable for finite automata.
This is later characterized by the bialgebraic approach~\cite{jacobs_BialgebraicReviewDeterministic_2006}, although we will not dive deep into coalgebra, many of the techniques used in this paper are inspired by this framework.

The decidability and completeness of Kleene Algebra has inspired a suite of applications in programming languages and verifications; specifically, in the areas of network system~\cite{Anderson_Foster_Guha_Jeannin_Kozen_Schlesinger_Walker_2014,Foster_Kozen_Milano_Silva_Thompson_2015, Smolka_Kumar_Kahn_Foster_Hsu_Kozen_Silva_2019},
concurrent programs~\cite{hoare_ConcurrentKleeneAlgebra_2009,Kappé_Brunet_Silva_Wagemaker_Zanasi_2020,Kappé_Brunet_Silva_Zanasi_2018}, 
probabilistic systems~\cite{mciver_UsingProbabilisticKleene_2006, McIver_Rabehaja_Struth_2011}, 
relational verification~\cite{Antonopoulos_Koskinen_Le_Nagasamudram_Naumann_Ngo_2022},
and program schematology~\cite{Angus_Kozen_2001}.
Notice that despite the decidability of KA and KAT, the decision procedure is PSPACE-complete~\cite{Cohen_Kozen_Smith_1999}, which should be infeasible for large systems.
However, in real-world testing, network systems based on KAT was able to out-perform state of the art network verifier~\cite{Smolka_Kumar_Kahn_Foster_Hsu_Kozen_Silva_2019}.
This phenomenon is explained by an efficient fragment of KAT named \emph{Guarded Kleene Algebra with Tests} (GKAT)~\cite{Smolka_Foster_Hsu_Kappé_Kozen_Silva_2020}. Concretely, the particular fragment of GKAT enjoys a more efficient automaton structure that enables fast equivalence checking.

\paragraph{Our Contributions}

\begin{itemize}
    \item Proved that KAT cannot encode incorrectness logic. Demonstrated the insufficiency of domain reasoning in KAT.
    \item Developed the system TopKAT, and proven it is sound and complete with respect to its trace/language interpretation and general relational model.
    \item Developed a new notion of reduction, which is based on homomorphism from the free algebra.
    \item Proved TopKAT is complete with respect to domain comparisons.
    \item Designed the syntax and semantic of CF-GKAT, enabling control-flow verification of \command{while}-programs.
    \item Designed the decision procedure of CF-GKAT based on GKAT automaton.
    \item Proving the decision procedure is sound and complete, and also enjoys nearly-linear time complexity, assuming the number of primitive tests are fixed.
\end{itemize}

\section{Technical Background}

\subsection{Extensions of Kleene algebra And Their Models}

A \emph{Kleene algebra} is an idempotent semiring with a star operation, written
$p^*$, that satisfies the following \emph{unfolding}, \emph{left induction},
and \emph{right induction} rules:
\[
    p^* = 1 + p p^* = 1 + p^* p, \\
    p r + q  \leq  r  \implies  p^* q  \leq  r, \\
    r p + q  \leq  r  \implies  q p^*  \leq  r;
\]
the ordering here is the conventional ordering in idempotent semirings: \(p  \leq  q  \triangleq  p + q = q.\)
It is known that the right-hand version of unfolding and induction rule 
can be removed while preserving the same equational theory~\cite{Kozen_Silva_2020}.
Yet, we will focus on the standard definition of KA in this paper.
\begin{lemma}\label{the: well known fact about KA}
    Following are well-known facts in Kleene algebra
    \begin{itemize}
        \item All the Kleene algebra operations preserve order.
        \item The following equations are true for the star operation:
              \[ p^*  \cdot  p^* = p^* \\ (p^*)^* = p^*.\]
    \end{itemize}
\end{lemma}

A Kleene algebra with tests (KAT) is a Kleene algebra with an embedded Boolean algebra,
where the conjunction, disjunction, and identities in the Boolean algebra coincide with 
the addition, multiplication, and the identities of Kleene algebra.  
We refer to elements of this embedded Boolean algebra as \emph{tests}.

Given an algebraic theory, we can construct its \emph{free model} 
over a finite set \( Σ \), 
called the \emph{alphabet}~\cite{burrisCourseUniversalAlgebra1981}.  
The free model consists of all the terms formed by \( Σ \) modulo  
provable equivalences of the algebra. The operations of the free model are obtained 
by lifting the term-level operations to equivalence classes.

The above construction can be extended to the case of KAT and TopKAT, 
suppose that we are given two disjoint finite sets $K$ (the
\emph{action alphabet}) and $B$ (the \emph{test alphabet}).  Elements of $K$ and
$B$ are called \emph{primitive actions and primitive tests}, respectively. 
KAT terms over the alphabet \(K, B\) are defined with the following grammar:
\[e  ≜  b  ∈  B  ∣  p  ∈  K  ∣  1  ∣  0  ∣  e_1 + e_2  ∣  e_1  ⋅  e_2  ∣  e^*  ∣  \overline{e_b},\]
where \(e_b\) does not contain primitive actions.
The \emph{free KAT} over \(K, B\), written $\KAT_{K,B}$, 
consists of terms over \(K, B\) modulo provable KAT equivalences.  
The tests of the free KAT are Boolean terms, i.e. terms formed by
primitive tests and Boolean operations modulo Boolean axioms.  A similar
construction applies to TopKAT, where an additional symbol \( ⊤ \) was added 
as the largest element in the theory; we denote the free TopKAT over $K,B$ as
\(\TopKAT_{K, B}\).  We sometimes omit the alphabets \(K\) and \(B\) when they
are irrelevant or can be inferred.

In the paper, we frequently consider terms modulo provable equalities, i.e. in the
context of its corresponding free model.  For example, given \(e_1, e_2  \in  \KAT\),
we will say \(e_1 = e_2\) when they are provably equal using the theory of KAT.
Although the free model seems trivial, it leads to simpler and more modular
proofs of some properties of algebraic theories, as we will see in~\Cref{chapter:TopKAT}.

Other important models that we will use in this paper are language (Top)KATs and
relational (Top)KATs, which we review here.  An \emph{atom} (short for ``atomic
test'') over a test alphabet \(B = \{b_{1}, b_{2},  … , b_{n}\}\) is a sequence of the form
\[\hat{b_{1}}  \cdot  \hat{b_{2}}  \cdot   \cdots   \cdot \hat{b_{n}} \text{ where } \hat{b_{i}}  \in  \{b_{i}, \bar{b_{i}}\}.\] 
We denote atoms as \( α ,  β ,  γ ,  … \) and the set of all atoms as \(\At\).

A \emph{guarded string} (or \emph{guarded word}) over \(K, B\) is an alternation 
between atoms and primitive actions that starts and ends in atoms: 
\[ \alpha _{0}p_{1} \alpha _{1}  \cdots  p_{n}  \alpha _{n} \text{ where } p_{i}  \in  K,  \alpha _{i}  \in  \At;\] 
where each action is ``guarded'' by an atom.
A guarded string is similar to a program trace. These traces record the initial, intermediate and final machine states observed during execution, as well as the actions that occurred between those states. % chktex 36
Because the value of the indicator variable matters only for control flow, we do not consider indicators to be part of the machine state; hence, machine states in a guarded word are drawn from $\At$.

\begin{example} 
    Let $B = \{ b₁, b₂ \}$ and $K = \{ p₁, p₂ \}$.
    Now the guarded word \[b₁ \overline{b₂} ⋅ p₁ ⋅ \overline{b₁} b₂ ⋅ p₂ ⋅ \overline{b₁} \overline{b₂}\] represents a program trace that starts out in a machine state where $b₁$ is true (but $b₂$ is not).
    The program then executes the action $p₁$, after which $b₂$ is true (but $b₁$ is not).
    Finally, the program goes on to execute the action $p₂$, and halts in a state where neither $b₁$ nor $b₂$ is true.
\end{example}

We denote the set of all guarded
strings over alphabet \(K, B\) as \(GS_{K, B}\), and we will omit the alphabet
\(K, B\) when it is irrelevant or can be inferred from context.  The notation
\( α  w\) denotes a guarded string starting with atom \( α \) with the rest of the string
\(w\); similarly, \(w  α \) denotes a guarded string that ends with atom \( α \) with
rest of the string being \(w\).

\begin{definition}[Language/trace KAT~\cite{Kozen_Smith_1997}]
  The \emph{language KAT} (also called ``\emph{trace KAT}'') over an alphabet \(K, B\) is
  denoted as \(\mathcal{G}_{K, B}\), or simply \(\mathcal{G}\) if no confusion can arise.

  The elements are sets of guarded strings (called \emph{guarded languages}), 
  and the tests are sets of atoms.
  The additive identity 0 is the empty set, and the multiplicative identity 1 is
  the set of all the atoms \(\At\).  The addition operator is set union, and the
  multiplication operator is defined as follows:
    \[W_{1}  ⋄  W_{2}  ≜  \{w_{1}  α  w_{2}  ∣  w_{1}  α   ∈  W_{1},  α  w_{2}  ∈  W_{2}\}.\]
    The star operation is defined non-deterministically 
    iterating the multiplication operator:
    \[W^*  ≜   ⋃_{i  ∈ ℕ} W^i \text{ where } W^0 = \At, W^{k+1} = W  ⋄  W^k.\]
\end{definition}


Another useful type of KAT are relational ones, where each element is a relation
\(R  \subseteq  X  \times  X\) over a fixed set \(X\).  In applications, the set $X$ typically
represents the set of all possible program states, and each relation $R$
represents a program by relating each possible input to the corresponding
output.

\begin{definition}[Relational KAT]
  A relational KAT is a KAT $\mathcal{R}$ consists of relations over a fixed set \(X\) 
  (though $\mathcal{R}$ need not contain every relation over $X$),
  and it is closed under the following operations. 
  The tests are all the relations that are subsets of the identity relation.  
  The additive identity 0 is the empty set, and
  the multiplicative identity is the identity relation:
  \[1  \triangleq  \{(x, x)  \mid  x  \in  X\}.\] The addition operator is set union, and the
  multiplication operation is relational composition:
  \[R_{1} ; R_{2} = \{(x, z)  \mid   \exists  y  \in  X, (x, y)  \in  R_{1}, (y, z)  \in  R_{2}\}.\] 
  Finally, the star operation is defined as:
  \[R^*  \triangleq   \bigcup _{i  \in  \mathbb{N}} R^i \text{ where } R^0 = 1, R^{k+1} = R ; R^k.\] We denote the
  class of all relational KATs as \(\REL\).
\end{definition}

We are also interested in maps between models:
A \emph{KAT homomorphism} \(h\) is a map between two KATs \(\mathcal{K}\) and \(\mathcal{K}'\)
s.t. it preserves the sorts and operations:
given a test \(b\) in \(\mathcal{K}\) then \(h(b)\) is a test in \(\mathcal{K}'\);
and all the KAT operations (complement, identities, addition, multiplication, and star) are preserved:
\begin{align*}
    h & : 𝒦  →  𝒦'\\
    h(b̄) & = \overline{f(b)} \\  
    h(1) & = 1 \\  
    h(0) & = 0 \\
    h(p + q) & = h(p) + h(q) \\  
    h(p ⋅ q) & = h(p) ⋅ h(q) \\  
    h(p^*) & = h(p)^*.
\end{align*}

\subsection{Interpretation, Completeness, and Injectivity}\label{sec: completeness background}

Consider a KAT equation such as \(p ⋅ b ⋅ b̄ = 0\). To determine its
validity in a particular KAT \(\mathcal{K}\), we need to assign meaning to it by
interpreting each primitive as an element in \(\mathcal{K}\); that is, by defining a map
\(\hat{I}\) of type \(K + B  →  𝒦\).  Such a map \(\hat{I}: K + B  →  𝒦\) induces a
unique KAT homomorphism \(I : \KAT_{K,B}  →  \mathcal{K}\) inductively defined on the term 
as follows:
\begin{equation}
    \begin{aligned}
        I(p)       &  \triangleq  \hat{I}(p)    & \text{ where } p  \in  K + B \\
        I(\overline{e_b}) &  \triangleq  \overline{I(e_b)} 
            & \text{\(e_b\) does not contain primitive actions} \\
        I(e_1 + e_2) &  \triangleq  I(e_1) + I(e_2)                     \\
        I(e_1  \cdot  e_2) &  \triangleq  I(e_1)  \cdot  I(e_2)                     \\
        I(e^*)     &  \triangleq  I(t)^*
    \end{aligned}
\end{equation}
In fact, every KAT homomorphism from a free model arises this way: there is a
bijection between functions of type \(K + B  \to  \mathcal{K}\) and KAT homomorphisms of type
\(\KAT_{K, B}  \to  \mathcal{K}\), for any KAT \(\mathcal{K}\).  
Because the homomorphism \(I\) and the function \(\hat{I}\) are equivalent, 
we will refer to them interchangeably as \emph{KAT interpretations} 
and denote both of them as \(I\).

The above result enables us to define a homomorphism from the free KAT just by
defining its action on the primitives; saving us time to check the equations
that a homomorphism must satisfy.  It also allows us to prove that two
interpretations are equal by arguing that they map the primitives to
equal values.

Given a KAT \(𝒦\), and two terms \(e_1, e_2  ∈  \KAT_{K, B}\) we say that \(𝒦  ⊧  e_1 = e_2\) if
\[ ∀  I : \KAT_{K, B}  →  𝒦, I(e_1) = I(e_2).\] In particular, 
for two terms in the free model \(e_1, e_2  ∈  \KAT_{K, B}\),
\(\KAT_{K, B}  ⊧  e_1 = e_2\) is equivalent to \(e_1 = e_2\).  
For a collection of models \(\mathsf{K}\), 
we say that \(𝖪  ⊧  e_1 = e_2\) if for all \(\mathcal{K}  ∈  𝖪\),
\(𝒦  ⊧  e_1 = e_2\).  For example, \(\REL  ⊧  e_1 = e_2\) means that \(e_1 = e_2\) is
valid in all relational KATs.  All the above notations and terminologies can be
similarly extended to TopKAT.

Theories like KAT and TopKAT are designed to model practical
programs, so it is important to know if they can model all the desirable
equations between programs. If the theory of KAT can derive all the equalities
for a particular interpretation \(I\), namely:
\[\KAT_{K, B}  \models  e_1 = e_2  \iff  I(e_1) = I(e_2),\]
we say that the theory of KAT is \emph{complete} with respect to \(I\).
Recall that \(\KAT_{K, B}  \models  e_1 = e_2\) is equivalent to \(e_1 = e_2\);  
thus, by definition, an interpretation \(I\) is complete if and only if it is injective.
One of such interpretation is the guarded string interpretation
\(G: \KAT_{K, B}  →  𝒢_{K, B}\)~\cite{Kozen_Smith_1997},
defined by lifting the following action on the primitives:
\[
    G(b) = \{ α   ∣  \text{\(b\) appears positively in \( α \)}\}, \\
    G(p) = \{ α  p  β   ∣   α ,  β   ∈  \At\}.
\]

In several previous works, the term ``free model'' refers to the range (set of reachable elements) of a complete interpretation.  
Since a complete interpretation is an injective homomorphism, such interpretation induces an isomorphism on its range, thus our definition of free model is equivalent to these definitions.

Many previous proofs can also be explained by seeing complete interpretations as injective homomorphisms: the proof for completeness of relational KATs constructs an injective homomorphism $h$ from a language KAT into a relational
KAT~\cite{Kozen_Smith_1997}.  
Since both \(G\) and \(h\) are injective homomorphisms, \(h  ∘  G\) is also an injective homomorphism, hence a complete interpretation.  Since \(h  ∘  G\) is a relational interpretation:
\[\KAT_{K, B}  ⊧  e_1 = e_2  ⟹  \REL  ⊧  e_1 = e_2  ⟹  h  ∘  G(e_1) = h  ∘  G(e_2);\]
then the completeness of \(h  ∘  G\) implies
\((h  ∘  G)(e_1) = (h  ∘  G)(e_2)  ⟺  \KAT_{K, B}  ⊧  e_1 = e_2\). Hence,
\[\KAT_{K, B}  ⊧  e_1 = e_2  ⟺  \REL  ⊧  e_1 = e_2,\]
i.e. the theory of KAT is complete with respect to relational KAT.

% Function/homomorphism composition will be used frequently in this paper,
% but sometimes simply writing \(f  \circ  g\) will be confusing without knowing
% the domain and codomain of function \(f\) and \(g\).
% Therefore, we will use the notation \(X \xrightarrow{g} Y \xrightarrow{f} Z\)
% to denote the composition of \(f: Y  \to  Z\) and \(g: X  \to  Y\) when it is desirable.

% AAA: I think the above notation is standard enough; we don't have to explain
% what it means.
% CZ: Removed

Besides using composition of injective homomorphisms, another technique commonly
used to prove injectivity is to construct a left inverse: 
if a (Top)KAT homomorphism \(f: 𝒦  →  𝒦'\) has a left inverse homomorphism \(g: 𝒦'  →  𝒦\) 
i.e. \(g  ∘  f = id_{𝒦}\), then \(f\) is injective.  
Notice that \(g\) does not need to be a homomorphism for \(f\) to be injective,
however, in the case where \(f\) is an interpretation, 
\(g\) being a homomorphism makes the equality \(g  ∘  f = id_{𝒦}\) easier to check.
Because both \(g  ∘  f\) and \(id_{𝒦}\) are all interpretations,
they are equal if and only if they have the same action on all the primitives.
% AAA: This is only true because

\section{Guarded Kleene Algebra With Tests}

Guarded Kleene Algebra with Tests (GKAT)~\cite{Smolka_Foster_Hsu_Kappé_Kozen_Silva_2020} is a efficient fragment of Kleene Algebra with tests. Specifically, GKAT uses tests to ``guard'' the non-deterministic operations like \(+\) and \(*\), to make it more akin to simply \command{while}-programs. 
A GKAT expression over an alphabet \(K, B\) is defined as follow:
\begin{align*}
    \BExp ∋ e_b, f_b & ≜ 
        b ∈ B ∣ 1 ∣ 0 ∣ e_{b} ∧ f_{b} ∣ e_{b} ∨ f_{b} ∣ \overline{e_{b}} ; \\
    \GKAT ∋ e, f & ≜ 
        p ∈ K ∣ e_b ∣ e ⋅ f ∣  \comITE{e_b}{e}{f} ∣ \comWhile{e_b}{e} .
\end{align*}
Sometimes, like in~\cref{tab: thompson's construction}, we will use the compact notation \(e +_{e_b} f\) for \(\comITE{e_b}{e}{f}\) and \(e^{(e_b)}\) for \(\comWhile{e_b}{e}\).
In fact, these new \command{if}-statement and \command{while}-loop can be embedded back into KAT in the usual manner~\cite{Kozen_1997}:
\begin{align*}
    \comITE{e_b}{e}{f} & ≜ e_b ⋅ e + \overline{e_b} ⋅ f \\  
    \comWhile{e_b}{e} & ≜ (e_b ⋅ e)* ⋅ \overline{e_b}
\end{align*}
This embedding into KAT gives Guarded Kleene Algebra with Tests a natural trace semantics: by computing the trace of the underlying KAT term; and a PSPACE decision procedure for trace-equivalence. Because the trace semantics of GKAT is obtained by the embedding into KAT we will overload the notation \(G(e)\) as the trace interpretation of a GKAT expression \(e\). 

The main advantage of GKAT, compare to KAT, is its efficient decision procedure. Such decision procedure is enabled by its deterministic nature, namely all the non-deterministic operations are guarded by tests. Thus, endowing GKAT a different automaton structure.

\begin{definition}[GKAT automaton]\label{def:GKAT-automaton}
    A GKAT automaton \(A ≜ ⟨S, δ, ŝ⟩\) over an alphabet \(K, B\) consists of a state set \(S\), a transition function \(δ: S → 2 + K × S\), and a start state \(ŝ ∈ S\).
\end{definition}
where \(2\) denotes \(\{\accept, \reject\}\), which represents accept and reject respectively.
Intuitively, the transition function of a GKAT automaton tells us, given a state $q$ and an atom $α$ accounting for the truth value of each primitive test, to either \emph{reject} the input, represented by $δ(q, α) = \reject$, \emph{accept} the input, represented by $δ(q, α) = \accept$, or to \emph{transition} to a new state in $S$ after executing an action from $K$, represented by $δ(q, α) ∈ K × S$.

People familiar with coalgebra~\cite{jacobs_IntroductionCoalgebraMathematics_2016,rutten_UniversalCoalgebraTheory_2000} might realize that this definition is a pointed coalgebra over the following \emph{signature} (or \emph{dynamics}):
\[2 + K × (-).\]

A GKAT automaton induces a guarded language in a fairly straightforward manner.
\begin{definition}
 Given a GKAT automaton $A ≜ ⟨ S, δ, \hat{s} ⟩$, we define $G( - )_A: S → 𝒢$ as the (pointwise) smallest function satisfying the following rules for all $s ∈ S$ and $α ∈ \At$:
 \begin{mathpar}
  \inferrule{%
   δ(s, α) = \accept
  }{%
   α ∈ G( s )_A
  }
  \and
  \inferrule{%
   δ(s, α) = (p, s') \\
   w ∈ G( s' )_A
  }{%
   αpw ∈ G( s )_A
  }
 \end{mathpar}
 Finally, we define the guarded language semantics of $A$ by setting $G( A ) = G( \hat{s} )_A$.
\end{definition}

\citet{Smolka_Foster_Hsu_Kappé_Kozen_Silva_2020} was able to design a efficient trace equivalence decision procedure based on the classical Hopcroft-Karp algorithm~\cite{hopcroft_LinearAlgorithmTesting_1971}

\begin{theorem}[Decidability for GKAT~\cite{Schmid_Kappé_Kozen_Silva_2021}]
    Given two finite GKAT automata $A₀$ and $A₁$, it is decidable whether they represent the same guarded language, i.e., whether $G( A₀ ) = G( A₁ )$.
    The algorithm to do this has a complexity that is nearly-linear\footnote{$𝒪(\hat{α}(n))$, where $\hat{α}$ is the inverse Ackermann function; c.f.~\cite{Tarjan75}.} in the total number of states.
\end{theorem}

When given two expression \(e\) and \(f\), the decision procedure will first convert it into two trace equivalent automaton via the \emph{Thompson's Construction}~\cite{Smolka_Foster_Hsu_Kappé_Kozen_Silva_2020}. Concretely, two GKAT automata \(A_e\) and \(A_f\) s.t.
\[G(A_e) = G(e); \qquad G(A_f) = G(f).\]
Then, apply the efficient decision procedure to decide the trace-equivalence of \(A_e\) and \(A_f\)




\cleardoublepage

\chapter{Domain Reasoning In TopKAT}
\label{chapter:TopKAT}
\thispagestyle{myheadings}

\section{Motivation and Overview}

% 1 - Program logics are a fundamental tool for program reasoning
% 2 - Hoare logic have guided the development of formal reasoning frameworks
% 3 - KAT subsumes Hoare Logic
% 4 - Incorrectness Logic is a new deductive logic for reasoning about programs
% 5 - Can we encode Incorrectness Logic in KAT
% 6 - We show that this is not possible
% 7 - We consider TopKat and show that we can encode incorrectness triples in it
% 8 - We consider Fail top kat for reasoning about failure
% 9 - We study the relation among different models. 

Since the seminal work of \citet{Floyd67} and \citet{Hoare69}, logic has become
an essential tool for program verification.  A program logic provides a system
of \emph{deduction rules} to prove \emph{Hoare triples} \(\hoareTriple{b}{p}{c}\),
where \(p\) is a program, and \(b\) and \(c\) are assertions describing the pre-
and post-conditions.  Such Hoare triples are (partial) correctness
specifications, which state that, if we run \(p\) on an initial state satisfying
\(b\), all the final states (if reached) will satisfy \(c\). In other words, \(c\)
\emph{over-approximates} the set of final states of \(p\) starting from \(b\). 

Despite their popularity, \emph{equational reasoning}, like Kleene Algebra with Tests has also prevailed in parallel.
The two approaches turned out to be not so different. \citet{Kozen_2000}
showed that KAT can express the validity of a Hoare triple as an equation, in
such a way that the deduction rules of a large fragment of Hoare Logic can be
obtained by equational reasoning.  \citeauthor{Kozen_2000}'s work
established a clear bridge between deductive and equational reasoning for
program verification, thus shedding light on the expressive power of KAT\@.

In this work, we are interested in extending this correspondence to other
deductive formalisms for reasoning about programs.  One such formalism is
\emph{incorrectness logic} (IL), a recent proposal
by~\citet{OHearn_2020} that relates to earlier works
by~\citet{devries_ReverseHoareLogic_2011}. 
Instead of correctness, as in the Floyd-Hoare tradition, the system
revolves around \emph{incorrectness specifications} of the form
\(\incorTriple{b}{p}{c}\), which state that \(p\) can produce \emph{any} final
state satisfying \(c\) from \emph{some} input state satisfying \(b\)---or,
equivalently, that \(c\) \emph{under-approximates} the set of final states of
\(p\) starting from \(b\).  Though less conventional than traditional Hoare
logic, incorrectness logic has already found its way to several applications,
such as variants of separation logic~\cite{Raad_Berdine_Dang_Dreyer_OHearn_Villard_2020} 
and relational verification for noninterference~\cite{murray_UnderApproximateRelationalLogic_2020a}.

\begin{figure}
    \centering
    \begin{tikzcd}[row sep=5em, column sep=10em, math mode = false]
        \FailTopREL \arrow[r, two heads, 
            "\Cref{the: FailTopKAT can express incorrectness logic}"]
        & \text{IL} \\
        \TopREL \arrow[r, two heads, 
            "\Cref{the: topkat can express incorrectness logic normal termination}"]
        & \text{codomain inclusion} 
            \arrow[d, "defines"description]
            \arrow[dd, bend left=90, "defines"description]\\
        \TopGREL \arrow[d, leftrightarrow, "\Cref{the: TopGREL expressive power} (equiexpressive)"' ]
            \arrow[r, "/" marking, two heads, "\Cref{the: TopGREL expressive power}"]
        &  \text{IL (without errors)} \\
        \REL \arrow[r, two heads, "\cite{Kozen_2000}"']
            \arrow[ur, "/" marking, two heads, "\Cref{the: KAT not able to express incorrectness logic}"']
        & \text{HL} 
    \end{tikzcd}
    \caption{Expressiveness of different systems.}\label{fig:expressiveness-of-different-kat}
\end{figure}

It is natural to wonder whether \citeauthor{Kozen_2000}'s idea could be
adapted to encode incorrectness logic in Kleene algebra with tests.
Unfortunately, this is not the case. As we will show in this paper, there are
incorrectness triples that cannot be expressed by any KAT equation.  This might
appear surprising, given the symmetry between over- and under-approximation in
the formulation of Hoare logic and incorrectness logic.  However, the symmetry
involves the \emph{image} of a set by a relation, an operation that is not part
of the syntax of KAT\@.  Several prior works have considered enlarging KAT with
similar
operations~\cite{Desharnais_Möller_Struth_2004,Fahrenberg_Johansen_Struth_Ziemiánski_2021,Desharnais_Möller_Struth_2006},
but we show here that a smaller extension also serves our purposes: namely,
adding a top element \(\top\) to KAT\@.   We call such a structure a \emph{Kleene
  algebra with tests and top}, or a TopKAT\@.  We show that such structures can
encode inequalities between images, which we use to express incorrectness
triples.  The encoding allows us to prove the rules of incorrectness logic
equationally, thus extending \citeauthor{Kozen_2000}'s correspondence to
incorrectness reasoning. In fact, the use of TopKAT to encode incorrectness logic was also hinted in the original incorrectness logic paper~\cite{OHearn_2020}.

For clarity of exposition, our main focus is on the fragment of incorrectness
logic that handles normal program termination.  However,
\citet{OHearn_2020} also considered triples of the form
\([p]\, c\, [er: q]\), whose interpretation is similar to the one we described
above, except that they assume that execution can terminate with a fatal error
(e.g.\ a failed assertion).  Following \citet{Mamouras_2017}, we show
that our encoding carries over to such triples by considering FailTopKAT, an
extension of TopKAT that includes an element \(\fail\) for representing failure.
We prove that the abnormal termination rules of incorrectness logic follow from
the equations of FailTopKAT\@.

We summarize our encodings in \Cref{fig:expressiveness-of-different-kat}.  We
use the notation \(A ↠ B\) to mean that the logic \(B\) can be
expressed in the equational theory of \(A\).  More formally, we model the
ground-truth notion of validity in each logic as a statement about sets
(assertions about program states) and relations (the input-output graph of a
program).  The encodings show that such statements are equivalent to equations
involving operations in \emph{relational} algebraic structures, where the
carrier of the structure is some set of relations between program states (for
example, \(\FailTopREL\) is the class of relational FailTopKATs in
\Cref{def:relational-failtopkat}).  Moreover, we prove that the equational
theory alone (i.e., not specialized to relations) suffices to derive the rules
of each logic.  We also use the arrow \(\to\) to denote the fact that 
systems capable of expressing codomain can express both Hoare logic and incorrectness logic.

\aaa{The explanation about the \(\to\) notation in
  \Cref{fig:expressiveness-of-different-kat} seems a bit ad hoc.}

\begin{figure}
    \centering
    \begin{tikzcd}[math mode = false, row sep=4em, column sep=9em]
        & \TopREL \arrow[d, "/" marking, "incomplete"'near start, 
        "\Cref{the: topkatstar incomplete over relational model}"near start]
        & \\
        \TopGREL \arrow[r, "complete", "\Cref{the: general relational TopKAT for free}"']
        & \TopKAT & 
        \text{language \TopKAT} \arrow[l, "complete"', "\Cref{the: language TopKAT for free}"] \\
        \REL \arrow[r, "complete", "\cite{Kozen_1997}"'] & 
        \KAT & 
        \text{language \KAT} \arrow[l, "complete"', "\cite{Kozen_1997}"]
    \end{tikzcd}
    \caption{Completeness relationships between classes of Kleene Algebras with Tests.}\label{fig:completeness}
\end{figure}

To evaluate the usefulness of these encodings, we investigate two basic
properties of TopKAT\@: \emph{completeness} and \emph{decidability}.  We say
that an equational theory is complete for a certain class of structures if it
can derive any equation that is valid in the class.  We are particularly
interested in completeness with respect to relational structures, since they are
the natural setting for formulating program logics. 
However, it is well-known that the addition of a top element can be problematic for completeness of relational structures~\cite{pousAutomataRelationAlgebra2016}, and we show that this is the case for TopKAT as well: the theory is incomplete for \(\TopREL\), the class of relational structures where the top element is the complete relation.  
However, we were able to obtain (co)domain completeness: all (co)domain comparison inequalities, which subsumes both the encoding of incorrectness logic and Hoare logic, are derivable if and only if they are true in relational model. Setting a solid foundation for reasoning about program logic using KAT.

The (co)domain comparison result is built upon several lower-level completeness result. Specifically, the completeness regarding language TopKAT, which naturally induces the completeness of general relational TopKAT (\TopGREL), a larger class of TopKAT where the elements are relations.
Given the relational nature of TopGREL, it is natural to wonder if we can encode incorrectness logic in it. However, we have shown that it is impossible: every predicate that is encodable in TopGREL is already encodable in relational KAT.
All of the above completeness results and equi-expressiveness results are simple corollaries of our new structured version of reduction~\cite{Pous_Rot_Wagemaker_2021}, where instead of relating relating both the language model and the theory, we characterize the reduction simply as an injective interpretation.

\section{Preliminary on Program Logics}

\subsection{Hoare Logic}\label{sec:incorrectness-and-hoare}

Hoare logic is a fundamental tool for specifying and proving the correctness of
while-like programs. Following \citet{Kozen_2000}, we consider here
\emph{propositional Hoare logic}, which involves partial correctness Hoare
triples \(\hoareTriple{b}{p}{c}\) consisting of atomic propositions,
propositional connectives and while-like programs.  As usual, a Hoare triple
\(\hoareTriple{b}{p}{c}\) means that if the program \(p\) terminates when run on
a memory satisfying \(b\), it will result in a memory satisfying \(c\).
\Cref{fig:HL} shows the rules of propositional Hoare logic, which differ from
the classical setting in their omission of the assignment rule.


\begin{figure}
    \begin{mathpar}
        \inferrule[Composition]
        {\hoareTriple{a}{p}{b} \\ \hoareTriple{b}{q}{c}}
        {\hoareTriple{a}{p;q}{c}}
    
        \and 
        \inferrule[Conditional]
        {\hoareTriple{b\land c}{p}{d} \\ \hoareTriple{\neg b\land c}{q}{d}}
        {\hoareTriple{c}{\comITE{b}{p}{q}}{d}}

        \and
        \inferrule[While]
        {\hoareTriple{b\land c}{p}{c}}
        {\hoareTriple{c}{\comWhile{b}{p}}{\neg b\land c}}

        \and
        \inferrule[Consequence]
        {b' → b \\ \hoareTriple{b}{p}{c} \\ c → c'}
        {\hoareTriple{b'}{p}{c'}}

      \end{mathpar}
      \caption{proof rules of propositional Hoare logic}\label{fig:HL}
\end{figure}

In its essence, Hoare logic is an \emph{over-approximation} logic.  To see this,
it is convenient to think about a program \(p\) as a relation between input
memories and output memories, and to think about predicates \(b\) and \(c\) as
sets of states.  Given a program \(p\) and a predicate \(b\), we can write
\(\post(p)(b)\) for the set of post-states, that is
%
\[\post(p)(b)=\{ x \mid \exists y\in b, (y,x)\in p \}.\]
%
A partial-correctness Hoare triple \(\hoareTriple{b}{p}{c}\) is valid
iff \[\post(p)(b) \subseteq c.\] In words, \(c\) over-approximates the set of
memories which can be obtained from \(b\) by running the program \(p\).  This
condition can be expressed by means of the codomain of a relation: if we set
\[\cod(r) ≜ \{y \mid \exists x \in X, (x, y) \in r\},\]
then, for all a relational KAT \((ℛ, ℬ)\), 
\(b, c \in ℬ\) and \(p \in ℛ\),
\[\hoareTriple{b}{p}{c} ≜ \cod(b p) \subseteq \cod(c).\]


\citet{Kozen_2000} showed that we can reason about the 
partial correctness of propositional Hoare logic in KAT\@. 
To do this  we can use tests to represent pre and post-conditions,
thus encode a partial correctness propositional Hoare triple
\(\hoareTriple{b}{p}{c}\) as the KAT equality:
\[b p \overline{c} = 0, \text{ or equivalently, } b p = b p c.\]

Indeed, we can show that in all relational KATs \((ℛ, ℬ)\) and
\(p \in ℛ, b, c \in ℬ\),
\[\hoareTriple{b}{p}{c} \iff bp = bpc \iff b p \overline{c} = 0.\]
Given that KAT is complete over relational KAT, we can determine the relational
validity of some propositional Hoare triples by the equational theory of KAT\@.

\subsection{Incorrectness Logic}
% The formulation provided by \citet{Kozen_2000} turns out to be very useful 
% in reasoning about program correctness using equational theory of KAT;
% but what if the given program is incorrect? 
%
\citet{OHearn_2020} proposed incorrectness logic to reason about
incorrect programs.  This logic is also related to earlier works by
\citet{devries_ReverseHoareLogic_2011}.  Like Hoare logic, incorrectness logic is built on
triples of the form \(\incorTriple{b}{p}{c}\), which denote \emph{incorrectness
  specifications}, where \(c\) is a set of undesirable final states and \(b\) is a
precondition. Intuitively, such a triple says that every memory in \(c\) needs
to be reachable from \(b\) by executing \(p\).

\begin{figure}
    \begin{mathpar}
        \inferrule[Empty]
        {\\}{\incorTriple{b}{p}{\gamma: ⊥}}

        \and
        \inferrule[Consequence]
        {b → b' \\ \incorTriple{b}{p}{\gamma: c} \\ c' → c}
        {\incorTriple{b'}{p}{\gamma: c'}}

        \and
        \inferrule[Disjunction]
        {\incorTriple{b_1}{p}{\gamma: c_1} \\ \incorTriple{b_2}{p}{\gamma: c_2}}
        {\incorTriple{b_1 \lor b_2}{p}{\gamma: (c_1 \lor c_2)}}

        \and
        \inferrule[Identity]
        {\\} {\incorTriple{b}{\comSkip}{\okState: b, \errState: 0}}

        \and
        \inferrule[Composition-Fail]
        {\incorTriple{a}{p}{\errState: b}}
        {\incorTriple{a}{p;q}{\errState: b}}

        \and
        \inferrule[Composition-Normal]
        {\incorTriple{a}{p}{\okState: b} \\ \incorTriple{b}{p}{\gamma: c}}
        {\incorTriple{a}{p;q}{\gamma: c}}

        \and
        \inferrule[Choice-Left]
        {\incorTriple{b}{p}{\gamma: c}}
        {\incorTriple{b}{p + q}{\gamma: c}}

        \and
        \inferrule[Choice-Right]
        {\incorTriple{b}{q}{\gamma: c}}
        {\incorTriple{b}{p + q}{\gamma: c}}

        \and
        \inferrule[Assume]
        {\\}
        {\incorTriple{a}{\comAssert{b}}{\okState: a \land b, \errState: 0}}

        \and
        \inferrule[Error]
        {\\}{\incorTriple{b}{\fail}{\errState: b}}

        \and
        \inferrule[Iter-Zero]
        {\\}{\incorTriple{b}{{p}^*}{\okState: b}}

        \and
        \inferrule[Iter-NonZero]
        {\incorTriple{b}{{p}^*;p}{\gamma: c}}
        {\incorTriple{b}{{p}^*}{\gamma: c}}

        \and
        \inferrule[Iter-Dependent]
        {∀ n ∈ ℕ,~ \incorTriple{b(n)}{p}{\okState: b(n+1)}}
        {\incorTriple{b(0)}{{p}^*}{\okState: \exists n, b(n)}}
    \end{mathpar}
    \caption{Generic incorrectness logic proof rules~\cite{OHearn_2020}}\label{fig: orignal proof rules of inc logic}
\end{figure}

If Hoare logic is an ``over-approximation'' logic, incorrectness logic is an
``under-approximation'' logic: an incorrectness triple \(\incorTriple{b}{p}{c}\)
is valid if and only if
\[\post(p)(b) \supseteq c.\]

In other words, the post-condition \(c\) ``under-approximates'' the canonical
post condition of \(b\) after executing \(p\).
When we only look at program that terminates normally, 
the previous definition can be expressed in relational KAT as 
\[\cod(bp) \supseteq \cod(c).\]
As mentioned by \citet{OHearn_2020},
the definitions of incorrectness and Hoare triples are highly symmetric:
\begin{align*}
    \hoareTriple{b}{p}{c} & ≜ \cod(b p) ⊆ \cod(c) \\
    \incorTriple{b}{p}{c} & ≜ \cod(b p) ⊇ \cod(c)
\end{align*}

Since incorrect programs often lead to explicit errors,
\citet{OHearn_2020} also considered incorrectness triples
\(\incorTriple{b}{p}{\errState: c}\), which mean that, in addition to satisfying
$c$, we require that the final states of $p$ lead to an error.  More generally,
we use the notation $\incorTriple{b}{p}{γ: c}$, where the \emph{error
  code} \(γ ∈ \{\okState, \errState\}\) signals whether the program
terminated normally or not; hence, the unqualified notation
\(\incorTriple{b}{p}{c}\) is simply a shorthand for
\(\incorTriple{b}{p}{\okState: c}\), when abnormal termination is not a
concern. Informally, such general triples mean that
\[\incorTriple{b}{p}{\gamma: c} ≜ \cod(b p) \supseteq \cod((\gamma: c)).\]
In \Cref{sec: failkat and error}, we will give a more concrete definition of
this semantics in an extension of KAT\@.

The generic proof rules of incorrectness logic are listed in \Cref{fig: orignal
  proof rules of inc logic}.
Following \citet{OHearn_2020}, we formulate
incorrectness logic for a language of commands which is essentially
the same of KAT terms. Conditionals and loops as their embedding from GKAT~\cite{Kozen_2000,Smolka_Foster_Hsu_Kappé_Kozen_Silva_2020}, which we will recall here:
\begin{align*}
    \comITE{b}{p}{q} & ≜  (\comAssert{b};p) + (\comAssert{\overline{b}}; q) \\
    \comWhile{b}{p} & ≜ {(\comAssert{b}^*; p)} ;\comAssert{\overline{b}}
\end{align*}
Following \citet{OHearn_2020}
we also use
\(\incorTriple{b}{p}{\okState: c_1, \errState: c_2}\) as a shorthand for two
different rules. For example, the rule
\begin{mathpar}
    \inferrule[Unit]
    {\\}{\incorTriple{a}{1}{\okState: a, \errState: 0}}
\end{mathpar}
stands for the two rules
\begin{mathpar}
    \inferrule[Unit-Ok]
    {\\}{\incorTriple{a}{1}{\okState: a}}

    \inferrule[Unit-Er]
    {\\}{\incorTriple{a}{1}{\errState: 0}}.
  \end{mathpar}
The fragment of incorrectness logic we consider here is \emph{propositional} in the sense of \citet{Kozen_2000}. In particular, this presentation omits  rules for variables and mutation~\cite{OHearn_2020}. 

\section{Formulating Incorrectness Logic}\label{sec: formulating incorrectness logic}

We might hope that the symmetry between Hoare logic and incorrectness logic
would help us express incorrectness triples by adapting the formulation of
\citet{Kozen_2000}. However,
it is not obvious how we can exploit this symmetry, since it involves the
codomain operation, which does not appear in the formulation of
\citet{Kozen_2000}.  
This difficulty, unfortunately, is fundamental: KAT cannot express
incorrectness logic.

\begin{definition}\label{def: relational validity of incorrectness triple}
  Given a relational KAT \((ℛ, ℬ)\), \(p \in ℛ\)
  and \(b, c ∈ ℬ\), an Incorrectness Triple
  \(\incorTriple{b}{p}{c}\) is \emph{valid} with respect to
  \((ℛ, ℬ)\), denoted
  \((ℛ, ℬ) ⊧ \incorTriple{b}{p}{c}\), if
  \[\cod(bp) ⊇ \cod(c)\]
  we consider the predicate of incorrectness triple:
  \(inc(b, p, c) ≜ \incorTriple{b}{p}{c}\).  We write \(inc\) over
  primitive tests \(b, c\) and primitive action \(p\) as
  \(\incorTriple{b}{p}{c}\).
\end{definition}

To show that incorrectness triple cannot be formulated using equality of KAT
terms, we only need to show that \(\incorTriple{b}{p}{c}\)
cannot be expressed in \(\REL\).  More explicitly, we need to show that for
all \(K, B\) where \(p ∈ K\) and \(b, c ∈ B\), there
does \emph{not} exist a pair of terms
\(e₁, e₂ ∈ \KAT_{K, B}\) s.t.\ for all relational
Interpretation \(I\):
\[I(e₁) = I(e₂) ⟺ \incorTriple{I(b)}{I(p)}{I(c)}\] 
To prove such a pair of expressions don't exist, we will first establish some lemmas,
which will help us narrow down our goal.

\begin{lemma}[redundancy of alphabet]\label{the: redundancy of alphabet}
    If for some \(K, B\) where \(p ∈ K\) and \(b, c ∈ B\), 
    and a pair of terms \(e_1, e_2 ∈ \KAT_{K, B}\),
    that for all relational KAT interpretations \(I: \KAT_{K, B} → ℛ\),
    \[I(e_1) = I(e_2) ⟺ \incorTriple{I(b)}{I(p)}{I(c)}\]
    then there exist \(\hat{e_1}, \hat{e_2} ∈ \KAT_{\{p\},\{b, c\}}\) 
    such that, for every relational interpretation \(I': \KAT_{\{p\}, \{b, c\}} → ℛ\), 
    \[I'(\hat{e_1}) = I'(\hat{e_2}) ⟺ \incorTriple{I(b)}{I(p)}{I(c)}\]
\end{lemma}

\begin{proof}
    We define the following interpretation \(\hat{(-)}: \KAT_{K, B} → \KAT_{\{p\}, \{b, c\}}\) by lifting the following action on the primitives:
    \begin{align*}
        \hat{p'} & = p & p' ∈ K \\  
        \hat{b'} & = \begin{cases}
            c & b' ≠ b \\  
            b & b' = b
        \end{cases} 
    \end{align*}
    Notice that this definition preserves the primitives \(p, b, c\):
    \begin{mathpar}
        \hat{p} = p \and \hat{c} = c \and \hat{b} = b
    \end{mathpar}
    Because \(I' ∘ \hat{(-)}: \KAT_{K, B} → ℛ\) is a interpretation, by the premise, we obtain our conclusion,
    \[I'(\hat{e_1}) = I'(\hat{e_2}) ⟺ \incorTriple{I'(b̂)}{I'(p̂)}{I'(ĉ)} ⟺ \incorTriple{I'(b)}{I'(p)}{I'(c)}. \qedhere\]
\end{proof}

Thus, in order to show that there does not exist a pair of terms
\(e_1, e_2 ∈ \KAT_{K,B}\) to express incorrectness logic,
we only need to show that there does not exist
\(\hat{e_1}, \hat{e_2} ∈ \KAT_{\{p\},\{b, c\}}\) 
that are capable of formulating incorrectness logic.


\begin{lemma}\label{the: only one action var}
    Given a expression \(e ∈ \KAT_{\{p\}, \{b, c\}}\)
    and a relational KAT interpretation \(I\) with \(I(p) = ∅\), if
    \((x, y) ∈ I(e)\), we have \(x = y\).
\end{lemma}

\begin{proof}
    By induction on the structure of \(e\).
\end{proof}

\begin{lemma}\label{the: KAT inccorrectness logic core lemma}
    Given a expression \(e ∈ \KAT_{\{p\},\{b, c\}}\)
    and a relational KAT interpretation \(I\), if \((x, y) ∈ I(e)\),
    then either:
    \begin{itemize}
    \item there exist \(z\) and \(z'\) such that \((x, z) ∈ I(p) \text{ and } (z', y) ∈ I(p)\); or
    \item \(x = y\) and \((x,y) ∈ I_∅(e)\), where the interpretation \(I_{∅}\) is lifted by the following action on primitives:
    \[I_{∅}(p') ≜ \begin{cases}
        ∅ & p' = p \\  
        I(p') & \text{otherwise}
    \end{cases}\]
\end{itemize}
\end{lemma}

\begin{proof}
    Intuitively, this lemma states
    if the element \((x, x)\) is generated by purely by some tests 
    in \(I(e_2)\),
    then we can ignore all the action variables in the expression by setting it to \(∅\).
    Otherwise, \((x, x)\) must be generated by composing some actions together,
    since tests only \emph{filters out} elements when composed 
    and cannot generated new elements.
    Thus we will need at least a action to start with \(x\), 
    and a action to end with \(x\).
    
    This lemma can be proven by induction on the structure of \(e\): 
    \begin{itemize}
        \item If \(e\) is an element of the alphabet, i.e. \(e = p\), \(e = b\), or \(e = c\).
        \begin{itemize}
        \item If \(e = p\), then \((x, y) ∈ I(e)\) 
            iff \((x, y) ∈ I(p)\)
            thus, we can have \(z = y, z' = x\), hence 
            \[(x, z) ∈ I(p)  \text{ and }  (z', y) ∈ I(p)\]
        \item If \(e = b\) or \(e = c\),
            then the interpretation of \(p\) will not matter, 
            hence assigning \(p\) to empty will also contain \((x, y)\).
            Therefore \((x, y) ∈ I_∅(e)\).
            And by \Cref{the: only one action var}, \(x = y\)
        \end{itemize}
        \item If \(e\) is of the form \(e_1 + e_2\),
        then 
        \begin{align*}
            (x, y) ∈ I(e_1 + e_2)
            & ⟹ (x, y) ∈ I(e_1) ∪ I(e_2) \\
            & ⟹ (x, y) ∈ I(e_1) \text{ or } (x, y) ∈ I(e_2)
        \end{align*}
        Without loss of generality, assume \((x, y) ∈ I(e_1)\).
        Then by induction hypothesis, we have 
        \begin{itemize}[nosep]
            \item either \(I_∅(e_1)\) and \(x = y\)
            \item or \(∃ z, z', \text{ s.t. } (x, z), (z', y) ∈ I(p)\)
        \end{itemize}
        Because \(I_∅(e) = 
        I_∅(e_1 + e_2) = 
        I_∅(e_1) ∪ I_∅(e_2)\),
        therefore 
        \[(x, y) ∈ I_∅(e_1) ⟹ (x, y) ∈ I_∅(e)\]
        Thus we have
        \begin{itemize}[nosep]
            \item either \((x, y) ∈ I_∅(e)\) and \(x = y\)
            \item or \(\exists z, z', \text{ s.t. } (x, z), (z', y) ∈ I(p)\)
        \end{itemize}
        \item If \(e\) is of the form \(e_1 e_2\),
        then 
        \[
            (x, y) ∈ I(e_1 e_2)
            ⟹ ∃ k, \text{ s.t. } (x, k) ∈ I(e_1)  \text{ and }  (k, y) ∈ I(e_2) 
        \]
        Then by induction hypothesis for \(e_1\), we have 
        \begin{itemize}[nosep]
            \item either \((x, k) ∈ I_∅(e_1)\) and \(x = k\)
            \item or \(\exists z, z', \text{ s.t. } (x, z), (z', k) ∈ I(p)\)
        \end{itemize}
        by induction hypothesis for \(e_2\)
        \begin{itemize}[nosep]
            \item either \((k, y) ∈ I_∅(e_2)\) and \(k = y\)
            \item or \(\exists z, z', \text{ s.t. } (k, z), (z', y) ∈ I(p)\)
        \end{itemize}
        
        Then there are 4 different cases:
        \begin{itemize}
            \item If both 
            \begin{itemize}[nosep]
                \item \((x, k) ∈ I_∅(e_1)\) and \(x = k\)
                \item \((k, y) ∈ I_∅(e_2)\) and \(k = y\)
            \end{itemize}
            are true, then \((x, y) ∈ I_∅(e)\) by rule of composition;
            and \(x = y\) by transitivity of equality.
            \item if both 
            \begin{itemize}[nosep]
                \item \((x, k) ∈ I_∅(e_1)\) and \(x = k\)
                \item \(∃ z, z', \text{ s.t. } (k, z), (z', y) ∈ I(p)\)
            \end{itemize}
            are true, because \(x = k\), thus from the second point we have 
            \[\exists z, z', \text{ s.t. } (x, z), (z', y) ∈ I(p)\]
            hence the result is true.
            \item if both 
            \begin{itemize}[nosep]
                \item \(\exists z, z', \text{ s.t. } (x, z), (z', k) \in I(p)\)
                \item \((k, y) ∈ I_∅(e_2)\) and \(k = y\)
            \end{itemize}
            are true, because \(k = y\), thus from the first point we have 
            \[\exists z, z', \text{ s.t. } (x, z), (z', y) \in I(p)\]
            hence the result is true.
            \item if both 
            \begin{itemize}[nosep]
                \item \(∃ z, z', \text{ s.t. } (x, z), (z', k) ∈ I(p)\)
                \item \(∃ z, z', \text{ s.t. } (k, z), (z', y) ∈ I(p)\)
            \end{itemize}
            is true, then there exists \((x, z)\) and \((z', y)\) in \(I(p)\)
            hence the result is valid
        \end{itemize}
        \item If \(e\) is of the form \((e_1)*\) for some \(e_1\).
        Then by definition of the star operator in \(\REL\),
        \((x, y) ∈ (e_1)*\) means there exists \(n ∈ ℕ\),
        s.t. \((x, y) ∈ {(e_1)}^{n}\).
        Then we can prove this result by induction on \(n\), 
        using a strategy similar to the multiplication case.
        \item If \(e\) is of the form \(\overline{e_1}\),
        then by definition \(e_1\) cannot contain primitive action.
        Hence \(e_1\) and \(e\) will not contain \(p\).
        Therefore the interpretation of \(p\) do not matter to the interpretation.
        Thus we have \[(x, y) ∈ I_∅(e)\]
        and by \Cref{the: only one action var}, \(x = y\) \qedhere
    \end{itemize}
\end{proof}

Finally, with all the lemma in place, we can prove that KAT is insufficient in encoding incorrectness logic. 
This result demonstrates crucial limitation about domain and reachability reasoning in KAT.

\begin{theorem}\label{the: KAT not able to express incorrectness logic}
    \(\incorTriple{b}{p}{c}\)
    cannot be expressed in \(\REL\).
\end{theorem}

\begin{proof}
    First, by \Cref{the: redundancy of alphabet},
    we only need to show that there does not exist
    \[e₁, {e₂} ∈ \KAT_{\{p\},\{b, c\}}\]
    that can express incorrectness logic.

    Let's assume that there exist \({e₁}\) and \({e₂}\) 
    in \(\KAT_{\{p\},\{b, c\}}\)
    such that
    \[I(e₁) = I(e₂) ⟺ \incorTriple{I(b)}{I(p)}{I(c)}\]  holds for all relational interpretation \(I\).

    Consider the relational KAT \(ℛ\) 
    that contains all the relations and predicates over \(\{0, 1\}\).
    We will construct a pair of interpretation on \(b, c, p\) 
    to show a contradiction: 
    \begin{align*}
        I_{\emptyset}(p) & ≜ ∅ 
        & I(p) & ≜ \{(0, 1)\} \\
        I_{\emptyset}(b) & ≜ \{(0, 0)\}
        & I(b) & ≜ \{(0, 0)\} \\
        I_{\emptyset}(c) & ≜ \{(1, 1)\}
        & I(c) & ≜ \{(1, 1)\}.
    \end{align*}
    Where the incorrectness triple
    \(\incorTriple{b}{p}{c}\) is valid with interpretation
    \(I\), but not with \(I_{∅}\).  And the only difference between
    \(I\) and \(I_{∅}\) is that \(p\) is mapped to \(∅\)
    in \(I_{∅}\).

    Since \(I_{∅}\) valuates the only action variable \(p\) as \(∅\),
    By \Cref{the: only one action var} 
    all the elements of \(I_∅(e₁)\) must be of the form \((x, x)\).
    Then because the incorrectness triple \(\incorTriple{b}{p}{c}\)
    is invalid with \(I_{∅}\),
    \[I_∅(e₁) ≠ I_∅(e₂).\]
    Without loss of generality, assume that
    \((x, x) ∈ I_∅(e₁)\), but not in
    \(I_∅(e₂)\).  Then by monotonicity of KAT operators,
    \((x, x) ∈ I(e₁)\).  We will derive a contradiction
    from the fact that \((x, x) ∈ I(e₂)\) but
    \((x, x) ∈ I_∅(e₂)\).  We can summarize the
    above strategy using \Cref{fig: proof of KAT cannot encode IL}.

    Because \((x, x) ∉ I_∅(e₂)\)
    and the only element of the action \(I(p)\) is \((1, 0)\),
    none of the conditions in \Cref{the: KAT inccorrectness logic core lemma} is satisfied,
    therefore \((x, x)\) cannot be in \(I(e₂)\),
    which contradicts the earlier result stating \((x, x) ∈ I(e₂)\).
\end{proof}

\begin{figure}
    \centering
    \begin{tikzcd}[row sep=1.5em, column sep=1.5em]
        (x, x) \in I(e₁)
            \arrow[rrr, "I(e₁) = I(e₂)"] 
            &&& 
        (x, x) \in I(e₂) \\
        & I(e₁) 
            \arrow[r, equal]
        & I(e₂) & \\
        & I_∅(e₁) 
            \arrow[u, hook, "\text{monotonicity}"]
            \arrow[r, equal, "/" marking]
        & I_∅(e₂) 
            \arrow[u, hook, "\text{monotonicity}"']
        & \\
        (x, x) ∈ I_∅(e₁) 
            \arrow[uuu, "\text{by monotonicity}"] &&& 
        (x, x) ∉ I_∅(e₂) 
            \arrow[uuu, dash, "\text{contradiction}"', "×"marking]\\
    \end{tikzcd}
    \caption{Relation of interpretations \(I\) and \(I_{∅}\)}\label{fig: proof of KAT cannot encode IL}
\end{figure}

One strategy for exploiting the symmetry between Hoare and incorrectness logic
is to extend KAT with a codomain operator.  Similar ideas have been explored in prior work~\cite{Möller_O’Hearn_Hoare_2021}. However,
rather than adopting a full-fledged codomain operator, it suffices for our
purposes to consider a equational theory that only extends KAT with a top
element.  Similar algebraic structure has also been considered in prior
work~\cite[Section~4]{Mamouras_2017}, though for a different purpose.

\begin{definition}[Kleene Algebra with Top and Tests]
  A Kleene Algebra with Top and Tests, or TopKAT, is a KAT \(𝒦\) that contains a largest
  element \(⊤\); that is, for all elements \(p ∈ 𝒦\),
  \[⊤ ≥ p.\]
  A relational TopKAT over \(X\) is a relational KAT that includes the complete relation \(X × X\).
\end{definition}

\begin{theorem}[TopKAT can Express Codomain]\label{the: TopKAT encodes domain PRIMITIVE}
    For all relational TopKATs \(ℛ\),
    and \(p, q ∈ ℛ\) The following is true:
    \[⊤ p = ⊤ q ⟺ \cod(p) = \cod(q)
    \text{ and }
    ⊤ p ≤ ⊤ q ⟺ \cod(p) ⊆ \cod(q).\]
\end{theorem}

\begin{proof}
  For all relations \(r \subseteq X \times X\):
    \begin{align*}
        \top ⋅ r 
        & = \{(z, y) \mid z \in X \land \exists x, (x, y) \in r\} 
        = \{(z, y) \mid z \in X \land y \in \cod(r)\} 
    \end{align*}
    Therefore for two relation \(p, q\), we will have 
    \begin{align*}
        ⊤ ⋅ q = ⊤ ⋅ p & ⟺ 
        \{(z, y) ∣ z ∈ X ∧ y ∈ \cod(q)\}
        = \{(z, y) ∣ z ∈ X ∧ y ∈ \cod(p)\} \\
        & ⟺ \cod(q) = \cod(p) \\
        ⊤ ⋅ q ≤ ⊤ ⋅ p & ⟺ 
        \{(z, y) ∣ z ∈ X ∧ y ∈ \cod(q)\}
        \subseteq \{(z, y) \mid z \in X \land y \in \cod(p)\} \\
        & \iff \cod(q) \subseteq \cod(p) \qedhere
    \end{align*} 
\end{proof}

We also provide a shorthand for domain reasoning. 
For two terms \(e_1, e_2  ∈  \KAT\), we write
\[\REL  ⊧  \dom(e_1)  ≥  \dom(e_2),\] when 
\(\dom(I(e_1))  ⊇  \dom(I(e_1))\) for all relational KAT interpretations \(I\);  
and similarly for relational TopKAT and general relational TopKAT.  
Then \Cref{the: TopKAT encodes domain PRIMITIVE} implies the following:
\begin{lemma}\label{the: top element represent domain}
    For two KAT terms \(e_1, e_2  ∈  \KAT_{K, B}\):
    \begin{align*}
        \TopREL  ⊧  e_1  ⊤   ≥  e_2  ⊤  &  ⟺ \REL  ⊧  \dom(e_1)  ≥  \dom(e_2) \\
        \TopREL  ⊧   ⊤  e_1  ≥   ⊤  e_2 &  ⟺  \REL  ⊧  \cod(e_1)  ≥  \cod(e_2)
    \end{align*}
\end{lemma}

\begin{corollary}\label{the: topkat can express incorrectness logic normal termination}
    For all relational TopKATs \((ℛ, ℬ)\),
    and \(p ∈ ℛ\), \(b, c \in ℬ\)
    we have the following:
    \[{⊤ bp ≥ ⊤ c} ⟺ {\incorTriple{b}{p}{c}}\]
\end{corollary}

Notice that the left hand side of this equivalence makes sense in all TopKATs,
not just relational ones.
Thus it is natural to generalize the definition of incorrectness triple to all TopKATs
using the inequality \(\top bp \geq \top c\).
As a matter of fact, just by using the equational theory of TopKAT,
we can find several equivalent formulations of incorrectness triple:

\begin{theorem}[Equivalent Formulation of Incorrectness Logic]
  Given a TopKAT \((𝒦, ℬ)\), where \(p ∈ 𝒦\) and
  \(b, c ∈ ℬ\), we say that an incorrectness triple
  \(\incorTriple{b}{p}{c}\) is valid if the following equivalent conditions are
  met:
  \[⊤ b p ≥ ⊤ c ⟺ ⊤ b p ≥ c ⟺ ⊤ b p c = ⊤ c\]
\end{theorem}

\begin{proof}
We show that these conditions are equivalent in all TopKATs.
\begin{itemize}
    \item First show \[⊤ b p ≥ ⊤ c ⟺ ⊤ b p ≥ c,\]
    the \(⟹\) is true because \(⊤ ≥ 1\):
    \[\top b p \geq \top c \geq c,\]
    the \(\impliedby\) can be proven by idempotency of \(\top\):
    \[\top b p = \top \top b p \geq \top c.\]
    \item Then show \[\top b p \geq \top c \iff \top b p c = \top c,\]
    the \(\impliedby\) can be shown by \(1 \geq c\):
    \[\top b p \geq \top b p c = \top c,\]
    the \(\implies\) is by two inequalities:
    \begin{align*}
        \top b p \leq \top & \implies \top b p c \leq \top c \\
        c \leq 1 \land \top b p \geq \top b p c & \implies \top b p \geq \top b p c = \top c.\\
    \end{align*}
    Therefore we have our conclusion \(\top b p c = \top c\). \qedhere
\end{itemize}
\end{proof}

A similar encoding involving \(\top\) was also mentioned by \citet{OHearn_2020}.
We want to show that this abstraction is enough to capture all the rules of
incorrectness logic.  Indeed most of the the rules are formulated using program
operations expressible in \(\KAT\) ~\cite{OHearn_2020}.  We
focus here on the proof rules concerning normal program termination, and we will
further investigate the rules with error in \Cref{sec: failkat and error}.


\begin{figure}
    \begin{mathpar}
        \inferrule[Empty]
        {\\}{\incorTriple{b}{p}{0}}
    
        \and
        \inferrule[Consequence]
        {b \leq b' \\ \incorTriple{b}{p}{c} \\ c' \leq c}
        {\incorTriple{b'}{p}{c'}}
    
        \and 
        \inferrule[Disjunction]
        {\incorTriple{b_1}{p}{c_1} \\ \incorTriple{b_2}{p}{c_2}}
        {\incorTriple{b_1 + b_2}{p}{c_1 + c_2}}
    
        \and 
        \inferrule[Identity]
        {\\} {\incorTriple{b}{1}{b}}
    
        \and 
        \inferrule[Composition]
        {\incorTriple{a}{p}{b} \\ \incorTriple{b}{q}{c}}
        {\incorTriple{a}{pq}{c}}
    
        \and 
        \inferrule[Choice-Left]
        {\incorTriple{a}{p}{b} }
        {\incorTriple{a}{p + q}{b}}
    
        \and 
        \inferrule[Choice-Right]
        {\incorTriple{a}{q}{b} }
        {\incorTriple{a}{p + q}{b}}
    
        \and 
        \inferrule[Assume]
        {\\}
        {\incorTriple{b}{c}{bc}}
    
        \and 
        \inferrule[Iter-Zero]
        {\\}{\incorTriple{b}{p^*}{b}}
    
        \and 
        \inferrule[Iter-NonZero]
        {\incorTriple{b}{p^* p}{c}}{\incorTriple{b}{{p}^*}{c}}
    
        \and
        \inferrule[Iter-Dependent]
        {\forall n \in ℕ, \incorTriple{b_n}{p}{b_{n + 1}}}
        {\incorTriple{b_0}{{p}^*}{\sup_{n \in ℕ} b_{n}}}
    \end{mathpar}
    \caption{Incorrectness logic with normal termination}\label{fig: normal terminating Inc rule}
\end{figure}

In \Cref{fig: normal terminating Inc rule}, we present all the rules of
incorrectness logic with normal termination using the syntax of KAT.  (Note
that this differs slightly from \citeauthor{OHearn_2020}'s
original notation; for instance, the sequencing operator \(p;q\) corresponds to
multiplication \(pq\) in KAT, and the logical implication \(b → b'\)
corresponds to order \(b \leq b'\).)  For the Iter-dependent rule (Backwards
Variant in~\cite{OHearn_2020}), the function \(b(n)\) corresponds
to a sequence of tests \((b_n)_{n \in ℕ}\), and the existential
\(∃ n, b(n)\) corresponds to the infinite disjunction of all these
\(b_n\), which we express as \(\sup_{n ∈ ℕ} b_n\).  (Note that \(\sup\)
does not exist in KATs, TopKATs, or boolean algebras in general, and this rule
implicitly assumes that \(\sup_{n \in ℕ} b(n)\) exists.)

\begin{theorem}[Soundness of \(\okState\) State Rules]\label{the: incor logic rule sound ok}
    All the rules of \Cref{fig: normal terminating Inc rule}
    are derivable in all TopKATs.
\end{theorem}

\begin{proof}
    The proofs of each individual rules as follows:
    \begin{itemize}
        \item \textbf{Empty Rule:}
            because 0 is the smallest element 
        \item \textbf{Consequence Rule:}
            because multiplication preserves order
            and \(b' \geq b\), therefore \(\top b' p \geq \top b p\).
            Therefore 
            \[\top b' p \geq \top b p \geq c \geq c'\] 
        \item \textbf{Disjunction Rule:}
            because addition preserves order, therefore 
            \[\top b_1 p + \top b_2 p \geq c_1 + c_2\]
            and by distributivity, we have 
            \[\top (b_1 + b_2) p \geq c_1 + c_2\]
        \item \textbf{Identity Rule:}
            because \(\top \geq 1\),
            therefore \[\top b 1 \geq 1 b 1 \geq b\]
        \item \textbf{Composition Rule:}
            first, by idempotency of \(\top\)
            and associativity of multiplication, we have 
            \[\top a (p q) = \top \top a p q\]
            By order preserving of multiplication, 
            and \(\top a p \geq b\),
            we have \[\top a (p q) = \top \top a p q \geq \top b q \geq c\]
        \item \textbf{Choice Left and Choice Right Rule:}
            by distributivity, we have 
            \[\top b (p + q) 
            = \top b p + \top b q 
            \geq \top b p \geq c\]
            and 
            \[\top b (p + q) 
            = \top b p + \top b q 
            \geq \top b q \geq c\]
        \item \textbf{Assume Rule:}
            since \(\top \geq 1\), 
            we have \[\top b c = \top b c \geq 1 b c = bc\]
        \item \textbf{Iter-Zero Rule:}
            since \({p}^* = p {p}^* + 1\),
            we have 
            \[\top b {p}^* 
            = \top b (p {p}^* + 1)
            \geq \top b 1 \geq b\]
        \item \textbf{Iter-NonZero Rule:}
            since \({p}^* = p {p}^* + 1\),
            we have 
            \[\top b {p}^* 
            = \top b (p {p}^* + 1)
            \geq \top b p {p}^* \geq c\]
        \item \textbf{Iter-Dependent Rule:}
            by definition of \(\sup\),
            in order to show \(\top b_{0} {p}^* \geq \sup_{n \in ℕ} b_{n}\)
            all we need to show is that for all \(m \in ℕ\)
            \(\top b_{0} {p}^* \geq b_{m}\).
            We prove this by induction on \(m\).
            \begin{itemize}
                \item \textbf{Base Case:}
                    we need to show \(\top b_{0} {p}^* \geq b_{0}\).
                    Because \({p}^* = 1 + p{p}^*\),
                    therefore 
                    \[\top b_{0} {p}^* 
                    = \top b_{0} (1 + p{p}^*)
                    \geq \top b_{0} 1 \geq 1 b_{0} 1 = b_{0}\]
                \item \textbf{Inductive Case:}
                    if \(\top b_{0} {p}^* \geq b_{n}\),
                    we need to show \(\top b_{0} {p}^* \geq b_{n+1}\).
                    By premise we have \(\top b_{n} p \geq b_{n + 1}\)
                    Therefore by idempotency of the top element,
                    we have 
                    \begin{align*}
                        \top b_{0} {p}^* 
                        & = \top \top b_{0} (1 + {p}^* p) \\
                        & \geq \top \top b_{0} {p}^* p
                        & 1 + {p}^* p \geq {p}^* \\
                        & \geq \top b_{n} p
                        & \text{by induction hypothesis} \\
                        & \geq b_{n+1}
                        & \text{by premise}
                    \end{align*}
            \end{itemize}   
    \end{itemize}
\end{proof}

\paragraph*{An alternative encoding for Hoare logic}

Since \(\TopKAT\) can express codomain (\Cref{the: top element represent domain}), we can exploit the symmetry between incorrectness logic and Hoare logic (\Cref{sec:incorrectness-and-hoare}) to give another encoding of Hoare logic in \(\TopKAT\): 
\[\top b p \leq \top c.\] 
This is equivalent to the original encoding $bp = bpc$ proposed for KATs~\cite{Kozen_2000} in \emph{all TopKAT}, not just relational ones.  
Since the proof rules of propositional Hoare logic are sound for that encoding~\cite[Theorem~3.1]{Kozen_2000}, they are also sound for ours.

\begin{theorem}[Equivalence of Hoare Logic Formulations]\label{the: hoare logic formulation equivalent}
  For all TopKATs \((𝒦, ℬ)\), and three elements
  \(p \in 𝒦\) and \(b, c \in ℬ\), the following inequalities
  are equivalent:
    \[b p \leq \top c  \iff  \top b p \leq \top c \iff b p = b p c.\]
\end{theorem}

\begin{proof}
\begin{itemize}
    \item We first show the equivalence:
        \[b p ≤ ⊤ c  ⟺  ⊤ b p ≤ ⊤ c.\]
        \(⟸\) is by \(1 ≤ ⊤\),
        thus \[b p \leq \top b p \leq \top c,\]
        and \(⟹\) is by the idempotency of \(\top\),
        multiply both side with \(\top\), we have 
        \[\top b p \leq \top \top c = \top c.\]
    \item Then we show the equivalence
        \[b p \leq \top c  \iff  b p = b p c.\]
        Start with \(⟸\), because \(⊤ ≥ b p\), we have
        \[b p = b p c ≤ ⊤ c.\]
        Then \(⟹\), we start from \(b p ≤ ⊤ c\),
        we first multiply \(\overline{c}\) on both side,
        \[b p \overline{c} \leq \top c \overline{c} = \top 0 = 0,\]
        and because \(b p \overline{c} \geq 0\) (\(0\) is the smallest element),
        we have \[b p \overline{c} = 0.\]
        Finally, add \(b p c\) to both side, we have
        \[b p = b p (\overline{c} + c) = 0 + b p c = b p c.\qedhere\]
\end{itemize}
\end{proof}

\section{(In)completeness and Decidability of TopKAT}\label{sec: properties of TopKAT}


In the previous section, we have shown that the theory of TopKAT subsumes incorrectness logic. In this section, we study some of the properties of its algebraic equational theory.
%
First, we will show that TopKAT is \emph{incomplete} with respect to relational
TopKATs: there are identities that are valid for every relational TopKAT that
cannot be proven using the TopKAT identities alone. This contrasts with what
happens for KAT, which is complete for relational KATs.
%
We will then show the completeness of TopKATs with respect to a class of
\emph{language-based} TopKATs and also with respect to a more general class of
\emph{relational} TopKATs, as well as the decidability of TopKAT equalities.
Finally we will introduce the concept of equational expressiveness,
and show that the general relational TopKAT has the same expressiveness as relational KAT,
hence cannot express incorrectness logic.

\subsection{Incompleteness with Respect to Relational TopKATs}

We start with our negative result, which is the incompleteness over relational TopKATs.

\begin{theorem}\label{the: topkatstar incomplete over relational model}
  \(\TopKAT\) is incomplete over \(\TopREL\): the formulas
  \[(⊤ p) (⊤ p) = ⊤ p \text{ and } p ⊤ p ≥ p\] 
  are valid in every relational TopKAT, but not in every TopKAT\@.
\end{theorem}

\begin{proof}
  We can show that the \((⊤ p) (⊤ p) = ⊤ p\) 
  holds in every relational TopKAT simply by
  unfolding the definitions.  To exhibit a TopKAT where it does not hold,
  consider the following counterexample.  We take a relational KAT over
  \(\{0,1\}\) whose largest element is
  \begin{align*}
    ⊤ & ≜ \{(0,0), (1,1), (0,1)\}.
  \end{align*}
  (Note that this KAT is \emph{not} a relational TopKAT\@: the largest
  element is not the complete relation.)
  Then, by taking a interpretation lifted by \(I(p) ≜\{(0, 1)\}\), we see that
  \begin{align*}
    I(⊤ p) & = \{(0, 1)\} \\
    I((⊤ p) (⊤ p)) & 
        = \{(0, 1)\} ⋅ \{(0, 1)\} 
        = ∅ ≠ I(⊤ p)
  \end{align*}

  And the same counter example also works for 
  \(p ⊤ p ≥ p\).
\end{proof}

In fact, \citeauthor{Pous_Wagemaker_2022}~\cite{Pous_Wagemaker_2023,Pous_Wagemaker_2022} has shown that with the additional axiom \(p ⊤ p ≥ p\), the theory of TopKAT is indeed complete with relational model.

Without this additional axiom, the incompleteness result might be discouraging, since we might not be able to derive some valid theorems of incorrectness logic using TopKAT. However in \Cref{the: incor logic rule sound ok} and \Cref{the: incorrectness logic fail rule sound}, we show that all the rules of incorrectness logic are derivable using either the equational theory of TopKAT or FailTopKAT\@.  Thus,
our framework is at least as strong as the structural proof rules presented by~\citet{OHearn_2020}.

In fact, this power of TopKAT is not accidental: as we will show in~\cref{sec: domain completeness of TopKAT}, TopKAT without the additional axiom \(p ⊤ p ≥ p\), is still complete with respect to domain comparison inequalities. Concretely, given any KAT terms \(e₁, e₂ ∈ \KAT\), the inequalities of the following form:
\[⊤ e₁ ≥ ⊤ e₂ \text{ and } e₁ ⊤ ≥ e₂ ⊤\]
are called domain comparison inequalities; because they corresponds to domain comparisons in relational models (\cref{the: top element represent domain}).
And these forms of inequalities subsumes the encoding of both propositional Hoare and incorrectness logic.

\subsection{Reduction on free models}\label{sec: reduction on free models}

Our goal for \Cref{sec: reduction on free models,sec: complete model for free} is to construct a complete interpretation for TopKAT, by reducing its theory to that of plain KAT.  
In other words, any equation between two TopKAT expressions is logically equivalent to another equation between a pair of corresponding KAT terms.  
We present a streamlined proof that hinges on the universal properties of free KATs and TopKATs, without relying explicitly on language models, unlike previous definitions of reduction~\cite{Pous_Rot_Wagemaker_2021,Kozen_Smith_1997}.  
Similar to previous works, we obtain the
decidability of the equational theory of TopKAT as a corollary of reduction.
However, because of the new notion of reduction,
our decidability result no longer depends on the completeness of the language TopKAT.  
Moreover, our technique helps us to construct complete models and interpretations 
simply by computation, as well as simplifying proofs of other results about TopKAT.


We first note that any free KAT over an alphabet \(K, B\) is also a TopKAT,
where the largest element is \((∑ K)^*\). This fact can be seen by
straightforward induction.

\begin{lemma}\label{the: every free KAT is a TopKAT}
    Every free KAT over alphabet \(K, B\) forms a TopKAT.
\end{lemma}

\begin{proof}
    Since \(\KAT_{K, B}\) is a KAT, we only need to show 
    the expression \((∑ K)^*\) is the largest element of \(\KAT_{K, B}\),
    i.e. \[(∑ K)^* ≥ e, ∀ e ∈ \KAT_{K, B}.\] 
    The above fact can be shown by induction on \(e\);
    some algebraic manipulations below use facts in~\Cref{the: well known fact about KA}:
    \begin{itemize}
        \item \((∑ K)^* ≥ 1\) (by unfolding rule),
              thus \((∑ K)^*\) is larger than \(0, 1\) and every Boolean term.
        \item \((∑ K)^*\) is larger than \(∑ K\),
              which is larger than every primitive action.
        \item Given two expressions \(e₁\) and \(e₂\),
              assume \((∑ K)^*\) is larger than both.
              Because \((∑ K)^* = (∑ K)^* + (∑ K)^*\)
              and addition preserves order,
              \[(∑ K)^* = (∑ K)^* + (∑ K)^* ≥ e₁ + e₂\] 
        \item Given two expressions \(e₁\) and \(e₂\),
              assume \((∑ K)^*\) is larger than both.
              Because \((∑ K)^* = (∑ K)^* ⋅ (∑ K)^*\)
              and multiplication preserves order, 
              \[(∑ K)^* = (∑ K)^* ⋅ (∑ K)^* ≥ e₁ ⋅ e₂.\]
        \item Given a expression \(e\),
              if \((∑ K)^* ≥ t\), then \((∑ K)^* ≥ t^*\).
              Since \((∑ K)^* = ((∑ K)^*)^*\) and star preserves order:
              \[(∑ K)^* = ((∑ K)^*)^* ≥ t^*. \qedhere\]
    \end{itemize}
\end{proof}

Since every free KAT is a TopKAT, every KAT interpretation
\(I : \KAT → \mathcal{K}\) induces a sub-KAT $\Img(I) ⊆ 𝒦$,
and this sub-KAT happens to be a \emph{TopKAT}. Specifically, the image of $(∑ K)^*$
in $𝒦$ is the largest element of $\Img(I)$, and the restricted
$I : \KAT → \Img(I)$ is a TopKAT homomorphism.

This gives us a powerful tool to construct complete TopKAT interpretations.
Since we already know that the KAT interpretations \(G: \KAT → 𝒢\) and
\(h ∘ G: \KAT → \Img(h)\) are injective TopKAT homomorphisms, we can
construct complete TopKAT interpretations by \emph{composition}, 
if we can construct an injective TopKAT interpretation \(r\) of type
\(\TopKAT_{K, B} → \KAT_{K_⊤, B}\):
\begin{align*}
    \TopKAT_{K, B} \xrightarrow{r} \KAT_{K_⊤, B} \xrightarrow{G} 𝒢_{K_⊤, B},\\
    \TopKAT_{K, B} \xrightarrow{r} \KAT_{K_⊤, B} \xrightarrow{G} 𝒢_{K_⊤, B} \xrightarrow{h} \Img(h).
\end{align*} 

In fact, such an injective homomorphism can be obtained by lifting 
the embedding map \(K + B ↪ \KAT_{K_⊤, B}\):
\begin{align*}
    r   & : K + B → \KAT_{K_⊤, B}             \\
    r(p) & ≜ p.                   
\end{align*}
This homomorphism coincides with the \emph{reduction maps} of the same name in
previous works~\cite{Zhang_de_Amorim_Gaboardi_2022, Pous_Wagemaker_2023}.  More
concretely, we can picture $r$ as simply replacing the symbol \(⊤\) in a TopKAT
term with \((∑ K_⊤)^*\), the largest element in \(\KAT_{K_⊤, B}\).

We will show that \(r\) is injective by constructing a left inverse for it.  
In fact, the left inverse \([-]_⊤\) simply interprets the \(⊤\) primitive in \(\KAT_{K_⊤, B}\)
as the largest element.
\begin{lemma}\label{the: equivalence class is the inverse of reduction}
  The map \([-]_⊤: \KAT_{K_⊤, B} → \TopKAT_{K, B}\), where each expression is
  mapped to its corresponding equivalence class, 
  is defined by lifting the following action on the primitives:
  \begin{align*}
    [p]_⊤ & ≜ p & p ∈ K + B \\  
    [⊤]_⊤ & ≜ ⊤.
  \end{align*}
  The map \([-]_⊤\) is a TopKAT homomorphism.
\end{lemma}

\begin{proof}
  Because this map defined by lifting on the primitives,
  it is automatically a KAT homomorphism.
  All we need to show is that \([-]_⊤\) preserves the top element, that is
  \([(∑ K_⊤)^*]_⊤ = (∑ K_⊤)^*\) is the largest element in \(\TopKAT_{K, B}\).

  By construction of \(\TopKAT_{K, B}\), \(⊤\) is the largest element in \(\TopKAT_{K, B}\). 
  Thus, to prove that \((∑ K_⊤)^*\) is also the largest element in \(\TopKAT_{K, B}\),
  it suffices to prove \((∑ K_⊤)^* ≥ ⊤\): \[(∑ K_⊤)^* ≥ ∑ K_⊤ = ⊤ + ∑ K ≥ ⊤. \qedhere\]
\end{proof}

\begin{theorem}[Reduction]
    \([-]_⊤\) is the right inverse of \(r\): \([-]_⊤ ∘ r  = id_{\TopKAT_{K, B}}\).
    More explicitly for all \(e ∈ \TopKAT_{K, B}\): \[\TopKAT_{K, B} ⊧ [r(e)]_⊤ = e.\]
\end{theorem}

\begin{proof}
    Since \([-]_⊤ ∘ r : \TopKAT_{K, B} → \TopKAT_{K, B}\) is a TopKAT interpretation,
    the action on the primitives uniquely determines the interpretation:
    because both \(r\) and \([-]_⊤\) are identity on the primitives,
    therefore \([-]_⊤ ∘ r\) is the identity interpretation on \(\TopKAT_{K, B}\).
\end{proof}

The above theorem matches one of the soundness condition of reductions in 
previous works~\cite{Zhang_de_Amorim_Gaboardi_2022,Kozen_Smith_1997,Pous_Rot_Wagemaker_2021},
which was typically proven by a monolithic induction on the structure of terms.
Our approach, on the other hand, relies on establishing fine-grained 
algebraic properties, like~\cref{the: every free KAT is a TopKAT,the: equivalence class is the inverse of reduction};
then the theorem follows simply by computing the action of \([-]_⊤ ∘ r\) on primitives.

Since \(r\) has a right inverse, it is indeed the injective interpretation we desired, 
and it is also a complete interpretation:
\[\TopKAT_{K, B} ⊧ e₁ = e₂ ⟺ r(e₁) = r(e₂),\]
With the completeness of \(r\), we can already show the complexity of TopKAT.
The complexity results echos previous proofs~\cite{Zhang_de_Amorim_Gaboardi_2022,Pous_Wagemaker_2023},
but we are able to obtain this result without completeness of TopKAT language interpretation,
which is essential in previous proofs. 

\begin{corollary}[Complexity]\label{the: PSPACE-completeness of TopKAT}
  Given two expressions \(e₁, e₂ ∈ \TopKAT_{K, B}\), deciding whether these two terms
  are equal is PSPACE-complete.
\end{corollary}

\begin{proof}
    Deciding KAT equality is a sub-problem of deciding TopKAT equality,
    and KAT equality is PSPACE-hard \cite{Cohen_Kozen_Smith_1999};
    therefore TopKAT equality is PSPACE-hard.

    To decide the equality of \(e₁, e₂\),
    we first remove all the redundant primitives that do not appear in \(e₁, e₂\)
    from the alphabet \(K, B\). Then we compute \(r(e₁)\) and \(r(e₂)\),
    each taking polynomial space (of \(|e₁| + |e₂|\)) to store;
    and we use the standard algorithm \cite{Cohen_Kozen_Smith_1999}
    to decide whether \(r(e₁) = r(e₂)\) in \(\KAT_{K_⊤, B}\),
    this will also take polynomial space.
    Hence, the decision procedure for TopKAT equality in PSPACE.

    Thus deciding TopKAT equality is PSPACE-complete.
\end{proof}


\subsection{Computing the complete interpretations}\label{sec: complete model for free}

Designing complete interpretations and models was not always easy.
In fact, in previous works \cite{Zhang_de_Amorim_Gaboardi_2022_POPL},
the authors made a mistake in the definition of language TopKAT,
which was fixed later \cite{Zhang_de_Amorim_Gaboardi_2022} 
by suggestion of Pous et al. \cite{Pous_Wagemaker_2022}.
However, with the results in~\Cref{sec: reduction on free models},
we can construct the complete interpretation just by composition,
and compute the complete model by computing the range of the complete interpretation.

We already know that there are two complete interpretations of TopKAT defined as follows:
\begin{align*}
    \TopKAT_{K, B} \xrightarrow{r} \KAT_{K_⊤, B} \xrightarrow{G} 𝒢_{K_⊤, B},\\  
    \TopKAT_{K, B} \xrightarrow{r} \KAT_{K_⊤, B} \xrightarrow{G} 𝒢_{K_⊤, B} \xrightarrow{h} \Img(h),
\end{align*}
with a complete language model \(𝒢_{K_⊤, B}\), 
and a complete model consisting of relations \(\Img(h)\).

The operations in these models can be recovered by computing these maps.
For example, the multiplication operation in the language TopKAT can be computed as follows:
\[G ∘ r(e₁ ⋅ e₂) = G(r(e₁) ⋅ r(e₂)) = G(r(e₁)) ⋄ G(r(e₂)).\]
Since \(r\) does not change the multiplication operation,
the multiplication in the language TopKAT is the same as in language KAT.
In fact, as \(r\) does not change any operation in KAT,
most operations in language TopKAT are the same as language KAT.
Thus, we only need to compute the top element in language TopKAT.

The top element in language TopKAT can be computed in the same fashion:
\[G ∘ r(⊤) = G((∑ K_⊤)^*) = GS_{K_⊤, B},\]
i.e. the top element is just the complete language.

\begin{corollary}\label{the: language TopKAT for free}
    The language TopKAT inherits all the operations in language KAT,
    except the top element, which is defined as the full language.
    And such models are complete with \(G ∘ r\) as a complete interpretation.
\end{corollary}

In the same way, we know that complete models consisting of relations (a.k.a. general relational TopKAT) 
will have the same operations as relational KATs.
However, in this case the characterization of the computed top: \(h ∘ G ∘ r(⊤)\)
is not as simple as the full language,
but we know it is the largest relation in the range of \(h ∘ G ∘ r\):

\begin{corollary}\label{the: general relational TopKAT for free}
    The general relational TopKAT inherits all the operations in relational KAT,
    except the top element is the largest relation.
    And such models are complete with \(h ∘ G ∘ r\) as a complete interpretation.
\end{corollary}

Finally, to investigate whether we can use general relational TopKAT
to encode incorrectness logic,
we will provide a short proof that general relational TopKATs
are as expressive as relational KATs~\cite{Zhang_de_Amorim_Gaboardi_2022};
that is, every property on relations that can be encoded using general relational TopKAT,
is already encodable in the relational KAT.
Hence, adding a top element does not give extra expressive power in general relational TopKAT.

The original proof~\cite[Lemma 2]{Zhang_de_Amorim_Gaboardi_2022} 
encodes every TopKAT expression using a KAT term,
and then uses two pages to prove the soundness of this encoding.
Here we show the aforementioned encoding is simply the reduction \(r\).

\begin{definition}
    Given two expressions \(e₁, e₂ ∈ \TopKAT\), and n primitives \(p₁, p₂, … , pₙ ∈ K + B\),
    we say that an n-ary predicate \(P\) is \emph{expressible} by 
    equation \(e₁ = e₂\) for a class of TopKATs \(\mathsf{K}\) 
    when for all interpretations \(I\) into TopKATs in \(\mathsf{K}\),
    the following equivalence holds:
    \[I(e₁) = I(e₂) ⟺ P(I(p₁), I(p₂), …, I(pₙ)).\]
\end{definition}

\begin{theorem}[Expressiveness of general relational TopKAT]\label{the: TopGREL expressive power}
    Given an alphabet \(K, B\), an n-ary predicate \(P\) on relations,
    the predicate \(P\) over primitives \(p₁, p₂, … , pₙ ∈ K\) is expressible in
    general relational TopKAT if and only if it is expressible in relational KAT.
\end{theorem}

\begin{proof}
    A predicate expressible in relational KAT is also expressible 
    in general relational TopKAT using the same pair of terms,
    we only need to show the converse.
    Assume a predicate \(P\) is expressible in general relational TopKAT,
    then there exists two TopKAT expressions \(e₁, e₂ ∈ \TopKAT_{K, B}\) s.t. 
    for all general relational TopKAT interpretations \(I_⊤\):
    \[I_⊤(e₁) = I_⊤(e₂) ⟺ P(I_⊤(p₁), I_⊤(p₂), … , I_⊤(pₙ));\]

    We take an arbitrary relational KAT interpretation \(I\) from \(\KAT_{K_⊤, B}\).
    Notice \(\Img(I)\), the range of \(I\), 
    is a relational KAT with the largest element \(I((∑ K)^*)\),
    i.e. \(\Img(I)\) is a general relational TopKAT.
    Because \(I\) is a KAT interpretation, 
    it preserves all the KAT operations and the largest element.
    Hence, \(I\) is a TopKAT homomorphism from \(\KAT_{K_⊤, B}\) to \(\Img(I)\).

    Then we can construct \(I ∘ r: \TopKAT_{K, B} → \Img(I)\),
    a general relational interpretation:
    \begin{align*}
        I(r(e₁)) = I(r(e₂))
         & ⟺ I ∘ r(e₁) = I ∘ r(e₂)                           \\
         & ⟺ P(I ∘ r(p₁), … , I ∘ r(pₙ))
            & \text{\(I ∘ r\) is a \(\TopGREL\) interpretation} \\
         & ⟺ P(I(p₁), … , I(pₙ))
            & r(pᵢ) = pᵢ
    \end{align*}
    Thus the two KAT expressions \(r(e₁), r(e₂) ∈ \KAT_{K_⊤, B}\) also can express the predicate \(P\).
\end{proof}

Since the image of \(I\) is not necessarily a relational TopKAT, where the top element is interpreted as the complete relation, the above trick does not work for relational TopKAT.
In fact, in~\Cref{the: topkat can express incorrectness logic normal termination,the: KAT not able to express incorrectness logic}, we have already showed that relational TopKAT is strictly more expressive than relational KAT,
since relational TopKAT can encode incorrectness logic, where relational KAT cannot.


\subsection{(Co)domain Completeness}\label{sec: domain completeness of TopKAT}

As we have shown in~\Cref{the: topkatstar incomplete over relational model}, TopKAT is not complete over relational models, which are crucial for applications in program logics (\Cref{the: top element represent domain}).  
However, \citet{Pous_Wagemaker_2023} later showed that we can obtain a complete theory for relational models by simply adding the axiom \(p ⊤ p ≥ p\) to the theory of TopKAT. 

Here, we take a different approach than Pous et al.~\cite{Pous_Wagemaker_2023}: instead of extending the TopKAT framework, we will restrict the completeness result.
In particular, the encoding of incorrectness logic and Hoare Logic in TopKAT~\cite{Zhang_de_Amorim_Gaboardi_2022}
relies only on the ability of TopKAT to compare the domain and codomain of two
relations.  This raises the question of whether TopKAT suffices for proving such
properties; that is, whether the following completeness results hold: for
\(e₁, e₂ ∈ \KAT_{K, B}\) (i.e. \(⊤\) does not appear in \(e₁\) and \(e₂\))
\begin{align*}
    \REL ⊧ \cod(e₁) ≥ \cod(e₂) & ⟺ \TopKAT ⊧ ⊤ e₁ ≥ ⊤ e₂ & \text{codomain completeness} \\
    \REL ⊧ \dom(e₁) ≥ \dom(e₂) & ⟺ \TopKAT ⊧ e₁ ⊤ ≥ e₂ ⊤ & \text{domain complete}
\end{align*}

In this section, we prove that these equivalences hold, even without the additional axiom.
However, they do \emph{not} hold if we allow expressions that contain top.
For example, let \(e₁ ≜ p ⊤ p\), and \(e₂ ≜ p\). Since \(p ⊤ p ≥ p\) holds in
relational TopKAT, thus \(\dom(p ⊤ p) ≥ \dom(p)\). 
However, \(p ⊤ p ⊤ ≥ p ⊤\) is not provable in TopKAT, 
because the inequality is not valid with the language interpretation.
The incompleteness of codomain comparison can also be shown using the same example.

\subsubsection{Codomain completeness}

The core insight to prove the domain completeness result is 
to construct a specific relational interpretation \(h ∘ i ∘ G\),
where its codomain is equivalent to the complete TopKAT interpretation \(G ∘ r\):
\[\cod(h ∘ i ∘ G(e)) = G ∘ r(e),\]
where \(i\) is the natural inclusion homomorphism \(i: 𝒢_{K, B} ↪ 𝒢_{K_⊤, B}\), 
that maps every language to itself;
and \(h\) is the classical embedding of language KAT into relational KAT~\cite{Kozen_Smith_1997},
which we will recall as follows:
\[h(W) = \{(w, w ⋄ w') ∣ w ∈ GS, w' ∈ W\}.\]
Although \(i\) will not change the outcome of \(G\),
it will add a new primitive action \(⊤\) to the alphabet, hence changing the outcome of \(h\).
Such addition will equate the codomain of \(h ∘ i ∘ G(e)\) 
with the complete TopKAT interpretation \(G ∘ r\) of \(⊤ e\).
The proof of this equality is by simply computing both sides of the equation.

\begin{lemma}\label{the: codomain completeness core lemma}
    For any expression \(e ∈ \KAT_{K, B}\),
    \[\cod(h ∘ i ∘ G(e)) = G ∘ r(e).\]
\end{lemma}

\begin{proof}
    We explicitly write out the domain and codomain of the functions in
    the relational KAT interpretation \(h ∘ i ∘ G\) for the ease of the reader:
    \[\KAT_{K, B}
        \xrightarrow{G} 𝒢_{K, B}
        \xrightarrow{i} 𝒢_{K_⊤, B}
        \xrightarrow{h} 𝒫(𝒢_{K_⊤, B} × 𝒢_{K_⊤, B}).\]
    In this case, \(h\) is a KAT homomorphism from \(𝒢_{K_⊤, B}\):
    \[h(W) = \{(w, w ⋄ w₁) ∣ w ∈ GS_{K_⊤, B}, w₁ ∈ S\}.\]
    Since the reduction \(r\) preserves expressions without \(⊤\),
    let \(e ∈ \KAT_{K, B}\) (i.e. \(e\) does not contain \(⊤\)),
    \[G ∘ r(⊤) = GS_{K_⊤, B} \\ G ∘ r(e) = G(e).\]
    Therefore, for any expression \(e ∈ \KAT_{K, B}\)
    \begin{align*}
        \cod(h ∘ i ∘ G(e))
         & = \{w α w₁ ∣ w α ∈ GS_{K_⊤, B}, α w₁ ∈ G(e)\} \\
         & = GS_{K_⊤, B} ⋄ G(e)                          \\
         & = (G ∘ r(⊤)) ⋄ (G ∘ r(e))                     \\
         & = G ∘ r(e). \qedhere
    \end{align*}
\end{proof}

\Cref{the: codomain completeness core lemma} established a connection between 
the codomain operator and the language interpretation of TopKAT.
Then by completeness of the language interpretation, 
we will obtain the completeness of codomain comparison.

\begin{theorem}[Codomain completeness]\label{the: codomain completeness}
    Given two expressions \(e₁, e₂ ∈ \KAT_{K, B}\) (i.e. expressions without \(⊤\)),
    then codomain comparison is complete:
    \begin{align*}
        \REL ⊧ \cod(e₁) ≥ \cod(e₂) & ⟺ \TopKAT ⊧ ⊤ e₁ ≥ ⊤ e₂.
    \end{align*}
\end{theorem}

\begin{proof}
    Given the natural inclusion homomorphism: \(i: \KAT_{K, B} → \KAT_{K_⊤, B}\),
    we show that the following are equivalent:
    \begin{enumerate}
        \item \(\REL ⊧ \cod(e₁) ≥ \cod(e₂).\)
        \item \(\cod(h ∘ i ∘ G(e₁)) ≥ \cod(h ∘ i ∘ G(e₂)).\)
        \item \(\TopKAT ⊧ ⊤ e₁ ≥ ⊤ e₂.\)
    \end{enumerate}

    We first show that \(1 ⟹ 2\), by definition, \(\REL ⊧ \cod(e₁) ≥ \cod(e₂)\)
    implies \(\cod(I(e₁)) ≥ \cod(I(e₂))\) for all relational KAT interpretations \(I\).
    Because \(h ∘ i ∘ G\) is a relational KAT interpretation, so \(1 ⟹ 2\).

    We show \(2 ⟹ 3\), which uses the equality discussed above, 
    and proved in~\Cref{the: codomain completeness core lemma}:
    \begin{align*}
             & \cod(h ∘ i ∘ G(e₁)) ≥ \cod(h ∘ i ∘ G(e₂))           \\
        ⟺ {} & G ∘ r(⊤ e₁) ≥ G ∘ r(⊤ e₂)
             & \text{\Cref{the: codomain completeness core lemma}} \\
        ⟺ {} & \TopKAT ⊧ ⊤ e₁ ≥ ⊤ e₂.
             & \text{Completeness of \(G ∘ r\)}
    \end{align*}

    Finally, we show \(3 ⟹ 1\), by \Cref{the: top element represent domain}:
    \[\TopKAT ⊧ ⊤ e₁ ≥ ⊤ e₂ ⟹ \TopREL ⊧ ⊤ e₁ ≥ ⊤ e₂ ⟹ \REL ⊧ \cod(e₁) ≥ \cod(e₂). \qedhere\]
\end{proof}

\subsubsection{Domain completeness}

The domain completeness result can be derived from codomain completeness 
by observing properties of opposite TopKAT and the converse operator \((-)^{∨}\), 
both of which we will recall below.

For every TopKAT \(𝒦\), we can construct the opposite TopKAT \(𝒦^{\op}\) 
by reversing the multiplication operation, keeping the sorts and other operations unchanged:
\[p \mathbin{\hat{⋅}} q ≜ q ⋅ p,\]
where \(\hat{⋅}\) is multiplication in \(𝒦^{\op}\) and \(⋅\) is multiplication in \(𝒦\).
By definition, \((-)^{\op}\) is a involution, that is \({(𝒦^{\op})}^{\op} = 𝒦\).
Furthermore, \((-)^{\op}\) is a TopKAT functor,
this means all TopKAT homomorphisms \(h: 𝒦 → 𝒦'\) 
can be lifted to a TopKAT homomorphism on the opposite TopKAT \(h^{\op}: 𝒦^{\op} → {𝒦'}^{\op}\). 
The lifting \(h^{\op}\) is point-wise equal to \(h\):
\[∀ p ∈ 𝒦, h^{\op}(p) ≜ h(p).\]
The fact that \(h^{\op}\) is a TopKAT homomorphism can be proven by unfolding the definition,
and the functor laws are satisfied because \(h^{\op}\) is point-wise equal to \(h\).

There are two important homomorphisms involving opposite TopKAT:
\begin{align*}
    (-)^{∨} & : (X × X)^{\op} → (X × X) &
    \op & : \TopKAT_{K, B} → \TopKAT^{\op}_{K, B} \\  
    (R)^{∨} & = \{(b, a) ∣ (a, b) ∈ R\}, & 
    ∀ p ∈ K + B, \op & (p) = p.
\end{align*}
The \((-)^{∨}\) is the relational converse operator, 
the rules of homomorphism can simply be proven by unfolding of definitions.
The crucial property of \((-)^{∨}\) is that it flips the domain and codomain:
\begin{equation}\label{the: converse flips domain to codomain}
    \dom(R^{∨}) = \cod(R).
\end{equation}
Hence, allowing us to flip the result about codomains and apply it to domains.

\(\op\) is a homomorphism from free TopKAT to its opposite TopKAT;
it can be defined by lifting the embedding function \(K + B ↪ \TopKAT_{K, B}\) on primitives.
Intuitively, given a expression \(e ∈ \TopKAT\), 
\(\op(e)\) will flip all the multiplications in \(e\) recursively.
\begin{lemma}\label{the: injectivity of op}
    the left inverse of \(op\) can be obtained by lifting itself through the \((-)^{\op}\) functor,
    \[\op^{\op}: \TopKAT^{\op} → (\TopKAT^{\op})^{\op} = \TopKAT.\]
    Recall \(\op^{\op}\) is pointwise equal to \(\op\), 
    thus \(\op^{\op} ∘ \op: \TopKAT → \TopKAT\) is the identity interpretation 
    because it preserves all the primitives.
    Thus, \(\op\) has a left inverse, hence it is injective:
    \[e₁ = e₂ ⟺ \op(e₁) = \op(e₂).\]
\end{lemma}

Finally, since the elements in \(\TopKAT^{\op}\) are the same as \(\TopKAT\), 
which are TopKAT expressions modulo provable TopKAT equalities,
theorems about TopKAT expressions are also true for elements in \(\TopKAT^{\op}\).
In particular, codomain completeness (\Cref{the: codomain completeness})
also holds in \(\TopKAT^{\op}\): 
for all expressions \(e₁, e₂ ∈ \TopKAT\),
\begin{equation}\label[equiv]{the: op codomain completeness}
    ⊤ ⋅ \op(e₁) ≥ ⊤ ⋅ \op(e₂) ⟺ \REL ⊧ \cod(\op(e₁)) = \cod(\op(e₂)).
\end{equation}

\begin{theorem}[Domain Completeness]\label{the: domain completeness}
    For all expressions \(e₁, e₂ ∈ \KAT\), the following equivalence hold:
    \[\REL ⊧ \dom(e₁) = \dom(e₂) ⟺ \TopKAT ⊧ e₁ ⊤ ≥ e₂ ⊤.\]
\end{theorem}

\begin{proof}
    \(⟸\) direction is trivial by \Cref{the: top element represent domain};  
    and \(⟹\) direction can be derived as follows:
    let \(I\) be some relational interpretation,
    then \(I^{\op}(\op(-))^∨\) is also a relational interpretation:
    \[I^{\op}(\op(-))^∨: 
        \TopKAT \xrightarrow{\op} \TopKAT^{\op} \xrightarrow{I^{\op}} 
        (X × X)^{\op} \xrightarrow{(-)^{∨}} (X × X).\]
    Thus, we let \(I\) range over all relational interpretations:
    \begin{align*}
        & \REL ⊧ \dom(e₁) ⊇ \dom(e₂)  \\
        & ⟹ ∀ I, \dom(I(e₁)) ⊇ \dom(I(e₂)) \\
        & ⟹ ∀ I, \dom(I^{\op}(\op(e₁))^∨) ⊇ \dom(I^{\op}(\op(e₂))^∨) 
            &\text{specialize \(I\) as \(I^{\op}(\op(-))^∨\)}\\  
        & ⟹ ∀ I, \cod(I^{\op}(\op(e₁))) ⊇ \cod(I^{\op}(\op(e₁))) 
            &\text{\Cref{the: converse flips domain to codomain}}\\
        & ⟹ ∀ I, \cod(I(\op(e₁))) ⊇ \cod(I(\op(e₁))) 
            &\text{\(I^{\op}\) is pointwise equal to \(I\)}\\
        & ⟹ ⊤ ⋅ \op(e₁) ≥ ⊤ ⋅ \op(e₂) 
            &\text{\Cref{the: op codomain completeness}}\\
        & ⟹ \op(⊤ ⋅ e₁) ≥ \op(⊤ ⋅ e₂) 
            & \text{Definition of \(\op\)}\\
        & ⟹ e₁ ⊤ ≥ e₂ ⊤ & \text{\Cref{the: injectivity of op}}
    \end{align*}
\end{proof}

\begin{remark}
    Alternatively, \Cref{the: domain completeness} can also be proven 
    by constructing the following \(h'\):
    \begin{align*}
        h' & : 𝒢_{K, B} → 𝒫(𝒢_{K, B} × 𝒢_{K, B})\\
        h' & (W₁) ≜ \{(w₁ α w, α w) ∣ w₁ α ∈ W₁, α w ∈ GS_{K, B}\}.
    \end{align*}
    Then the proof would mirror that of \Cref{the: codomain completeness},
    replacing \(h\) with \(h'\) and replacing \(\cod\) with \(\dom\).
    However, the proof of \Cref{the: domain completeness} reveals more properties
    of maps like \((-)^{∨}\) and \(\op\), 
    thus we choose to present the current proof of \Cref{the: domain completeness} 
    instead of the alternative proof.
\end{remark}


\section{Modeling Errors in Incorrectness Logic}\label{sec: failkat and error}

One of the advantages of algebraic methods is the ease of extension.
In this section, we show how extending TopKAT with failure can naturally give rise to incorrectness triples that express abnormal termination.  
The main difference arises from the short-circuiting behavior of sequencing with errors.
This can be seen in the following rule for sequential composition, which states that if an error already occurred in \(p\), \(q\) will not be executed.
\[
\inferrule[Composition-Fail]
{ \incorTriple{b}{p}{\errState: c}}
{ \incorTriple{b}{pq}{\errState: c}}.
\]

To capture this type of control flow, we adapt the ideas from
\citet{Mamouras_2017}, who investigated similar issues
in the setting of KAT\@.
\begin{definition}[FailTopKAT]
  A FailTopKAT is a tuple \((\mathcal{F},\mathcal{K},\mathcal{B},\fail)\), where
  \((𝒦, ℬ)\) is a TopKAT, \(\fail ∈ ℱ\), and
  \(𝒦 ⊊ ℱ\). The set \(ℱ\) has the
  structure of a KAT that extends that of \(\mathcal{K}\), except that the right
  annihilation rule \(p ⋅ 0 = 0\) need not hold.  Instead,
  \[\fail ⋅ p = \fail,\] where \(p\) is any element of \(ℱ\).
  (Crucially, we do \emph{not} assume \(⊤ ≥ \fail\).) We call
  \((𝒦,ℬ)\) the \emph{fail-free subalgebras}, 
  which model programs that do not fail.  
  We will omit some of \(𝒦,ℬ,\fail\),
  if they are not used or can be inferred from the context.

  The class of all FailTopKATs is denoted \(\FailTopKAT\).
\end{definition}
Note that the original definition of \citet{Mamouras_2017} allows for try-catch statements and different types of errors.  
We omit these features for simplicity, since they are not needed in incorrectness logic~\cite{OHearn_2020}. 

There exists a canonical procedure for extending a TopKAT \(\mathcal{K}\) with
failures.  The idea, which we adapt from Construction
F~\cite[Definition~4]{Mamouras_2017} is to consider elements of the
form \((p,p') ∈ 𝒦 × 𝒦\), where \(p\) represents
executions that terminate normally, and \(p'\) represents executions that fail.
\begin{definition}[Construction F for \(\FailTopKAT\)]
  Given a TopKAT \((\mathcal{K}, \mathcal{B})\), we construct a FailTopKAT
  \((\mathcal{F}, \mathcal{K'}, \mathcal{B'})\).  The carrier sets are defined
  as 
  \begin{align*}
    ℱ & ≜ 𝒦 × 𝒦 &
    𝒦' & ≜ 𝒦 × \{0_{𝒦}\} &
    \mathcal{B'} & ≜ ℬ × \{0_{𝒦}\},
  \end{align*}
  where \(0_{\mathcal{K}}\) is the additive identity in \(\mathcal{K}\).  The
  operations of \(\mathcal{F}\) are defined as follows
  \begin{align*}
      0_{\mathcal{F}} & ≜ (0_{𝒦}, 0_{𝒦}) \\
      1_{ℱ} & ≜ (1_{𝒦}, 0_{𝒦}) \\
      ⊤_{ℱ} & ≜ (⊤_{𝒦}, 0_{𝒦}) \\
    (p, p') (q, q') & = (p q, p' + pq') \\
    (p, p') + (q, q') & = (p + q, p' + q') \\
    {(p, p')}^* & = ({p}^*, ({p}^*) p') \\
    \fail & = (0_{\mathcal{K}}, 1_{\mathcal{K}}) \\
    \overline{(b, 0_{\mathcal{K}})} & = (\overline{b}, 0_{\mathcal{K}}).
  \end{align*}
\end{definition}
To develop some intuitions for this construction, suppose that \(𝒦\) is a relational TopKAT over \(X\).  
There exists a canonical embedding of \(ℱ\) in \(𝒫(X × X × \{\okState,\errState\})\) that identifies \((p,p') ∈ \mathcal{F}\) with the relation
\(r = p × \{\okState\} ∪ p' × \{\errState\} ⊆ X × X × \{\okState,\errState\}\).
Intuitively, \((x,y,γ) ∈ r\) means that a program took the input state \(x\) to
the output state \(y\), and the bit \(\gamma ∈ \{\okState,\errState\}\) signals whether an error has
occurred.  By looking at the definition of sequential composition under this
reading, it says that we get an error either by getting an error when running
the first command (\(p'\)), or if we successfully run the first command, but get
an error when running the second (\(pq'\));
and we terminates normally only we sequentially execute \(p\) and then \(q\).

The semantics of~\citet{OHearn_2020} follows the same pattern, except that he considered the cases \(\okState\) and \(\errState\) in
separate relations.  We can merge them back into a tuple, for example the
semantics of \(\comSkip\)
\[(\{(x, x) ∣ x ∈ X\}, \emptyset)\] 
coincides with the multiplicative identity \(1\) of 
applying construction F to a relational TopKAT \(\mathcal{K}\): 
\[(1_{\mathcal{K}}, 0).\]
In the same way, the sequential composition is multiplication,
the choice operator is addition, the star operator is the Kleene star,
and the \(\comError\) command is \(\fail\).
Thus applying F construction on a relational TopKAT will
capture the semantics of programs with abnormal termination:

\begin{definition}[Relational FailTopKAT]\label{def:relational-failtopkat}
  A relational FailTopKAT is a FailTopKAT constructed by applying construction F
  to a relational TopKAT\@. The class of all relational FailTopKATs is denoted
  \(\FailTopREL\).
\end{definition}

To better understand how to encode an incorrectness triple using FailTopKAT, we
propose a definition of incorrectness triple equivalent to the original one
\cite[Definition 1 and 4]{OHearn_2020}:
\begin{definition}\label{def: relational validity of incor triple with fail}
  Given a relational FailTopKAT \((\mathcal{F}, \mathcal{B})\),
  \(p \in \mathcal{F}\), and \(b, c \in \mathcal{B}\), for an error code
  \(\gamma \in \{\okState, \errState\}\) an incorrectness triple
  \(\incorTriple{b}{p}{\gamma: c}\) is valid if
    \[\cod(b p) ⊇ \cod(c ⋅ \hat{\gamma)}\] 
    where
    \[ \hat{\gamma} ≜
      \begin{cases}
        1 & \text{if \(\gamma = \okState\)} \\
        \fail & \text{if \(\gamma = \errState\).}
      \end{cases}\]
    and the function \(\cod(-)\) is extended entry-wise:
    \begin{align*}
        \cod((r, q)) & ≜ (\cod(r), \cod(q)).
    \end{align*}
\end{definition}

Following the development of \Cref{sec: formulating incorrectness logic}, we can
obtain a formulation of incorrectness triple with abnormal termination.

\begin{theorem}[Relational Validity]
  For all relational FailTopKATs \((\mathcal{F}, \mathcal{B})\), and \(p, q \in \mathcal{F}\),
  we have
\begin{align*}
    \top p = \top q & \iff \cod(p) = \cod(q) \\
    \top p \leq \top q & \iff \cod(p) \subseteq \cod(q),
\end{align*}
\end{theorem}

\begin{proof}
  Write \(p = (p_1,p_2)\) and \(q = (q_1,q_2)\).  Then, by definition,
  \(\top p = (\top p_1, \top p_2)\) and \(\top q = (\top q_1, \top q_2)\).  Thus,
  \(\top p \leq \top q\) is equivalent to \(\top p_1 \leq \top q_1\) and
  \(\top p_2 \leq \top q_2\), which allows us to conclude.
\end{proof}

\begin{corollary}\label{the: FailTopKAT can express incorrectness logic}
  For all relational FailTopKATs \(\mathcal{F}\), for all \(p \in \mathcal{F}\)
  and for all tests \(b, c \in \mathcal{B}\), and for all
  \(\gamma \in \{\okState,\errState\}\), the following holds
    \[\incorTriple{b}{p}{\gamma: c} \iff \top b p \geq (c \cdot \hat{\gamma}).\]
  where \(\hat{\gamma}\) is defined in \Cref{def: relational validity of incor triple with fail}.
\end{corollary}

Thus, we can generalize incorrectness logic with errors to an arbitrary
FailTopKAT\@.

\begin{definition}[Abstract Incorrectness Triple With Failure]
  Given a \(FailTopKAT\) \((\mathcal{F}, \mathcal{K}, \mathcal{B})\),
  \(b, c \in \mathcal{B}\), \(p \in \mathcal{F}\) and
  \(\gamma \in \{\okState, \errState\}\), we define
  \[\incorTriple{b}{p}{\gamma: c} ≜ \top b p \geq c \cdot
    \hat{\gamma}, \] where \(\hat{\gamma}\) is defined as in 
    \Cref{def: relational validity of incor triple with fail}.
\end{definition}

\begin{figure}
    \begin{mathpar}
        \inferrule[Empty]
        {\\}{\incorTriple{b}{p}{\gamma: 0}}

        \and
        \inferrule[Consequence]
        {b \leq b' \\ \incorTriple{b}{p}{\gamma: c} \\ c' \leq c}
        {\incorTriple{b'}{p}{\gamma: c'}}

        \and
        \inferrule[Disjunction]
        {\incorTriple{b_1}{p}{\gamma: c_1} \\ \incorTriple{b_2}{p}{\gamma: c_2}}
        {\incorTriple{b_1 + b_2}{p}{\gamma: (c_1 + c_2)}}

        \and
        \inferrule[Identity]
        {\\} {\incorTriple{b}{1}{\okState: b, \errState: 0}}

        \and
        \inferrule[Composition-Fail]
        {\incorTriple{a}{p}{\errState: b}}
        {\incorTriple{a}{p q}{\errState: b}}

        \and
        \inferrule[Composition-Normal]
        {\incorTriple{a}{p}{\okState: b} \\ \incorTriple{b}{p}{\gamma: c}}
        {\incorTriple{a}{pq}{\gamma: c}}

        \and
        \inferrule[Choice-Left]
        {\incorTriple{b}{p}{\gamma: c}}
        {\incorTriple{b}{p + q}{\gamma: c}}

        \and
        \inferrule[Choice-Right]
        {\incorTriple{b}{q}{\gamma: c}}
        {\incorTriple{b}{p + q}{\gamma: c}}

        \and
        \inferrule[Assume]
        {\\}
        {\incorTriple{a}{b}{\okState: ab, \errState: 0}}

        \and
        \inferrule[Error]
        {\\}{\incorTriple{b}{\fail}{\errState: b}}

        \and
        \inferrule[Iter-Zero]
        {\\}{\incorTriple{b}{{p}^*}{\okState: b}}

        \and
        \inferrule[Iter-NonZero]
        {\incorTriple{b}{{p}^*p}{\gamma: c}}
        {\incorTriple{b}{{p}^*}{\gamma: c}}

        \and
        \inferrule[Iter-Dependent]
        {\forall n \in ℕ,~ \incorTriple{b_{n}}{p}{\okState: b_{n+1}}}
        {\incorTriple{b_{0}}{{p}^*}{\okState: \sup_{n \in ℕ}b_n}}
    \end{mathpar}
    \caption{Complete set of incorrectness logic proof rule with both normal and
      abnormal termination}\label{fig: all proof rules of inc logic}
\end{figure}

\begin{theorem}[Soundness of Incorrectness Logic Rules in FailTopKAT]\label{the: incorrectness logic fail rule sound}
  The rules in \Cref{fig: all proof rules of inc logic} are valid for any
  FailTopKAT \((\mathcal{F}, \mathcal{K}, \mathcal{B})\),
  \(a, b, c \in \mathcal{B}\) and \(p, q \in \mathcal{F}\).
\end{theorem}

\begin{proof}
    All of the \(\okState\) cases is proved in~\Cref{the: incor logic rule sound ok},
    therefore we just prove all the \(\errState\) case.

    We list all the failure cases of these rules
    and remove all the rules without failure cases here,
    for convenience of the reader:
    \begin{mathpar}
        \inferrule[Empty]
        {\\}{\incorTriple{b}{p}{\errState: 0}}

        \and
        \inferrule[Consequence]
        {b \leq b' \\ \incorTriple{b}{p}{\errState: c} \\ c' \leq c}
        {\incorTriple{b'}{p}{\errState: c'}}

        \and
        \inferrule[Disjunction]
        {\incorTriple{b_1}{p}{\errState: c_1} \\ \incorTriple{b_2}{p}{\errState: c_2}}
        {\incorTriple{b_1 + b_2}{p}{\errState: (c_1 + c_2)}}

        \and
        \inferrule[Identity]
        {\\} {\incorTriple{b}{1}{\errState: 0}}

        \and
        \inferrule[Composition-Fail]
        {\incorTriple{a}{p}{\errState: b}}
        {\incorTriple{a}{p q}{\errState: b}}

        \and
        \inferrule[Composition-Normal]
        {\incorTriple{a}{p}{\okState: b} \\ \incorTriple{b}{p}{\errState: c}}
        {\incorTriple{a}{p}{\errState: c}}

        \and
        \inferrule[Choice-Left]
        {\incorTriple{b}{p}{\errState: c}}
        {\incorTriple{b}{p + q}{\errState: c}}

        \and
        \inferrule[Choice-Right]
        {\incorTriple{b}{q}{\errState: c}}
        {\incorTriple{b}{p + q}{\errState: c}}

        \and
        \inferrule[Assume]
        {\\}
        {\incorTriple{a}{b}{\errState: 0}}

        \and
        \inferrule[Error]
        {\\}{\incorTriple{b}{\fail}{\errState: b}}

        \and
        \inferrule[Iter-NonZero]
        {\incorTriple{b}{{p}^*p}{\errState: c}}
        {\incorTriple{b}{{p}^*}{\errState: c}}
    \end{mathpar}

    The proof are basically the same as~\Cref{the: incor logic rule sound ok},
    since \(FailTopKAT\) share most of the properties of \(\TopKAT\):
    \begin{itemize}
        \item \textbf{Empty:}
            By left-annihilation and \(0\) is the least element (additive identity):
            \[\top b p \geq 0 = 0 \fail\]
        \item \textbf{Consequence Rule:}
            because multiplication preserves order
            and \(b' \geq b\), therefore \(\top b' p \geq \top b p\);
            and because \(c' \leq c\), then \(c' \fail \leq c \fail\).
            Therefore
            \[\top b' p \geq \top b p \geq c \fail \geq c' \fail\]
        \item \textbf{Disjunction Rule:}
            because addition preserves order, therefore
            \[\top b_1 p + \top b_2 p \geq c_1 \fail + c_2 \fail\]
            and by distributivity, we have
            \[\top (b_1 + b_2) p \geq (c_1 + c_2) \fail\]
        \item \textbf{Identity Rule:}
            By left-annihilation and \(0\) is the least element (additive identity):
            \[\top b 1 \geq 0 = 0 \fail\]
        \item \textbf{Composition-Fail Rule:}
            Given \(\top a p \geq b \fail\),
            then by order preserving and associativity of multiplication,
            we have
            \[\top a p q \geq b \fail q = b \fail\]
        \item \textbf{Composition-Normal Rule:}
            first, by idempotency of \(\top\)
            and associativity of multiplication, we have
            \[\top a (p q) = \top \top a p q\]
            By order preserving of multiplication,
            and the two premises \(\top a p \geq b\), \(\top b p \geq c \fail\),
            we have \[\top a (p q) = \top \top a p q \geq \top b q \geq c \fail\]
        \item \textbf{Choice-Left and Choice-Right Rule:}
            by distributivity, we have
            \[\top b (p + q)
            = \top b p + \top b q
            \geq \top b p \geq c \fail\]
            and
            \[\top b (p + q)
            = \top b p + \top b q
            \geq \top b q \geq c \fail\]
        \item \textbf{Error:}
            because \(\top \geq 1\), by order preserving of multiplication
            \[\top b \fail \geq b \fail\]
        \item \textbf{Iter-NonZero Rule:}
            since \({p}^* = p {p}^* + 1\),
            we have
            \[\top b {p}^*
            = \top b (p {p}^* + 1)
            \geq \top b p {p}^* \geq c \fail\]
    \end{itemize}

    Note from the proof we can see all the rules with failure case \(\errState: 0\),
    like Assume and Identity rule,
    are redundant, since the failure case rule can be derived from the failure case of empty rule.
\end{proof}


\section{Examples: Reasoning Using TopKAT and FailTopKAT}

In this section, we show some concrete examples of algebraic program reasoning.
We take the assignment language of \citet{OHearn_2020},
and regard assignments as primitive actions and assume statements as primitive tests.
The relational semantics of this language forms a relational FailTopKAT,
and it also forms a relational TopKAT if we do not consider the \(\comError\) command and
the \(\errState\) post-condition.

\begin{example}[Incorrect Absolute Value Procedure]\label{exp: incorrect absolute value}
    Here is an incorrect procedure for finding the absolute value of \(x\):
    \[\comITE{x < 0}{\comSkip}{\comAssign{x}{-x}},\]
    To have a correct procedure for computing the absolute value the condition of the if statement should be \(x > 0\).
    We can use TopKAT to show that every negative number is reachable by using 
    the following incorrectness triple: 
    \[\incorTriple{x < 0}{\comITE{x < 0}{\comSkip}{\comAssign{x}{-x}}}{x < 0}\]

This triple can be proven using just the theory of TopKAT. 
First, we can unfold the if statement:
\[\incorTriple{x < 0}{ (x < 0) 1 + \overline{(x < 0)} (\comAssign{x}{-x})}{x < 0}\]
Then convert the triple to TopKAT encoding: 
\[\top(x < 0) ((x < 0) 1 + \overline{(x < 0)} (\comAssign{x}{-x})) \geq (x < 0)\]
Finally, we prove the above inequality:
\begin{align*}
    & \top(x < 0) ((x < 0) 1 + \overline{(x < 0)} (\comAssign{x}{-x}))\\
    & \geq \top (x < 0) (x < 0) 1 
        & (p + q) \geq p\\
    & = \top (x < 0) 
        & \text{idempotency of test} \\
    & \geq (x < 0)  & \top \geq 1
\end{align*}
Thus we have shown that 
\[\incorTriple{x < 0}{ (x < 0) 1 + \overline{(x < 0)} (\comAssign{x}{-x})}{x < 0}\]
is valid, and the non-desirable results in \(x < 0\) can be reached.
As this triple can be shown just using the equational theory of TopKAT, 
this triple can be automatically decided using the algorithm in \Cref{the: PSPACE-completeness of TopKAT}.
\end{example}     

\citeauthor{OHearn_2020} motivated the under-approximate triple 
as a way to reason about incorrect programs. 
However the under-approximation logic can have other use cases.
For example, we can mix under-approximation and over-approximation triples to
prove a certain post condition is the strongest
(as in Hoare logic) without a relational semantics.

\begin{example}[Reasoning With Hoare And Incorrectness Logic]\label{exp: hoare and incorrect together}
    The assertion \(x ≥ 0\) is the strongest post condition of program 
    \(\comWhile{x < 0}{\comAssign{x}{x + 1}}\) with precondition \(\true\).

    We can show this by the following two triples:
\begin{align*}
    & \incorTriple{\true}{\comWhile{x < 0}{\comAssign{x}{x + 1}}}{x \geq 0}; \\
    & \hoareTriple{\true}{\comWhile{x < 0}{\comAssign{x}{x + 1}}}{x \geq 0}.
\end{align*}

Because for all \(p\) in some TopKAT, \({p}^* \geq 1\),
the incorrectness triple can be shown as follows
\[\top 1 {((x < 0) (\comAssign{x}^*{x + 1}))}\overline{(x < 0)}
≥ ⊤ 1 1 \overline{(x < 0)}
≥ (x ≥ 0).\]
And because \(⊤ ≥ 1 {((x < 0) (\comAssign{x}^*{x + 1}))}\),
then the Hoare triple can be shown
\[
    1 {((x < 0) (\comAssign{x}^*{x + 1}))}\overline{(x < 0)}
    ≤ ⊤ \overline{(x < 0)}
    = ⊤ (x ≥ 0)
  \]
  \end{example}


\begin{example}[Theorem Proving In Hoare And Incorrectness Logic]
    For all TopKATs \((\mathcal{K}, \mathcal{B})\), and \(b, c \in \mathcal{B}\), \(p \in \mathcal{K}\),
    if \(c \geq \overline{b}\), then following incorrectness and Hoare triples are valid
    \[\incorTriple{c}{\comWhile{b}{p}}{\overline{b}} 
    \text{ and }
    \hoareTriple{c}{\comWhile{b}{p}}{\overline{b}}\]

  This example is a generalization of \Cref{exp: hoare and incorrect together}.
If we have a while loop with condition \(b\),
and the precondition \(c\) is larger than \(\overline{b}\),
then \(\overline{b}\) is the strongest post-condition (in the sense of Hoare logic).

We first show a proof in a relational setting, as a comparison to the algebraic proof.
\begin{itemize}
    \item \(\cod(c {(b p))^* \overline{b}} \supseteq \cod(\overline{b})\):
        because on the left hand side, every output needs to go through the final check 
        of \(\overline{b}\), it will also be in \(\cod(\overline{b})\);
    \item \(\cod(c {(b p)}^* \overline{b}) ⊆ \cod(\overline{b})\):
        if the input of the left hand side is in \(\cod(\overline{b})\), 
        it will not be filtered out by \(c\), will not go into loop \({(b p)}^*\),
        will not be filtered out by \(\overline{b}\), and will be outputted unchanged. 
        thus everything in \(\cod(\overline{b})\) will 
        be a output of \( {(b p)}^* \overline{b}\),
        hence in \(\cod(c {(b p)}^* \overline{b})\).
\end{itemize}

Alternatively, we prove this example algebraically.
Because \({(bp)}^* \geq 1\) and \(c \geq \overline{b}\): 
\[\top c {(b p)}^* \overline{b} \geq \top c 1 \overline{b} = \top \overline{b} \geq \overline{b},\]
and because \(\top \geq \top c {(b p)}^*\):
\[\overline{b} \leq \top \overline{b} \geq \top c {(b p)}^* \overline{b}.\]
We can also show that the Hoare triple is also valid with Kozen's encoding:
\[c {(bp)}^* \overline{b} (\overline{\overline{b}}) = c {(bp)}^* \overline{b} b = c {(bp)}^* 0 = 0.\]
The fact that we can also use Kozen's encoding to reach the same conclusion is not surprising,
as we have shown that Kozen's encoding is equivalent to our encoding in all TopKAT.
 \end{example}

\begin{example}[Error In Loop]
    This example simulates a while loop where the body will encounter an error when \(x \leq 0\),
    and it will do some useful computation \(p\) if it does not encounter the error:
    \[
        \incorTriple{\true}
        {\comWhile{x \geq 0}{\comITE{x \leq 0}{\comError}{p}}}
        {\errState: x = 0}
    \]
    the incorrect loop condition will trigger the possible error in the loop body,
    which is undesirable.


To show this triple, we need to show the following FailTopKAT inequalities:
\[
    \top 1 {((x \geq 0) ((x \leq 0) (\fail) + (\overline{x \leq 0}^*) (p)))} (\overline{x \geq 0}) \geq 
    (x = 0) \fail
\]
The proof is as follow: 
\begin{align*}
    & \top 1 {((x \geq 0) ((x \leq 0) (\fail) + (\overline{x \leq 0}^*) (p)))} (\overline{x \geq 0})\\
    & \geq \top 1 {((x \geq 0) (x \leq 0) (\fail))}^* (\overline{x \geq 0})
        & q + r \geq q\\
    & = \top 1 (x \geq 0) (x \leq 0) (\fail) (\overline{x \geq 0})
        & {q}^* \geq q \\
    & = \top 1 (x \geq 0) (x \leq 0) (\fail) 
        & \fail p = \fail \\
    & = \top (x = 0) \fail 
        & (x \geq 0 \land x \leq 0) = (x = 0)\\
    & \geq (x = 0) \fail & \top \geq 1
\end{align*}

Notice that in the second last step we used the fact 
\[(x \geq 0 \land x \leq 0) = (x = 0).\]
We invoked the logical meaning of \(x \geq 0\), \(x \leq 0\), and \(x = 0\).
Thus we are \emph{not} purely using the theory of TopKAT.
\end{example}

We can use more than just logical implications.
Since our encoding of incorrectness logic is conservative (relationally valid), 
all the proof rules of incorrectness logic will play nicely with algebraic reasoning.
The next example demonstrates the mix of equational reasoning and assignment rule.

\begin{example}[Assignment]
    We have the same program as \Cref{exp: incorrect absolute value}
    with the precondition changed to \(x \geq 0\):
    \[\incorTriple{x > 0}{\comITE{x < 0}{\comSkip}{\comAssign{x}{-x}}}{x < 0}.\]

    Even though the assignment rule from Incorrectness Logic is not propositional, 
    in the sense we use in this paper,  
    we can use specific inequalities to represent specific instances of these rules. 
    For example, we can use an instance represented by the following incorrectness triple:
\[\incorTriple{x > 0}{\comAssign{x}{-x}}{x < 0},\]
This triple corresponds to the following TopKAT inequality:
\[\top (x > 0) (\comAssign{x}{-x}) \geq (x < 0).\]
With the above inequality, we can derive the incorrectness triple in the example as follow:
\begin{align*}
    & \top (x > 0) ((x < 0) 1 + \overline{(x < 0)} (\comAssign{x}{-x})) \\
    & \geq \top (x > 0) (\overline{(x < 0)} (\comAssign{x}{-x})) 
        & p + q \geq q\\
    & \geq \top (x > 0) (\comAssign{x}{-x}) 
        & (x > 0) \leq \overline{(x < 0)}\\
    & \geq (x < 0) 
        & \text{above inequality}
\end{align*}
\end{example}

Thus, as we can see, even though TopKAT does not contain an assignment axiom,
we can still reason about programs with assignments.

\section{Related Works}

\textbf{Extensions of Kleene algebra and reduction:}
soon after the completeness of Kleene algebra was proven~\cite{Kozen_1994},
it was realized that adding an embedded Boolean algebra can help reasoning
about control structures, such system is referred to as
Kleene algebra with tests (KAT)~\cite{Kozen_Smith_1997,Cohen_Kozen_Smith_1999}.
Later KAT was further extended to reason about failure~\cite{Mamouras_2017},
indicator variables~\cite{Grathwohl_Kozen_Mamouras_2014},
domain~\cite{Desharnais_Möller_Struth_2006}, networks~\cite{Anderson_Foster_Guha_Jeannin_Kozen_Schlesinger_Walker_2014},
and relational reasoning~\cite{Antonopoulos_Koskinen_Le_Nagasamudram_Naumann_Ngo_2022}.
Kleene algebra has also been extended to reason about 
concurrency, as concurrent Kleene algebra~\cite{Hoare_van_Staden_Möller_Struth_Zhu_2016, Kappé_Brunet_Silva_Zanasi_2018}
and concurrent Kleene algebra with observations~\cite{Kappé_Brunet_Silva_Wagemaker_Zanasi_2020}.
Many of these extensions can be seen as Kleene algebra with extra hypotheses~\cite{Cohen_1995,Doumane_Kuperberg_Pous_Pradic_2019}.
Although many hypotheses make the theory undecidable~\cite{Kozen_1996,Kozen_2002,Doumane_Kuperberg_Pous_Pradic_2019},
many useful hypotheses can be eliminated via reduction~\cite{Pous_Rot_Wagemaker_2021}.
Thus, our new perspective on reduction could potentially lead to streamlining of various previous proofs, 
and more general proofs of completeness results.

\textbf{Top element:}
Tarski's relational algebra~\cite{tarski_CalculusRelations_1941} contains the addition, 
mulitiplication, and identity operation of KA;  
in addition, relational algebra also include a top element. 
Hence attempts to incorporat Kleene star into relational algebra 
effectively create a super theory of TopKAT.
Unfortuantly, several attempts at these algebras turn out to be undecidable
because of the presence of intersection and 
converse operations~\cite{andrekaAxiomatizabilityPositiveAlgebras2011, pous_PositiveCalculusRelations_2018}.
With the intersection and converse operators removed, 
top element is proven to be individually useful in Kleene algebra:
for example, Mamouras~\cite{Mamouras_2017} uses the top element to forget program states,
and Antonopoulos et al.~\cite{Antonopoulos_Koskinen_Le_Nagasamudram_Naumann_Ngo_2022} 
uses top to design forward simulation rules for relational verification, 
and claim that relational incorrectness logic~\cite{murray_UnderApproximateRelationalLogic_2020a} 
can be encoded using BiKAT extended with top.
The completeness and decidability of TopKAT was first studied by Zhang et al.~\cite{Zhang_de_Amorim_Gaboardi_2022},
and concluded that TopKAT is not complete with relational models.
Later, Pous et al.~\cite{Pous_Wagemaker_2022,Pous_Wagemaker_2023} showed that 
both TopKA and TopKAT is complete with relational model with one additional axiom: \(p ⊤ p ≥ p\),
and the theory remains PSPACE-complete, like KAT and TopKAT.
In this paper, we showed that TopKAT without the additional axiom is complete 
for a specific form of inequalities, namely when top only appears in the front or the end of the term.
Although this form of inequalities seem restrictive, 
they are enough to encode both Hoare and incorrectness logic~\cite{Zhang_de_Amorim_Gaboardi_2022}.

\textbf{Domain in KAT:}
The study of axiomatizing (co)domain in KAT has a long and rich history. 
Domain semiring~\cite{Desharnais_Struth_2011} 
and Kleene algebra with domain~\cite{Desharnais_Möller_Struth_2006}
were two popular yet different axiomatizations of (co)domain in Kleene algebra with tests.
These two axiomitizations turn out to coincide in a large class of semirings~\cite{Fahrenberg_Johansen_Struth_Ziemiánski_2021}.
Various applications for domain in KAT have been discovered, including modeling
program correctness, predicate transformers, temporal logics, 
termination analysis, and many more~\cite{Desharnais_Möller_Struth_2004}.
Many of these applications can even be efficiently automated~\cite{hofner_AutomatedReasoningKleene_2007}.
However, although the free relational model of these theories has been characterized~\cite{mclean_FreeKleeneAlgebras_2020},
the search for general complete interpretation remains unfruitful.
The complexity of these theories was recently shown to be EXPTIME-complete~\cite{Sedlár_2023},
a worse complexity class than PSPACE-complete for TopKAT.

\cleardoublepage 
 

% continuations
\newcommand{\conti}[1]{\mathbf{#1}}
\newcommand{\acc}[1]{{\conti{acc} ~ #1}}  % accept with an output indicator value
\newcommand{\ret}{{\conti{ret}}}  % return, terminate the program
\newcommand{\brk}[1]{\conti{brk} ~ #1}  % break with an output indicator value
\newcommand{\jmp}[1]{{\conti{jmp} ~ #1}} % goto a label

% paper specific
\newcommand{\contWith}{\mathbf{cont}}
\newcommand{\exitWith}{\mathbf{exit}}
\newcommand{\iter}{\mathrm{iter}}
\newcommand{\const}{\mathrm{const}}


\chapter{CF-GKAT, fast control-flow verification}
\label{chapter:Conclusions}
\thispagestyle{myheadings}

% set this to the location of the figures for this chapter. it may
% also want to be ../Figures/2_Body/ or something. make sure that
% it has a trailing directory separator (i.e., '/')!
\graphicspath{{4_Conclusion/Figures/}}

\section{Motivation and Overview}

The usual baseline to reason about correctness of control-flow manipulation is \emph{trace semantics}, which can be represented by guarded languages. 
Intuitively, trace semantics abstracts the meaning of the primitive tests and actions that occur in a program, and instead focus on how tests determine which actions are performed, and in what order.

\emph{Guarded Kleene Algebra with Tests}~\cite{kozen_BohmJacopiniTheorem_2008a,Schmid_Kappé_Kozen_Silva_2021},
provides a nice (co)algebraic framework to reason about trace semantics.
For example, GKAT is able to verify nontrivial  program equivalences like
\[
 \{\ \comWhile{b}{p}\ \};\ \comWhile{s}{\{\ q;\ \comWhile{b}{p}\ \}} ≡
 \comWhile{b ∨ c}{\{\ \comITE{b}{p}{q}\ \}}
\]
Furthermore, unlike KAT, GKAT equivalence is decidable in nearly-linear time (assuming the set of primitive tests is fixed)~\cite{Schmid_Kappé_Kozen_Silva_2021}, making GKAT a practical framework to reason about trace semantics of simple \command{while}-programs.

Nevertheless, GKAT still lacks important constructs that are ubiquitous among control-flow transformation algorithms.
First, as mentioned, GKAT disregards the meaning of primitive programs and tests.
For instance, when given a program like
\begin{equation}
 \comITE{y \neq 0}{\{\ x := 42;\ p\ \}}{\{\ x := 42;\ q\ \}}%
 \label[prog]{prog: assignment inside branches}
\end{equation}
we can note that a change in the value of $x$ does not have effect on whether or not $y \neq 0$.
Hence, it should be possible to factor the assignment to $x$ out of the branches, and obtain
\begin{equation}
 x := 42;\ \comITE{y \neq 0}{p}{q}%
 \label[prog]{prog: assignment outside branches}
\end{equation}
Unfortunately, GKAT does not admit this equivalence, precisely because it is agnostic with respect to the meaning of primitive actions.
However, moving to a setting that accounts for the semantics of actions is hard, because Turing completeness — and by extension, undecidability — lurks nearby.
In fact, just considering commutativities of non-interfering statements, like 
\[(x := x + 1); (y := y + 1) = (y := y + 1); (x := x + 1), \]
the theory can be quickly become undecidable~\cite{Kozen_1996, kuznetsov_ComplexityReasoningKleene_2023, azevedodeamorim_KleeneAlgebraCommutativity_2024}.

Second, GKAT excludes non-local control-flow structures like \(\command{goto}\), \(\comBrk\), and \(\comRet\).
In a general imperative language, lacking these commands will not limit its expressivity -- indeed, these control structures can be recovered using variables~\cite{erosa-hendren-1994}.

However, lacking both variables and non-local control structures, GKAT is not able to express all control flows of real-world programs.
As a very concrete example, consider the programs below.
\Cref{prog: goto version two state automaton} has a control flow strategy based purely on labels and $\texttt{goto}$.
Meanwhile, \Cref{prog: break version two state automaton} is structured as a loop with the option to terminate early using $\comBrk$.
These programs happen to be trace equivalent (i.e., they always execute the same actions in the same order) but represent behavior not expressible in plain GKAT~\cite{kozen_BohmJacopiniTheorem_2008a,schmid_GuardedKleeneAlgebra_2021}.
\begin{align}
  & \begin{aligned}
      & \comLabel{ℓ₀};\ \comIT{\neg b}{\comGoto{ℓ₁}};\ p;\ \comIT{b}{\comGoto{ℓ₁}};\ q;\ \comGoto{ℓ₀};\ \comLabel{ℓ₁}
    \end{aligned}\label[prog]{prog: goto version two state automaton}
 \\[3pt]
  & \comWhile{b}{\{\ p;\ \comITE{¬ b}{q}{\comBrk}\ \}}
 \label[prog]{prog: break version two state automaton}
\end{align}

As it turns out, deciding equivalence between these complex programs is essential in verifying control-flow manipulation procedures.
Specifically, consider the control flow structuring phase of a decompiler~\cite{cifuentes-1994}, which is tasked with converting conditional and unconditional jumps into more conventional control flow constructs as best as possible.
\Cref{prog: goto version two state automaton} can be thought of as pseudo-assembly that models the input of this process, and \Cref{prog: break version two state automaton} is a plausible outcome.
Thus the control-flow structuring process is correct when \Cref{prog: goto version two state automaton,prog: break version two state automaton} are equivalent.

To overcome these limitations of GKAT, we propose control flow GKAT (CF-GKAT), a extension of GKAT that is capable of equating programs making use of non-local control flow and indicator variables.

First, \emph{indicator variables} can be assigned and tested against hardcoded values, and do not appear in other primitive actions and tests.
Thus, assignments like $x := 42$ are allowed, but assignments like $x := y + 1$ are not.
This addition strikes a delicate balance: it is strong enough to verify well-known control-flow transformation algorithms~\cite{yakdan_NoMoreGotos_2015,erosa-hendren-1994}, yet weak enough to still exclude general computation (keeping program equivalence decidable).

Second, we extend GKAT with non-local control-flow constructs, including \(\comGoto{\!}\), \(\comBrk\) and \(\comRet\).
However, the non-local nature of these commands prevents a compositional semantics --- after all, the precise meaning of a statement like \(\comBrk\) depends on its context.
To overcome these challenges, we propose a intermediate semantics, named \emph{continuation semantics}, which appends a continuation to every trace (\Cref{sec:continuation-semantics}). 
Specifically, at the end of the trace, the program can either accept (terminate normally), break, return, or go to a label.
Then, the trace semantics of the program can be obtained by resolving these continuations.

%FIXME: CZ: This paragraph is still bit out of place to me
Inspired by the triangular correspondence between deterministic trace semantics, GKAT, and GKAT automaton, we were able to design an automaton model for CF-GKAT, where every CF-GKAT expression can be unfolded into an CF-GKAT automaton through Thompson's construction (\Cref{tab: thompson's construction}), while preserving the continuation semantic (\Cref{the:thompson-correctness}). 
Furthermore, CF-GKAT automata and continuation semantics can be lowered into GKAT automata and trace semantics respectively, while preserving their semantic correspondence (\Cref{the:cf-gkat-automaton-lowering-correctness}). 
With the Thompson's construction and the lowering, we are able to reduce the problem of deciding trace equivalence of programs into deciding the bisimulation of two GKAT automata, which is known to be efficient~\cite{Smolka_Foster_Hsu_Kappé_Kozen_Silva_2020}. 

\smallskip
As a result of these extensions, CF-GKAT is able to soundly and completely verify trace equivalence of a larger class of programs, while preserving the nearly-linear efficiency of GKAT.
For instance, it can automatically check that \Cref{prog: break version two state automaton,prog: goto version two state automaton} are equivalent to each other, and also to \Cref{prog: indicator version two state automaton} (below), which is their single-loop equivalent obtained via the Böhm-Jacopini theorem~\cite{DBLP:journals/cacm/BohmJ66}.
\begin{align}
 \begin{aligned}
   & x := 1;\ \comWhile{x \neq 0}{\{ \\
   & \qquad \comITE{x = 1 ∧ b}{\{\ p;\ x := 2\ \} \\
   & \qquad\qquad}{\comITE{x = 2 ∧ ¬ b} {\{\ q;\ x := 1\ \} \\
   & \qquad\qquad}{x := 0} }\ \}}                                 \\
 \end{aligned} \label[prog]{prog: indicator version two state automaton}
\end{align}

To put this theory to work, we implemented an proof-of-concept equivalence checker for CF-GKAT\@.
This checker is able to validate highly non-trivial program transformations, such as the aforementioned Böhm-Jacopini conversion~\cite{DBLP:journals/cacm/BohmJ66}.


\section{CF-GKAT Expression and Semantics}
%FIXME: CZ: multiple indicator variable?

In this section, we introduce the language of CF-GKAT, and gradually develop its semantics.
We begin by explaining the syntax of CF-GKAT;\@ after that, we delve into the semantics of its tests.
We then introduce the model of \emph{(labeled and indexed families of) guarded languages with continuations}, which is flattened into to a model based on \emph{guarded languages}.
Having defined these tools, we then conclude by giving a semantics to CF-GKAT programs in this model.
Along the way, we will single out and explain some of the finer points using examples.

% We fix a single indicator variable $x$, as well as a finite set $I$ of possible indicator values.

\subsection{Syntax}

The syntax of CF-GKAT consists of two levels, similar to GKAT\@.
At the bottom level, there are \emph{tests}; these are Boolean assertions that can occur as guards inside conditional statements, or within assertions that occur in the program text.
To model them, we fix a finite set of primitive tests $B$, which represent uninterpreted expressions that may or may not hold.
The full syntax is as follows.
\[
 \BExp_I ∋ e_{b}, f_{b} ::=
 \false ∣ \true ∣ b ∈ B ∣ {\color{blue}x = i\ (i ∈ I)}
 ∣ e_{b} ∨ f_{b} ∣ e_{b} ∧ f_{b} ∣ \overline{e_{b}}
\]
Compared to GKAT and KAT, tests in CF-GKAT include the \emph{indicator variable test} $x = i$ (highlighted in \textcolor{blue}{blue}) for each \emph{indicator value} $i$ drawn from a finite but fixed set of possible indicator values $I$.
As the notation suggests, this test holds when the indicator variable $x$ currently has the value $i$.
% Since the set \(I\) is finite, any complex predicate \(P\) on \(I\) can be encoded as a disjunction that enumerates all the values in \(P\)\ : \[⋁ \{(x = i) ∣ P(i)\}.\]

The top level syntax of GKAT is built using a finite set of uninterpreted commands \(K\) (the \emph{primitive actions}), as well as \emph{assertions} of the form $\comAssert{e_b}$, where $e_b \in \BExp_I$ is a boolean expression.
Expressions are composed using sequencing, \texttt{if} statements, and \texttt{while} loops.
CF-GKAT extends the base elements of the syntax with indicator variable assignments \(x := i\) (for each $i \in I$), which changes the value of the indicator variable \(x\) to \(i\).
In addition, it adds the non-local control flow commands $\comBrk$ and $\comRet$, as well as $\comGoto{\ell}$ and $\comLabel{\ell}$, where $\ell$ is taken from a fixed but finite set of labels $L$.
The full syntax is given below; constructs new compared to GKAT are highlighted in \textcolor{blue}{blue} again.
\begin{align*}
 \CFGKAT ∋ e, f ::= {}&
 \comAssert{e_b}
 ∣ p ∈ K
 ∣ {\color{blue}x := i\ (i ∈ I)}
 ∣ e; f
 ∣ \comITE{e_b}{e}{f} ∣ {} \\
 &
 \comWhile{e_b}{e}
 ∣ {\color{blue} \comBrk}
 ∣ {\color{blue} \comRet}
 ∣ {\color{blue} \comGoto{ℓ}\ (ℓ ∈ L)}
 ∣ {\color{blue} \comLabel{ℓ}\ (ℓ ∈ L)}
\end{align*}

A \emph{valid program}, or \emph{program} for short, is an expression without (1)~duplicate labels, (2)~$\command{goto}$ commands with an undefined label, or (3)~$\comBrk$ statements that occur outside a loop.
For the sake of simplicity, we assume that the reader does not require a more formal definition of this notion.

\begin{example}
 Any GKAT expression is a valid program.
 Also, \cref{prog: break version two state automaton,%
  prog: goto version two state automaton,%
  prog: indicator version two state automaton}
 from the introduction are all valid CF-GKAT expressions.
 The following expressions are \emph{not} valid programs:
 \begin{align*}
  \comLabel{ℓ}; (\comITE{t}{\comLabel{ℓ}; p}{q}) \tag{the label \(ℓ\) is defined twice} \\
  (\comWhile{\true}{p}); \comGoto{ℓ} \tag{the label \(ℓ\) is undefined} \\
  \comITE{t}{\comBrk}{p} \tag{$\comBrk$ appears outside a loop}
 \end{align*}
\end{example}

%FIXME: move to semantics
\begin{remark}
 For soundness, it is important that the indicator variable $x$ does not
 occur in any primitive test $b ∈ B$ or action $p ∈ K$.
 In other words, $x$ is completely divorced from the other actions in the program,
 and may influence execution only by affecting flow control.
\end{remark}

\subsection{Boolean semantics}

To assign a semantics to CF-GKAT expressions, we first need to talk about the semantics of the Boolean sublanguage.
Similar to GKAT~\cite{Smolka_Foster_Hsu_Kappé_Kozen_Silva_2020}, the semantics of a boolean is defined as a set of program states that satisfy the boolean expression. 
The difference however is that we not only need to keep track of which primitive tests are satisfied under the boolean expression (represented by atoms), but also need to record the indicator variables that satisfy the boolean expression.

Thus, the semantics of a boolean expression is a set of indicator-atom pairs.
Formally, we can calculate the semantics of a given Boolean expression \(e_b ∈ \BExp_I\) by induction.
\begin{definition}
 We define the \emph{Boolean semantics} function $C(-): \BExp_I → 2^{I  × \At}$ inductively,
 as follows.
 \begin{align*}
  C( \false ) & ≜ ∅
    & C( t )& ≜ \{ (i, α) ∣ i ∈ I, b ≤ α \}
    & C( e_{b} ∨ f_{b} ) & ≜ C( e_{b} ) ∪ C( f_{b} ) \\
  C( \true )  & ≜ I × \At
    & C( x = i ) & ≜ \{ (i, α) ∣ α ∈ \At \}
    & C( e_{b} ∧ f_{b} ) & ≜ C( e_{b} ) ∩ C( f_{b} ) \\
  & & & & C( ¬ e_{b} ) & ≜ I × \At \setminus C( e_b )
 \end{align*}
\end{definition}
We will overload the notation \(C\) later for continuation semantics/interpretation, because the boolean semantics is a restriction of the continuation semantics to assertions.

\begin{example}
 Take $B = \{ b_1, b_2 \}$ and $I = \{ 1, 2, 3 \}$; then we can calculate that
 \[
  C( (b_1 ∨ ¬ b_2) ∧ (x = 2) ) = \{
  (b_1 b_2, 2),
  (b_1 \overline{b₂}, 2),
  (\overline{b₁} \overline{b₂}, 2)
  \}
 \]
 In other words, the test above holds in execution contexts where $b₁$ and $b₂$ are both true (first element) or both false (last element), and those where $b₁$ is true but $b₂$ is false (middle element).
 In contrast, $C( x = 1 ∧ x = 3) = ∅$, which is to say that this test does not hold in any execution context, because the indicator variable \(x\) cannot be both 1 and 3 at the same time.
\end{example}

\subsection{Guarded languages with continuations}\label{sec:continuation-semantics}

We can now turn our attention to the semantics of CF-GKAT\@.
Like (G)KAT, the semantics of CF-GKAT is given in terms of \emph{guarded languages}~\cite{Schmid_Kappé_Kozen_Silva_2021,Kozen_1997}, which are best thought of as sets of symbolic traces of the program.
Our semantics of a CF-GKAT expression will ultimately be a guarded language.
To get there, however, we will need continuation semantics, a semantics that can account for the indicator variables as well as the non-local flow control statements.
The remainder of this subsection is dedicated to explaining the domain of this continuation semantics, based on \emph{guarded words with continuations}.
Intuitively, these are guarded words equipped with a piece of information called a \emph{continuation}, which contains relevant information about how flow control continues after the program ends.
This could, for instance, tell us that the execution will continue at a location marked by a label.

The possibility of including continuation information at the end of a trace allows us to define a semantics of CF-GKAT expressions inductively.
This is especially necessary in the case of non-local control flow, because the label may occur in an entirely different part of the program whose traces have not yet been computed.
Once the continuation semantics of a CF-GKAT program is known, we can flatten it into a guarded language.

\begin{definition}
 A \emph{guarded word with continuation} is a pair $w ⋅ c$,
 where $w$ is a guarded word and $c$ is a \emph{continuation},
 which can take on one of the following forms for $i ∈ I$ and $ℓ ∈ L$:
 \begin{mathpar}
  \acc{i} \and
  \brk{i} \and
  \ret \and
  \jmp{(ℓ, i)}
 \end{mathpar}
 We write $C$ for the set of all continuations.
 A set of guarded words with continuations is a \emph{guarded language with continuations}; the set of guarded languages with continuations is written $𝒞$.
\end{definition}
Intuitively, the different types of continuation may be interpreted as follows:
\begin{itemize}
    \item
    The continuation $\acc{i}$ represents that the trace has successfully reached the end of this part of the program, with indicator value \(i\).
    Execution can be picked up if the program is put in a larger context --- e.g., if $w \cdot \acc{i}$ is a trace of $e$, then it may be combined with a trace found when $f$ is executed with indicator value $i$ to compute the semantics of $e; f$.
    \item
    A continuation of the form $\brk{i}$ signals that the trace ends by halting the loop in which it occurs.
    Execution can resume only after this loop (with indicator value $i$).
    This kind of trace cannot be composed on the right, as is done for traces with accepting continuations, because we first need to enclose it in a loop to halt; it will then be converted into $\acc{i}$.
    \item
    The continuation $\ret$ represents a trace that ends in the program halting completely.
    Traces of this kind will percolate upwards in the semantics, without changing their continuation.
    These are intended to model the $\comRet$ statement, which halts the program no matter how deeply it is nested.
    In this case, the indicator value does not matter any more.
    \item
    Finally, the continuation $\jmp{(\ell, i)}$ is put on traces that will continue executing from label $\ell$, with indicator value $i$.
    Like $\brk{i}$ and $\ret$, these traces do not compose on the right, but unlike $\brk{i}$ this continuation does not change, as jump resolution happens only at the end, when the semantics is known for the entire program.
\end{itemize}

\begin{example}
 Let $w$ be the guarded word from the previous example;
 the guarded word with continuation $w ⋅ \jmp{(ℓ₁, 2)}$
 represents a partial program trace that takes the steps represented by $w$,
 and will continue executing at the label $ℓ₁$ with an indicator value of $2$.
\end{example}

%FIXME: change L to G to avoid conflict with sets of labels

\subsection{Indexed families and sequencing}
The continuation semantics of a CF-GKAT expression takes a starting indicator value, and produces a guarded language with continuations representing the traces of that program when started with this indicator value.
This semantics is modeled by the following.
\begin{definition}
An \emph{indexed family} of guarded languages (with continuations), or ``indexed family'' for the sake of brevity, is function from $I$ to guarded languages (with continuations).
Similar to guarded languages, we use \(W, V\) to denote a family of languages.
To lighten notation, we write \(Wᵢ\) to denote \(W(i)\).
\end{definition}

Similar to guarded languages, indexed families can be composed in several ways.
In particular, we are interested in the sequencing operation and the Kleene star operation of indexed families, because these will turn out to be useful when defining the continuation semantics of CF-GKAT\@.

When sequencing two families \(W\) and \(V\), the traces in \(W_i\) with a continuation of the form $\acc{j}$ will be composed with traces in \(V_j\); traces with different continuations are copied over in full, because they do not compose on the right.
Formally, this operation is defined as follows.

\begin{definition}%
\label{def:sequencing}
 Let $W, V: I \to 𝒞$.
 We write $W ⋄ V$ for the \emph{sequencing} (or \emph{concatenation}) operation of \(W\) and \(V\), which is defined as the smallest family of guarded languages with continuations (in the pointwise order) satisfying the following rules for all $i,j ∈ I$ as well as all $ℓ ∈ L$:
 \begin{mathpar}
  \inferrule{%
   wα ⋅ \acc{j} ∈ Wᵢ \\
   αx ⋅ c ∈ Vⱼ
  }{%
   wαx ⋅ c ∈ (W ⋄ V)ᵢ
  }
  \and
  \inferrule{%
   w ⋅ c ∈ Wⱼ \\
   c \in \{ \brk{i}, \acc{i}, \jmp{(\ell, i)} \}
  }{%
   w ⋅ c ∈ (W ⋄ V)ⱼ
  }
 \end{mathpar}
\end{definition}
The first rule composes accepting traces in $W$ with traces in $V$, picking up with the indicator value where the first trace left off.
Note also that this rule requires the last atom in the trace on the left to match the first atom in the trace on the right, because we want the second trace to start from the machine state computed in the first trace.
This mirrors the \emph{coalesced product} used to define the sequential composition of guarded languages (without continuations).
The last rule ensures that traces that encountered non-local control flow within $W$ are preserved in $W ⋄ V$.

\begin{example}%
\label{example:sequencing}
 Let $I = \{1,2\}$, and let $W$ and $V$ be indexed families given by:
 \begin{align*}
  W₁ & = \{ αpβ ⋅ \brk{1},\; βpα ⋅ \acc{2} \}
    & W₂& = \{ αqβ \cdot \acc{1} \} \\
  V₁ & = \{ γqβ ⋅ \ret \}
    & V₂ & = \{ αrβ ⋅ \jmp{(ℓ₁, 1)} \}
  \intertext{
   Then we can compute that the sequencing $W ⋄ V$ is the following indexed family:
  }
  (W ⋄ V)₁ & = \{ αpβ ⋅ \brk{1},\; βpαrβ ⋅ \jmp{(ℓ₁, 1)} \}
           & (W ⋄ V)₂ & = ∅
 \end{align*}
 Here, we find that $(W ⋄ V)_1$ contains $α p β ⋅ \brk{1}$ by the second rule, because $W_1$ does.
 Furthermore, the trace $β p α ⋅ \acc{2}$ in $W_1$ is composed with $α r β ⋅ \jmp{(\ell_1, 1)}$ from $V_2$ to form $β p α r β ⋅ \jmp {(\ell_1, 1)}$ in $(W ⋄ V)_1$, by the first rule.
 The set $(W ⋄ V)_2$ is empty, because despite the fact that $αqβ ⋅ \acc{1} ⋅ \acc{1} ∈ W_2$, there is no trace in $V_2$ that starts with $β$, and so neither rule can apply.
\end{example}

\subsection{continuation semantics, from the start}
With the theory of indexed families in place, we can now define the continuation semantics $C(e)^♯$ of a CF-GKAT program $e$ in terms of an indexed family.
We start with the base cases.

\begin{definition}[continuation semantics, base]
 For all $i, j ∈ I$, we define the following sets:
 \begin{align*}
  C(\comAssert{e_b})ᵢ^♯ & ≜ \{ α ⋅ \acc{i} ∣ (i, α) ∈ C( e_b ) \}
    & C( \comGoto{ℓ} )ᵢ^♯ & ≜ \{ α ⋅ \jmp{(ℓ, i)} ∣ α ∈ \At \} \\
  C(p)ᵢ^♯             & ≜ \{α p β ⋅ \mathbf{acc}\ i ∣ α, β ∈ \At\}
    & C( \comLabel{ℓ} )ᵢ^♯ & ≜ \{ α ⋅ \acc{i} ∣ α ∈ \At \} \\
  C( x := j )ᵢ^♯ & ≜ \{ α ⋅ \acc{j} ∣ α ∈ \At \}
    & C( \comBrk )ᵢ^♯ & ≜ \{ α ⋅ \brk{i} ∣ α ∈ \At \} \\
  C( \comRet )ᵢ^♯ & ≜ \{ α ⋅ \ret ∣ α ∈ \At \}
 \end{align*}
 \end{definition}

Each of these base syntax elements yields a simple (finite) indexed family.
For the constructs $\command{return}$, $\command{goto}$, and $\command{break}$, all traces terminate immediately in the corresponding continuation.

We inherit the semantics of assertions and primitive actions from (G)KAT~\cite{Kozen_1997,Schmid_Kappé_Kozen_Silva_2021}. % chktex 36
Assertions have traces that accept when their only atom satisfies the test.
A primitive action $p$ yields traces of the form $\alpha p \beta \cdot \acc{i}$ for all $\alpha, \beta \in \At$ to witness that $p$ is uninterpreted: we could reach any other machine state by running $p$.
The only information retained is the value of the indicator variable, because primitive actions cannot interact with indicators.
In contrast with primitive actions, an assignment like $x := j$ has traces that accept immediately, without changing the machine state; however, each trace ends with the indicator value $j$ --- regardless of the initial indicator value $i$.

Finally, labels are encoded as no-operations, which makes them neutral for sequencing operator, i.e., we have $C( \comLabel{ℓ} )^♯ ⋄ W = W = C( \comLabel{ℓ} )^♯ ⋄ W$ for all indexed families $W$.
This is because labels serve only as potential starting points of execution; we will leverage them in the next subsection.

We now turn our attention to the program composition operators.
These are generalizations of the guarded language semantics of GKAT~\cite{Schmid_Kappé_Kozen_Silva_2021}.
First of all, the $\comITE{b}{e}{f}$ filters out traces in the semantics of the $e$ that satisfy the guard $b$, as well as the traces in $f$ that invalidate it.

 \begin{definition}[continuation semantics, branching]
 Let $e, f \in \CFGKAT$.
 We define $C( \comITE{b}{e}{f} )^♯$ as the least indexed family that satisfies the following rules for all $i \in I$:
 \begin{mathpar}
    \inferrule{
        α ∈ C( e_b ) \\
        \alpha{}w ⋅ c ∈ C( e )^♯ᵢ
    }{%
        \alpha{}w ⋅ c ∈ C( \comITE{e_b}{e}{f} )^♯ᵢ
    }
    \and
    \inferrule{
        α ∉ C( e_b ) \\
        \alpha{}w ⋅ c ∈ C( f )^♯ᵢ
    }{%
        \alpha{}w ⋅ c ∈ C( \comITE{e_b}{e}{f} )^♯ᵢ
    }
 \end{mathpar}
 \end{definition}

 The semantics of the sequencing operator is easy: it just composes the semantics of the operands with the sequencing operator we have for indexed families.
 For loops, some more care is needed because traces can be iterated, and we need to account for early termination.

 \begin{definition}[continuation semantics, sequencing and loops]%
 \label{def:intermediate-sequencing-loops}
 Let $e, f ∈ \CFGKAT$.
 We define
 \(
    C(e; f)^♯ ≜ C( e )^♯ ⋄ C( f )^♯
 \).
 Also, for all $e_b ∈ \BExp_I$, we define $C(\comWhile{e_b}{e})^♯$ as the least indexed family satisfying:
 \begin{mathpar}
    \inferrule{%
        i ∈ I \\
        α ∉ C( e_b )
    }{%
        α ⋅ \acc{i} ∈ C(\comWhile{e_b}{e})^♯ᵢ
    }
    \and
    \inferrule{%
        α ∈ C( e_b ) \\
        α w ⋅ c ∈ (C( e )^♯ ⋄ C(\comWhile{e_b}{e})^♯)ᵢ
    }{%
        α w ⋅ ⌊ c ⌋ ∈ C(\comWhile{e_b}{e})^♯_i
    }
 \end{mathpar}
 The operation $\lfloor - \rfloor$ in the last rule is defined by $\lfloor c \rfloor = \acc{i}$ when $c = \brk{i}$, and $\lfloor c \rfloor = c$ otherwise.
\end{definition}

%FIXME: We need some kind of example around here, preferably based on a program from the introduction.

The first rule accounts for traces that halt immediately because the loop guard is false.
The second rule allows prepending traces from the loop body that satisfy the guard.
Because of the way sequencing works, body traces that end in $\brk{i}$ may occur; the second rule converts their continuations to $\acc{i}$, signaling that the loop has been exited and normal control flow can resume.

\smallskip
The semantics we have so far defines the traces of a program starting from the beginning.
However, a CF-GKAT program can be started from any label.
To obtain these traces for a given label $\ell$, we must descend into the program until we encounter the corresponding label statement.
For the base cases, this is relatively simple to accomplish: just check if we start at the label.

\begin{definition}[continuation semantics starting from a label, base]
 Let $e ∈ \CFGKAT$.
 For each $ℓ ∈ L$, we define the following guarded languages with continuations:
 \begin{align*}
  C(\comAssert{e_b})ᵢ^ℓ & ≜ ∅
    & C( \comGoto{ℓ'} )ᵢ^ℓ  & ≜ ∅ \\
  C(p)ᵢ^ℓ             & ≜ ∅
    & C( \comLabel{ℓ'} )ᵢ^ℓ & ≜ \{ α ⋅ \mathbf{acc}\ i ∣ α ∈ \At,\ ℓ = ℓ' \} \\
  C(x := j)ᵢ^ℓ        & ≜ ∅
    & C( \comBrk )ᵢ^ℓ     & ≜ ∅ \\
  C( \comRet )ᵢ^ℓ     & ≜ ∅
 \end{align*}
\end{definition}
Note how none of these cases has a trace, except the one for $C( \comLabel{ℓ'} )ᵢ^ℓ$ when $\ell' = \ell$, which accepts immediately.
With these cases covered, we can then treat the inductive step.

\begin{definition}[continuation semantics starting from a label, sequencing and branching]
Let $e, f \in \CFGKAT$, $e_b \in \BExp_I$ and $\ell \in L$.
We define the following indexed families to cover the traces of CF-GKAT programs starting from the label $\ell$ when composed using branching or sequencing:
\begin{mathpar}
    C(\comITE{e_b}{e}{f})^ℓᵢ ≜ C( e )^ℓᵢ \cup C( f )^ℓᵢ
    \and
    C(e; f)^ℓᵢ ≜ (C( e )^ℓ ⋄ C( f )^♯)ᵢ \cup C( f )^ℓᵢ
\end{mathpar}
\end{definition}
For branching, the semantics starting from $\ell$ disregards the guard and descends into the operands.
The sequencing case is more interesting: here, we still need to account for the traces that start from the beginning of $f$ after executing a trace in $e$ starting from the label $ℓ$.

The only remaining case to cover is the loop.
In this case, if we start execution from a label somewhere in the body, we may need to start the loop again after completing the loop body.
On the other hand, early termination in the loop body still needs to be turned into an accepting trace.

\begin{definition}[continuation semantics starting from a label, loops]
Let $e \in \CFGKAT$ and $e_b \in \BExp_I$.
We define the indexed family $C(\comWhile{b}{e})^ℓ$ below, where $\lfloor - \rfloor$ is as in \Cref{def:intermediate-sequencing-loops}:
\[
    C(\comWhile{e_b}{e})^ℓ_i = \{ w \cdot \lfloor c \rfloor \mid w \cdot c \in (C( e )^ℓ ⋄ C(\comWhile{e_b}{e})^♯)ᵢ \}
\]
\end{definition}

\subsection{trace semantics}

The continuation semantics of a CF-GKAT program $e$ in terms of indexed families $C( e )^♯$ uses continuations to record how a trace ends.
In particular, some traces may end with the continuation of the form $\jmp{(\ell, i)}$, signaling that computation needs to continue from the label $\ell$.
But we have just seen that we can also obtain the traces of $e$ starting from $\ell$, in the form of the indexed family $C( e )^ℓ$.
This means that we have the information we need to resolve the jumping continuations, if we just put together the right traces.
We will end this section by doing just that.

To formalize our approach, we need a way to refer to the continuation semantics of a program as a whole, i.e., for all indicator values, starting from either the beginning or some label.

% FIXME: I don't think using super script here is not a good idea,
% as it doesn't align with the notation later used for λ.
% I think we should just call this jump map, name it λ and use the bracket notation.
% TK: Does the above still apply? I don't see how superscripts clash with a different notation..?

\begin{definition}
 A \emph{labeled family of guarded languages (with continuations)}, or \emph{labeled family} for short, is a function $W$ from $L + ♯$ to indexed families of guarded languages (with continuations), e.g., $W: L + ♯ → I → 𝒞$.
 We often use superscripts to denote the value at a given label $ℓ$, writing $W^ℓ$ for $G(ℓ)$.
 Note that under this convention, $W^ℓ$ is an indexed family, which means that we may further unravel by writing $W^ℓ_i$ to obtain the guarded language with continuations $G(ℓ, i)$.
\end{definition}

Crucially, we can retrofit the continuation semantics $C( e )$ as a labeled family; after all, $C( e )^♯$ is an indexed family, and so is $C( e )^ℓ$ for each $ℓ ∈ L$.
We will thus treat $C( e )$ as such from this point on.

\smallskip
To resolve the jumps in a labeled family of guarded languages with continuations, we resolve the traces ending in $\jmp{(ℓ, i)}$ by looking up the traces that originate from label $ℓ$ with indicator value $i$.
We also remove the continuations $\acc{i}$ and $\ret$, because those come with traces that either reached the end of the program, or encountered a $\comRet$ statement respectively.
Continuations of the form $\brk{i}$ should not occur at the top level when computing the semantics of a program, as they will be resolved when computing the semantics of a loop; so we can ignore them.
The result is a labeled family of guarded languages (without continuations).

\begin{definition}
 Let $W: L + ♯ → I → 𝒞$ be a labeled family of guarded languages with continuations.
 We write $W\!↓$ for the (point-wise) least labeled family of guarded languages, such that the following rules are satisfied for all $k ∈ L + ♯$, $ℓ ∈ L$, and $i, j ∈ I$:
 \begin{mathpar}
  \inferrule{%
   w ⋅ \acc{i} ∈ Wᵢᵏ
  }{%
   w ∈ W\!↓ᵢᵏ
  }
  \and
  \inferrule{%
   w ⋅ \ret ∈ Wᵢᵏ
  }{%
   w ∈ W\!↓ᵢᵏ
  }
  \and
  \inferrule{%
   wα ⋅ \jmp{(ℓ, j)} ∈ Wᵢᵏ \\
   αx ∈ W\!↓ⱼ^ℓ
  }{%
   wαx ∈ W\!↓ᵢᵏ
  }
 \end{mathpar}
\end{definition}
The first two rules take care of flattening guarded words with continuations that in acceptance,
while the third rule strings together guarded words continuations that jump to a different label.

\begin{example}
 Let $W$ be the labeled family of guarded languages with continuations defined by
 \begin{align*}
  W₁^♯ & = \{ α ⋅ \jmp{(ℓ, 1)} \}
    & W₂^♯ & = ∅ \\
  W₁^{ℓ} & = \{ α p α ⋅ \jmp{(ℓ', 1)}, β ⋅ \acc{1} \}
    & W₂^{ℓ} & = \{ α ⋅ \jmp{(ℓ', 2)} \} \\
  W₁^{ℓ'} & = \{ α q α ⋅ \jmp{(ℓ, 1)}, α r β ⋅ \jmp{(ℓ, 1)} \}
    & W₂^{ℓ'} & = \{ α ⋅ \jmp{(ℓ, 1)}\}
 \end{align*}
 Now $W\!↓₁^♯$ contains, among other things,
 the guarded word $α p α q α p α r β$.

 Note furthermore that $W\!↓₂^{ℓ}$ is empty, despite $W₂^{\ell}$ containing a guarded word with a continuation that has a mutual jump with another guarded word with continuation in $W₂^{\ell'}$, as these can never be concatenated into one guarded word with continuation of the form $\acc{i}$ or $\ret$.
\end{example}

In total, we can then obtain the semantics of a CF-GKAT term as $C( e )\!\downarrow$, in the form of a labeled family of guarded languages.
This concludes our discussion of the semantics of CF-GKAT\@.

% FIXME: Do we need this here..?
\begin{lemma}\label{the: label missing causes empty semantics}
    If \(\comLabel{ℓ}\) does not appear in expression \(e\), then \(∀ i ∈ I, C(e)ᵢ^ℓ = ∅\).
\end{lemma}

% FIXME: I think this should be conversion to GKAT automaton
\section{Decision procedure}
\label{section:decision procedure}

To establish the decision procedure, we propose CF-GKAT automata, completing the classical correspondence between program, semantics, and automaton. 
Specifically, every CF-GKAT expression can be converted to a CF-GKAT automaton via the Thompson's construction, while preserving its continuation semantics. 

Unlike GKAT automata, directly performing bisimulation on CF-GKAT automata will not yield the desired trace equivalence. 
Indeed, there exist two programs with the same trace semantics but different continuation semantics:
\begin{mathpar}
  x := 1 \and \comAssert{\true}.
\end{mathpar} 
The first program sets the indicator variable to 1, and the second program simply skips. 
The trace semantics of the above two programs are the same: both indicator assignment and skip are ``unproductive'', i.e. they will terminate immediately without executing any action.
Yet, their continuation semantics are different: \(C(x := 1)ᵢ^♯\) will constantly emit the continuation \(\acc{1}\) regardless of \(i\), but \(C(\comAssert{\true})ᵢ^♯\) will preserve starting indicator by outputting the continuation \(\acc{i}\).

Thus, our decision procedure cannot rely on bisimulation between CF-GKAT automaton; instead, we lower the CF-GKAT automata into GKAT automata. This process allows us to reuse the nearly-linear decision algorithm for GKAT automata equivalences.
Finally, the soundness and completeness of our decision procedure can be derived from a sequence of correctness results: first the correctness of Thompson's construction (\Cref{the:thompson-correctness}), then the correctness of the lowering (\Cref{the:cf-gkat-automaton-lowering-correctness}), and finally the soundness and completeness of GKAT automata equivalence~\cite{Schmid_Kappé_Kozen_Silva_2021}.

\subsection{CF-GKAT automata}

To leverage the efficient decision algorithm for GKAT automata, we will need to convert each CF-GKAT expression $e$ into a GKAT automaton that implements $C( e )\!↓$.
As we have discussed before, this process is separated into two steps, and CF-GKAT automaton serves as an crucial intermediate between CF-GKAT expressions and GKAT automata. 
In this section, we will formally define CF-GKAT automata and their continuation semantics.

Like GKAT automaton, CF-GKAT automaton is defined by a dynamics/signature~\cite{rutten_UniversalCoalgebraTheory_2000,jacobs_IntroductionCoalgebraMathematics_2016}
\begin{definition}[CF-GKAT dynamics]
 Given a set $X$, we write $D(X)$ for the set
 \[D(X) ≜ I × \At → \{\reject\} + C + K × X × I.\]
\end{definition}

Intuitively, the elements of \(D(S)\) represent possible transition behaviors in a CF-GKAT automaton over a state set $S$.
Given a current indicator value \(i ∈ I\) and an atom $α$ accounting for the truth value of each primitive test, a dynamic $ρ ∈ D(S)$ may either:
\begin{itemize}
 \item
       \emph{reject} the input, represented by $ρ(i, α) = \reject$;
 \item
       offer a \emph{continuation}, represented by $ρ(i, α) ∈ C$; or
 \item
       execute a primitive action in $K$ and set a new indicator value from $I$ while transitioning to a new state in $S$, represented by $ρ(i, α) ∈ K × X × I$.
\end{itemize}

Then the definition of CF-GKAT automaton is similar to GKAT automaton, except we will need a function \(λ: L → D(S)\) where \(λ(ℓ)\) provides a dynamics representing the ``entry point'' for label \(ℓ\).
\begin{definition}
 A \emph{CF-GKAT automaton} \(A ≜ ⟨S, δ, \hat{s}, λ⟩\) consists of a set of \emph{states} \(S\), a \emph{transition function} \(δ: S → D(S)\),
 a \emph{start state} \(\hat{s} ∈ S\), and a \emph{jump map} \(λ: L → D(S)\).
\end{definition}

Intuitively, the transition map \(δ\) assigns every state in $S$ a dynamics from $D(S)$. The jump map $λ$, on the other hand, assign a dynamics for each label $ℓ ∈ L$, indicating how to resume the computation after a \(\jmp{ℓ, i}\) continuation is reached.

\begin{example}[A simple CF-GKAT automaton]
  Consider the following program \[\comITE{b}{\{\comLabel{ℓ}; p\}}{\comGoto{ℓ}},\] then we can construct the following automaton that have the same behavior as the program 
  %FIXME: probably better diagram?
  \[\begin{tikzcd}[row sep=small]
    {} & {\hat{s}} & s & {} \\
    & {} & {}
    \arrow[shorten <=8pt, from=1-1, to=1-2]
    \arrow["{b/p}", from=1-2, to=1-3]
    \arrow["\overline{b}/\jmp{ℓ}"'{pos=1}, Rightarrow, from=1-2, to=2-2]
    \arrow["b/\acc{i}"{pos=0.5}, shorten >=15pt, Rightarrow, from=1-3, to=1-4]
    \arrow["\overline{b}/\acc{i}"{pos=1}, Rightarrow, from=1-3, to=2-3]
  \end{tikzcd}\]
  where \(ŝ \xrightarrow{b/p} s\) means that \(δ(ŝ, i, b) = (s, p)\) and \(ŝ ⇒^{\overline{b}/\acc{i}}\) means that \(δ(ŝ, i, \overline{b}) = \acc{i}\).
  In the above automaton, where the start state is \(ŝ\), 
  \begin{itemize}
    \item If the input atom is \(b\), then it will transition to the state \(s\), while executing \(p\);
    then the state \(s\) will always accept.
    \item If the input atom is \(\overline{b}\), then it will simply accept the input without executing any action.
  \end{itemize}
  As we can see, the behavior of \(ŝ\) indeed matches the behavior of the program when executing from the start.
  Then the entry dynamics for \(ℓ\) can be defined as follows:
  \[λ(ℓ, i, α) ≜ (p, s, i).\]
  To put the above definition into words: when jump to the label \(ℓ\), we will reach the state \(s\) while executing \(p\); then \(s\) will halt regardless of the condition.
  Thus, the behavior of \(λ(ℓ)\) matches the behavior of the program when executing starting from the label \(ℓ\).
\end{example}

To formalize the intuition of ``behaviors'' in the previous example. We can assign a continuation semantics to each CF-GKAT automaton \(A ≜ ⟨S, δ, ŝ, λ⟩\).
Before that, it is convenient to first define the semantics for each dynamics in \(D(S)\).

\begin{definition}[continuation semantics]
 Given an automaton \(A ≜ ⟨S, δ, \hat{s}, λ⟩\),
 the continuation semantics of each dynamics \(ρ ∈ D(S)\) is
 a family \(C(ρ)_A: I → 𝒞\),
 defined as the (point-wise) smallest set satisfying the following rules for $i, j ∈ I$ and $α ∈ \At$:
 \begin{mathpar}
  \inferrule{%
   ρ(i, α) = \acc{j}
  }{%
   α ⋅ \acc{j} ∈ (C(ρ)_A)ᵢ
  }
  \and
  \inferrule{%
   ρ(i, α) = \brk{j}
  }{%
   α ⋅ \brk{j} ∈ (C(ρ)_A)ᵢ
  }
  \and
  \inferrule{%
   ρ(i, α) = \ret
  }{%
   α ⋅ \ret ∈ (C(ρ)_A)ᵢ
  }
  \\
  \inferrule{%
   ρ(i, α) = \jmp{(ℓ, j)}
  }{%
   α ⋅ \jmp{(ℓ, j)} ∈ (C(ρ)_A)ᵢ
  }
  \and
  \inferrule{%
   ρ(i, α) = (p, s, j) \\
   w ∈ (C(δ(s))_A)ⱼ
  }{%
   αpw ∈ (C(ρ)_A)ᵢ
  }
 \end{mathpar}
 Similar to the continuation semantics of expressions, the continuations semantics of automata are also labeled families of guarded languages with continuations. Specifically, the semantics from the start \(C(A)^♯\) is defined by the dynamics of the start state, and the semantics of a label \(ℓ ∈ L\) is defined by the jump map: 
 \begin{mathpar}
  C( A )^♯ = C( δ( ŝ ) )_A, \and 
  C( A )^ℓ = C( λ(ℓ) )_A \text{ for } ℓ ∈ L.
 \end{mathpar}
\end{definition}

\subsection{Lowering CF-GKAT automata to GKAT automata}\label{sec:lowering-cf-gkat-automata-to-gkat}

The process to lower a CF-GKAT automaton $⟨S, δ, \hat{s}, λ⟩$ into a GKAT automaton consists of two different components.
First, we "embed" the indicator values into the state set; the new state set then becomes $S × I$.
Second, we resolve all the continuations in transition results.
In particular, we need to resolve the jump continuations using the jump map $λ$: when $δ(s, i, α) = \jmp{(ℓ, j)}$, the $α$-transition leaving the state $(s, i)$ in the resulting GKAT automaton is determined by looking at the $α$-behavior starting from the label $ℓ$ with indicator value $j$, given by $λ(ℓ, j, α)$.

The main obstacle to properly resolve jump continuation is that $λ(ℓ, j, α)$ may itself point to a different label by returning $\jmp{(ℓ', k)}$, then \(λ(ℓ', k, α)\) may also yield another jump, et cetera.
These jump sequences can be resolved by iterating the jump map, and terminate when either the result is no longer a jump, or an infinite loop is detected.

\begin{definition}[iteration lifting]\label{def: iteration lifting}
  Given a function \(h: X → X + \{\reject\} + E\), where \(X\) is a finite set
  and \(\{\reject\} + E\) specifies the ``exit results'',
  then this function can be lifted to \(\iter(h)\) by iterating \(h\).
  We will use \(M\) to keep track of the explored value of \(M\):
  \begin{align*}
  \iter'(h) & : 2^X → X → \{\reject\} + E \\
  \iter'(h) & (M)(m) ≜ \begin{cases}
      \reject & \text{if } m ∈ M  \\
      h(m) & \text{if } m ∉ M \text{ and } h(m) ∈ \{\reject\} + E \\
      \iter'(h)(M ∪ \{m\})(h(m)) & \text{if } m ∉ M \text{ and } h(m) ∈ X
    \end{cases};
  \intertext{and \(\iter\) defined as supplying \(∅\) as the starting point of \(M\):}
    \iter(h) & : X → \{\reject\} + E \\
    \iter(h) & ≜ \iter'(h)(∅)
  \end{align*}
  To improve clarity, in the definition of \(h\),
  we will write \(\injL : X → X + \{\reject\} + E\) as \(\contWith\),
  to indicate the iteration will continue;
  and write \(\injR: \{\reject\} + E → X + \{\reject\} + E\) as \(\exitWith\),
  to indicate the iteration will be exited.
\end{definition}
Intuitively, \(M\) keeps track of all the explored value in \(X\),
and if a input has already been explored,
then rejection \(\reject\) will be returned to indicate a infinite loop;
as unproductive infinite iterations will not produce any observable trace.
Indeed, GKAT treats un-productive infinite iterations
in the \command{while} loops with the same strategy~\cite{Schmid_Kappé_Kozen_Silva_2021}.
On the other hand, if \(h(m)\) falls into the exit set \(\{\reject\} + E\),
then \(\iter(h)\) will stop and return \(h(m)\).
Finally, if the \(h(m)\) fall into \(X\), then \(\iter(h)\) will continue the loop with \(h(m)\) as input, and mark \(m\) as explored.
Notice that \(\iter(h)\) will be total when \(h\) is total,
because \(X\) is a finite set.

In the specific case of jump resolution,
the iteration will continue when the result of \(λ\) is a jump, but exit otherwise.

\begin{definition}[Jump resolution]
 Let $S$ be a finite set, and let $λ∶ L → D(S)$ be a jump function.
 We define the resolved jump map ${λ\!↓}: L → D(S)$, as follows:
 \[
  λ\!↓ = \iter \left(
    (ℓ, i, α) ↦ \begin{cases}
      \contWith (ℓ', i', α) & λ(ℓ, i, α) = \jmp{(ℓ', i')} \\
      \exitWith (λ(ℓ, i, α)) & \text{otherwise}
    \end{cases}
  \right)
 \]
\end{definition}

Notice that \(λ\!↓\) resolves the internal jumps continuation in \(λ\); concretely the return of \(λ\!↓\) will never be a jump continuation. 
Intuitively, the continuation resolution is separated into two procedure: we first replace all the jump continuation with the dynamics \(λ\!↓\) to obtain \(δ'\), and then we will resolve the other continuation in \(δ'\) as both accept or reject to obtain the lowered transition function, namely \(δ\!↓\). Formally, the lowering is defined as follows:

\begin{definition}[Lowering CF-GKAT automata]
 Given a CF-GKAT automaton \(A ≜ ⟨S, δ, \hat{s}, λ⟩\), and $i ∈ I$, we define the GKAT automaton \({𝐴\!↓ᵢ} ≜ ⟨S × I, δ\!↓, (s, i)⟩\), where $δ\!↓$ is given in two steps:
 \begin{align*}
  δ'((s, i), α) & ≜
    \begin{cases}
      λ\!↓(ℓ, i, α) & δ(q, i, α) = \jmp{(ℓ, i)} \\
      δ(q, i, α) & \text{otherwise}
    \end{cases}\\
  δ\!↓((s, i), α) & ≜
  \begin{cases}
    % \mathrlap and \hphantom is used for alignment purpose
    % the \mathrlap is the displayed expression
    % and \hphantom contains the "longest expression" for alignment
    \mathrlap{\reject}\hphantom{λ\!↓(ℓ, i, α)} & δ'(s, i, α) ∈ \{ \brk{j} \}\\
   \accept & δ'(s, i, α) ∈ \{ \ret, \acc{j} : j ∈ I \} \\
   δ(q, i, α) & \text{otherwise}
  \end{cases}
 \end{align*}
\end{definition}
Notice \(λ\!↓\) is total and \(δ'((s, i), α)\) will not return a jump continuation, therefore $δ\!↓$ is well-defined; in other words, for all $s ∈ S$, $i ∈ I$ and $α ∈ \At$, it holds that $δ\!↓((s, i), α) ∈ \{\reject, \accept\} + K × (S × I)$, as expected for a GKAT automaton on state set $S × I$.

Having defined our lowering operation, we can state its correctness as follows.

\begin{theorem}[Correctness of lowering]\label{the:cf-gkat-automaton-lowering-correctness}
 Let \(A ≜ ⟨S, δ, \hat{s}, λ⟩\) be a CF-GKAT automaton.
 The translation from CF-GKAT automata to GKAT automata commutes with the semantic jump resolution operator, in the sense that for $i ∈ I$, it holds that $G(A\!↓_i) = C(A)\!↓^♯ᵢ$.
\end{theorem}

\subsection{Converting expressions to CF-GKAT automata}

The final piece of our puzzle is to convert CF-GKAT expressions to CF-GKAT automata.
To accomplish this, we generalize a construction proposed for GKAT, which turns a GKAT expression into a GKAT automaton in a trace-equivalent manner~\cite{Smolka_Foster_Hsu_Kappé_Kozen_Silva_2020}.
This construction proceeds by induction on the structure of the expression, and was inspired by Thompson's construction to obtain a non-deterministic finite automaton from a regular expression~\cite{thompson_ProgrammingTechniquesRegular_1968}, which is why we refer to it as the \emph{Thompson construction for CF-GKAT}.

In contrast to the original Thompson's construction, the Thompson's construction for GKAT produces a GKAT automaton with a \emph{start dynamics} instead of an explicit start state. 
Although automata with start dynamics are equivalent to automata with start states; using start dynamics will help us efficiently compose automata, avoiding the silent transitions present in the original algorithm.
To take advantage of start dynamics, we will define \emph{CF-GKAT automata with start dynamics} in the following definition:

\begin{definition}
 A CF-GKAT automaton with start dynamics \(A ≜ ⟨S, δ, ι, λ⟩\) consists of $S$, $δ$ and $λ$ as in a CF-GKAT automaton, in addition to a start dynamics \(ι ∈ G(s)\).
\end{definition}

We elide the definition of the semantics for CF-GKAT automata with start dynamics for the sake of brevity.
Suffice it to say that they can be easily converted to a plain CF-GKAT automata by adding a start state \(\hat{s}\) that takes the behavior of the start dynamics \(ι\):
\begin{equation}\label{cons: CF-GKAT pseudo start to CF-GKAT automata}
 ⟨S, δ, ι, λ⟩ ↦ ⟨S + \hat{s}, δ_ι, \hat{s}, λ⟩,
 \qquad
 \text{where}
 \qquad
 δ_ι(s, i, α) ≜
 \begin{cases}
  ι(i, α)    & \text{if } s = \hat{s} \\
  δ(s, i, α) & \text{if } s ≠ \hat{s}
 \end{cases}
\end{equation}

Thompson's construction turns a CF-GKAT expression \(e\) to a CF-GKAT automaton with start dynamics.
We call the result of said construction \emph{the Thompson's automaton} for \(e\).
The following paragraphs describe the construction and intuition behind Thompson's construction by cases. 
We will use \(A₁ ≜ ⟨S₁, δ₁, ι₁, λ₁⟩\) and \(A₂ ≜ ⟨S₂, δ₂, ι₂, λ₂⟩\) to denote the Thompson's automata for \(e₁\) and \(e₂\) respectively; and use \({!}\) to be the unique function from the empty set \({!}: ∅ → X\).
For brevity, we will use the compact notation~\cite{Smolka_Foster_Hsu_Kappé_Kozen_Silva_2020} for \command{if}-statements and \command{while}-loop:
\begin{mathpar}
  e₁ +_b e₂ ≜ \comITE{b}{e₁}{e₂}, \and 
  e₁^{(b)} ≜ \comWhile{b}{e₁}.
\end{mathpar}
Finally, to obtain a CF-GKAT automaton with the same continuation semantics as \(e\), we will convert the Thompson's automaton of \(e\) to CF-GKAT automaton by construction \labelcref{cons: CF-GKAT pseudo start to CF-GKAT automata}.

%FIXME: CZ: I think all the following paragraphs should have formally defined the thompson's construction using notation. And the table should only be used as a reference.
% in this way we can make the table more compact.
\paragraph{Converting \comBrk, \comRet, \command{goto}, and indicator assignment:}
recall the semantics of \(\comBrk\), \(\comRet\), \(\comGoto{ℓ}\), and indicator assignments will simply emit the corresponding continuations.
Thus, the non-trivial part of these Thompson's automata are the start dynamics \(ι\), which output said continuations:
\begin{align*}
  S_{\comBrk} & ≜ ∅ &
    S_{\comRet} & ≜ ∅ &
    S_{\comGoto{ℓ}} & ≜ ∅ &
    S_{x := i} & ≜ ∅ \\
  δ_{\comBrk} & ≜ {!} &
    δ_{\comRet} & ≜ {!} &
    δ_{\comGoto{ℓ}} & ≜ {!} &
    δ_{x := i} & ≜ {!}\\
  ι_{\comBrk}(i, α) & ≜ \brk{i} &
    ι_{\comRet}(i, α) & ≜ \ret &
    ι_{\comGoto{ℓ}}(i, α) & ≜ \jmp{(ℓ, i)} &
    ι_{x := i}(i, α) & ≜ \acc{i}\\
  λ_{\comBrk}(ℓ) & ≜ ⊥ &
    λ_{\comRet}(ℓ) & ≜ ⊥ &
    λ_{\comGoto{ℓ}}(ℓ) & ≜ ⊥ &
    λ_{x := i}(ℓ) & ≜ ⊥
\end{align*}

\paragraph{Converting tests and primitive actions:}
the conversions of primitive tests and primitive actions largely inherit the Thompson's construction for GKAT.
The Thompson's automaton for tests \(\comAssert{e_b}\) will contain only a start dynamic, which accepts the input indicator-atom pairs if and only if they satisfy \(e_b\).
The Thompson's automata for primitive actions \(p\) will contain a start dynamic that always transition to the unique state while executing the action \(p\), then the unique state will accept all inputs.
\begin{align*}
  S_{\comAssert{e_b}} & ≜ ∅ &
    S_{p} & ≜ \{*\} \\
  δ_{\comAssert{e_b}} & ≜ {!} &
    δ_{p}(s, i, α) & ≜ \acc{i} \\
  ι_{\comAssert{e_b}}(i, α) & ≜ \begin{cases}
      \acc{i} & (i, α) ∈ ⟦e_b⟧ \\  
      ⊥ & (i, α) ∉ ⟦e_b⟧
    \end{cases} &
    ι_{p}(i, α) & ≜ (p, *, i) \\
  λ_{\comAssert{e_b}}(ℓ) & ≜ ⊥ &
    λ_{p}(ℓ) & ≜ ⊥ 
\end{align*}

\paragraph{Converting labels:}
recall that \(\comLabel{ℓ'}\) is a non-operation when computing the semantics from the start of the program, i.e. its semantics coincides with the sequential identity: \(\comAssert{\true}\).
However, the behavior of \(\comLabel{ℓ'}\) and \(\comAssert{\true}\) diverges when we consider the semantics starting from labels.
Specifically, when starting from a label \(ℓ ≠ ℓ'\),
\begin{mathpar}
  C(\comLabel{ℓ'})ᵢ^ℓ = ∅; \and
  C(\comAssert{\true})ᵢ^ℓ = \{α ⋅ \acc{i} ∣ α ∈ \At \}.
\end{mathpar}
because the continuation \(\jmp{(ℓ,i)}\) will not resume from \(\comLabel{ℓ'}\) when \(ℓ' ≠ ℓ\).
This difference is reflected in the jump map \(λ: L → G(S)\), which specifies the entry point for each label.
In this case, the jump map will map \(ℓ'\) to the behavior of the identity operation \(\comAssert{\true}\), and map every other label to constant rejection.
\begin{align*}
  S_{\comLabel{ℓ'}} & ≜ ∅ & 
  δ_{\comLabel{ℓ'}} & ≜ {!} \\
  ι_{\comLabel{ℓ'}}(i, α) & ≜ \acc{i} &
  λ_{\comLabel{ℓ'}}(ℓ, i, α) & ≜ \begin{cases}
    \acc{i} & ℓ = ℓ'  \\
    ⊥ & ℓ ≠ ℓ'
  \end{cases}
\end{align*}

% FIXME: I think this is too on the nose... Basically just reiterating the definition
\paragraph{Converting \command{if} statements:}
the Thompson automaton for \(\comITE{b}{e₁}{e₂}\) is also similar to that of GKAT:
If the input indicator-atom pair satisfies \(b\), the start \(ι\) will enter the Thompson automaton of \(e₁\) by taking on the behavior of \(ι₁\); and when the starting indicator-atom pair doesn't satisfy \(b\), then \(ι\) will take on the behavior of \(ι₂\).
The jump map \(λ\) assigns the entry point for label \(ℓ\) based on where \(ℓ\) appears: namely if \(ℓ\) appears in \(e₁\), then \(ℓ\) will take its entry point in \(A₁\); and similarly, if \(ℓ\) appears in \(e₂\), \(ℓ\) will take its entry point in \(A₂\).
\begin{align*}
  S_{e₁ +_b e₂} & ≜ S₁ + S₂ \\
  δ_{e₁ +_b e₂}(s) & ≜ \begin{cases}
    δ₁(s) & \text{if } s ∈ S₁ \\
    δ₂(s) & \text{if } s ∈ S₂ \\
  \end{cases} \\
  ι_{e₁ +_b e₂}(i, α) & ≜ \begin{cases}
    ι₁(i, α) & (i, α) ∈ ⟦b⟧ \\  
    ι₂(i, α) & (i, α) ∉ ⟦b⟧
  \end{cases} & \\
  λ_{e₁ +_b e₂}(ℓ, i, α) & ≜ \begin{cases}
    λ₁(ℓ, i, α) & (\comLabel{ℓ}) \text{ appears in } e₁\\
    λ₂(ℓ, i, α) & (\comLabel{ℓ}) \text{ appears in } e₂\\
    ⊥ & \text{otherwise}
  \end{cases}
\end{align*}

\paragraph{Converting Sequencing:}
Sequencing of automata can be defined by \emph{uniform continuations}, which combines two dynamics $h₁, h₂ ∈ G(S)$ into a new dynamic \(h₁[h₂]\): the resulting dynamic acts like $h₁$ in almost all cases, except when $h₁$ accepts, then it will take on the behavior of \(h₂\).
In other word, \(h₁[h₂]\) connects all the accepting transition of \(h₁\) to \(h₂\).
Uniform continuation is typically used to compose two automata or add self-loops to an automaton; and can be formally defined as follows.
\begin{definition}[Uniform Continuation]
  Let $S$ be a set and given two dynamics $h₁, h₂ ∈ G(S)$,
  their \emph{uniform continuation} is the dynamic $h₁[h₂] ∈ G(S)$, defined as follows:
  \[
    h₁[h₂](i, α) ≜
    \begin{cases}
    h₂(i', α) & \text{if } h₁(i, α) = \acc{i'} \\
    h₁(i, α)  & \text{otherwise}
    \end{cases}
  \]
\end{definition}
To construct the Thompson's automaton for \(e₁; e₂\),
we will simply connect all the accepting transition in \(A₁\) to \(A₂\),
by applying uniform continuations on start dynamics \(ι₁\), transitions \(δ₁\), and jump map \(λ₁\); while preserving the dynamics in \(A₂\).
\begin{align*}
  S_{e₁; e₂} & ≜ S₁ + S₂ &
  δ_{e₁; e₂}(s) & ≜ \begin{cases}
    δ₁(s)[ι₂] & \text{if } s ∈ S₁ \\
    δ₂(s) & \text{if } s ∈ S₂ \\
  \end{cases} \\
  ι_{e₁; e₂} & ≜ ι₁[ι₂] &
  λ_{e₁; e₂}(ℓ, i, α) & ≜ \begin{cases}
    λ₁(ℓ, i, α) & (\comLabel{ℓ}) \text{ appears in } e₁\\
    λ₂(ℓ, i, α) & (\comLabel{ℓ}) \text{ appears in } e₂\\
    ⊥ & \text{otherwise}
  \end{cases}
\end{align*}

\paragraph{Converting \command{while} loops:}
Like GKAT automata, CF-GKAT automata require every transition between states to execute a primitive action.
This characteristic presents a unique challenge in defining the start dynamics for while loops.
Namely, the primitive action is not necessarily encountered in the first iteration of the loop; for example, consider 
\begin{equation}\label[prog]{prog:loop-head-iter-example}
  \begin{aligned}
    \comWhile{ & \true}{\{\\
      & \comITE{x = 0}{x := 1 \\
      &}{\comITE{x = 1}{\comBrk \\
      &}{\{\comAssert{\true}\}}}\\ 
      \}}
  \end{aligned}
 \end{equation}
with the input indicator \(0\), then the first continuation \(\brk{1}\) will be encountered on the second iteration of the loop.
Even worse, when starting with an indicator value that does not appear in the program, like \(x = 2\), the program will enter an infinite loop and never encounter its first primitive action or continuation. 

Fortunately, these difficulties can be resolved by the \(\iter\) function (\cref{def: iteration lifting}). We can iterate the loop body until we encounter a primitive action or non-local control.
\begin{definition}[Iterated Start Dynamics]
  Let $S$ be a set, let $h ∈ G(S)$, and $e_b ∈ \BExp$.
  We can use the \(\iter\) function to define $h^{(e_b)}: G(S)$, as follows:
  \begin{align*}
   h^{(e_b)} & : G(S)\\
   h^{(e_b)} & ≜
   \iter\left(
     (i, α) ↦
     \begin{cases}
       \exitWith(\acc{i}) & \text{if } (i, α) ∉ ⟦e_b⟧ \\
       \contWith(i', α) & \text{if } (i, α) ∈ ⟦e_b⟧ \text{ and } h(i, α) = \acc{i'} \\
       \exitWith(h(i, α)) & \text{otherwise}
     \end{cases}
   \right)
  \end{align*}
\end{definition}
In the first case, the input \((i, a)\) doesn't satisfy \(b\), causing the while loop to terminate.
In the second case, the loop body accepts \((i, α)\) immediately and returns the exit indicator value \(i'\), thus the iteration of loop body will continue with \((i', α)\).
And the final case is reached when the program executes an action or encounters a non-local control, then the iteration can also be stopped.
\begin{example} [Iterated Start Dynamics]
  Consider~\cref{prog:loop-head-iter-example} above with indicator set \(\{0, 1, 2\}\), no primitive action, no label, and no primitive test.
  We use \(\true\) to denote the only atom over empty set. Thus, the  Thompson's automaton \(A₁ ≜ ⟨S₁, δ₁, ι₁, λ₁⟩\) of the loop body
  \begin{align*}
    & \comITE{x = 0}{x := 1 \\
    &}{\comITE{x = 1}{\comBrk \\
    &}{\{\comAssert{\true}\}}}
  \end{align*}
  can be computed as the following
  \begin{align*}
    && ι₁(0, \true) & ≜ \acc{1} &&&\\
    S₁ & ≜ ∅ & 
    ι₁(1, \true) & ≜ \brk{1} &
    δ₁ & ≜ {!} & 
    λ & ≜ {!}\\
    && ι₁(2, \true) & ≜ \acc{2} &&&
  \end{align*}
  %FIXME: this derivation does not exactly follow the definition of iter, but a intuitive account
  Then we compute iterated start dynamics \(ι^{(\true)}\) with input \((0,\true)\) and \((2,\true)\):
  \begin{align*}
    {ι₁}^{(\true)}(0, \true) 
    & = {ι₁}^{(\true)}(1, \true) 
      & \text{because }(0, \true) ∈ ⟦\true⟧ \text{ and } {ι₁}(0, \true) = \acc{1} \\  
    & = \brk{1}
      & \text{because } {ι₁}(1, \true) = \brk{1} \\[10px]
    {ι₁}^{(\true)}(2, \true) 
    & = {ι₁}^{(\true)}(2, \true) 
      & \text{because }(2, \true) ∈ ⟦\true⟧ \text{ and } ι₁(2, \true) = \acc{2} \\  
    & = ⊥
      & \text{because the input \((2, \true)\) is already explored}
  \end{align*}
\end{example}

With the start dynamics defined, we still need to resolve structures within the loop body, like the \(\comBrk\)-continuation.
To perform $\comBrk$-resolution, we extend the \(⌊-⌋\) operator to dynamics.
\begin{definition}
Let $S$ be a set, and let $h ∈ G(S)$.
We define $⌊h⌋ ∈ G(S)$ by lifting \(h\) via \(⌊-⌋\) when it returns a continuation:
\[
  ⌊h⌋(i, α) = \begin{cases}
  ⌊h(i, α)⌋ & \text{if } h(i, α) ∈ C \\
  h(i, α)   & \text{otherwise}
  \end{cases}
\]
\end{definition}

Finally, the transition function \(δ\) and jump map \(λ\) can be defined by first connecting \(δ₁\) and \(λ₁\) back to start dynamics \(ι\), forming a loop in the automaton;
then resolving the \(\brk{i}\) continuations using the break resolution function \(⌊-⌋\).
Formally, the Thompson's automaton for loop \(e^{(e_b)}\) can be defined as follows:
\begin{align*}
  S_{e₁^{(e_b)}} & ≜ S₁ & 
  δ_{e₁^{(e_b)}} & ≜ ⌊ δ₁(s)[{ι₁}^{(e_b)}] ⌋ &  
  ι_{e₁^{(e_b)}} & ≜ ⌊ {ι₁}^{(e_b)} ⌋ &
  λ_{e₁^{(e_b)}}(ℓ) & ≜ ⌊ λ(ℓ)[{ι₁}^{(e_b)}] ⌋
\end{align*}


% FIXME: I think this is good to say, but should be in a separate remark after the construction is fully presented - T
% In the worst case, it is possible for \(I'\)
% to exhaust all of \(I\) before an infinite loop is found,
% which means computing \(γ\) can take \(|I|\) time
% for every indicate \(i\) and atoms \(α\).
% However, for a fixed atom \(α\), we can cache the result of each \(i\),
% leading to a \(|I|\)-timed algorithm to compute \(γ\) for every input \(i\).

We present a summary of the Thompson's construction in~\cref{tab: thompson's construction}, and we will state the correctness of the Thompson's construction as follows.
\begin{theorem}[Thompson's construction preserves continuation semantics]\label{the:thompson-correctness}
  Given an expression $e ∈ \CFGKAT$, let $A_e$ be the Thompson automaton for $e$, then \(C(e) = C(A_e)\). By unfolding the definition of continuation semantics:
  \[∀ i ∈ I, ℓ ∈ \{♯\} + L, C(e)ᵢ^ℓ = C(A_e)ᵢ^ℓ\]
 \end{theorem}


% FIXME: The syntax in the expression column is the compact one... we probably want to change that - T
% FIXME: CZ: we don't have enough space, we need to think about this.
\begin{table}
  \centering
  \renewcommand\arraystretch{2}
  \begin{tabular}{c || c | l | l | l}
   \(e\)  & \(S\) & \(δ(s)\) & \(ι(i, α)\) & \(λ(ℓ)\) \\
   \hline
   \(x := i'\) & \(∅\) & \(!\) & \(\acc{i}\) & \(\const(⊥)\) \\
   \(\comRet\) & \(∅\) & \(!\) & \(\ret\) & \(\const(⊥)\) \\
   \(\comBrk\) & \(∅\) & \(!\) & \(\brk{i}\) & \(\const(⊥)\) \\
   \(\comGoto{ℓ}\) & \(∅\) & \(!\) & \(\jmp{(ℓ, i)}\) & \(\const(⊥)\) \\
   \(\comAssert{b}\)& \(∅\)& \(!\)& \(
   \begin{cases}
    \acc{i} & (i, α) ∈ ⟦ b ⟧   \\
    ⊥       & (i, α) ∉ ⟦ b ⟧ 
   \end{cases}
   \) & \(\const(⊥)\) \\
   \(p\) & \(\{*\}\) & \(λ i.\const(\acc{i})\) & \((p, *, i)\) & \(\const(⊥)\) \\
   %
   \(\comLabel{ℓ'}\) & \(∅\) & \(!\)  & \(\acc{i}\) &
    \(\begin{cases}
      λ i.\const(\acc{i}) & ℓ = ℓ' \\
      \const(⊥)    & ℓ ≠ ℓ'
      \end{cases}
   \)\\
   %
   \(e₁ +_{e_b} e₂\) & \(S₁ + S₂\) & \(
   \begin{cases}
    δ₁(s) & s ∈ S₁ \\
    δ₂(s) & s ∈ S₂
   \end{cases}
   \) & \(
   \begin{cases}
    ι₁(i, α) & (i, α) ∈ ⟦ b ⟧ \\
    ι₂(i, α) & (i, α) ∉ ⟦ b ⟧
   \end{cases}
   \) & \(
   \begin{cases}
    λ₁(ℓ) & \text{\(ℓ\) in \(e₁\)} \\
    λ₂(ℓ) & \text{\(ℓ\) in \(e₁\)} \\
    0 & \text{otherwise}
   \end{cases}
   \)\\
   %
   \(e₁ ⋅ e₂\) & \(S₁ + S₂\) & \(
   \begin{cases}
    δ₁(s)[ι₂] & s ∈ S₁ \\
    δ₂(s)      & s ∈ S₂
   \end{cases}
   \) & \( ι₁[ι₂](i, α) \) & \(
   \begin{cases}
    λ₁(ℓ)[ι₂] & \text{\(ℓ\) in \(e₁\)} \\
    λ₂(ℓ) & \text{\(ℓ\) in \(e₁\)} \\
    0 & \text{otherwise}
   \end{cases}
   \) \\
   %
   \({e₁}^{(b)}\)
   & \(S_{₁}\)
   & \(⌊δ₁(s)[{ι₁}ᵇ]⌋\)
   & \(⌊{ι₁}ᵇ⌋(i, α)\)
   & \( ⌊ λ(ℓ)[{ι₁}ᵇ] ⌋ \) \\
  \end{tabular}
  %FIXME: CZ: I don't feel like we need a new notation 0
  \caption{Thompson's construction for CF-GKAT. Here,
  \(!\) denotes the unique function \(!: ∅ → X\) for any $X$, and \(\const(x)\) denotes the constant function that always return \(x\)}
  \label{tab: thompson's construction}
\end{table}
\subsection{Algorithm and Complexity}

With the definitions of lowering and Thompson's construction established, the decision procedure mostly follows.
Nevertheless, it remains essential to define the alphabet: \((K, B, I, L)\), representing the set of primitive actions, primitive tests, indicator values, and labels, respectively. 
We may safely restrict primitive actions, primitive tests, and labels to those explicitly present in the expression, as expanding the alphabet beyond these will preserve trace semantics. This trace perseverance can be validated through induction on the expression itself.

Indicator variables, however, exhibit unique behavior within the alphabet. If the initial indicator value is absent from the program, the program's traces may diverge from traces starting from the present indicator values.
\Cref{prog:loop-head-iter-example} is one of the witness of this phenomenon: if the initial indicator value is 0, 1, or 2, the program terminates; however, if starting from an indicator value that doesn't appear in the program, then the program will loop indefinitely. 
An even simpler example is:
\[(x = 1); q,\]
where the program executes \(p\) if the initial indicator value is 1, but rejects for indicator values not present in the program.

Fortunately, given an expression \(e\), we can demonstrate that if neither \(i\) nor \(i'\) appears in \(e\), then 
\[∀ ℓ, C(e)ᵢ^{ℓ} = C(e)_{i'}^{ℓ}.\]
Therefore, when compiling the set of indicator values, it suffices to gather indicator values that appear explicitly in the program and augment this set with a special indicator \(*\) that does not appear in the program.

We summarize our decision procedure as follows:
\begin{enumerate}
  \item Given two CF-GKAT programs \(e, f\), we first collect their alphabet \((K, B, I, L)\). We gather the sets of primitive actions \(K\), primitive tests \(B\), and labels \(L\) that are present in either program \(e\) or \(f\). 
  Additionally, we identify indicator values, encompassing those found in either \(e\) or \(f\), along with an additional indicator \(*\) that is exclusive to neither program.
  \item We then proceed to compute Thompson's automata of \(e\) and \(f\) and convert them into CF-GKAT automata, denoted as \(A_e\) and \(A_f\). 
  It is noteworthy that these automata preserve the continuation semantics (\cref{the:thompson-correctness}). Formally,
  \begin{mathpar}
    C(A_e) = C(e) \and C(A_f) = C(f)
  \end{mathpar}
  \item Subsequently, we lower both CF-GKAT automata \(A_e\) and \(A_f\) to GKAT automata \(A_e\!↓ᵢ\) and \(A_f\!↓ᵢ\) for each \(i ∈ I\). 
  According to~\cref{the:cf-gkat-automaton-lowering-correctness}, these GKAT automata exhibits the same traces as \(e\) and \(f\) starting from \(i\):
  \begin{mathpar}
    G(A_e \!↓ᵢ) = C(A_e)\!↓ᵢ = C(e)\!↓ᵢ, \and 
    G(A_f \!↓ᵢ) = C(A_f)\!↓ᵢ = C(f)\!↓ᵢ.
  \end{mathpar}
  \item Finally, run the equivalence algorithm for GKAT automata~\cite{Smolka_Foster_Hsu_Kappé_Kozen_Silva_2020} on \(A_e \!↓ᵢ\) and \(A_f \!↓ᵢ\) for each \(i ∈ I\). 
  The current algorithm will return true, when all the GKAT automata equivalence checks return true.
\end{enumerate}

The soundness and completeness of this algorithm now follow as a corollary of the corresponding properties for the decision procedure in GKAT.  
We denote the algorithm introduced above as \(\mathrm{equiv}_{\CFGKAT}\), while the decision algorithm for GKAT automata is denoted as \(\mathrm{equiv}_{\GKAT}\). Thus, we establish the equivalence:
\begin{align*}
  \mathrm{equiv}_{\CFGKAT}(e, f) 
  & ⟺ ∀ i ∈ I, \mathrm{equiv}_{\GKAT}(A_e \!↓ᵢ, A_f \!↓ᵢ)  \\
  & ⟺ ∀ i ∈ I, G(A_e \!↓ᵢ) = G(A_f \!↓ᵢ) \\
  & ⟺ ∀ i ∈ I, C(e)\!↓ᵢ = C(f)\!↓ᵢ   
  ⟺ C(e)\!↓ = C(f)\!↓.
\end{align*}
This equivalence ensures that the algorithm \(\mathrm{equiv}_{\GKAT}\) not only correctly determines whether \(e\) and \(f\) have the same trace semantics; but also guarantees that if \(e\) and \(f\) are trace equivalent, then \(\mathrm{equiv}_{\GKAT}(e, f)\) will return true.

The complexity analysis of this algorithm is straightforward. Starting with an expression \(e\), we can observe through induction that the number of states in the Thompson's automaton \(A_e\) is bounded by \(|e|\), where the size \(|e|\) represents the number of primitive actions in \(e\).
After the lowering process, each GKAT automaton \(A_e\!↓\) contains at most \(|I|×|e|\) states. 
Notably, for each \(i ∈ I\), determining the equivalence between \(A_e\!↓ᵢ\) and \(A_f\!↓ᵢ\) is accomplished in nearly linear time relative to their states (assuming a constant number of primitive tests)~\cite{Smolka_Foster_Hsu_Kappé_Kozen_Silva_2020}, which is bounded by \(|I|×(|e|+|f|)\).
To verify the trace equivalence between \(e\) and \(f\), equivalence checks are required for \(A_e\!↓ᵢ\) and \(A_f\!↓ᵢ\) across all \(i∈I\). 
Therefore, the overall time complexity is nearly linear with respect to \(|I|^2 × (|e|+|f|)\). 
This implies that the algorithm's complexity scales nearly linearly with the sizes of \(e\) and \(f\), but nearly quadratically with respect to the number of the indicator values \(|I|\).
\cleardoublepage

\chapter{}
\label{chapter:}
\thispagestyle{myheadings}

\cleardoublepage

%\appendix
\begin{appendices}
\chapter{Proof of xyz}
\label{appendix}
\thispagestyle{myheadings}

This is the appendix.
\end{appendices}
%==========================================================================%
% Bibliography
\newpage
\printbibliography
\singlespace

% each subdirectory can have its own BiBTeX file
\printbibheading
\cleardoublepage

%==========================================================================%
% Curriculum Vitae
\phantomsection\addcontentsline{toc}{chapter}{Curriculum Vitae}
\hypersetup{ urlcolor=black,linkcolor=black }

\begin{center}
{\LARGE {\bf CURRICULUM VITAE}}\\
\vspace{0.5in}
{\large {\bf Cheng Zhang}}
\end{center}

\section*{Education}

\begin{itemize}
    \item 
    2018 --- 2024, Computer Science, Doctor of Philosophy, Boston University, Boston, MA

    Primary Interest: Kleene Algebra and its application in program verification\\
    I am broadly interested in application of mathematics in computer science,
    especially in programming languages.
    I have worked on Kleene algebra, automata thoery, semantics, type systems, and their application in program verification.

    \item 2014 --- 2018, Mathematics, Bachelor of Art, \emph{with department honor, magna cum laude}, Wheaton College, Norton, MA

    Honor Thesis in Graph Theory: \href{http://hdl.handle.net/11040/24570}{King in Generalized Tournaments}.\\
    Minors: Computer Science, Economics.

    \item
    2016 --- 2017, Economics, Study Aboard, London School Of Economics, London, United Kingdom
\end{itemize}

\section*{Preprints And Drafts}

\begin{itemize}
    \item 
    Cheng Zhang, Hang Ji, Ines Santacruz, Marco Gaboardi, 
    \emph{A Symbolic Decision Procedure For GKAT, and its Complexity}, Draft

    \item 
    Cheng Zhang, Tobias Kappé, David E. Narváez, Nico Naus,
    \emph{CF-GKAT: Control Flow Verification in Nearly Linear Time}, Draft

    \item 
    Arthur Azevedo de Amorim, Cheng Zhang, Marco Gaboardi,
    \emph{\href{https://hal.science/hal-04534715/}{Kleene algebra with commutativity conditions is undecidable}}, Preprint
\end{itemize}

\section*{Publications}

\begin{itemize}
    \item 
    Cheng Zhang, Arthur Azevedo de Amorim, Marco Gaboardi,
    \emph{\href{https://arxiv.org/abs/2404.18417}{Domain Reasoning In TopKAT}}, ICALP 2024
    
    \item 
    Mark Lemay, Qiancheng Fu, William Blair, Cheng Zhang, Hongwei Xi,
    \emph{\href{https://doi.org/10.1145/3609027.3609407}{A Dependently Typed Language with Dynamic Equality}}, TyDe 2023

    \item 
    Cheng Zhang, Arthur Azevedo de Amorim, Marco Gaboardi,
    \emph{\href{https://arxiv.org/abs/2108.07707}{On Incorrectness Logic and Kleene Algebra With Top and Tests}}, POPL 2022

    \item 
    Cheng Zhang, \emph{\href{http://hdl.handle.net/11040/24570}{King in Generalized Tournaments}}, Undergraduate Thesis 

    \item 
    Cheng Zhang, Weiqi Feng, Emma Steffens, Alvaro de Landaluce, Scott Kleinman, Mark D. LeBlanc
    \emph{\href{https://dl.acm.org/doi/10.5555/3205191.3205205}{Lexos 2017: Building Reliable Software in Python}}, JCSC 2018
\end{itemize}


\section*{Honors And Fellowships}

\begin{itemize}
    \item 2020, Meta PhD Research Fellowship Finalist in Programminag Languages
    \item 2018 --- now, Phi Beta Kappa Honor Society Member.
    \item 2018, Boston University Dean's Fellowship.
    \item 2018, Phi Beta Kappa Graduate Scholarship.
    \item 2018, Fred Kollett Prize in Mathematics \& Computer Science.
    \item 2018, Madeleine F. Clark Wallace Mathematics Prize.
    \item 2017, Weaton College Faculty-Student Research Awards.
    \item 2016, Wheaton Fellows.
    \item 2014 --- 2018, Wheaton College International Scholarship.
    \item 2014 --- 2018, Wheaton College Dean's Lists.
\end{itemize}


% \section{Publications}

% \cventry{2024}
% {Cheng Zhang, Arthur Azevedo de Amorim, Marco Gaboardi}
% {\href{https://arxiv.org/abs/2404.18417}{Domain Reasoning In TopKAT}}
% {International Colloquium on Automata, Languages and Programming (ICALP)}
% {}{}

% \cventry{2023}
% {Mark Lemay, Qiancheng Fu, William Blair, Cheng Zhang, Hongwei Xi}
% {\href{https://doi.org/10.1145/3609027.3609407}{A Dependently Typed Language with Dynamic Equality}}
% {The workshop on Type-Driven Development (TyDe)}
% {}{}

% \cventry{2022}
% {Cheng Zhang, Arthur Azevedo de Amorim, Marco Gaboardi}
% {\href{https://arxiv.org/abs/2108.07707}{On Incorrectness Logic and Kleene Algebra With Top and Tests}}
% {Principle Of Programming Language (POPL)}
% {}{}

% \cventry{2020}
% {Mark Lemay, Cheng Zhang, William Blair}
% {\href{https://icfp20.sigplan.org/details/tyde-2020-papers/7/Developing-a-Dependently-Typed-Language-with-Runtime-Proof-Search-Extended-Abstract-}
% {Developing a Dependently Typed Language with Runtime Proof Search (Extended Abstract)}}
% {Workshop on Type-Driven Development (TyDe)}
% {}{}

% \cventry{2018}
% {Cheng Zhang}
% {\href{http://hdl.handle.net/11040/24570}{King in Generalized Tournaments}}
% {Wheaton College Honor Thesis}
% {}{}

% \cventry{2018}
% {Cheng Zhang, Weiqi Feng, Emma Steffens, Alvaro de Landaluce, Scott Kleinman, Mark D. LeBlanc}
% {\href{https://dl.acm.org/doi/10.5555/3205191.3205205}{Lexos 2017: Building Reliable Software in Python}}
% {Journal of Computing Sciences in Colleges}
% {}{}


% \section{Research Talks}

% \cventry{2023}
% {Cheng Zhang}
% {GKAT with Indicator Variables, Fast Decompilation Verification}
% {BU Principles of Programming and Verification (POPV) Seminar}
% {}{}

% \cventry{2023}
% {Cheng Zhang}
% {A Practical Tutorial to KAT and its Extensions}
% {Systems Software Research Group at Virginia Tech}
% {}{}

% \cventry{2022}
% {Cheng Zhang}
% {Kleene Algebra and Its Applications in Verification}
% {Boston Computation Club}
% {}{}

% \cventry{2022}
% {Cheng Zhang}
% {On Incorrectness Logic Kleene Algebra With Test}
% {Cornell Programming Language Discussion Group (PLDG)}
% {}{}

% \cventry{2022}
% {Cheng Zhang}
% {On Incorrectness Logic Kleene Algebra With Test}
% {Principle Of Programming Languages (POPL)}
% {}{}

% \cventry{2018}
% {Cheng Zhang, Mark D. LeBlanc}
% {Lexos 2017: Building Reliable Software in Python}
% {Journal of Computing Sciences in Colleges}
% {}{}

% \cventry{2018}
% {Cheng Zhang}
% {Kings in Quasi-transitive Oriented Graph}
% {Wheaton Summit For Woman In STEM}
% {}{}



% \section{Honors And Fellowships}
% \cvitem{2022}{Meta PhD Research Fellowship Finalist in Programminag Languages}
% \cvitem{2018 --- Now} {Phi Beta Kappa Honor Society Member.}
% \cvitem{2018}{Boston University Dean's Fellowship.}
% \cvitem{2018}{Phi Beta Kappa Graduate Scholarship.}
% \cvitem{2018} {Madeleine F. Clark Wallace Mathematics Prize.}
% \cvitem{2018}{Fred Kollett Prize in Mathematics \& Computer Science.}
% \cvitem{2017}{Weaton College Faculty-Student Research Awards.}
% \cvitem{2016}{Wheaton Fellows.}
% \cvitem{2014 --- 2018}{Wheaton College International Scholarship.}
% \cvitem{2014 --- 2018}{Wheaton College Dean's Lists.}
 

%==========================================================================%
\end{document}
