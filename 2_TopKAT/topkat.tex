\chapter{Domain Reasoning In TopKAT}
\label{chapter:TopKAT}
\thispagestyle{myheadings}




\section{Reduction, A New Perspective}
\label{sec: general completeness}

Our goal in this section is to construct a complete interpretation for TopKAT,
by reducing its theory to that of plain KAT.  In other words, any equation
between two TopKAT terms is logically equivalent to another equation between a
pair of corresponding KAT terms.  While this result is not
new~\cite{Zhang_de_Amorim_Gaboardi_2022, Zhang_de_Amorim_Gaboardi_2022_POPL,
  Pous_Wagemaker_2022}, we present a more streamlined proof that hinges on the
universal properties of free KATs and TopKATs, without relying explicitly on
language models.  Similar to previous works, we obtain the
decidability of the equational theory of TopKAT as a corollary of reduction.
However, because of the new notion of reduction,
our decidability result no longer depends on the completeness of the language TopKAT.  
Moreover, our technique helps us to construct complete models and interpretations 
simply by computation, as well as simplifying proofs of other results about TopKAT.

% AAA: It would be nice if we could mention explicitly how this helps the next
% section.

\subsection{Reduction on free models}\label{sec: reduction on free models}

We first note that any free KAT over an alphabet \(K, B\) is also a TopKAT,
where the largest element is \((∑ K)^*\). This fact can be seen by
straightforward induction.

\begin{lemma}\label{the: every free KAT is a TopKAT}
    Every free KAT over alphabet \(K, B\) forms a TopKAT.
\end{lemma}

\begin{proof}
    Since \(\KAT_{K, B}\) is a KAT, we only need to show 
    the term \((∑ K)^*\) is the largest element of \(\KAT_{K, B}\),
    i.e. \[(∑ K)^* ≥ t, ∀ t ∈ \KAT_{K, B}.\] 
    The above fact can be shown by induction on \(t\);
    some algebraic manipulations below use facts in~\Cref{the: well known fact about KA}:
    \begin{itemize}
        \item \((∑ K)^* ≥ 1\) (by unfolding rule),
              thus \((∑ K)^*\) is larger than \(0, 1\) and every Boolean term.
        \item \((∑ K)^*\) is larger than \(∑ K\),
              which is larger than every primitive action.
        \item Given two terms \(t₁\) and \(t₂\),
              assume \((∑ K)^*\) is larger than both.
              Because \((∑ K)^* = (∑ K)^* + (∑ K)^*\)
              and addition preserves order,
              \[(∑ K)^* = (∑ K)^* + (∑ K)^* ≥ t₁ + t₂\] 
        \item Given two terms \(t₁\) and \(t₂\),
              assume \((∑ K)^*\) is larger than both.
              Because \((∑ K)^* = (∑ K)^* ⋅ (∑ K)^*\)
              and multiplication preserves order, 
              \[(∑ K)^* = (∑ K)^* ⋅ (∑ K)^* ≥ t₁ ⋅ t₂.\]
        \item Given a term \(t\),
              if \((∑ K)^* ≥ t\), then \((∑ K)^* ≥ t^*\).
              Since \((∑ K)^* = ((∑ K)^*)^*\) and star preserves order:
              \[(∑ K)^* = ((∑ K)^*)^* ≥ t^*. \qedhere\]
    \end{itemize}
\end{proof}

Since every free KAT is a TopKAT, every KAT interpretation
\(I : \KAT → \mathcal{K}\) induces a sub-KAT $\Img(I) ⊆ 𝒦$,
and this sub-KAT happens to be a \emph{TopKAT}. Specifically, the image of $(∑ K)^*$
in $𝒦$ is the largest element of $\Img(I)$, and the restricted
$I : \KAT → \Img(I)$ is a TopKAT homomorphism.

This gives us a powerful tool to construct complete TopKAT interpretations.
Since we already know that the KAT interpretations \(G: \KAT → 𝒢\) and
\(h ∘ G: \KAT → \Img(h)\) are injective TopKAT homomorphisms, we can
construct complete TopKAT interpretations by \emph{composition}, 
if we can construct an injective TopKAT interpretation \(r\) of type
\(\TopKAT_{K, B} → \KAT_{K_⊤, B}\):
\[\TopKAT_{K, B} \xrightarrow{r} \KAT_{K_⊤, B} \xrightarrow{G} 𝒢_{K_⊤, B},\\  
  \TopKAT_{K, B} \xrightarrow{r} \KAT_{K_⊤, B} \xrightarrow{G} 𝒢_{K_⊤, B}
  \xrightarrow{h} \Img(h).\] 

In fact, such an injective homomorphism can be obtained by lifting 
the embedding map \(K + B ↪ \KAT_{K_⊤, B}\):
\begin{align*}
    r   & : K + B → \KAT_{K_⊤, B}             \\
    r(p) & ≜ p.                   
\end{align*}
This homomorphism coincides with the \emph{reduction maps} of the same name in
previous works~\cite{Zhang_de_Amorim_Gaboardi_2022, Pous_Wagemaker_2023}.  More
concretely, we can picture $r$ as simply replacing the symbol \(⊤\) in a TopKAT
term with \((∑ K_⊤)^*\), the largest element in \(\KAT_{K_⊤, B}\).

We will show that \(r\) is injective by constructing a left inverse for it.  
In fact, the left inverse \([-]_⊤\) simply interprets the \(⊤\) primitive in \(\KAT_{K_⊤, B}\)
as the largest element.
\begin{lemma}\label{the: equivalence class is the inverse of reduction}
  The map \([-]_⊤: \KAT_{K_⊤, B} → \TopKAT_{K, B}\), where each term is
  mapped to its corresponding equivalence class, 
  is defined by lifting the following action on the primitives:
  \begin{align*}
    [p]_⊤ & ≜ p & p ∈ K + B \\  
    [⊤]_⊤ & ≜ ⊤.
  \end{align*}
  The map \([-]_⊤\) is a TopKAT homomorphism.
\end{lemma}

% AAA: I think it would be more elegant to define this map using the universal
% property of \KAT :) -- that is, by mapping ⊤ to ⊤.
% CZ: DONE.

\begin{proof}
  Because this map defined by lifting on the primitives,
  it is automatically a KAT homomorphism.
  All we need to show is that \([-]_⊤\) preserves the top element, that is
  \([(∑ K_⊤)^*]_⊤ = (∑ K_⊤)^*\) is the largest element in \(\TopKAT_{K, B}\).

  By construction of \(\TopKAT_{K, B}\), \(⊤\) is the largest element in \(\TopKAT_{K, B}\). 
  Thus, to prove that \((∑ K_⊤)^*\) is also the largest element in \(\TopKAT_{K, B}\),
  it suffices to prove \((∑ K_⊤)^* ≥ ⊤\): \[(∑ K_⊤)^* ≥ ∑ K_⊤ = ⊤ + ∑ K ≥ ⊤. \qedhere\]
\end{proof}

\begin{theorem}[Reduction]
    \([-]_⊤\) is the right inverse of \(r\): \([-]_⊤ ∘ r  = id_{\TopKAT_{K, B}}\).
    More explicitly for all \(t ∈ \TopKAT_{K, B}\): \[\TopKAT_{K, B} ⊧ [r(t)]_⊤ = t.\]
\end{theorem}

\begin{proof}
    Since \([-]_⊤ ∘ r : \TopKAT_{K, B} → \TopKAT_{K, B}\) is a TopKAT interpretation,
    the action on the primitives uniquely determines the interpretation:
    because both \(r\) and \([-]_⊤\) are identity on the primitives,
    therefore \([-]_⊤ ∘ r\) is the identity interpretation on \(\TopKAT_{K, B}\).
\end{proof}

The above theorem matches one of the soundness condition of reductions in 
previous works~\cite{Zhang_de_Amorim_Gaboardi_2022,Kozen_Smith_1997,Pous_Rot_Wagemaker_2021},
which was typically proven by a monolithic induction on the structure of terms.
Our approach, on the other hand, relies on establishing fine-grained 
algebraic properties, like~\cref{the: every free KAT is a TopKAT,the: equivalence class is the inverse of reduction};
then the theorem follows simply by computing the action of \([-]_⊤ ∘ r\) on primitives.

Since \(r\) has a right inverse, it is indeed the injective interpretation we desired, 
and it is also a complete interpretation:
\[\TopKAT_{K, B} ⊧ t₁ = t₂ ⟺ r(t₁) = r(t₂),\]
With the completeness of \(r\), we can already show the complexity of TopKAT.
The complexity results echos previous proofs~\cite{Zhang_de_Amorim_Gaboardi_2022,Pous_Wagemaker_2023},
but we are able to obtain this result without completeness of TopKAT language interpretation,
which is essential in previous proofs. 

\begin{corollary}[Complexity]\label{the: PSPACE-completeness of TopKAT}
  Given two terms \(t₁, t₂ ∈ \TopKAT_{K, B}\), deciding whether these two terms
  are equal is PSPACE-complete.
\end{corollary}

\begin{proof}
    Deciding KAT equality is a sub-problem of deciding TopKAT equality,
    and KAT equality is PSPACE-hard \cite{Cohen_Kozen_Smith_1999};
    therefore TopKAT equality is PSPACE-hard.

    To decide the equality of \(t₁, t₂\),
    we first remove all the redundant primitives that do not appear in \(t₁, t₂\)
    from the alphabet \(K, B\). Then we compute \(r(t₁)\) and \(r(t₂)\),
    each taking polynomial space (of \(|t₁| + |t₂|\)) to store;
    and we use the standard algorithm \cite{Cohen_Kozen_Smith_1999}
    to decide whether \(r(t₁) = r(t₂)\) in \(\KAT_{K_⊤, B}\),
    this will also take polynomial space.
    Hence, the decision procedure for TopKAT equality in PSPACE.

    Thus deciding TopKAT equality is PSPACE-complete.
\end{proof}


\subsection{Computing the complete interpretations}\label{sec: complete model for free}

Designing complete interpretations and models was not always easy.
In fact, in previous works \cite{Zhang_de_Amorim_Gaboardi_2022_POPL},
the authors made a mistake in the definition of language TopKAT,
which was fixed later \cite{Zhang_de_Amorim_Gaboardi_2022} 
by suggestion of Pous et al. \cite{Pous_Wagemaker_2022}.
However, with the results in~\Cref{sec: reduction on free models},
we can construct the complete interpretation just by composition,
and compute the complete model by computing the range of the complete interpretation.

We already know that there are two complete interpretations of TopKAT defined as follows:
\[\TopKAT_{K, B} \xrightarrow{r} \KAT_{K_⊤, B} \xrightarrow{G} 𝒢_{K_⊤, B},\\  
\TopKAT_{K, B} \xrightarrow{r} \KAT_{K_⊤, B} \xrightarrow{G} 𝒢_{K_⊤, B} \xrightarrow{h} \Img(h),\]
with a complete language model \(𝒢_{K_⊤, B}\), 
and a complete model consisting of relations \(\Img(h)\).

The operations in these models can be recovered by computing these maps.
For example, the multiplication operation in the language TopKAT can be computed as follows:
\[G ∘ r(t₁ ⋅ t₂) = G(r(t₁) ⋅ r(t₂)) = G(r(t₁)) ⋄ G(r(t₂)).\]
Since \(r\) does not change the multiplication operation,
the multiplication in the language TopKAT is the same as in language KAT.
In fact, as \(r\) does not change any operation in KAT,
most operations in language TopKAT are the same as language KAT.
Thus, we only need to compute the top element in language TopKAT.

The top element in language TopKAT can be computed in the same fashion:
\[G ∘ r(⊤) = G((∑ K_⊤)^*) = GS_{K_⊤, B},\]
i.e. the top element is just the complete language.

\begin{corollary}\label{the: language TopKAT for free}
    The language TopKAT inherits all the operations in language KAT,
    except the top element, which is defined as the full language.
    And such models are complete with \(G ∘ r\) as a complete interpretation.
\end{corollary}

In the same way, we know that complete models consisting of relations (a.k.a. general relational TopKAT) 
will have the same operations as relational KATs.
However, in this case the characterization of the computed top: \(h ∘ G ∘ r(⊤)\)
is not as simple as the full language,
but we know it is the largest relation in the range of \(h ∘ G ∘ r\):

\begin{corollary}\label{the: general relational TopKAT for free}
    The general relational TopKAT inherits all the operations in relational KAT,
    except the top element is the largest relation.
    And such models are complete with \(h ∘ G ∘ r\) as a complete interpretation.
\end{corollary}

Finally, to investigate whether we can use general relational TopKAT
to encode incorrectness logic,
we will provide a short proof that general relational TopKATs
are as expressive as relational KATs~\cite{Zhang_de_Amorim_Gaboardi_2022};
that is, every property on relations that can be encoded using general relational TopKAT,
is already encodable in the relational KAT.
Hence, adding a top element does not give extra expressive power in general relational TopKAT.

% AAA: We should recall what "expressible" means here, to make the paper
% self-contained.

The original proof~\cite[Lemma 2]{Zhang_de_Amorim_Gaboardi_2022} 
encodes every TopKAT term using a KAT term,
and then uses two pages to prove the soundness of this encoding.
Here we show the aforementioned encoding is simply the reduction \(r\).

\begin{definition}
    Given two terms \(t₁, t₂ ∈ \TopKAT\), and n primitives \(p₁, p₂, … , pₙ ∈ K + B\),
    we say that an n-ary predicate \(P\) is \emph{expressible} by 
    equation \(t₁ = t₂\) for a class of TopKATs \(\mathsf{K}\) 
    when for all interpretations \(I\) into TopKATs in \(\mathsf{K}\),
    the following equivalence holds:
    \[I(t₁) = I(t₂) ⟺ P(I(p₁), I(p₂), …, I(pₙ)).\]
\end{definition}

\begin{theorem}[Expressiveness of general relational TopKAT]\label{the: TopGREL expressive power}
    Given an alphabet \(K, B\), an n-ary predicate \(P\) on relations,
    the predicate \(P\) over primitives \(p₁, p₂, … , pₙ ∈ K\) is expressible in
    general relational TopKAT if and only if it is expressible in relational KAT.
\end{theorem}

\begin{proof}
    A predicate expressible in relational KAT is also expressible 
    in general relational TopKAT using the same pair of terms,
    we only need to show the converse.
    Assume a predicate \(P\) is expressible in general relational TopKAT,
    then there exists two TopKAT terms \(t₁, t₂ ∈ \TopKAT_{K, B}\) s.t. 
    for all general relational TopKAT interpretations \(I_⊤\):
    \[I_⊤(t₁) = I_⊤(t₂) ⟺ P(I_⊤(p₁), I_⊤(p₂), … , I_⊤(pₙ));\]

    We take an arbitrary relational KAT interpretation \(I\) from \(\KAT_{K_⊤, B}\).
    Notice \(\Img(I)\), the range of \(I\), 
    is a relational KAT with the largest element \(I((∑ K)^*)\),
    i.e. \(\Img(I)\) is a general relational TopKAT.
    Because \(I\) is a KAT interpretation, 
    it preserves all the KAT operations and the largest element.
    Hence, \(I\) is a TopKAT homomorphism from \(\KAT_{K_⊤, B}\) to \(\Img(I)\).

    Then we can construct \(I ∘ r: \TopKAT_{K, B} → \Img(I)\),
    a general relational interpretation:
    \begin{align*}
        I(r(t₁)) = I(r(t₂))
         & ⟺ I ∘ r(t₁) = I ∘ r(t₂)                           \\
         & ⟺ P(I ∘ r(p₁), … , I ∘ r(pₙ))
            & \text{\(I ∘ r\) is a \(\TopGREL\) interpretation} \\
         & ⟺ P(I(p₁), … , I(pₙ))
            & r(pᵢ) = pᵢ
    \end{align*}
    Thus the two KAT terms \(r(t₁), r(t₂) ∈ \KAT_{K_⊤, B}\) also can express the predicate \(P\).
\end{proof}

Since the image of \(I\) is not necessarily a relational TopKAT,
where the top element is interpreted as the complete relation,
the above trick does not work for relational TopKAT.
It is also known that relational TopKAT is strictly more expressive than general relational TopKAT,
since relational TopKAT can encode incorrectness logic,
where general relational TopKAT cannot~\cite{Zhang_de_Amorim_Gaboardi_2022}.


\section{(Co)domain Completeness}\label{sec: domain completeness of TopKAT}

In general, TopKAT is not complete over relational models, which are crucial for
applications in program logics~\cite{Zhang_de_Amorim_Gaboardi_2022}.  However, it
was later showed that we can obtain a complete theory for relational
models by simply adding the axiom \(p ⊤ p ≥ p\) to the theory of
TopKAT~\cite{Pous_Wagemaker_2023}. 

In this paper, we take a different approach than Pous et al.~\cite{Pous_Wagemaker_2023}:
instead of extending the TopKAT framework, we will restrict the completeness result.
In particular, the encoding of incorrectness logic and Hoare Logic in TopKAT~\cite{Zhang_de_Amorim_Gaboardi_2022}
relies only on the ability of TopKAT to compare the domain and codomain of two
relations.  This raises the question of whether TopKAT suffices for proving such
properties; that is, whether the following completeness results hold: for
\(t₁, t₂ ∈ \KAT_{K, B}\) (i.e. \(⊤\) does not appear in \(t₁\) and \(t₂\))
\begin{align*}
    \REL ⊧ \cod(t₁) ≥ \cod(t₂) & ⟺ \TopKAT ⊧ ⊤ t₁ ≥ ⊤ t₂ & \text{codomain completeness} \\
    \REL ⊧ \dom(t₁) ≥ \dom(t₂) & ⟺ \TopKAT ⊧ t₁ ⊤ ≥ t₂ ⊤ & \text{domain complete}
\end{align*}

In this section, we prove that these equivalences hold, even without the additional axiom.
However, they do \emph{not} hold if we allow terms that contain top.
For example, let \(t₁ ≜ p ⊤ p\), and \(t₂ ≜ p\). Since \(p ⊤ p ≥ p\) holds in
relational TopKAT, thus \(\dom(p ⊤ p) ≥ \dom(p)\). 
However, \(p ⊤ p ⊤ ≥ p ⊤\) is not provable in TopKAT, 
because the inequality is not valid with the language interpretation.
The incompleteness of codomain comparison can also be shown using the same example.

\subsection{Codomain completeness}

The core insight to prove the domain completeness result is 
to construct a specific relational interpretation \(h ∘ i ∘ G\),
where its codomain is equivalent to the complete TopKAT interpretation \(G ∘ r\):
\[\cod(h ∘ i ∘ G(t)) = G ∘ r(⊤ t),\]
where \(i\) is the natural inclusion homomorphism \(i: 𝒢_{K, B} ↪ 𝒢_{K_⊤, B}\), 
that maps every language to itself;
and \(h\) is the classical embedding of language KAT into relational KAT~\cite{Kozen_Smith_1997},
which we will recall as follows:
\[h(L) = \{(s, s ⋄ s') ∣ s ∈ GS, s' ∈ L\}.\]
Although \(i\) will not change the outcome of \(G\),
it will add a new primitive action \(⊤\) to the alphabet, hence changing the outcome of \(h\).
Such addition will equate the codomain of \(h ∘ i ∘ G(t)\) 
with the complete TopKAT interpretation \(G ∘ r\) of \(⊤ t\).
The proof of this equality is by simply computing both sides of the equation.

\begin{lemma}\label{the: codomain completeness core lemma}
    For any term \(t ∈ \KAT_{K, B}\),
    \[\cod(h ∘ i ∘ G(t)) = G ∘ r(⊤ t).\]
\end{lemma}

\begin{proof}
    We explicitly write out the domain and codomain of the functions in
    the relational KAT interpretation \(h ∘ i ∘ G\) for the ease of the reader:
    \[\KAT_{K, B}
        \xrightarrow{G} 𝒢_{K, B}
        \xrightarrow{i} 𝒢_{K_⊤, B}
        \xrightarrow{h} 𝒫(𝒢_{K_⊤, B} × 𝒢_{K_⊤, B}).\]
    In this case, \(h\) is a KAT homomorphism from \(𝒢_{K_⊤, B}\):
    \[h(S) = \{(s, s ⋄ s₁) ∣ s ∈ GS_{K_⊤, B}, s₁ ∈ S\}.\]
    Since the reduction \(r\) preserves terms without \(⊤\),
    let \(t ∈ \KAT_{K, B}\) (i.e. \(t\) does not contain \(⊤\)),
    \[G ∘ r(⊤) = GS_{K_⊤, B} \\ G ∘ r(t) = G(t).\]
    Therefore, for any term \(t ∈ \KAT_{K, B}\)
    \begin{align*}
        \cod(h ∘ i ∘ G(t))
         & = \{s α s₁ ∣ s α ∈ GS_{K_⊤, B}, α s₁ ∈ G(t)\} \\
         & = GS_{K_⊤, B} ⋄ G(t)                          \\
         & = (G ∘ r(⊤)) ⋄ (G ∘ r(t))                     \\
         & = G ∘ r(⊤ t). \qedhere
    \end{align*}
\end{proof}

\Cref{the: codomain completeness core lemma} established a connection between 
the codomain operator and the language interpretation of TopKAT.
Then by completeness of the language interpretation, 
we will obtain the completeness of codomain comparison.

\begin{theorem}[Codomain completeness]\label{the: codomain completeness}
    Given two terms \(t₁, t₂ ∈ \KAT_{K, B}\) (i.e. terms without \(⊤\)),
    then codomain comparison is complete:
    \begin{align*}
        \REL ⊧ \cod(t₁) ≥ \cod(t₂) & ⟺ \TopKAT ⊧ ⊤ t₁ ≥ ⊤ t₂.
    \end{align*}
\end{theorem}

\begin{proof}
    Given the natural inclusion homomorphism: \(i: \KAT_{K, B} → \KAT_{K_⊤, B}\),
    we show that the following are equivalent:
    \begin{enumerate}
        \item \(\REL ⊧ \cod(t₁) ≥ \cod(t₂).\)
        \item \(\cod(h ∘ i ∘ G(t₁)) ≥ \cod(h ∘ i ∘ G(t₂)).\)
        \item \(\TopKAT ⊧ ⊤ t₁ ≥ ⊤ t₂.\)
    \end{enumerate}

    We first show that \(1 ⟹ 2\), by definition, \(\REL ⊧ \cod(t₁) ≥ \cod(t₂)\)
    implies \(\cod(I(t₁)) ≥ \cod(I(t₂))\) for all relational KAT interpretations \(I\).
    Because \(h ∘ i ∘ G\) is a relational KAT interpretation, so \(1 ⟹ 2\).

    We show \(2 ⟹ 3\), which uses the equality discussed above, 
    and proved in~\Cref{the: codomain completeness core lemma}:
    \begin{align*}
             & \cod(h ∘ i ∘ G(t₁)) ≥ \cod(h ∘ i ∘ G(t₂))           \\
        ⟺ {} & G ∘ r(⊤ t₁) ≥ G ∘ r(⊤ t₂)
             & \text{\Cref{the: codomain completeness core lemma}} \\
        ⟺ {} & \TopKAT ⊧ ⊤ t₁ ≥ ⊤ t₂.
             & \text{Completeness of \(G ∘ r\)}
    \end{align*}

    Finally, we show \(3 ⟹ 1\), by \Cref{the: top element represent domain}:
    \[\TopKAT ⊧ ⊤ t₁ ≥ ⊤ t₂ ⟹ \TopREL ⊧ ⊤ t₁ ≥ ⊤ t₂ ⟹ \REL ⊧ \cod(t₁) ≥ \cod(t₂). \qedhere\]
\end{proof}

\subsection{Domain completeness}

The domain completeness result can be derived from codomain completeness 
by observing properties of opposite TopKAT and the converse operator \((-)^{∨}\), 
both of which we will recall below.

For every TopKAT \(𝒦\), we can construct the opposite TopKAT \(𝒦^{\op}\) 
by reversing the multiplication operation, keeping the sorts and other operations unchanged:
\[p \mathbin{\hat{⋅}} q ≜ q ⋅ p,\]
where \(\hat{⋅}\) is multiplication in \(𝒦^{\op}\) and \(⋅\) is multiplication in \(𝒦\).
By definition, \((-)^{\op}\) is a involution, that is \({(𝒦^{\op})}^{\op} = 𝒦\).
Furthermore, \((-)^{\op}\) is a TopKAT functor,
this means all TopKAT homomorphisms \(h: 𝒦 → 𝒦'\) 
can be lifted to a TopKAT homomorphism on the opposite TopKAT \(h^{\op}: 𝒦^{\op} → {𝒦'}^{\op}\). 
The lifting \(h^{\op}\) is point-wise equal to \(h\):
\[∀ p ∈ 𝒦, h^{\op}(p) ≜ h(p).\]
The fact that \(h^{\op}\) is a TopKAT homomorphism can be proven by unfolding the definition,
and the functor laws are satisfied because \(h^{\op}\) is point-wise equal to \(h\).

There are two important homomorphisms involving opposite TopKAT:
\begin{align*}
    (-)^{∨} & : (X × X)^{\op} → (X × X) &
    \op & : \TopKAT_{K, B} → \TopKAT^{\op}_{K, B} \\  
    (R)^{∨} & = \{(b, a) ∣ (a, b) ∈ R\}, & 
    ∀ p ∈ K + B, \op & (p) = p.
\end{align*}
The \((-)^{∨}\) is the relational converse operator, 
the rules of homomorphism can simply be proven by unfolding of definitions.
The crucial property of \((-)^{∨}\) is that it flips the domain and codomain:
\begin{equation}\label{the: converse flips domain to codomain}
    \dom(R^{∨}) = \cod(R).
\end{equation}
Hence, allowing us to flip the result about codomains and apply it to domains.

\(\op\) is a homomorphism from free TopKAT to its opposite TopKAT;
it can be defined by lifting the embedding function \(K + B ↪ \TopKAT_{K, B}\) on primitives.
Intuitively, given a term \(t ∈ \TopKAT\), 
\(\op(t)\) will flip all the multiplications in \(t\) recursively.
\begin{lemma}\label{the: injectivity of op}
    the left inverse of \(op\) can be obtained by lifting itself through the \((-)^{\op}\) functor,
    \[\op^{\op}: \TopKAT^{\op} → (\TopKAT^{\op})^{\op} = \TopKAT.\]
    Recall \(\op^{\op}\) is pointwise equal to \(\op\), 
    thus \(\op^{\op} ∘ \op: \TopKAT → \TopKAT\) is the identity interpretation 
    because it preserves all the primitives.
    Thus, \(\op\) has a left inverse, hence it is injective:
    \[t₁ = t₂ ⟺ \op(t₁) = \op(t₂).\]
\end{lemma}

Finally, since the elements in \(\TopKAT^{\op}\) are the same as \(\TopKAT\), 
which are TopKAT terms modulo provable TopKAT equalities,
theorems about TopKAT terms are also true for elements in \(\TopKAT^{\op}\).
In particular, codomain completeness (\Cref{the: codomain completeness})
also holds in \(\TopKAT^{\op}\): 
for all terms \(t₁, t₂ ∈ \TopKAT\),
\begin{equation}\label[equiv]{the: op codomain completeness}
    ⊤ ⋅ \op(t₁) ≥ ⊤ ⋅ \op(t₂) ⟺ \REL ⊧ \cod(\op(t₁)) = \cod(\op(t₂)).
\end{equation}

\begin{theorem}[Domain Completeness]\label{the: domain completeness}
    For all terms \(t₁, t₂ ∈ \KAT\), the following equivalence hold:
    \[\REL ⊧ \dom(t₁) = \dom(t₂) ⟺ \TopKAT ⊧ t₁ ⊤ ≥ t₂ ⊤.\]
\end{theorem}

\begin{proof}
    \(⟸\) direction is trivial by \Cref{the: top element represent domain};  
    and \(⟹\) direction can be derived as follows:
    let \(I\) be some relational interpretation,
    then \(I^{\op}(\op(-))^∨\) is also a relational interpretation:
    \[I^{\op}(\op(-))^∨: 
        \TopKAT \xrightarrow{\op} \TopKAT^{\op} \xrightarrow{I^{\op}} 
        (X × X)^{\op} \xrightarrow{(-)^{∨}} (X × X).\]
    Thus, we let \(I\) range over all relational interpretations:
    \begin{align*}
        & \REL ⊧ \dom(t₁) ⊇ \dom(t₂)  \\
        & ⟹ ∀ I, \dom(I(t₁)) ⊇ \dom(I(t₂)) \\
        & ⟹ ∀ I, \dom(I^{\op}(\op(t₁))^∨) ⊇ \dom(I^{\op}(\op(t₂))^∨) 
            &\text{specialize \(I\) as \(I^{\op}(\op(-))^∨\)}\\  
        & ⟹ ∀ I, \cod(I^{\op}(\op(t₁))) ⊇ \cod(I^{\op}(\op(t₁))) 
            &\text{\Cref{the: converse flips domain to codomain}}\\
        & ⟹ ∀ I, \cod(I(\op(t₁))) ⊇ \cod(I(\op(t₁))) 
            &\text{\(I^{\op}\) is pointwise equal to \(I\)}\\
        & ⟹ ⊤ ⋅ \op(t₁) ≥ ⊤ ⋅ \op(t₂) 
            &\text{\Cref{the: op codomain completeness}}\\
        & ⟹ \op(⊤ ⋅ t₁) ≥ \op(⊤ ⋅ t₂) 
            & \text{Definition of \(\op\)}\\
        & ⟹ t₁ ⊤ ≥ t₂ ⊤ & \text{\Cref{the: injectivity of op}}
    \end{align*}
\end{proof}

\begin{remark}
    Alternatively, \Cref{the: domain completeness} can also be proven 
    by constructing the following \(h'\):
    \begin{align*}
        h' & : 𝒢_{K, B} → 𝒫(𝒢_{K, B} × 𝒢_{K, B})\\
        h' & (S₁) ≜ \{(s₁ α s, α s) ∣ s₁ α ∈ S₁, α s ∈ GS_{K, B}\}.
    \end{align*}
    Then the proof would mirror that of \Cref{the: codomain completeness},
    replacing \(h\) with \(h'\) and replacing \(\cod\) with \(\dom\).
    However, the proof of \Cref{the: domain completeness} reveals more properties
    of maps like \((-)^{∨}\) and \(\op\), 
    thus we choose to present the current proof of \Cref{the: domain completeness} 
    instead of the alternative proof.
\end{remark}



% \section{A Coalgebraic Theory}\label{sec: TopKCT}

% Coalgebraic theory is not only theoretically rich in structures~\cite{Kozen_Silva_2020,Silva_2010,ruttenUniversalCoalgebraTheory2000},
% but also produces decision procedures with great performance in real-world 
% applications~\cite{Foster_Kozen_Milano_Silva_Thompson_2015, Smolka_Kumar_Kahn_Foster_Hsu_Kozen_Silva_2019, Pous_2015}.
% In this section we will develop the coalgebraic theory of TopKAT, 
% named Kleene Coalgebra with Tests and Top (TopKCT).
% It should come as no surprise that the reduction interpretation \(r: \TopKAT_{K, B} → \KAT_{K, B}\) 
% is central in this endeavor: the reduction map helps us show both the completeness and decidability 
% by reducing the problem about TopKCT into KCT.


% \subsection{Kleene Coalgebra with Tests, Revisited}\label{sec: revisit KCT}

% Kozen's definition of derivative and empty word operation is inductive on terms,
% which means we cannot use the universality of interpretation to reason about equalities,
% and needs to resort to induction proofs.
% Fortunately, we can recreate Kozen's definition using the homomorphism \(I_{α p}\).
% We first define the the action of \(I_{α p}\) on primitives: 
% \begin{align*}
%     I_{α p} & : K + B → ℳ₂(\KATExp_{K, B})\\
%     I_{α p} & (b) ≜ \begin{bmatrix}
%         [α ≤ b] & 0 \\  
%         0 & b
%     \end{bmatrix} & b ∈ B \\
%     I_{α p} & (q) ≜ \begin{bmatrix}
%         0 & 0 \\  
%         0 & q
%     \end{bmatrix} & q ∈ K, q ≠ p \\
%     I_{α p} & (p) ≜ \begin{bmatrix}
%         0 & 1 \\  
%         0 & p
%     \end{bmatrix},
% \end{align*}
% where \([α ≤ b]\) will be \(1\) if \(α ≤ b\) is true, and \(0\) otherwise.

% Indeed such action can be lifted to a function on expressions and a KAT homomorphism
% using lifting by matrix operations:
% \begin{lemma}\label{the: lifting of a fun from term to matrices of terms}
%     A function \(f: K + B → ℳₙ(\KATExp_{K, B})\)
%     can be lifted to a function on terms \(f: \KATExp_{K, B} → ℳₙ(\KATExp_{K, B})\),
%     by construction similar to \labelcref{the: lifting to interpretation};
%     and a KAT homomorphism \(f: \KAT_{K, B} → ℳₙ(\KAT_{K, B})\),
%     by lifting \(K + B \xrightarrow{f} ℳₙ(\KATExp_{K, B}) \xrightarrow{[-]_{\KAT}} ℳₙ(\KAT_{K, B})\).
%     The following diagram commutes by definition of both liftings:
%     \[
%         \begin{tikzcd}[column sep=1.4cm]
%             K + B \ar[swap]{rd}{f} \ar[hookrightarrow]{r}{i}
%                 & \KATExp_{K, B} \ar[dashed]{d}{f} \ar{r}{[-]_{\KAT}}
%                 & \KAT_{K, B} \ar[dashed]{d}{f}\\  
%             & ℳₙ(\KATExp_{K, B}) \ar{r}{ℳₙ([-]_{\KAT})} & ℳₙ(\KAT_{K, B})
%         \end{tikzcd}
%     \]
%     Similarly for TopKAT:
%     \[
%         \begin{tikzcd}[column sep=1.4cm]
%             K + B \ar[swap]{rd}{f} \ar[hookrightarrow]{r}{i}
%                 & \TopKATExp_{K, B} \ar[dashed]{d}{f} \ar{r}{[-]_{\TopKAT}}
%                 & \TopKAT_{K, B} \ar[dashed]{d}{f}\\  
%             & ℳₙ(\TopKATExp_{K, B}) \ar{r}{ℳₙ([-]_{\TopKAT})} & ℳₙ(\TopKAT_{K, B})
%         \end{tikzcd}
%     \]
% \end{lemma}
% In this case instead of lifting by the syntactical operations in terms and the free model,
% the operation we are lifting by is the matrix operation.
% We take the lifting of the star operation as an example:
% \[f(t) = \begin{bmatrix}
%     t₁ & t₂ \\  
%     0 & t₃
% \end{bmatrix} ⟹
% f(t^*) ≜ f(t)^* = \begin{bmatrix}
%     t₁ & t₂ \\  
%     0 & t₃
% \end{bmatrix}^* = 
% \begin{bmatrix}
%     t₁^* & t₁^* t₂ t₃^* \\  
%     0 & t₃^*
% \end{bmatrix}\]
% Thus we can extend \(I_{α p}\) to all KAT expressions,
% and this generates a function on terms \(I_{α p}: \KATExp_{K, B} → \KATExp_{K, B}\),
% and a homomorphism on the free model \(I_{α p}: \KAT_{K, B} → \KAT_{K, B}\).
% In fact the function \(I_{α p}\) on terms exactly corresponds to Kozen's definition 
% of operations in the KCT formed by KAT terms:
% \begin{theorem}
%    Let \(ϵ_{α}: \KATExp → 2\) and \(δ_{α p}: \KATExp → \KATExp\) be the definition of 
%    derivatives by Kozen~\cite{Kozen_2008}, and the following equation holds:
%    \[\begin{bmatrix}
%     ϵ_{α}(e) & δ_{α p}(e) \\  
%     0 & e
%    \end{bmatrix} = I_{α p}(e).\]
% \end{theorem}

% \begin{proof}
%     By induction on the structure of \(e\).
%     TODO
% \end{proof}

% We can similarly define \(ϵ_{α}\) and \(δ_{α p}\) on the free KAT 
% using the homomorphism version \(I_{α p}: \KAT → \KAT\):
% \[
%     ϵ_{α}(e) ≜ π_{1, 1}(I_{α p}(e)), \qquad δ_{α p}(e) ≜ π_{1, 2}(I_{α p}(e)).
% \]
% Then we can establish a obvious connection between the KCT formed by KAT expression and free KAT:
% \begin{corollary}
%     By~\Cref{the: lifting of a fun from term to matrices of terms},
%     the following diagram commutes
%     \[
%         \begin{tikzcd}[column sep=1.3cm]
%             \KATExp \ar{r}{[-]_{\KAT}} \ar{d}[swap]{I_{α p}}& \KAT \ar{d}{I_{α p}} \\ 
%             2 × \KATExp \ar[swap]{r}{(id, [-]_{\KAT})} & 2 × \KAT
%         \end{tikzcd}
%     \]
%     Since \(ϵ_{α}\) and \(δ_{α p}\) are components of \(I_{α p}\),
%     the following diagram commutes for all \(β ∈ \At\) and \(α p ∈ \At × K\):
%     \[
%         \begin{tikzcd}[column sep=1.3cm]
%             \KATExp \ar{r}{[-]_{\KAT}} \ar{d}[swap]{⟨ϵ_β, δ_{α p}⟩}& \KAT \ar{d}{⟨ϵ_β, δ_{α p}⟩} \\ 
%             2 × \KATExp \ar[swap]{r}{(id, [-]_{\KAT})} & 2 × \KAT
%         \end{tikzcd}
%     \]
%     Namely, \([-]_{KAT}\) is a KCT homomorphism.
% \end{corollary}
% % TODO: weaker theories are also a homomorphism... 
% % TODO: homomorphism preserves bisimulation...


% \subsection{Coalgebra From Reduction}

% Similar to the case of KCT, our definition of Kleene Coalgebra with Top and Tests (TopKCT) 
% is based on the structure of guarded language interpretation for TopKATs.
% Although the guarded language interpretation for TopKATs also results in sets of guarded strings,
% a fresh primitive \(⊤\) is added to the action alphabet, which is not the case in KAT.
% To handle the primitive \(⊤\),
% we need to extend the derivative operation \(δ\) to also consume the top symbol.

% \begin{definition}[TopKCT]
%     A TopKCT \(𝒮\) over \(K, B\) consists of two families of operations,
%     indexed by \(α ∈ \At\), \(α p ∈ \At × K_⊤\) and \(α ∈ \At\) respectively:
%     \[ϵ_α: 𝒮 → 2, \\  δ_{α p}: 𝒮 → 𝒮.\]
% \end{definition}

% Despite the slight differences in the definition,
% The connection between TopKCT and KCT is rather clear:
% \begin{corollary}
%     Every TopKCT over alphabet \(K, B\) exactly correspond to a KCT over alphabet \(K_⊤, B\);
%     since \(𝒢_{K_⊤, B}\) is the final KCT over \(K_⊤, B\), it is the final TopKCT over \(K, B\).
% \end{corollary}

% TODO: this definition is wrong, it doesn't handle Top,
% We need to restrict it by a matrix.
% Similar to the case of KAT, the TopKAT terms \(\TopKATExp\) and free TopKAT both 
% form a TopKCT. The homomorphism that defines such TopKCT can be lifted by the 
% same action on primitives as KCT:
% \begin{align*}
%     I_{α p} & : K_⊤ + B → ℳ₂(\TopKATExp_{K, B})ᵤ\\
%     I_{α p} & (b) ≜ \begin{bmatrix}
%         [α ≤ b] & 0 \\  
%         0 & b
%     \end{bmatrix} & b ∈ B \\
%     I_{α p} & (q) ≜ \begin{bmatrix}
%         0 & 0 \\  
%         0 & p
%     \end{bmatrix} & q ∈ K, q ≠ p \\
%     I_{α p} & (p) ≜ \begin{bmatrix}
%         0 & 1 \\  
%         0 & p
%     \end{bmatrix},
% \end{align*}
% And the coalgebraic operations in TopKAT terms and free TopKAT are defined similarly:
% \begin{align*}
%     \begin{bmatrix}
%         ϵ_α(e) & δ_α(e) \\  
%         0 & e'
%     \end{bmatrix} ≜ I_{α p}(e)
% \end{align*}
% This definition gives the TopKCT over TopKAT terms when 
% \(I_{α p}\) is a function between TopKAT terms \(\TopKATExp → \TopKATExp\),
% and gives the TopKCT over free TopKAT when 
% \(I_{α p}\) is a homomorphism between free TopKAT \(\TopKAT → \TopKAT\).
% And by \Cref{the: lifting of a fun from term to matrices of terms},
% the following diagram commutes:
% \[
%     \begin{tikzcd}
%         \TopKATExp \ar{d}{I_{α p}} \ar{r}{[-]_\TopKAT}
%             & \TopKAT \ar{d}{I_{α p}} \\  
%         ℳ₂(\TopKATExp) \ar{r}{[-]_\TopKAT}
%             & ℳ₂(\TopKAT)
%     \end{tikzcd}  
% \]
% Since \(δ()\)



% Similar to before, we can \emph{reduce} TopKCT over free TopKAT \(\TopKAT_{K, B}\) 
% to TopKCT over free KAT \(\KAT_{K_⊤, B}\):
% \begin{lemma}\label{the: reduction in coalgebra}
%     The following diagram commutes:
%     \[
%         \begin{tikzcd}
%             \TopKATExp_{K, B} \ar{r}{r} \ar{d}[swap]{⟨ϵ_β, δ_{α p}⟩} 
%                 & \KAT_{K_⊤,B} \ar{d}[swap]{⟨ϵ_β, δ_{α p}⟩} \\
%             2 × \TopKATExp_{K, B} \ar[swap]{r}{(id, r)} 
%                 & 2 × \KAT_{K_⊤,B}
%         \end{tikzcd}
%     \]
% \end{lemma}

% \begin{proof}
%     To prove the commutativity, notice the following equality holds,
%     since they have the same action on primitives:
%     \[r ∘ [-]_{⊤} ∘ I_{α p} ∘ r = I_{α p} ∘ r.\]
%     then recall for a term \(e ∈ \TopKAT_{K, B}\):
%     \[
%     \begin{bmatrix}
%         ϵ_α(e) & δ_{α p}(e) \\  
%         0 & e
%     \end{bmatrix} ≜ [-]_{⊤} ∘ I_{α p} ∘ r(e)
%     \]
%     We can unfold the above equality on any TopKAT term \(e\) as follows:
%     \[\begin{bmatrix}
%         r ∘ ϵ_α(e) & r ∘ δ_{α p}(e) \\  
%         0 & e
%     \end{bmatrix} = 
%     \begin{bmatrix}
%         ϵ_α ∘ r(e) & δ_{α p} ∘ r(e) \\  
%         0 & r(e)
%     \end{bmatrix} ⟹ r ∘ δ_{α p} = δ_{α p} ∘ r.\]
%     And similarly, for all \(β ∈ \At\), we take any \(p ∈ K\):
%     \[\begin{bmatrix}
%         r ∘ ϵ_β(e) & r ∘ δ_{β p}(e) \\  
%         0 & e
%     \end{bmatrix} = 
%     \begin{bmatrix}
%         ϵ_β ∘ r(e) & δ_{β p} ∘ r(e) \\  
%         0 & r(e)
%     \end{bmatrix} ⟹ r ∘ ϵ_β = ϵ_β ∘ r.\]
%     Finally because \(r\) is a homomorphism, hence preserves identities,
%     thus \(r ∘ ϵ_β(e) = id ∘ ϵ_β(e)\), 
%     thus we have obtained the commutativity result:
%     \[id ∘ ϵ_β = r ∘ ϵ_β = ϵ_β ∘ r \quad \text{and} \quad  
%     r ∘ δ_{α p} = δ_{α p} ∘ r. \qedhere\]
% \end{proof}

% Therefore the language interpretation of TopKAT is indeed final.

% \begin{theorem}\label{the: finality of guarded language interpretation}
%     The language interpretation of TopKAT is the unique 
%     coalgebra homomorphism from \(\TopKAT_{K,B}\) to \(𝒢_{K_⊤, B}\):
%     \[
%     \begin{tikzcd}
%         \TopKATExp \ar{r}{} &
%             \TopKAT_{K, B} \ar[dashed]{r}{G ∘ r} \ar{d}[swap]{⟨ϵ_β, δ_{α p}⟩} 
%             & 𝒢_{K_⊤, B} \ar{d}{⟨ϵ_β, δ_{α p}⟩} \\
%         2 × \TopKATExp
%             & 2 × \TopKAT_{K, B} \ar[swap, dashed]{r}{(id, G ∘ r)} 
%             & 2 × 𝒢_{K_⊤, B} 
%     \end{tikzcd}  
% \]
% \end{theorem}
% \begin{proof}
%     The uniqueness is trivial by the finality of the language model,
%     all we need to show is commutativity of the diagram.

%     We can split the diagram into two:
%     \[
%         \begin{tikzcd}
%             \TopKAT_{K, B} \ar{r}{r} \ar{d}[swap]{⟨ϵ_β, δ_{α p}⟩} 
%                 & \KAT_{K_⊤,B} \ar[dashed]{r}{G} \ar{d}[swap]{⟨ϵ_β, δ_{α p}⟩}
%                 & 𝒢_{K_⊤, B} \ar{d}{⟨ϵ_β, δ_{α p}⟩} \\
%             2 × \TopKAT_{K, B} \ar[swap]{r}{(id, r)} 
%                 & 2 × \KAT_{K_⊤,B} \ar[dashed]{r}{(id, G)}
%                 & 2 × 𝒢_{K_⊤, B} 
%         \end{tikzcd}
%     \]
%     The right rectangle commutes because of the finality of the 
%     language interpretation of KAT~\cite[Theorem 4.2, Section 4.1]{Kozen_2008};  
%     and the left rectangle commutes because \Cref{the: reduction in coalgebra}.
% \end{proof}

% The finality of guarded language interpretation (\Cref{the: finality of guarded language interpretation})
% leads directly to the completeness result of TopKCT, by standard theorems of universal coalgebra.

% \begin{theorem}[Completeness]
%     Two TopKAT term are bisimilar if and only if their equality is provable by TopKAT.
% \end{theorem}

% \begin{proof}
%     Similar to the case in KAT, finality of the guarded language interpretation of TopKAT
%     dictates that two TopKAT terms are bisimilar if and only if they have 
%     the same guarded language interpretations~\cite[Theorem 2.2.6, Theorem 2.2.7]{Silva_2010}.
%     And by completeness of the guarded language interpretation (\Cref{the: language TopKAT for free}),
%     two TopKAT terms are bisimilar if and only if they are equal in the theory of TopKAT.
%     In other word, the coalgebraic theory of TopKAT is complete with respect to both 
%     the algebraic theory of TopKAT and the guarded language interpretation.
% \end{proof}

% \subsection{Finiteness and Decidability}

% Kozen~\cite{Kozen_2008} defined the axioms of the algebraic structure \emph{right pre-semiring}, 
% which include the commutative monoid axiom with operation \(+\) 
% and identity \(0\), with the following axioms:
% \[1 ⋅ x = x, \quad 0 ⋅ x = x, \quad (x + y) ⋅ z = x ⋅ z + y ⋅ z.\]
% We denote the theory of right pre-semiring as \(\RP\), 
% and given a set of TopKAT or KAT terms \(E\), 
% we write \(E/\RP\) as the set modulo right pre-semiring equalities,
% where the identities and addition and multiplication operation of right pre-semiring
% coincide with that of KAT or TopKAT;  
% and we use \([e]_{\RP}\) to denote the equivalence class of (Top)KAT 
% expression \(e\) under right pre-semiring equalities.
% Specifically, \(\TopKATExp_{K, B}/\RP\) form a pre-semiring,
% with operations defined syntactically.

% Kozen~\cite{Kozen_2008} has also shown that 
% the set \(\{δ_{w}(e) ∣ w ∈ (\At ⋅ K)^*\}\) modulo is finite for 
% all KAT expression \(e ∈ \KAT_{K, B}\) over any alphabet \(K, B\).
% This result guarantees that every state in the coalgebra generated by the free KAT 
% has only finitely many reachable states,
% hence the bisimulation of every pair of KAT terms is decidable.

% We need show a similar result for TopKAT, namely \(\{δ_{w}(e) ∣ w ∈ (\At ⋅ K)^*\}\)
% modulo right pre-semiring is finite for all TopKAT expression \(e ∈ \TopKAT_{K, B}\) 
% over any alphabet \(K, B\).

% Notice the homomorphism \(r: \TopKAT_{K, B} → \KAT_{K_⊤, B}\) 
% is generated by a action on the primitives \(K + B\),
% therefore it will give a function between terms
% \(r: \TopKATExp_{K, B} → \KATExp_{K_⊤, B}\).
% Furthermore, since \(r\) preserves the syntactical addition, multiplication, and identities,
% \(r\) is a right semiring homomorphism:
% \[r: \TopKATExp_{K, B}/\RP → \KATExp_{K_⊤, B}/\RP.\]
% And we recall the action of \(r\) is simply replacing the \(⊤\) symbol with \((K + ⊤)^*\).
% This means that we can have a right inverse of this homomorphism:
% \begin{align*}
%     r^{-1} & : \KATExp_{K_⊤, B}/\RP → \TopKATExp_{K, B}/\RP\\  
%     r^{-1} & ((K + ⊤)^*) ≜ ⊤ \\  
%     r^{-1} & (p) ≜ p & p ∈ K + B + \{⊤, 1, 0\}\\   
%     r^{-1} & (t₁ + t₂) ≜ r^{-1}(t₁) + r^{-1}(t₂)\\   
%     r^{-1} & (t₁ ⋅ t₂) ≜ r^{-1}(t₁) ⋅ r^{-1}(t₂) \\
%     r^{-1} & (t^*) ≜ (r^{-1}(t))^*. 
% \end{align*}
% \(r^{-1}\) will recursively go through the structure of the input term,
% and replace all the occurrence of \((K + ⊤)^*\) to \(⊤\).
% It is quite easy to see that \(r^{-1}\) is a right pre-semiring homomorphism,
% since it preserves addition, multiplication, and identities.
% And \[r ∘ r^{-1} = id_{\TopKATExp_{K, B}/\RP},\]
% thus \(r\) is injective.

% The final piece of the puzzle is to show the image of 
% \(\{δ_{w}(e) ∣ w ∈ (\At ⋅ K)^*\}/\RP\) for all \(e ∈ \TopKATExp_{K, B}\)
% under \(r\) is \(\{δ_{w}(r(e)) ∣ w ∈ (\At ⋅ K)^*\}/\RP\).
% The stronger version of this theorem: 
% the image of \(\{δ_{w}(e) ∣ w ∈ (\At ⋅ K)^*\}\) under \(r\) is 
% \(\{δ_{w}(r(e)) ∣ w ∈ (\At ⋅ K)^*\}\) can be shown trivially by the completeness result.
% TODO: do this more formally.
% Thus by injectivity of \(r\) and the fact that \(\{δ_{w}(r(e)) ∣ w ∈ (\At ⋅ K)^*\}/\RP\) is finite,
% \(\{δ_{w}(e) ∣ w ∈ (\At ⋅ K)^*\}\) is finite for all TopKAT terms.







% And this is exactly given by the injectivity of function:
% \(r: \TopKATExp_{K, B} → \KATExp_{K_⊤, B}\).
% Recall that \(r: \TopKATExp_{K, B} → \KATExp_{K_⊤, B}\) is a TopKAT homomorphism,
% hence it preserves all the TopKAT operations, 
% which includes addition, multiplication, and identities.
% Thus \(r\) is a right pre-semiring homomorphism, 
% \(r: \TopKATExp_{K, B}/≈ → \KATExp_{K_⊤, B}/≈\).




% Proof strategy for decidability:
% \begin{itemize}
%     \item Kozen has shown that KAT under derivative from a finite number of term,
%         under a decidable equality \(≈\), which is pre-semiring equality.
%     \item To show that TopKAT can be reasoned using coalgebraic theory,
%         we need to make sure derivative is closed under finitely number of term.
%         I hypothesize pre-semiring equality is enough,
%         since \(δ_{α p}(⊤) = ⊤\).
%     \item We need to relate the TopKAT term derivative under pre-semiring equality 
%         with KAT term under pre-semiring equality to derive the finiteness result.
%     \item Basically we need to establish an bijection (injection?) between TopKAT term 
%         under pre-semiring equalities 
% \end{itemize}






% \section{Domain Hypotheses in TopKAT and Its Power}\label{sec: domain hypotheses}

% Kleene algebra with domain is a long-standing framework that axiomatizes the
% domain operation in
% Kleene algebra with tests~\cite{Möller_Struth_2006, Struth_2015, Desharnais_Möller_Struth_2004,
%     Desharnais_Möller_Struth_2006, Desharnais_Struth_2011, Fahrenberg_Johansen_Struth_Ziemiánski_2021}.
% This framework has found applications in encoding logic,
% reasoning about termination of rewriting system, encoding predicate transformers,
% and many other fields~\cite{Möller_O’Hearn_Hoare_2021, Desharnais_Möller_Struth_2004}.
% Recently the complexity of Kleene algebra with domain was shown to be EXPSPACE-complete
% by reduction to dynamic logic \cite{Sedlár_2023}.

% In this section, we establish the connection between TopKAT, domain comparison in TopKAT,
% and TopKAT with domain by showing that domain of any term \(t\) can be axiomatized
% using a restricted form of domain comparison: \(δ(t) ⊤ = t ⊤\),
% where \(δ(t)\) is considered a fresh primitive test.
% Hypotheses of this form are called \emph{domain hypotheses} in TopKAT, 
% and all the fresh primitive tests generated by a collection of domain hypotheses is denoted as \(D\).

% ``Domain hypotheses'' are different from ``domain axioms'':
% domain hypotheses defines the domains of given TopKAT terms 
% using a simple instance of the domain comparison equation;
% whereas domain axioms are the three properties that
% a domain operation needs to satisfy in Kleene algebra with domain.

% We will show that, despite the simplistic from of domain hypotheses,
% TopKAT with these domain hypotheses is strictly stronger than TopKAT with domain.
% In other word, for every equation of TopKAT with domain,
% we can always establish a finite set of domain hypotheses that suffice to prove the same result,
% but there are unprovable equations in TopKAT with domain 
% that is provable in TopKAT with a finite set of domain hypotheses.

% First we will reproduce an important property of \(δ\) in TopKAT with domain:
% since the domain \(δ\) in TopKAT with domain is an operation,
% hence each element automatically has a unique domain associated to it.
% This property does not come for free when we are using domain hypotheses.

% \begin{corollary}[Uniqueness of domain]
%   Given a TopKAT and an element \(p\), there is at most one test \(b\) s.t.
%   \(b ⊤ = p ⊤\).
% \end{corollary}

% \begin{proof}
%     This result is a direct consequence of~\Cref{the: b T = cT implies b = c}:
%     assume there exists two test \(a, b\) which are both domain of \(p\),
%     thus \(p ⊤ = a ⊤ = b ⊤\);
%     by~\Cref{the: b T = cT implies b = c}, \(a = b\), hence the domain is always unique.
% \end{proof}

% The above result means that with the hypothesis \(δ(t) ⊤ = t ⊤\), then \(δ(t)\)
% is the ``unique domain'' of \(t\) in the sense that: if there exists another
% test \(b\) that satisfied the same property, i.e. \(b ⊤ = t ⊤\), then
% \(b = δ(t)\). This suffices to emulate proofs that involve the domain operator:

% \begin{corollary}\label{the: TopKAT H subsumes TopKAT dom}
%     For a pair of term \(t₁, t₂ ∈ \TopKAT^{dom}_{K, B}\),
%     we can construct a set of domain hypotheses \(H\) with additional primitive tests denoted \(D\),
%     such that
%     \[\TopKAT^{\dom}_{K, B} ⊧ t₁ = t₂ ⟹ \TopKAT^{H}_{K, B + D} ⊧ t₁ = t₂.\]
% \end{corollary}

% \begin{proof}
%   Assume \(\TopKAT^{\dom} ⊧ t₁ = t₂\), then we have a finite proof \(P\) for
%   that result.  Thus, there are only finitely many terms of the form \(δ(t)\)
%   appears in that proof.  Consider the following
%   \(H ≜ \{δ(t) ⊤ = t ⊤ ∣ \text{\(δ(t)\) appears in the proof \(P\).}\}.\) This
%   suffices to prove all the instances of the domain axioms that might be used in
%   the proof:
%   \[p ≤ δ(p) ⋅ p \\ δ(b ⋅ p) ≤ b \\ δ(p ⋅ δ(q)) = δ(p ⋅ q).\] (This follows from
%   basic algebraic manipulations; see~\Cref{the: domain axiom is sound with
%     hypothesized domain} for details.)

%   Since all the relevant instances of the domain axioms for \(δ(t)\) can be
%   proved by having the hypotheses \(δ(t) ⊤ = t ⊤\), we can recreate the proof
%   \(P\) just by repeating it.
% \end{proof}

% However, the converse of the previous results does not hold;
% to show this, we will define a natural model of TopKAT with domain.

% \begin{definition}[language TopKAT with domain]
%     A language TopKAT with domain over alphabet \(K, B\)
%     is the language TopKAT \(𝒢_{K_⊤, B}\) with the following domain operator:
%     \[δ(S) = \{α ∣ ∃ s, α s ∈ S\}.\]
%     There is a language interpretation into the language TopKAT with domain
%     that is generated by the same action on the primitive as the
%     language TopKAT interpretation and the language KAT interpretation:
%     \begin{align*}
%         G^{\dom} & : \TopKAT^{\dom} → 𝒢_{K_⊤, B}                                   \\
%         G^{\dom} & (b) = \{α ∣ \text{\(b\) appears positively in \(α\)}\}
%                  & b ∈ B                                                           \\
%         G^{\dom} & (p) = \{α p β ∣ α, β ∈ \At\}                           & p ∈ K.
%     \end{align*}
% \end{definition}
% The soundness of this model can be proven just by unfolding the definition, 
% see~\Cref{the: soundess of TopKAT with domain}.
% However, unlike the language interpretation of KAT and TopKAT,
% the language interpretation of TopKAT with domain is not complete:
% \begin{align*}
%     G^{\dom}(δ(p)) & = δ(G^{\dom}(p)) = \At           \\
%     G^{\dom}(δ(1)) & = δ(G^{\dom}(1)) = δ(\At) = \At,
% \end{align*}
% therefore the equation \(δ(p) = δ(1)\) in the language interpretation,
% but it is clear that this equality will not hold in general.

% \begin{theorem}
%     For a pair of term \(t₁, t₂ ∈ \TopKAT^{dom}_{K, B}\),
%     given some set of domain hypotheses where the set of new primitive tests are denoted as \(D\).
%     Then
%     \[\TopKAT^{\dom}_{K, B} ⊧ t₁ = t₂ \not ⟸ \TopKAT^{H}_{K, B + D} ⊧ t₁ = t₂.\]
% \end{theorem}

% \begin{proof}
%     We consider the simple hypotheses \(H = \{δ(p) ⊤ = p ⊤\}\) for \(p ∈ K\),
%     which is sufficed to derive itself:
%     \[\TopKAT^{H}_{K, B + \{δ(p)\}} ⊧ δ(p) ⊤ = p ⊤.\]
%     However, the domain axiom are not enough to show this equation, i.e.
%     \[\TopKAT^{\dom}_{K, B} \mathrel{\not ⊧} δ(p) ⊤ = p ⊤,\]
%     We consider the language interpretation of the domain:
%     \begin{align*}
%         G^{\dom}(p ⊤)    & = G^{\dom}(p) ⋄ G^{\dom}(⊤)
%         = \{α p β s ∣ α, β ∈ \At\};                       \\
%         G^{\dom}(δ(p) ⊤) & = δ(G^{\dom}(p)) ⋄ G^{\dom}(⊤)
%         = \At ⋄ GS_{K_⊤, B} = GS_{K_⊤, B}.
%     \end{align*}
%     Thus, \(G^{\dom}(p ⊤) ≠ G^{\dom}(δ(p) ⊤)\),
%     since interpretation preserves equalities, thus \(p ⊤ = δ(p) ⊤\)
%     cannot be derivable via the equational theory of TopKAT with domain.
% \end{proof}

% This means that the domain hypotheses is strictly stronger
% than the domain axioms in TopKAT with domain:
% there are equations that are true in TopKAT with domain hypotheses,
% but unprovable using just TopKAT with domain.

\section{Related Works}

% \itemTitle{Kleene algebra and completeness:}
% Kleene algebra was developed as an axiomatically system
% for regular expressions and languages~\cite{Kleene_1956, Conway_2012}.
% It was later known that there is no finite variety
% that completely captures the equational theory of regular languages~\cite{redko_1964}.
% Thus an infinite axiomatization was proposed and proven complete by Krob~\cite{Krob_1991};
% and similarly for a quasi-equational theory by Kozen~\cite{Kozen_1994}
% using uniqueness of minimal automata~\cite{Hopcroft_Ullman_1979};
% this quasi-equational system is commonly refers to as \emph{Kleene algebra}
% as we have seen today.
% The connection between these systems was also established~\cite{Kozen_1990}.
% The original Kozen's completeness result was re-proven several times,
% where each proof either applies a different technique or shortens the proof:
% Kozen later reproved the result using the equation encoding of bi-simulation~\cite{Kozen_2001},
% this proof no longer relies on uniqueness of minimal automata;
% Silva made a coalgebraic version of Kozen's proof~\cite{Silva_2010}.
% Later the ``left-handed completeness'',
% a stronger result than the original completeness result,
% was proven by a shorter proof using the interaction of algebra and coalgebra~\cite{Kozen_Silva_2020};
% the left-handed completeness was later also proven using cyclic proofs~\cite{Das_Doumane_Pous}.

\textbf{Extensions of Kleene algebra and reduction:}
soon after the completeness of Kleene algebra was proven~\cite{Kozen_1994},
it was realized that adding an embedded Boolean algebra can help reasoning
about control structures, such system is referred to as
Kleene algebra with tests (KAT)~\cite{Kozen_Smith_1997,Cohen_Kozen_Smith_1999}.
Later KAT was further extended to reason about failure~\cite{Mamouras_2017},
indicator variables~\cite{Grathwohl_Kozen_Mamouras_2014},
domain~\cite{Desharnais_Möller_Struth_2006}, networks~\cite{Anderson_Foster_Guha_Jeannin_Kozen_Schlesinger_Walker_2014},
and relational reasoning~\cite{Antonopoulos_Koskinen_Le_Nagasamudram_Naumann_Ngo_2022}.
Kleene algebra has also been extended to reason about 
concurrency, as concurrent Kleene algebra~\cite{Hoare_van_Staden_Möller_Struth_Zhu_2016, Kappé_Brunet_Silva_Zanasi_2018}
and concurrent Kleene algebra with observations~\cite{Kappé_Brunet_Silva_Wagemaker_Zanasi_2020}.
Many of these extensions can be seen as Kleene algebra with extra hypotheses~\cite{Cohen_1995,Doumane_Kuperberg_Pous_Pradic_2019}.
Although many hypotheses make the theory undecidable~\cite{Kozen_1996,Kozen_2002,Doumane_Kuperberg_Pous_Pradic_2019},
many useful hypotheses can be eliminated via reduction~\cite{Pous_Rot_Wagemaker_2021}.
Thus, our new perspective on reduction could potentially lead to streamlining of various previous proofs, 
and more general proofs of completeness results.

\textbf{Top element:}
Tarski's relational algebra~\cite{tarski_CalculusRelations_1941} contains the addition, 
mulitiplication, and identity operation of KA;  
in addition, relational algebra also include a top element. 
Hence attempts to incorporat Kleene star into relational algebra 
effectively create a super theory of TopKAT.
Unfortuantly, several attempts at these algebras turn out to be undecidable
because of the presence of intersection and 
converse operations~\cite{andrekaAxiomatizabilityPositiveAlgebras2011, pous_PositiveCalculusRelations_2018}.
With the intersection and converse operators removed, 
top element is proven to be individually useful in Kleene algebra:
for example, Mamouras~\cite{Mamouras_2017} uses the top element to forget program states,
and Antonopoulos et al.~\cite{Antonopoulos_Koskinen_Le_Nagasamudram_Naumann_Ngo_2022} 
uses top to design forward simulation rules for relational verification, 
and claim that relational incorrectness logic~\cite{murray_UnderApproximateRelationalLogic_2020a} 
can be encoded using BiKAT extended with top.
The completeness and decidability of TopKAT was first studied by Zhang et al.~\cite{Zhang_de_Amorim_Gaboardi_2022},
and concluded that TopKAT is not complete with relational models.
Later, Pous et al.~\cite{Pous_Wagemaker_2022,Pous_Wagemaker_2023} showed that 
both TopKA and TopKAT is complete with relational model with one additional axiom: \(p ⊤ p ≥ p\),
and the theory remains PSPACE-complete, like KAT and TopKAT.
In this paper, we showed that TopKAT without the additional axiom is complete 
for a specific form of inequalities, namely when top only appears in the front or the end of the term.
Although this form of inequalities seem restrictive, 
they are enough to encode both Hoare and incorrectness logic~\cite{Zhang_de_Amorim_Gaboardi_2022}.

\textbf{Domain in KAT:}
The study of axiomatizing (co)domain in KAT has a long and rich history. 
Domain semiring~\cite{Desharnais_Struth_2011} 
and Kleene algebra with domain~\cite{Desharnais_Möller_Struth_2006}
were two popular yet different axiomatizations of (co)domain in Kleene algebra with tests.
These two axiomitizations turn out to coincide in a large class of semirings~\cite{Fahrenberg_Johansen_Struth_Ziemiánski_2021}.
Various applications for domain in KAT have been discovered, including modeling
program correctness, predicate transformers, temporal logics, 
termination analysis, and many more~\cite{Desharnais_Möller_Struth_2004}.
Many of these applications can even be efficiently automated~\cite{hofner_AutomatedReasoningKleene_2007}.
However, although the free relational model of these theories has been characterized~\cite{mclean_FreeKleeneAlgebras_2020},
the search for general complete interpretation remains unfruitful.
The complexity of these theories was recently shown to be EXPTIME-complete~\cite{Sedlár_2023},
a worse complexity class than PSPACE-complete for TopKAT.



\section{Conclusion And Open Problems}

In this paper, we exploit the homomorphic structure of reduction
to simplify the proof of various previous results~\cite{Zhang_de_Amorim_Gaboardi_2022}.
We have also showed that TopKAT is complete with respect to (co)domain comparison
in the relational models,
which lays a solid foundation for the use of TopKAT in (co)domain reasoning.

However, there are still several interesting unsolved problems about TopKAT.
Most of the incorrectness logic rules are written using hypotheses,
for example, the sequencing rule:
\[
    \frac{[a]~p~[b] \qquad [b]~q~[c]}{[a]~p ⋅ q~[c]}
\]
corresponds to the implication \(⊤ a p ≤ ⊤ b ∧ ⊤ b p ≤ ⊤ c ⟹ ⊤ a p q ≤ ⊤ c\).
Although each individual inequality in the implication fits the desired form \(⊤ t₁ ≥ ⊤ t₂\).
it is unclear whether implications of the form
\[⊤ t₁₁ ≤ ⊤ t₁₂ ∧ ⊤ t₂₁ ≤ ⊤ t₂₂ ∧ ⋯ ∧ ⊤ tₙ₁ ≤ ⊤ tₙ₂ ⟹ ⊤ t₁ ≤ ⊤ t₂\]
are complete with relational TopKAT or decidable.

Recently, there is an efficient fragment of KAT proposed, named 
\emph{Guarded Kleene algebra with tests}~\cite{Smolka_Foster_Hsu_Kappé_Kozen_Silva_2020}
or \emph{GKAT}.
This fragment not only enjoys nearly-linear time equality checking,
but also soundly models probabilistic computations as well. 
It would be interesting to see whether the completeness and decidability result of TopKAT
can be extended to GKAT, and whether the efficiency of GKAT will persist with the addition of top.

