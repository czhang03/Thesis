% ABSTRACT

Kleene Algebra (KA) is an equational system celebrated for its decidability 
and completeness with respect to regular language equalities.
Because of the nice properties of Kleene Algebra, 
numerous extensions of it are developed to reason about
network system~\cite{Anderson_Foster_Guha_Jeannin_Kozen_Schlesinger_Walker_2014,Foster_Kozen_Milano_Silva_Thompson_2015},
concurrent programs~\cite{Hoare_Möller_Struth_Wehrman_2009,Kappé_Brunet_Silva_Wagemaker_Zanasi_2020,Kappé_Brunet_Silva_Zanasi_2018}, 
probabilistic systems~\cite{McIver_Cohen_Morgan_2006, McIver_Rabehaja_Struth_2011}, 
relational verification~\cite{Antonopoulos_Koskinen_Le_Nagasamudram_Naumann_Ngo_2022},
and program schematology~\cite{Angus_Kozen_2001}.
Most of the above systems has been shown to be complete and decidable.

In this thesis, we focus on three variants of Kleene Algebra with real-world applications.

The first system is Kleene Algebra with atomic commutativity.
The question about the complexity of this system was raised over 25 years ago~\cite{Kozen_1996}.
Recently, a similar system named BiKA has been shown to be effective for alignment problem 
in relational reasoning~\cite{Antonopoulos_Koskinen_Le_Nagasamudram_Naumann_Ngo_2022}.
However, the completeness and decidability of these systems has still alluded us
after all these years. In this thesis we give a negative result:
although the word inhabitance problem 
(inequalities of the form \(w ≤ e\), where \(w\) is a word and \(e\) is a KA expression)
is complete and decidable, the general equalities in KA with atomic commutativity is 
neither complete nor decidable.

The second system, Kleene Algebra with tests and top (TopKAT), 
was developed to perform domain and reachablity reasoning.
We showed the conventional extension of Kleene Algebra with tests, 
despite able to encode Hoare logic, is inadequate for domain reasoning.
This leads to our development of TopKAT, which is complete for domain reasoning.
The main proof heavily relies on a technique called \emph{reduction},
we should that the reduction from TopKAT to KAT satisfy nice properties that allows 
us to generate complete interpretations for TopKAT for free,
and also gives us a complete and decidable coalgebraic theory with minimal efforts.

Finally, Guarded Kleene Algebra with Test and Indicator Variables (GKATI),
was designed to verify decompilation results. 
This system gives an efficient nearly linear time algorithm for decompilation verification.
However, we have also showed that this system is not equationally axiomatizable, 
since it failed to satisfy the congruence rule.

Readers might realize many results in this thesis are negative.
Although these results might be disappointing in the general development of these systems, 
the proof of these results are technically challenging, 
which necessitates new techniques and understandings of Kleene Algebras.
In the future, these techniques might lead to positive results for 
complex variants of Kleene Algebra or concise proofs for properties of existing systems.