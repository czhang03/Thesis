% This file contains all the necessary setup and commands to create
% the preliminary pages according to the buthesis.sty option.

\title{Two Variants of Kleene Algebra And Their Applications}

\author{Cheng Zhang}

% Type of document prepared for this degree:
%   1 = Master of Science thesis,
%   2 = Doctor of Philosophy dissertation.
\degree=2

\prevdegrees{B.A., Wheaton College, 2018}

\department{Department of Computer Science}

% Degree year is the year the diploma is expected, and defense year is
% the year the dissertation is written up and defended. Often, these
% will be the same, except for January graduation, when your defense
% will be in the fall of year X, and your graduation will be in
% January of year X+1
\defenseyear{2024}
\degreeyear{2024}

% For each reader, specify appropriate label {First, Second, Third},
% then name, and title. IMPORTANT: The title should be:
%   "Professor of Electrical and Computer Engineering",
% or similar, but it MUST NOT be:
%   Professor, Department of Electrical and Computer Engineering"
% or you will be asked to reprint and get new signatures.
% Warning: If you have more than five readers you are out of luck,
% because it will overflow to a new page. You may try to put part of
% the title in with the name.
\reader{First}{Marco Gaboardi, PhD}{Associate Professor Of Computer Science}
\reader{Second}{Arthur A. de Amorim, PhD}{Assistant Professor Of Computer Science, RIT}

% The Major Professor is the same as the first reader, but must be
% specified again for the abstract page. Up to 4 Major Professors
% (advisors) can be defined. 
\numadvisors=1
\majorprof{Marco Gaboardi, PhD}{Associate Professor Of Computer Science}
% \majorprofb{First M. Last, PhD}{{Professor of Computer Science}}
%\majorprofc{First M. Last, PhD}{{Professor of Astronomy}}
%\majorprofd{First M. Last, PhD}{{Professor of Biomedical Engineering}}

%%%%%%%%%%%%%%%%%%%%%%%%%%%%%%%%%%%%%%%%%%%%%%%%%%%%%%%%%%%%%%%%  

%                       PRELIMINARY PAGES
% According to the BU guide the preliminary pages consist of:
% title, copyright (optional), approval,  acknowledgments (opt.),
% abstract, preface (opt.), Table of contents, List of tables (if
% any), List of illustrations (if any). The \tableofcontents,
% \listoffigures, and \listoftables commands can be used in the
% appropriate places. For other things like preface, do it manually
% with something like \newpage\section*{Preface}.

% This is an additional page to print a boxed-in title, author name and
% degree statement so that they are visible through the opening in BU
% covers used for reports. This makes a nicely bound copy. Uncomment only
% if you are printing a hardcopy for such covers. Leave commented out
% when producing PDF for library submission.
%\buecethesistitleboxpage

% Make the titlepage based on the above information.  If you need
% something special and can't use the standard form, you can specify
% the exact text of the titlepage yourself.  Put it in a titlepage
% environment and leave blank lines where you want vertical space.
% The spaces will be adjusted to fill the entire page.
\maketitle
\cleardoublepage

% The copyright page is blank except for the notice at the bottom. You
% must provide your name in capitals.
\copyrightpage
\cleardoublepage

% Now include the approval page based on the readers information
% Once the approval page is approved by the Mugar Library staff, please
% comment out the "\approvalpagewithcomment" line and uncomment "\approvalpage"
% \approvalpagewithcomment
\approvalpage
\cleardoublepage

% Here goes your favorite quote. This page is optional.
\newpage
%\thispagestyle{empty}
\phantom{.}
\vspace{4in}

\begin{singlespace}
\begin{quote}
  \textit{Facilis descensus Averni;}\\
  \textit{Noctes atque dies patet atri janua Ditis;}\\*
  \textit{Sed revocare gradum, superasque evadere ad auras,}\\
  \textit{Hoc opus, hic labor est.}\hfill{Virgil (from Don's thesis!)}
\end{quote}
\end{singlespace}

% \vspace{0.7in}
%
% \noindent
% [The descent to Avernus is easy; the gate of Pluto stands open night
% and day; but to retrace one's steps and return to the upper air, that
% is the toil, that the difficulty.]

\cleardoublepage

% The acknowledgment page should go here. Use something like
% \newpage\section*{Acknowledgments} followed by your text.
\newpage
\section*{\centerline{Acknowledgments}}
Here go all your acknowledgments. You know, your advisor, funding agency, lab
mates, etc., and of course your family.

As for me, I would like to thank Jonathan Polimeni for cleaning up old LaTeX
style files and templates so that Engineering students would not have to suffer
typesetting dissertations in MS Word. Also, I would like to thank IDS/ISS
group (ECE) and CV/CNS lab graduates for their contributions and tweaks to this
scheme over the years (after many frustrations when preparing their final
document for BU library). In particular, I would like to thank Limor Martin who
has helped with the transition to PDF-only dissertation format (no more printing
hardcopies -- hooray !!!)

The stylistic and aesthetic conventions implemented in this LaTeX
thesis/dissertation format would not have been possible without the help from
Brendan McDermot of Mugar library and Martha Wellman of CAS.

Finally, credit is due to Stephen Gildea for the MIT style file off which this
current version is based, and Paolo Gaudiano for porting the MIT style to one
compatible with BU requirements.

\vskip 1in

\noindent
Janusz Konrad\\
Professor\\
ECE Department
\cleardoublepage

% The abstractpage environment sets up everything on the page except
% the text itself.  The title and other header material are put at the
% top of the page, and the supervisors are listed at the bottom.  A
% new page is begun both before and after.  Of course, an abstract may
% be more than one page itself.  If you need more control over the
% format of the page, you can use the abstract environment, which puts
% the word "Abstract" at the beginning and single spaces its text.

\begin{abstractpage}
% ABSTRACT

Kleene Algebra (KA) is an equational system celebrated for its decidability 
and completeness with respect to regular language equalities.
Because of the desirable properties of Kleene Algebra, 
numerous extensions were developed to reason about
network system~\cite{Anderson_Foster_Guha_Jeannin_Kozen_Schlesinger_Walker_2014,Foster_Kozen_Milano_Silva_Thompson_2015, Smolka_Kumar_Kahn_Foster_Hsu_Kozen_Silva_2019},
concurrent programs~\cite{hoare_ConcurrentKleeneAlgebra_2009,Kappé_Brunet_Silva_Wagemaker_Zanasi_2020,Kappé_Brunet_Silva_Zanasi_2018}, 
probabilistic systems~\cite{mciver_UsingProbabilisticKleene_2006, McIver_Rabehaja_Struth_2011}, 
relational verification~\cite{Antonopoulos_Koskinen_Le_Nagasamudram_Naumann_Ngo_2022},
and program schematology~\cite{Angus_Kozen_2001}.
In this thesis, we focus on two variants of Kleene Algebra with real-world applications.

The first system, Kleene Algebra with tests and top (TopKAT), 
was developed to perform domain and reachablity reasoning.
We showed the conventional extension of Kleene Algebra with tests, 
despite able to encode Hoare logic, is inadequate for domain reasoning.
This leads to our development of TopKAT, which is complete for domain reasoning.
TopKAT was able to soundly encode both propositional incorrectness and Hoare logic~\cite{OHearn_2020,Hoare69},
offering better complexity bound than alternative frameworks~\cite{Möller_O’Hearn_Hoare_2021, Sedlár_2023}.
The our completeness proof for TopKAT relies heavily on a technique called \emph{reduction},
we showed that the reduction from TopKAT to KAT satisfy nice properties that enable us to generate complete interpretations for TopKAT for free, and also gives us a complete decision procedure with minimal effort.

The second system, control-flow Guarded Kleene Algebra with Tests (CF-GKAT) verifies control-flow transformations.
Guarded Kleene Algebra with Tests~\cite{Smolka_Foster_Hsu_Kappé_Kozen_Silva_2020} provides a robust system that is not only sound and complete with respect to trace equivalence, but also enjoys a efficient decision procedure. 
Yet, GKAT remains insufficient as a system to verify several well-known control-flow algorithms~\cite{erosa-hendren-1994,yakdan_NoMoreGotos_2015,kozen_BohmJacopiniTheorem_2008a}, because it lacks important control-flow structures like indicator variable and non-local control-flow structures like \(\comBrk\), \(\comRet\), and \(\command{goto}\).
To obtain CF-GKAT, we extended the syntax and semantics of GKAT to incorporate these essential features. We have developed a efficient decision procedure for CF-GKAT program utilizing \emph{CF-GKAT automata}, an automata model that closely emulates CF-GKAT programs, and can be efficiently lowered into GKAT automata. 
Furthermore, this decision procedure is sound and complete: the algorithm will output true if and only if the two input programs are trace equivalent.
\end{abstractpage}
\cleardoublepage

% Now you can include a preface. Again, use something like
% \newpage\section*{Preface} followed by your text

% Table of contents comes after preface
\tableofcontents
\cleardoublepage

% If you do not have tables, comment out the following lines
\newpage
\listoftables
\cleardoublepage

% If you have figures, uncomment the following line
\newpage
\listoffigures
\cleardoublepage

% List of Abbrevs is NOT optional (Martha Wellman likes all abbrevs listed)
\chapter*{List of Abbreviations}

\begin{center}
  \begin{tabular}{lll}
    \hspace*{2em} & \hspace*{1in} & \hspace*{4.5in} \\
    CF-GKAT  & \dotfill & Control-Flow GKAT \\
    FailTopKAT & \dotfill & TopKAT with Failure \\
    FailTopREL & \dotfill & Relational TopKAT with Failure \\
    GKAT   & \dotfill & Guarded Kleene Algebra with Tests \\
    HL & \dotfill & Hoare Logic \\
    IL & \dotfill & Incorrectness Logic\\
    KA  & \dotfill & Kleene Algebra \\
    KAT & \dotfill & Kleene Algebra With Tests \\
    REL & \dotfill & Relational KAT \\
    TopGREL & \dotfill & General Relational TopKAT\\
    TopKAT & \dotfill & Kleene Algebra With Top and Tests\\
    TopREL & \dotfill & Relational TopKAT \\
  \end{tabular}
\end{center}
\cleardoublepage

% END OF THE PRELIMINARY PAGES

\newpage
\endofprelim
