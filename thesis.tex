% BU ECE template for MS thesis and PhD dissertation.
%
%==========================================================================%
% MAIN PREAMBLE 
%==========================================================================%
\documentclass[12pt,letterpaper]{report}          % Single-sided printing for the library
%\documentclass[12pt,twoside]{report} % Double-sided printing
\usepackage[intlimits]{amsmath}
\usepackage{amsfonts,amssymb}
\DeclareSymbolFontAlphabet{\mathbb}{AMSb}
\usepackage{natbib}
\usepackage{apalike}
\usepackage{float}
\usepackage[bf]{caption}       
\usepackage{fancyhdr}
%\usepackage{fancyheadings}
\usepackage{fancybox}
\usepackage{ifthen}
\usepackage{bu_ece_thesis}
\usepackage{url}
\usepackage{lscape,afterpage}
\usepackage{xspace}
\usepackage{epstopdf} 
\usepackage{subfig}


%==========================================================================%
%%% graphicx and pdf creation
\usepackage{graphicx}
\usepackage{appendix}
%\usepackage{psfrag}
%\DeclareGraphicsExtensions{.eps}   % extension for included graphics
%\usepackage{thumbpdf}              % thumbnails for ps2pdf
%\usepackage[ps2pdf,                % hyper-references for ps2pdf
%bookmarks=true,%                   % generate bookmarks ...
%bookmarksnumbered=true,%           % ... with numbers
%hypertexnames=false,%              % needed for correct links to figures !!!
%breaklinks=true,%                  % breaks lines, but links are very small
%linkbordercolor={0 0 1},%          % blue frames around links
%pdfborder={0 0 112.0}]{hyperref}%  % border-width of frames 
%                                   % will be multiplied with 0.009 by ps2pdf
%\hypersetup{
%  pdfauthor   = {Joe Graduate <joe.graduate@bu.edu>},
%  pdftitle    = {dissertation.pdf},
%  pdfsubject  = {doctoral dissertations},
%  pdfkeywords = {mathematics, science, technology},
%  pdfcreator  = {LaTeX with hyperref package},
%  pdfproducer = {dvips + ps2pdf}
%}
%==========================================================================%
% customized commands can be placed here
%\newcommand{\figref}[1]{Figure~\ref{#1}}
%\newcommand{\chapref}[1]{Chapter~\ref{#1}}
%\newcommand{\latex}{\LaTeX\xspace}
%==========================================================================%
\usepackage[dvipsnames]{xcolor}
\usepackage{hyperref}
\hypersetup{breaklinks=true,colorlinks=true,linkcolor=blue,urlcolor=magenta,citecolor=cyan}
\usepackage{breakurl}
\usepackage{algorithm}

%==========================================================================%
% MY PACKAGES
%==========================================================================%

% theorems
\usepackage{amsthm}
\newtheorem{corollary}{Corollary}
\newtheorem{theorem}{Theorem}
\newtheorem{lemma}{Lemma}
\newtheorem{definition}{Definition}
\newtheorem*{definition*}{Definition}
\newtheorem{example}{Example}
\newtheorem{remark}{Remark}

% For adding inline comments in the text.
\usepackage[margin=false,inline=true]{fixme}
\FXRegisterAuthor{aaa}{anaaa}{\color{cyan}AAA}
\FXRegisterAuthor{mg}{anmg}{\color{red}MG}
\FXRegisterAuthor{cz}{ancz}{\color{orange}CZ}
% \newcommand{\aaa}[1]{\aaanote{#1}}
% \newcommand{\mg}[1]{\mgnote{#1}}
% \newcommand{\cz}[1]{\cznote{#1}}
\newcommand{\aaa}[1]{}
\newcommand{\mg}[1]{}
\newcommand{\cz}[1]{}

\usepackage{annotate-equations}

% graphs
\usepackage{tikz}
\usetikzlibrary{shapes.geometric, positioning, arrows, matrix}

% commutative diagram
\usepackage{tikz-cd}

% inference rule
\usepackage{mathpartir}

% ref
\usepackage{hyperref}
\usepackage{cleveref}
\crefname{ineq}{inequality}{inequalities}
\creflabelformat{ineq}{#2{\upshape(#1)}#3} 
\crefname{equiv}{equivalence}{equivalences}
\creflabelformat{equiv}{#2{\upshape(#1)}#3} 

% item spacing
\usepackage{enumitem}

% unicode math symbols
\usepackage{fontspec}
\usepackage{unicode-math}
% support for hat, overline, underline, vec, and sim combining charactors
\directlua{
  local func = luatexbase.new_luafunction'afteracc'
  token.set_lua('afteracc', func, 'protected')

  local nest = tex.nest
  local noad_id = node.id'noad'
  local accent_id = node.id'accent'
  local math_char_id = node.id'math_char'

  lua.get_functions_table()[func] = function()
    local level = nest.top
    local last = level.tail
    if not (last and last.id == noad_id) then
      tex.error'I can only put accents on simple noads.'
      return
    end
    if last.sub or last.sup then
      tex.error'If you want accents on a superscript or subscript, please use braces.'
      return
    end
    local acc = node.new(accent_id, 1)
    acc.nucleus = last.nucleus
    last.nucleus = nil
    acc.accent = node.new(math_char_id)
    acc.accent.fam, acc.accent.char = 0, token.scan_int()

    level.tail = last.prev
    level.head = node.remove(level.head, last)
    node.flush_node(last)

    node.write(acc)
  end
}
\AtBeginDocument{
\begingroup
  \def\UnicodeMathSymbol#1#2#3#4{%
    \ifx#3\mathaccent
      \def\mytmpmacro{\afteracc#1 }%
      \global\letcharcode#1=\mytmpmacro
      \global\mathcode#1="8000
    \fi
  }
  \input{unicode-math-table}
\endgroup
}
% math font, this is needed to render several math command
% \setmathfont{STIX Two Math}

%==========================================================================%
% Macros
%==========================================================================%

% math
\DeclareMathOperator{\post}{\mathrm{post}}
\DeclareMathOperator{\dom}{\mathrm{dom}}
\DeclareMathOperator{\cod}{\mathrm{cod}}
\DeclareMathOperator{\Img}{\mathbf{Im}}

% theories
\newcommand{\Word}{\mathsf{Wrd}}
\newcommand{\Reg}{\mathsf{REG}}
\newcommand{\KA}{\mathsf{KA}}
\newcommand{\KAT}{\mathsf{KAT}}
\newcommand{\TopKAT}{\mathsf{TopKAT}}
\newcommand{\REL}{\mathbf{\mathrm{REL}}}
\newcommand{\TopREL}{\mathbf{\mathrm{TopREL}}}
\newcommand{\TopGREL}{\mathbf{\mathrm{TopGREL}}}

% two-counter machine command
\DeclareMathOperator{\Inc}{\mathtt{Inc}}
\DeclareMathOperator{\Dec}{\mathtt{Dec}}
\DeclareMathOperator{\If}{\mathtt{If}}
\DeclareMathOperator{\Halt}{\mathtt{Halt}}
% program commands 
\newcommand{\comSkip}{\mathtt{skip}}
\newcommand{\comError}{\mathtt{error ()}}
\newcommand{\comAssume}[1]{\mathtt{assume}~#1}
\newcommand{\comITE}[3]{\mathtt{if}~#1~\mathtt{then}~#2~\mathtt{else}~#3}
\newcommand{\comWhile}[2]{\mathtt{while}~#1~\mathtt{do}~#2}
\newcommand{\comAssign}[2]{#1:= #2}
\newcommand{\comLoop}[1]{\mathtt{trace}~#1}
\newcommand{\comIter}[1]{{#1}^{\star}}

\newcommand{\incorTriple}[3]{[#1]~#2~[#3]}
\newcommand{\hoareTriple}[3]{\{#1\}~#2~\{#3\}}


%==========================================================================%
% BEGIN
%==========================================================================%
\begin{document}

% The preliminary pages
% This file contains all the necessary setup and commands to create
% the preliminary pages according to the buthesis.sty option.

\title{Two Variants of Kleene Algebra And Their Applications}

\author{Cheng Zhang}

% Type of document prepared for this degree:
%   1 = Master of Science thesis,
%   2 = Doctor of Philosophy dissertation.
\degree=2

\prevdegrees{B.A., Wheaton College, 2018}

\department{Department of Computer Science}

% Degree year is the year the diploma is expected, and defense year is
% the year the dissertation is written up and defended. Often, these
% will be the same, except for January graduation, when your defense
% will be in the fall of year X, and your graduation will be in
% January of year X+1
\defenseyear{2024}
\degreeyear{2024}

% For each reader, specify appropriate label {First, Second, Third},
% then name, and title. IMPORTANT: The title should be:
%   "Professor of Electrical and Computer Engineering",
% or similar, but it MUST NOT be:
%   Professor, Department of Electrical and Computer Engineering"
% or you will be asked to reprint and get new signatures.
% Warning: If you have more than five readers you are out of luck,
% because it will overflow to a new page. You may try to put part of
% the title in with the name.
\reader{First}{Marco Gaboardi, PhD}{Associate Professor Of Computer Science}
\reader{Second}{Arthur A. de Amorim, PhD}{Assistant Professor Of Computer Science, RIT}

% The Major Professor is the same as the first reader, but must be
% specified again for the abstract page. Up to 4 Major Professors
% (advisors) can be defined. 
\numadvisors=1
\majorprof{Marco Gaboardi, PhD}{Associate Professor Of Computer Science}
% \majorprofb{First M. Last, PhD}{{Professor of Computer Science}}
%\majorprofc{First M. Last, PhD}{{Professor of Astronomy}}
%\majorprofd{First M. Last, PhD}{{Professor of Biomedical Engineering}}

%%%%%%%%%%%%%%%%%%%%%%%%%%%%%%%%%%%%%%%%%%%%%%%%%%%%%%%%%%%%%%%%  

%                       PRELIMINARY PAGES
% According to the BU guide the preliminary pages consist of:
% title, copyright (optional), approval,  acknowledgments (opt.),
% abstract, preface (opt.), Table of contents, List of tables (if
% any), List of illustrations (if any). The \tableofcontents,
% \listoffigures, and \listoftables commands can be used in the
% appropriate places. For other things like preface, do it manually
% with something like \newpage\section*{Preface}.

% This is an additional page to print a boxed-in title, author name and
% degree statement so that they are visible through the opening in BU
% covers used for reports. This makes a nicely bound copy. Uncomment only
% if you are printing a hardcopy for such covers. Leave commented out
% when producing PDF for library submission.
%\buecethesistitleboxpage

% Make the titlepage based on the above information.  If you need
% something special and can't use the standard form, you can specify
% the exact text of the titlepage yourself.  Put it in a titlepage
% environment and leave blank lines where you want vertical space.
% The spaces will be adjusted to fill the entire page.
\maketitle
\cleardoublepage

% The copyright page is blank except for the notice at the bottom. You
% must provide your name in capitals.
\copyrightpage
\cleardoublepage

% Now include the approval page based on the readers information
% Once the approval page is approved by the Mugar Library staff, please
% comment out the "\approvalpagewithcomment" line and uncomment "\approvalpage"
% \approvalpagewithcomment
\approvalpage
\cleardoublepage

% Here goes your favorite quote. This page is optional.
\newpage
%\thispagestyle{empty}
\phantom{.}
\vspace{4in}

\begin{singlespace}
\begin{quote}
  \textit{Facilis descensus Averni;}\\
  \textit{Noctes atque dies patet atri janua Ditis;}\\*
  \textit{Sed revocare gradum, superasque evadere ad auras,}\\
  \textit{Hoc opus, hic labor est.}\hfill{Virgil (from Don's thesis!)}
\end{quote}
\end{singlespace}

% \vspace{0.7in}
%
% \noindent
% [The descent to Avernus is easy; the gate of Pluto stands open night
% and day; but to retrace one's steps and return to the upper air, that
% is the toil, that the difficulty.]

\cleardoublepage

% The acknowledgment page should go here. Use something like
% \newpage\section*{Acknowledgments} followed by your text.
\newpage
\section*{\centerline{Acknowledgments}}
Here go all your acknowledgments. You know, your advisor, funding agency, lab
mates, etc., and of course your family.

As for me, I would like to thank Jonathan Polimeni for cleaning up old LaTeX
style files and templates so that Engineering students would not have to suffer
typesetting dissertations in MS Word. Also, I would like to thank IDS/ISS
group (ECE) and CV/CNS lab graduates for their contributions and tweaks to this
scheme over the years (after many frustrations when preparing their final
document for BU library). In particular, I would like to thank Limor Martin who
has helped with the transition to PDF-only dissertation format (no more printing
hardcopies -- hooray !!!)

The stylistic and aesthetic conventions implemented in this LaTeX
thesis/dissertation format would not have been possible without the help from
Brendan McDermot of Mugar library and Martha Wellman of CAS.

Finally, credit is due to Stephen Gildea for the MIT style file off which this
current version is based, and Paolo Gaudiano for porting the MIT style to one
compatible with BU requirements.

\vskip 1in

\noindent
Janusz Konrad\\
Professor\\
ECE Department
\cleardoublepage

% The abstractpage environment sets up everything on the page except
% the text itself.  The title and other header material are put at the
% top of the page, and the supervisors are listed at the bottom.  A
% new page is begun both before and after.  Of course, an abstract may
% be more than one page itself.  If you need more control over the
% format of the page, you can use the abstract environment, which puts
% the word "Abstract" at the beginning and single spaces its text.

\begin{abstractpage}
% ABSTRACT

Kleene Algebra (KA) is an equational system celebrated for its decidability 
and completeness with respect to regular language equalities.
Because of the desirable properties of Kleene Algebra, 
numerous extensions were developed to reason about
network system~\cite{Anderson_Foster_Guha_Jeannin_Kozen_Schlesinger_Walker_2014,Foster_Kozen_Milano_Silva_Thompson_2015, Smolka_Kumar_Kahn_Foster_Hsu_Kozen_Silva_2019},
concurrent programs~\cite{hoare_ConcurrentKleeneAlgebra_2009,Kappé_Brunet_Silva_Wagemaker_Zanasi_2020,Kappé_Brunet_Silva_Zanasi_2018}, 
probabilistic systems~\cite{mciver_UsingProbabilisticKleene_2006, McIver_Rabehaja_Struth_2011}, 
relational verification~\cite{Antonopoulos_Koskinen_Le_Nagasamudram_Naumann_Ngo_2022},
and program schematology~\cite{Angus_Kozen_2001}.
In this thesis, we focus on two variants of Kleene Algebra with real-world applications.

The first system, Kleene Algebra with tests and top (TopKAT), 
was developed to perform domain and reachablity reasoning.
We showed the conventional extension of Kleene Algebra with tests, 
despite able to encode Hoare logic, is inadequate for domain reasoning.
This leads to our development of TopKAT, which is complete for domain reasoning.
TopKAT was able to soundly encode both propositional incorrectness and Hoare logic~\cite{OHearn_2020,Hoare69},
offering better complexity bound than alternative frameworks~\cite{Möller_O’Hearn_Hoare_2021, Sedlár_2023}.
The our completeness proof for TopKAT relies heavily on a technique called \emph{reduction},
we showed that the reduction from TopKAT to KAT satisfy nice properties that enable us to generate complete interpretations for TopKAT for free, and also gives us a complete decision procedure with minimal effort.

The second system, control-flow Guarded Kleene Algebra with Tests (CF-GKAT) verifies control-flow transformations.
Guarded Kleene Algebra with Tests~\cite{Smolka_Foster_Hsu_Kappé_Kozen_Silva_2020} provides a robust system that is not only sound and complete with respect to trace equivalence, but also enjoys a efficient decision procedure. 
Yet, GKAT remains insufficient as a system to verify several well-known control-flow algorithms~\cite{erosa-hendren-1994,yakdan_NoMoreGotos_2015,kozen_BohmJacopiniTheorem_2008a}, because it lacks important control-flow structures like indicator variable and non-local control-flow structures like \(\comBrk\), \(\comRet\), and \(\command{goto}\).
To obtain CF-GKAT, we extended the syntax and semantics of GKAT to incorporate these essential features. We have developed a efficient decision procedure for CF-GKAT program utilizing \emph{CF-GKAT automata}, an automata model that closely emulates CF-GKAT programs, and can be efficiently lowered into GKAT automata. 
Furthermore, this decision procedure is sound and complete: the algorithm will output true if and only if the two input programs are trace equivalent.
\end{abstractpage}
\cleardoublepage

% Now you can include a preface. Again, use something like
% \newpage\section*{Preface} followed by your text

% Table of contents comes after preface
\tableofcontents
\cleardoublepage

% If you do not have tables, comment out the following lines
\newpage
\listoftables
\cleardoublepage

% If you have figures, uncomment the following line
\newpage
\listoffigures
\cleardoublepage

% List of Abbrevs is NOT optional (Martha Wellman likes all abbrevs listed)
\chapter*{List of Abbreviations}

\begin{center}
  \begin{tabular}{lll}
    \hspace*{2em} & \hspace*{1in} & \hspace*{4.5in} \\
    CF-GKAT  & \dotfill & Control-Flow GKAT \\
    FailTopKAT & \dotfill & TopKAT with Failure \\
    FailTopREL & \dotfill & Relational TopKAT with Failure \\
    GKAT   & \dotfill & Guarded Kleene Algebra with Tests \\
    HL & \dotfill & Hoare Logic \\
    IL & \dotfill & Incorrectness Logic\\
    KA  & \dotfill & Kleene Algebra \\
    KAT & \dotfill & Kleene Algebra With Tests \\
    REL & \dotfill & Relational KAT \\
    TopGREL & \dotfill & General Relational TopKAT\\
    TopKAT & \dotfill & Kleene Algebra With Top and Tests\\
    TopREL & \dotfill & Relational TopKAT \\
  \end{tabular}
\end{center}
\cleardoublepage

% END OF THE PRELIMINARY PAGES

\newpage
\endofprelim
        
\cleardoublepage

% Bodies
\chapter{Introduction}
\label{chapter:introduction}
\thispagestyle{myheadings}

\section{An overview on Kleene Algebra}
\label{sec:history}

Kleene algebra, named after the eminent mathematician Stephen Cole Kleene, represents a pivotal development in mathematical logic, computer science, and formal languages. 
Originating in the mid-20th century, Kleene algebra emerged from Kleene's seminal work on regular sets and expressions~\cite{Kleene_1956}, where he introduced algebraic structures to formalize fundamental operations on regular languages. 
Kleene left a important open question about Kleene Algebra: whether there exists \emph{complete} algebraic system for regular language equality: an algebraic system that is capable of derive all the language equalities of regular expression.
There are numerous systems proposed and are closely related~\cite{Kozen_1990}, yet the modern Kleene Algebra and its completeness proof is often attributed to Kozen~\cite{Kozen_2001,Kozen_1994}.

Given the close relation to regular languages, it is no surprise that the study of Kleene Algebra often makes heavy use of automata theory.
The automata perspective underpins the other important property of Kleene Algebra, that is, decidability.
Specifically, the behavioral equivalence of two automata (or more generally) coalgebra can be obtained by bisimulation on the automata~\cite{rutten_UniversalCoalgebraTheory_2000}.
And bisimulation usually is decidable for finite automata.
This is later characterized by the bialgebraic approach~\cite{jacobs_BialgebraicReviewDeterministic_2006}, although we will not dive deep into coalgebra, many of the techniques used in this paper are inspired by this framework.

The decidability and completeness of Kleene Algebra has inspired a suite of applications in programming languages and verifications; specifically, in the areas of network system~\cite{Anderson_Foster_Guha_Jeannin_Kozen_Schlesinger_Walker_2014,Foster_Kozen_Milano_Silva_Thompson_2015, Smolka_Kumar_Kahn_Foster_Hsu_Kozen_Silva_2019},
concurrent programs~\cite{hoare_ConcurrentKleeneAlgebra_2009,Kappé_Brunet_Silva_Wagemaker_Zanasi_2020,Kappé_Brunet_Silva_Zanasi_2018}, 
probabilistic systems~\cite{mciver_UsingProbabilisticKleene_2006, McIver_Rabehaja_Struth_2011}, 
relational verification~\cite{Antonopoulos_Koskinen_Le_Nagasamudram_Naumann_Ngo_2022},
and program schematology~\cite{Angus_Kozen_2001}.
Notice that despite the decidability of KA and KAT, the decision procedure is PSPACE-complete~\cite{Cohen_Kozen_Smith_1999}, which should be infeasible for large systems.
However, in real-world testing, network systems based on KAT was able to out-perform state of the art network verifier~\cite{Smolka_Kumar_Kahn_Foster_Hsu_Kozen_Silva_2019}.
This phenomenon is explained by an efficient fragment of KAT named \emph{Guarded Kleene Algebra with Tests} (GKAT)~\cite{Smolka_Foster_Hsu_Kappé_Kozen_Silva_2020}. Concretely, the particular fragment of GKAT enjoys a more efficient automaton structure that enables fast equivalence checking.

\paragraph{Our Contributions}

\begin{itemize}
    \item Proved that KAT cannot encode incorrectness logic. Demonstrated the insufficiency of domain reasoning in KAT.
    \item Developed the system TopKAT, and proven it is sound and complete with respect to its trace/language interpretation and general relational model.
    \item Developed a new notion of reduction, which is based on homomorphism from the free algebra.
    \item Proved TopKAT is complete with respect to domain comparisons.
    \item Designed the syntax and semantic of CF-GKAT, enabling control-flow verification of \command{while}-programs.
    \item Designed the decision procedure of CF-GKAT based on GKAT automaton.
    \item Proving the decision procedure is sound and complete, and also enjoys nearly-linear time complexity, assuming the number of primitive tests are fixed.
\end{itemize}

\section{Technical Background}

\subsection{Extensions of Kleene algebra And Their Models}

A \emph{Kleene algebra} is an idempotent semiring with a star operation, written
$p^*$, that satisfies the following \emph{unfolding}, \emph{left induction},
and \emph{right induction} rules:
\[
    p^* = 1 + p p^* = 1 + p^* p, \\
    p r + q  \leq  r  \implies  p^* q  \leq  r, \\
    r p + q  \leq  r  \implies  q p^*  \leq  r;
\]
the ordering here is the conventional ordering in idempotent semirings: \(p  \leq  q  \triangleq  p + q = q.\)
It is known that the right-hand version of unfolding and induction rule 
can be removed while preserving the same equational theory~\cite{Kozen_Silva_2020}.
Yet, we will focus on the standard definition of KA in this paper.
\begin{lemma}\label{the: well known fact about KA}
    Following are well-known facts in Kleene algebra
    \begin{itemize}
        \item All the Kleene algebra operations preserve order.
        \item The following equations are true for the star operation:
              \[ p^*  \cdot  p^* = p^* \\ (p^*)^* = p^*.\]
    \end{itemize}
\end{lemma}

A Kleene algebra with tests (KAT) is a Kleene algebra with an embedded Boolean algebra,
where the conjunction, disjunction, and identities in the Boolean algebra coincide with 
the addition, multiplication, and the identities of Kleene algebra.  
We refer to elements of this embedded Boolean algebra as \emph{tests}.

Given an algebraic theory, we can construct its \emph{free model} 
over a finite set \( Σ \), 
called the \emph{alphabet}~\cite{burrisCourseUniversalAlgebra1981}.  
The free model consists of all the terms formed by \( Σ \) modulo  
provable equivalences of the algebra. The operations of the free model are obtained 
by lifting the term-level operations to equivalence classes.

The above construction can be extended to the case of KAT and TopKAT, 
suppose that we are given two disjoint finite sets $K$ (the
\emph{action alphabet}) and $B$ (the \emph{test alphabet}).  Elements of $K$ and
$B$ are called \emph{primitive actions and primitive tests}, respectively. 
KAT terms over the alphabet \(K, B\) are defined with the following grammar:
\[e  ≜  b  ∈  B  ∣  p  ∈  K  ∣  1  ∣  0  ∣  e_1 + e_2  ∣  e_1  ⋅  e_2  ∣  e^*  ∣  \overline{e_b},\]
where \(e_b\) does not contain primitive actions.
The \emph{free KAT} over \(K, B\), written $\KAT_{K,B}$, 
consists of terms over \(K, B\) modulo provable KAT equivalences.  
The tests of the free KAT are Boolean terms, i.e. terms formed by
primitive tests and Boolean operations modulo Boolean axioms.  A similar
construction applies to TopKAT, where an additional symbol \( ⊤ \) was added 
as the largest element in the theory; we denote the free TopKAT over $K,B$ as
\(\TopKAT_{K, B}\).  We sometimes omit the alphabets \(K\) and \(B\) when they
are irrelevant or can be inferred.

In the paper, we frequently consider terms modulo provable equalities, i.e. in the
context of its corresponding free model.  For example, given \(e_1, e_2  \in  \KAT\),
we will say \(e_1 = e_2\) when they are provably equal using the theory of KAT.
Although the free model seems trivial, it leads to simpler and more modular
proofs of some properties of algebraic theories, as we will see in~\Cref{chapter:TopKAT}.

Other important models that we will use in this paper are language (Top)KATs and
relational (Top)KATs, which we review here.  An \emph{atom} (short for ``atomic
test'') over a test alphabet \(B = \{b_{1}, b_{2},  … , b_{n}\}\) is a sequence of the form
\[\hat{b_{1}}  \cdot  \hat{b_{2}}  \cdot   \cdots   \cdot \hat{b_{n}} \text{ where } \hat{b_{i}}  \in  \{b_{i}, \bar{b_{i}}\}.\] 
We denote atoms as \( α ,  β ,  γ ,  … \) and the set of all atoms as \(\At\).

A \emph{guarded string} (or \emph{guarded word}) over \(K, B\) is an alternation 
between atoms and primitive actions that starts and ends in atoms: 
\[ \alpha _{0}p_{1} \alpha _{1}  \cdots  p_{n}  \alpha _{n} \text{ where } p_{i}  \in  K,  \alpha _{i}  \in  \At;\] 
where each action is ``guarded'' by an atom.
A guarded string is similar to a program trace. These traces record the initial, intermediate and final machine states observed during execution, as well as the actions that occurred between those states. % chktex 36
Because the value of the indicator variable matters only for control flow, we do not consider indicators to be part of the machine state; hence, machine states in a guarded word are drawn from $\At$.

\begin{example} 
    Let $B = \{ b₁, b₂ \}$ and $K = \{ p₁, p₂ \}$.
    Now the guarded word \[b₁ \overline{b₂} ⋅ p₁ ⋅ \overline{b₁} b₂ ⋅ p₂ ⋅ \overline{b₁} \overline{b₂}\] represents a program trace that starts out in a machine state where $b₁$ is true (but $b₂$ is not).
    The program then executes the action $p₁$, after which $b₂$ is true (but $b₁$ is not).
    Finally, the program goes on to execute the action $p₂$, and halts in a state where neither $b₁$ nor $b₂$ is true.
\end{example}

We denote the set of all guarded
strings over alphabet \(K, B\) as \(GS_{K, B}\), and we will omit the alphabet
\(K, B\) when it is irrelevant or can be inferred from context.  The notation
\( α  w\) denotes a guarded string starting with atom \( α \) with the rest of the string
\(w\); similarly, \(w  α \) denotes a guarded string that ends with atom \( α \) with
rest of the string being \(w\).

\begin{definition}[Language/trace KAT~\cite{Kozen_Smith_1997}]
  The \emph{language KAT} (also called ``\emph{trace KAT}'') over an alphabet \(K, B\) is
  denoted as \(\mathcal{G}_{K, B}\), or simply \(\mathcal{G}\) if no confusion can arise.

  The elements are sets of guarded strings (called \emph{guarded languages}), 
  and the tests are sets of atoms.
  The additive identity 0 is the empty set, and the multiplicative identity 1 is
  the set of all the atoms \(\At\).  The addition operator is set union, and the
  multiplication operator is defined as follows:
    \[W_{1}  ⋄  W_{2}  ≜  \{w_{1}  α  w_{2}  ∣  w_{1}  α   ∈  W_{1},  α  w_{2}  ∈  W_{2}\}.\]
    The star operation is defined non-deterministically 
    iterating the multiplication operator:
    \[W^*  ≜   ⋃_{i  ∈ ℕ} W^i \text{ where } W^0 = \At, W^{k+1} = W  ⋄  W^k.\]
\end{definition}


Another useful type of KAT are relational ones, where each element is a relation
\(R  \subseteq  X  \times  X\) over a fixed set \(X\).  In applications, the set $X$ typically
represents the set of all possible program states, and each relation $R$
represents a program by relating each possible input to the corresponding
output.

\begin{definition}[Relational KAT]
  A relational KAT is a KAT $\mathcal{R}$ consists of relations over a fixed set \(X\) 
  (though $\mathcal{R}$ need not contain every relation over $X$),
  and it is closed under the following operations. 
  The tests are all the relations that are subsets of the identity relation.  
  The additive identity 0 is the empty set, and
  the multiplicative identity is the identity relation:
  \[1  \triangleq  \{(x, x)  \mid  x  \in  X\}.\] The addition operator is set union, and the
  multiplication operation is relational composition:
  \[R_{1} ; R_{2} = \{(x, z)  \mid   \exists  y  \in  X, (x, y)  \in  R_{1}, (y, z)  \in  R_{2}\}.\] 
  Finally, the star operation is defined as:
  \[R^*  \triangleq   \bigcup _{i  \in  \mathbb{N}} R^i \text{ where } R^0 = 1, R^{k+1} = R ; R^k.\] We denote the
  class of all relational KATs as \(\REL\).
\end{definition}

We are also interested in maps between models:
A \emph{KAT homomorphism} \(h\) is a map between two KATs \(\mathcal{K}\) and \(\mathcal{K}'\)
s.t. it preserves the sorts and operations:
given a test \(b\) in \(\mathcal{K}\) then \(h(b)\) is a test in \(\mathcal{K}'\);
and all the KAT operations (complement, identities, addition, multiplication, and star) are preserved:
\begin{align*}
    h & : 𝒦  →  𝒦'\\
    h(b̄) & = \overline{f(b)} \\  
    h(1) & = 1 \\  
    h(0) & = 0 \\
    h(p + q) & = h(p) + h(q) \\  
    h(p ⋅ q) & = h(p) ⋅ h(q) \\  
    h(p^*) & = h(p)^*.
\end{align*}

\subsection{Interpretation, Completeness, and Injectivity}\label{sec: completeness background}

Consider a KAT equation such as \(p ⋅ b ⋅ b̄ = 0\). To determine its
validity in a particular KAT \(\mathcal{K}\), we need to assign meaning to it by
interpreting each primitive as an element in \(\mathcal{K}\); that is, by defining a map
\(\hat{I}\) of type \(K + B  →  𝒦\).  Such a map \(\hat{I}: K + B  →  𝒦\) induces a
unique KAT homomorphism \(I : \KAT_{K,B}  →  \mathcal{K}\) inductively defined on the term 
as follows:
\begin{equation}
    \begin{aligned}
        I(p)       &  \triangleq  \hat{I}(p)    & \text{ where } p  \in  K + B \\
        I(\overline{e_b}) &  \triangleq  \overline{I(e_b)} 
            & \text{\(e_b\) does not contain primitive actions} \\
        I(e_1 + e_2) &  \triangleq  I(e_1) + I(e_2)                     \\
        I(e_1  \cdot  e_2) &  \triangleq  I(e_1)  \cdot  I(e_2)                     \\
        I(e^*)     &  \triangleq  I(t)^*
    \end{aligned}
\end{equation}
In fact, every KAT homomorphism from a free model arises this way: there is a
bijection between functions of type \(K + B  \to  \mathcal{K}\) and KAT homomorphisms of type
\(\KAT_{K, B}  \to  \mathcal{K}\), for any KAT \(\mathcal{K}\).  
Because the homomorphism \(I\) and the function \(\hat{I}\) are equivalent, 
we will refer to them interchangeably as \emph{KAT interpretations} 
and denote both of them as \(I\).

The above result enables us to define a homomorphism from the free KAT just by
defining its action on the primitives; saving us time to check the equations
that a homomorphism must satisfy.  It also allows us to prove that two
interpretations are equal by arguing that they map the primitives to
equal values.

Given a KAT \(𝒦\), and two terms \(e_1, e_2  ∈  \KAT_{K, B}\) we say that \(𝒦  ⊧  e_1 = e_2\) if
\[ ∀  I : \KAT_{K, B}  →  𝒦, I(e_1) = I(e_2).\] In particular, 
for two terms in the free model \(e_1, e_2  ∈  \KAT_{K, B}\),
\(\KAT_{K, B}  ⊧  e_1 = e_2\) is equivalent to \(e_1 = e_2\).  
For a collection of models \(\mathsf{K}\), 
we say that \(𝖪  ⊧  e_1 = e_2\) if for all \(\mathcal{K}  ∈  𝖪\),
\(𝒦  ⊧  e_1 = e_2\).  For example, \(\REL  ⊧  e_1 = e_2\) means that \(e_1 = e_2\) is
valid in all relational KATs.  All the above notations and terminologies can be
similarly extended to TopKAT.

Theories like KAT and TopKAT are designed to model practical
programs, so it is important to know if they can model all the desirable
equations between programs. If the theory of KAT can derive all the equalities
for a particular interpretation \(I\), namely:
\[\KAT_{K, B}  \models  e_1 = e_2  \iff  I(e_1) = I(e_2),\]
we say that the theory of KAT is \emph{complete} with respect to \(I\).
Recall that \(\KAT_{K, B}  \models  e_1 = e_2\) is equivalent to \(e_1 = e_2\);  
thus, by definition, an interpretation \(I\) is complete if and only if it is injective.
One of such interpretation is the guarded string interpretation
\(G: \KAT_{K, B}  →  𝒢_{K, B}\)~\cite{Kozen_Smith_1997},
defined by lifting the following action on the primitives:
\[
    G(b) = \{ α   ∣  \text{\(b\) appears positively in \( α \)}\}, \\
    G(p) = \{ α  p  β   ∣   α ,  β   ∈  \At\}.
\]

In several previous works, the term ``free model'' refers to the range (set of reachable elements) of a complete interpretation.  
Since a complete interpretation is an injective homomorphism, such interpretation induces an isomorphism on its range, thus our definition of free model is equivalent to these definitions.

Many previous proofs can also be explained by seeing complete interpretations as injective homomorphisms: the proof for completeness of relational KATs constructs an injective homomorphism $h$ from a language KAT into a relational
KAT~\cite{Kozen_Smith_1997}.  
Since both \(G\) and \(h\) are injective homomorphisms, \(h  ∘  G\) is also an injective homomorphism, hence a complete interpretation.  Since \(h  ∘  G\) is a relational interpretation:
\[\KAT_{K, B}  ⊧  e_1 = e_2  ⟹  \REL  ⊧  e_1 = e_2  ⟹  h  ∘  G(e_1) = h  ∘  G(e_2);\]
then the completeness of \(h  ∘  G\) implies
\((h  ∘  G)(e_1) = (h  ∘  G)(e_2)  ⟺  \KAT_{K, B}  ⊧  e_1 = e_2\). Hence,
\[\KAT_{K, B}  ⊧  e_1 = e_2  ⟺  \REL  ⊧  e_1 = e_2,\]
i.e. the theory of KAT is complete with respect to relational KAT.

% Function/homomorphism composition will be used frequently in this paper,
% but sometimes simply writing \(f  \circ  g\) will be confusing without knowing
% the domain and codomain of function \(f\) and \(g\).
% Therefore, we will use the notation \(X \xrightarrow{g} Y \xrightarrow{f} Z\)
% to denote the composition of \(f: Y  \to  Z\) and \(g: X  \to  Y\) when it is desirable.

% AAA: I think the above notation is standard enough; we don't have to explain
% what it means.
% CZ: Removed

Besides using composition of injective homomorphisms, another technique commonly
used to prove injectivity is to construct a left inverse: 
if a (Top)KAT homomorphism \(f: 𝒦  →  𝒦'\) has a left inverse homomorphism \(g: 𝒦'  →  𝒦\) 
i.e. \(g  ∘  f = id_{𝒦}\), then \(f\) is injective.  
Notice that \(g\) does not need to be a homomorphism for \(f\) to be injective,
however, in the case where \(f\) is an interpretation, 
\(g\) being a homomorphism makes the equality \(g  ∘  f = id_{𝒦}\) easier to check.
Because both \(g  ∘  f\) and \(id_{𝒦}\) are all interpretations,
they are equal if and only if they have the same action on all the primitives.
% AAA: This is only true because

\section{Guarded Kleene Algebra With Tests}

Guarded Kleene Algebra with Tests (GKAT)~\cite{Smolka_Foster_Hsu_Kappé_Kozen_Silva_2020} is a efficient fragment of Kleene Algebra with tests. Specifically, GKAT uses tests to ``guard'' the non-deterministic operations like \(+\) and \(*\), to make it more akin to simply \command{while}-programs. 
A GKAT expression over an alphabet \(K, B\) is defined as follow:
\begin{align*}
    \BExp ∋ e_b, f_b & ≜ 
        b ∈ B ∣ 1 ∣ 0 ∣ e_{b} ∧ f_{b} ∣ e_{b} ∨ f_{b} ∣ \overline{e_{b}} ; \\
    \GKAT ∋ e, f & ≜ 
        p ∈ K ∣ e_b ∣ e ⋅ f ∣  \comITE{e_b}{e}{f} ∣ \comWhile{e_b}{e} .
\end{align*}
Sometimes, like in~\cref{tab: thompson's construction}, we will use the compact notation \(e +_{e_b} f\) for \(\comITE{e_b}{e}{f}\) and \(e^{(e_b)}\) for \(\comWhile{e_b}{e}\).
In fact, these new \command{if}-statement and \command{while}-loop can be embedded back into KAT in the usual manner~\cite{Kozen_1997}:
\begin{align*}
    \comITE{e_b}{e}{f} & ≜ e_b ⋅ e + \overline{e_b} ⋅ f \\  
    \comWhile{e_b}{e} & ≜ (e_b ⋅ e)* ⋅ \overline{e_b}
\end{align*}
This embedding into KAT gives Guarded Kleene Algebra with Tests a natural trace semantics: by computing the trace of the underlying KAT term; and a PSPACE decision procedure for trace-equivalence. Because the trace semantics of GKAT is obtained by the embedding into KAT we will overload the notation \(G(e)\) as the trace interpretation of a GKAT expression \(e\). 

The main advantage of GKAT, compare to KAT, is its efficient decision procedure. Such decision procedure is enabled by its deterministic nature, namely all the non-deterministic operations are guarded by tests. Thus, endowing GKAT a different automaton structure.

\begin{definition}[GKAT automaton]\label{def:GKAT-automaton}
    A GKAT automaton \(A ≜ ⟨S, δ, ŝ⟩\) over an alphabet \(K, B\) consists of a state set \(S\), a transition function \(δ: S → 2 + K × S\), and a start state \(ŝ ∈ S\).
\end{definition}
where \(2\) denotes \(\{\accept, \reject\}\), which represents accept and reject respectively.
Intuitively, the transition function of a GKAT automaton tells us, given a state $q$ and an atom $α$ accounting for the truth value of each primitive test, to either \emph{reject} the input, represented by $δ(q, α) = \reject$, \emph{accept} the input, represented by $δ(q, α) = \accept$, or to \emph{transition} to a new state in $S$ after executing an action from $K$, represented by $δ(q, α) ∈ K × S$.

People familiar with coalgebra~\cite{jacobs_IntroductionCoalgebraMathematics_2016,rutten_UniversalCoalgebraTheory_2000} might realize that this definition is a pointed coalgebra over the following \emph{signature} (or \emph{dynamics}):
\[2 + K × (-).\]

A GKAT automaton induces a guarded language in a fairly straightforward manner.
\begin{definition}
 Given a GKAT automaton $A ≜ ⟨ S, δ, \hat{s} ⟩$, we define $G( - )_A: S → 𝒢$ as the (pointwise) smallest function satisfying the following rules for all $s ∈ S$ and $α ∈ \At$:
 \begin{mathpar}
  \inferrule{%
   δ(s, α) = \accept
  }{%
   α ∈ G( s )_A
  }
  \and
  \inferrule{%
   δ(s, α) = (p, s') \\
   w ∈ G( s' )_A
  }{%
   αpw ∈ G( s )_A
  }
 \end{mathpar}
 Finally, we define the guarded language semantics of $A$ by setting $G( A ) = G( \hat{s} )_A$.
\end{definition}

\citet{Smolka_Foster_Hsu_Kappé_Kozen_Silva_2020} was able to design a efficient trace equivalence decision procedure based on the classical Hopcroft-Karp algorithm~\cite{hopcroft_LinearAlgorithmTesting_1971}

\begin{theorem}[Decidability for GKAT~\cite{Schmid_Kappé_Kozen_Silva_2021}]
    Given two finite GKAT automata $A₀$ and $A₁$, it is decidable whether they represent the same guarded language, i.e., whether $G( A₀ ) = G( A₁ )$.
    The algorithm to do this has a complexity that is nearly-linear\footnote{$𝒪(\hat{α}(n))$, where $\hat{α}$ is the inverse Ackermann function; c.f.~\cite{Tarjan75}.} in the total number of states.
\end{theorem}

When given two expression \(e\) and \(f\), the decision procedure will first convert it into two trace equivalent automaton via the \emph{Thompson's Construction}~\cite{Smolka_Foster_Hsu_Kappé_Kozen_Silva_2020}. Concretely, two GKAT automata \(A_e\) and \(A_f\) s.t.
\[G(A_e) = G(e); \qquad G(A_f) = G(f).\]
Then, apply the efficient decision procedure to decide the trace-equivalence of \(A_e\) and \(A_f\)




\cleardoublepage

\chapter{Kleene Algebra with Atomic Commutativity}
\label{chapter:body}
\thispagestyle{myheadings}

% set this to the location of the figures for this chapter. it may
% also want to be ../Figures/2_Body/ or something. make sure that
% it has a trailing directory separator (i.e., '/')!
\graphicspath{{2_Commutativity/Figures/}}

\section{Free KA with Atomic Commutativity}

It is common to extend Kleene Algebra with additional equations 
to enrich the theory\cite{DBLP:conf/fossacs/DoumaneKPP19, Kozen_Mamouras_2014, Pous_Rot_Wagemaker_2022}.
In this paper we will consider atomic commutativity hypotheses,
where the equations in the hypotheses are of the form \(p q = q p\)
with \(p\) and \(q\) being primitives.

A \emph{commutable set} \((X, ∼)\) is a set with a reflexive symmetric relation \(∼: X × X\)
called \emph{commuting relation},
we typically omit \(∼\) and just denote the commutable set as \(X\).
In this paper we only consider finite commutable sets.

We say a commutable set \(X\) is \emph{discrete} if the relation \(∼\) is the identity relation.
A homomorphism \(h: X → Y\) between two commutable set \(X\) and \(Y\)
is a function that preserves the commuting relation:
\[x₁ ∼ x₂ ⟹ h(x₁) ∼ h(x₂).\]
The carrier of a commutable set \(X\) can be considered as a discrete commutable set,
and we denote this discrete commutable set as \(X_≁\).
There is a canonical homomorphism:
\begin{align*}
  [-]_∼ & : X_≁ → X \\  
  [x]_∼ & ≜ x.
\end{align*}

We can construct the free KA a commutable set \(X\) by 
taking all the KA terms over \(X\) modulo the equalities provable from KA axioms plus
the following equations \(\{p q = q p ∣ p ∼ q\}\).
Intuitively, the commuting relation of \(X\) specifies 
the atomic commutativity hypotheses in \(\KA(X)\).
Since the free KA over a set is just a free KA over discrete commutable set,
we abuse the notation to denote the free KA over a commutable set \(X\) as \(\KA(X)\).

Notice all Kleene Algebra form a commutable set, 
with the commuting relation defined as follows:
\[e₁ ∼ e₂ ⟺ e₁ ⋅ e₂ = e₂ ⋅ e₁.\]
We can show that the free KA over commutable set enjoys 
similar universal property as free KA over set.
We first prove the universal property without the uniqueness requirement:

\begin{theorem}\label{the: existence of KA with commutable set lifting}
  For all commutable set \(X\), a KA \(𝒦\), and a commutable set homomorphism \(Î: X → 𝒦\),
  then there is a KA interpretation \(I: \KA(X) → 𝒦\), s.t. the following diagram commutes:
  \[\begin{tikzcd}
    \KA(X) \ar[dashed]{r}{I} & 𝒦 \\ 
    X \ar[swap]{ur}{Î} \ar[hookrightarrow]{u}{i}
  \end{tikzcd}\]
\end{theorem}
\begin{proof}
  Given the function \(Î\), 
  we can apply the standard technique to generate the homomorphism \(I\) 
  by induction on the input:
  \begin{align*}
    I(a) & ≜ Î(a) & a ∈ X \\  
    I(1) & ≜ 1_𝒦 \\  
    I(0) & ≜ 0_𝒦 \\  
    I(e₁ + e₂) & ≜ I(e₁) + I(e₂) \\
    I(e₁ ⋅ e₂) & ≜ I(e₁) ⋅ I(e₂) \\
    I(e^*) & ≜ I(e)^*
  \end{align*}
  such a homomorphism exists, and makes the diagram commute:
  \[I(a) = Î(a), ∀ a ∈ X.\]
\end{proof}

To prove uniqueness, we will prove a stronger theorem first.
\begin{theorem}\label{the: ordered of homomorphism is determined by order on the primitives}
  Given two interpretation \(I, I': \KA(X) → 𝒦\),
  \[I(e) ≥ I'(e) ⟺ ∀ a ∈ X, I(a) ≥ I'(a),\]
  this result implies \(I(e) = I'(e) ⟺ ∀ a ∈ X, I(a) = I'(a).\)
\end{theorem}

\begin{proof}
  By induction on the structure of \(e\), and all KA operations preserve order.
  We show the star case as example: assume \(I(e) ≥ I'(e)\),
  we need to show \(I(e^*) ≥ I'(e^*)\).
  Since \(I\) is a homomorphism, and star preserves order:
  \[I(e^*) = (I(e))^* ≥ (I'(e))^* = I'(e^*).\]
\end{proof}

\begin{corollary}[Universal Property]
  For all commutable set \(X\), a KA \(𝒦\), and a commutable set homomorphism \(Î: X → 𝒦\),
  then there is a unique KA interpretation \(I: \KA(X) → 𝒦\), s.t. the following diagram commutes:
  \[\begin{tikzcd}
    \KA(X) \ar[dashed]{r}{I} & 𝒦 \\ 
    X \ar[swap]{ur}{Î} \ar[hookrightarrow]{u}{i}
  \end{tikzcd}\]
\end{corollary}

\begin{proof}
  By \Cref{the: existence of KA with commutable set lifting}, \(I\) exists. 
  By \Cref{the: ordered of homomorphism is determined by order on the primitives},
  if there exists another interpretation \(I'\) that makes the diagram commute,
  then \[I(a) = I'(a), ∀ a ∈ X ⟹ I(e) = I'(e).\]
\end{proof}
As usual, we will use the notation \(I\) for both \(I\) and \(Î\).

The words over a commutable set \(X\) are monoid terms modulo
monoid equations plus the commutativity axioms \(\{a b = b a ∣ a ∼ b\}\).
We still use \(ϵ\) as the identity of the monoid and call it the empty word;  
and we use the same notation \(\Word(X)\) for all the words over \(X\).
The language model over a commutable set \(X\) is the powerset of 
all words over \(X\), with operation defined by Kozen~\cite{DBLP:conf/lics/Kozen97},
denoted as \(ℒ_X\).
The language interpretation is generated by the same action on primitives
as in Kleene Algebra:
\begin{align*}
  L & : X → ℒ_X \\
  L & (a) = \{a\}
\end{align*}


\paragraph*{Notation}
In the rest of the article, notations \(\KA(X)\), \(\Word(X)\), and \(ℒ_X\)
always refers to the commutative variant, where \(X\) is a commutable set.
When we are referring to the non-commutative KA (word, language model, etc.), 
we will consider them as the KA (word, language model, etc.) over a discrete commutable set.
As we have mentioned before, \(2 ≜ \{0, 1\}\)
denotes the unique KA that only contains two distinct identities;  
this KA is also the free KA generated by the empty set \(\KA(∅)\).
Finally, when given a finite set of terms \(S ⊆ \KA(X)\), 
we will sometimes use \(S\) to denote 
the sum of all its elements \((∑_{e ∈ S} e) ∈ \KA(e)\).


\section{Word Inhabitant Problem}

Given a commutable set,
we allow a word in \(ℒ_X\) to be implicitly coerced into \(\KA(X)\),
where we pick the multiplication operator in \(\KA(X)\) as the monoidal multiplication.

Then given a word \(w ∈ ℒ_X\) and a KA expression \(e ∈ \KA(X)\),
the word inhabitant problem is the following inequality: \[w ≤ e.\]
The problem is complete with language interpretation when:
\[w ∈ L(e) ⟺ w ≤ e.\] 
We will show that the word inhabitance problem is complete and decidable
in Kleene Algebra with atomic commutativity hypotheses.

The core technique of this section is to construct 
a sound empty word predicate \(E: \KA(X) → 2\) 
and derivative operation \(δₐ: \KA(X) → \KA(X)\).

\subsection{Empty Word Predicate}

In this section we will prove a stronger result than 
the soundness of empty word predicate:
\[∀ e ∈ \KA(X), e = E(e) + e',\]
where \(E: \KA(X) → 2\) is the empty word predicate 
on the free KA over any commutable set \(X\), and \(ϵ ∉ L(e')\). 
This result is obtained by decomposing using the following matrix model.

\begin{theorem}
  For any Kleene Algebra \(𝒦\), 
  matrix of the following shapes forms a Kleene Algebra:
  \[D_E(𝒦) ≜ \{\begin{bmatrix}
    p & q \\
    0 & p + q
  \end{bmatrix} ∣ p, q ∈ 𝒦\}.\]
\end{theorem}

\begin{proof}
  We will only need to show that matrix of this shape is closed under all the KA operations.

  The identities and addition are easy to verify. 
  So we will only focus on verifying the closure under multiplication and star operation.

  The multiplication case:
  \[
    \begin{bmatrix}
      p₁ & q₁ \\
      0 & p₁ + q₁
    \end{bmatrix} 
    \begin{bmatrix}
      p₂ & q₂ \\
      0 & p₂ + q₂
    \end{bmatrix} = 
    \begin{bmatrix}
      p₁ p₂ & p₁ q₂ + q₁ (p₂ + q₂) \\
      0 & (p₁ + q₁) (p₂ + q₂)
    \end{bmatrix}.
  \]
  Since \(p₁p₂ + p₁ q₂ + q₁ (p₂ + q₂) = (p₁ + q₁) (p₂ + q₂)\),
  these matrices are closed under multiplication.

  The star case:
  \[
    \begin{bmatrix}
      p & q \\
      0 & p + q
    \end{bmatrix}^* = 
    \begin{bmatrix}
      p^* & p^* q (p + q)^* \\
      0 & (p + q)^*
    \end{bmatrix}.
  \]
  With a standard theorem of KA \((p + q)^* = p^*(q p^*)^*\),
  we are able to derive the closure under star operation:
  \begin{align*}
    p^* + p^* q (p + q)^* 
      & = p^* + p^* q p^* (q p^*)^* \\ 
      & = p^* (1 + q p^* (q p^*)^*) \\  
      & = p^* (q p^*)^* = (p + q)^*
  \end{align*}
\end{proof}

Given any commutable set \(X\), consider the following matrix: 
\[D_E(\KA(X)) ∋ u_E ≜ \begin{bmatrix}
  1 & X X^* \\  
  0 & X^*
\end{bmatrix},\]
where \(X\) is a shorthand for the expression \((∑_{x ∈ X} x)\).
By simply unfolding the definition, we can verify that \(u_E ⋅ u_E = u_E\)
and \(u_E ≥ 1\).
Therefore, all the matrices less than \(u_E\) in \(D_E(\KA(X))\) forms a Kleene Algebra.
We denote this Kleene Algebra as \(D_E(\KA(X))_{u_E}\).

In order to decompose an arbitrary expression,
we will define an interpretation into \(D_E(\KA(X))_{u_E}\) 
by lifting the following actions
\begin{align*}
  I_E & : \KA(X) → D_E(\KA(X))_{u_E} \\
  I_E & (a) ≜ 
  \begin{bmatrix}
    0 & a \\  
    0 & a
  \end{bmatrix}.
\end{align*}
Because the projection \(π_{2,2}\) is a homomorphism,
then \(π_{2,2} ∘ I_E\) is an interpretation.
Recall that interpretation is uniquely determined by the action on the primitives,
and \[π_{2,2} ∘ I_E(a) = a, ∀ a ∈ X.\]
Therefore, for all term in \(e ∈ \KA(X)\),
the \(2,2\) component of \(I_E(e)\) is exactly \(e\) itself:
\[π_{2,2} ∘ I_E(e) = e.\]

Then we define the empty word predicate as follows:
\[E(e) ≜ π_{1,1}(I_E(e)), \quad e' ≜ π_{1,2}(I_E(e)).\]
By \Cref{the: diagonal image of free model is closed under sub KA},
and \(π_{1,1}(I_E(a)) = 0 ∈ 2\) for all primitives \(a\),
\[∀e ∈ \KA(X), E(e) = π_{1,1}(I_E(e)) ∈ 2 ⊆ \KA(X).\]
Therefore, we can treat \(E\) as a homomorphism of the type \(\KA(X) → 2\).

\begin{corollary}[empty word decomposition]\label{the: empty word decomposition}
  All expression \(e ∈ \KA(X)\) over a commutable set \(X\)
  can be decomposed in the following way:
  \[e = E(e) + e' \text{ where } ϵ ∉ L(e').\]
\end{corollary}

\begin{proof}
  Recall that 
  \[\begin{bmatrix}
    E(e) & e' \\  
    0 & e 
  \end{bmatrix} ≜ I_E(e).\]
  Since  \(I_E(e) ∈ D_E(\KA(X))\), we have 
  \[e = E(e) + e'.\]

  Furthermore, since elements in \(D_E(\KA(X))_{u_E}\) is bounded by \(u_E\),
  \[e' = π_{2,2}(I_E(e)) ≤ X X^*.\]
  Because \(ϵ ∉ L(X X^*)\), and \(l\) is a homomorphism,
  we conclude \(ϵ ∉ L(e) ⊆ L(X X^*)\).
\end{proof}

\begin{corollary}[Soundness Of Empty Word Property]
  Let \(E: ℒ_X → 2\) the empty word predicate on the language over a commutable set
  \(E(l) = ϵ ∈ l,\) then the following diagram commute
  \[\begin{tikzcd}
    \KA(K) \ar{r}{E} \ar{d}{l} & 2 \\  
    ℒ_K \ar[swap]{ur}{E}
  \end{tikzcd}\]
\end{corollary}

\begin{proof}
  We only need to prove that for all \(e ∈ \KA(X)\), 
  \[E(e) = 1 ⟺ ϵ ∈ l.\]
  We show this by case analysis on \(E(e)\):
  \begin{itemize}
    \item If \(E(e) = 1\), then 
      \[L(e) = L(E(e)) ∪ L(e') = \{ϵ\} ∪ L(e') ∋ ϵ.\]
    \item If \(E(e) = 0\), recall that \(ϵ ∉ L(e')\),
      \[L(e) = L(E(e)) ∪ L(e') = L(e') ∌ ϵ.\]
  \end{itemize}
\end{proof}

\subsection{Derivative}

Similar to the last section, 
the derivative operation will also be defined by a decomposition:
for all \(e ∈ \KA(X)\) and \(a ∈ X\),
\[e = a ⋅ δₐ(e) + ρₐ(e),\]
where the language interpretation for \(a ⋅ δₐ(e)\) and \(ρₐ(e)\) are disjoint.
This result will imply the soundness of derivative.

\begin{theorem}
  Given a KA \(𝒦\) and an element \(t ∈ 𝒦\),
  the following matrices form a KA:
  \[Dₜ(𝒦) = \{\begin{bmatrix}
    a & b & c \\
    0 & d & 0 \\  
    0 & 0 & d
  \end{bmatrix} ∣ d = a + b + t c, a t = t a\}\]
\end{theorem}

\begin{proof}
  We need to show that these matrices are closed under KA operations.
  The closure under identities and addition are trivial, 
  we only show the multiplication case and the star case.

  The multiplication case:
  \begin{align*}
    & \begin{bmatrix}
      p₁ & q₁ & r₁ \\
      0 & s₁ & 0 \\  
      0 & 0 & s₁
    \end{bmatrix}
    \begin{bmatrix}
      p₂ & q₂ & r₂ \\
      0 & s₂ & 0 \\  
      0 & 0 & s₂
    \end{bmatrix} \\
    ={} & \begin{bmatrix}
      p₁ p₂ & p₁q₂ + q₁s₂ & p₁ r₂ + r₁ s₂ \\
      0 & s₁ s₂ & 0 \\  
      0 & 0 & s₁ s₂
    \end{bmatrix}
  \end{align*}
  We verify that the equation is preserved:
  \begin{align*}
    s₁ s₂ & = (p₁ + q₁ + p r₁) ⋅ s₂ \\  
    & = p₁ s₂ + q₁ s₂ + p r₁ s₂ \\  
    & = p₁ (p₂ + q₂ + p r₂) + q₁ s₂ + p r₁ s₂ \\  
    & = p₁ p₂ + (p₁q₂ + q₁s₂) + p₁ p r₂ + p r₁ s₂ \\ 
    & = p₁ p₂ + (p₁q₂ + q₁s₂) + p(p₁ r₂ + r₁ s₂) 
  \end{align*}
  The last step uses the commutativity of \(t\) and \(p₁\).
  Then we verify the commutativity condition:
  \[(p₁ p₂) t = p₁ t p₂ = t (p₁ p₂).\]
  Hence, \(Dₜ(𝒦)\) is closed under multiplication.

  The star case:
  \[
    \begin{bmatrix}
      p & q & r \\
      0 & s & 0 \\  
      0 & 0 & s
    \end{bmatrix}^*
    = 
    \begin{bmatrix}
      p^*  & p^*  q s^* & p^*  r s^* \\
      0 & s^* & 0 \\  
      0 & 0 & s^*
    \end{bmatrix}
  \]
  the equation is preserved:
  \begin{align*}
    s^* & = (p + q + t c)^* \\  
    & = p^* ((q + pc) p^*)^*\\
    & = p^* (1 + (q + pc) p^* ((q + pc) p^* )^*) \\
    & = p^* (1 + (q + pc) s^*) \\ 
    & = p^* + p^* q s^* + p^* t r s^* \\
    & = p^* + p^* q s^* + t p^* r s^*
  \end{align*}
  The last line is by standard KA theorem:
  \[p t = t p ⟹ p^*  t = t p^* .\]
  The commutativity condition \(p^*  t = t p^*\) is also implied by the above theorem.
  Therefore, \(Dₜ(𝒦)\) is closes under star operations.
\end{proof}

Given a commutable set \(X\), and an element \(a ∈ X\),
we can partition the rest of the elements in \(X\)
by whether they commute with \(a\):
\[X_{∼ a} ≜ \{b ∣ b ∼ a, b ≠ a\}, \quad X_{≁ a} = \{b ∣ b ≁ a\}.\]
Since \(a\) commutes with every element of \(X_{∼ a}\), 
\(a\) commutes with \(X_{∼ a}\): \(X_{∼ a} ⋅ a = a ⋅ X_{∼ a}\),
then by standard theorem of KA: \[X_{∼ a}^* ⋅ a = a ⋅ X_{∼ a}^*.\]

Consider the following matrix:
\[Dₐ(\KA(X)) ∋ uₐ = 
\begin{bmatrix}
  X_{∼ a}^* & X_{∼ a}^* X_{≁ a} X^* & X^* \\  
  0 & X^* & 0 \\  
  0 & 0 & X^*
\end{bmatrix}.\]
It is easy to verify that \(uₐ ≥ 1\) and \(uₐ ⋅ uₐ ≤ uₐ\).
Therefore, the elements under \(uₐ\) forms a KA: \(Dₐ(\KA(X))_{uₐ}\).
The purpose of model \(Dₐ(\KA(X))_{uₐ}\) is clear when we look at the 
language interpretation for each of the component,
let \[\begin{bmatrix}
  p & q & r \\  
  0 & s & 0 \\  
  0 & 0 & s 
\end{bmatrix} ∈ Dₐ(\KA(X))_{uₐ}\]
then 
\begin{itemize}
  \item \(L(p) ≤ L(X_{∼a}^*)\) contains only words with symbols that commutes with primitive \(a\),
    but is not \(a\).
  \item \(L(q) ≤ L(X_{∼a}^* X_{≁a} X^*)\) contains words that starts with 
    arbitrary number of primitives that commutes with \(a\), 
    then a primitive that does not commute with \(a\), followed by arbitrary primitives.
\end{itemize}
Both \(L(p)\) and \(L(q)\) do not contain words of the form \(a ⋅ w\) 
for any word \(w ∈ \Word(X)\); by the property of \(Dₐ(\KA(X))\):
\[L(s) = L(p) + L(q) + a ⋅ L(r).\]
Thus, \(L(r)\) will be the language derivative of \(L(s)\) with respect to primitive \(a\), 

To apply this decomposition on an arbitrary expression, we define an
interpretation by lifting the following action on primitives:
\begin{align*}
  I_a & : X → Dₐ(\KA(X))_{uₐ} \\
  I_a & (b) ≜ \begin{cases}
    \begin{bmatrix}
      b & 0 & 0 \\
      0 & b & 0 \\
      0 & 0 & b 
    \end{bmatrix} & b ∈ X_{∼ a} \\  
    \begin{bmatrix}
      0 & b & 0 \\
      0 & b & 0 \\
      0 & 0 & b 
    \end{bmatrix} & b ∈ X_{≁ a} \\
    \begin{bmatrix}
      0 & 0 & 1 \\
      0 & b & 0 \\
      0 & 0 & b 
    \end{bmatrix} & b = a 
  \end{cases}
\end{align*}

Again, since \(π_{3,3}\) and \(π_{2,2}\) are homomorphisms, 
\(π_{3,3} ∘ I_a\) and \(π_{2,2} ∘ I_a\) are interpretations.
Because interpretations are uniquely determined by the action on the primitives, 
and 
\[∀ b ∈ X, π_{3,3} ∘ I_a(b) = π_{2,2} ∘ I_a(b) = b,\]
then \(π_{3,3} ∘ I_a\) and \(π_{2,2} ∘ I_a\) are both identity homomorphisms.
This means the \(2,2\) and \(3,3\) component of \(I_a(e)\) are exactly \(e\) 
for all \(e ∈ \KA(X)\). Let 
\[\begin{bmatrix}
  p & q & r \\  
  0 & e & 0 \\  
  0 & 0 & e 
\end{bmatrix} ≜ Iₐ(e) ∈ Dₐ(\KA(X))_{uₐ}.\]
We can define the derivative \(δₐ\) and residual \(ρₐ\) as follows:
\[δₐ(e) ≜ r, \quad ρₐ(e) ≜ p + q.\]
Then the following corollary can be derived simply from 
the definition of \(Dₐ(\KA(X))_{uₐ}\).

\begin{corollary}[decomposition]\label{the: decomposition of derivative and residual}
  For all expressions \(e ∈ \KA(X)\), primitives \(a ∈ X\),
  and word \(w ∈ ℒ_X\),
  \[e = ρₐ(e) + a ⋅ δₐ(e) \text{ and } a ⋅ w ∉ L(ρₐ(e)).\]
\end{corollary}

\begin{theorem}[Soundness Property]
  For a primitive \(a\) in a commutable set \(X\),
  let the derivative on language \(δₐ\) defined as \(δₐ(l) ≜ \{s ∣ a ⋅ w ∈ l\}\),
  the following diagram commute:
  \[\begin{tikzcd}
    \KA(X) \ar{r}{δₐ} \ar{d}{L} & \KA(X) \ar{d}{L} \\  
    ℒ_X \ar{r}{δₐ} & ℒ_X 
  \end{tikzcd}\]
\end{theorem}

\begin{proof}
  Given any word \(w ∈ ℒ_X\) and in \(e ∈ \KA(X)\):
  \begin{align*}
    & s ∈ δₐ(L(e)) \\  
    & ⟺ a ⋅ w ∈ L(e) & \text{by definition of language \(δₐ\)}\\  
    & ⟺ a ⋅ w ∈ L(ρₐ(e) + a ⋅ δₐ(e)) & \text{\Cref{the: decomposition of derivative and residual}} \\  
    & ⟺ a ⋅ w ∈ L(a ⋅ δₐ(e)) & \text{\(a ⋅ w ∉ L(ρₐ(e))\)}\\  
    & ⟺ s ∈ L(δₐ(e)).
  \end{align*}
  Thus, for all \(e ∈ \KA(X)\), \(δₐ(L(e)) = L(δₐ(e))\), we have reached our conclusion.
\end{proof}

Finally, we prove a Galois connection that the derivative is expected to satisfy.

\begin{lemma}[Basic Algebraic Properties]
  Following basic algebraic properties are true,
  for all primitive \(a\) and expressions \(e, e'\):
  \begin{align*}
    δₐ(a e) & = e \\
    ρₐ(a e) & = 0 \\  
    δₐ(ρₐ(e)) & = 0 \\
    e ≥ e' & ⟹ δₐ(e) ≥ δₐ(e') \\  
    e ≥ e' & ⟹ ρₐ(e) ≥ ρₐ(e') 
  \end{align*}
\end{lemma}

\begin{proof}
  We first compute \(Iₐ(a e)\):
  \[Iₐ(a e) = Iₐ(a) ⋅ Iₐ(e) = 
  \begin{bmatrix}
    0 & 0 & 1 \\  
    0 & a & 0 \\  
    0 & 0 & a 
  \end{bmatrix}
  \begin{bmatrix}
    p & q & r \\  
    0 & e & 0 \\  
    0 & 0 & e 
  \end{bmatrix},
  \]
  for some expressions \(p, q, r ∈ \KA(X)\).
  Then 
  \[Iₐ(a e) = 
  \begin{bmatrix}
    0 & 0 & 1 \\  
    0 & a & 0 \\  
    0 & 0 & a 
  \end{bmatrix}
  \begin{bmatrix}
    p & q & r \\  
    0 & e & 0 \\  
    0 & 0 & e 
  \end{bmatrix}
  = \begin{bmatrix}
    0 & 0 & e \\  
    0 & a e & 0 \\
    0 & 0 & ae
  \end{bmatrix}.\]
  Therefore, we obtain the conclusion \(δₐ(a e) = e\) and \(ρₐ(a e) = 0\).

  Notice that \(ρ(e) ≤ X_{∼ a}^* + X_{∼ a}^* X_{≁ a} X^*\),
  therefore 
  \begin{align*}
    Iₐ(ρ(e)) & ≤ Iₐ(X_{∼ a}^* + X_{∼ a}^* X_{≁ a} X^*)\\ 
    & = \begin{bmatrix}
      X_{∼ a}^* & X_{∼ a}^* X_{≁ a} X^* & 0 \\  
      0 & X_{∼ a}^* + X_{∼ a}^* X_{≁ a} X^* & 0 \\
      0 & 0 & X_{∼ a}^* + X_{∼ a}^* X_{≁ a} X^*
    \end{bmatrix}
  \end{align*}
  Therefore \(Iₐ(ρ(e)) ≤ 0\), and since \(0\) is the smallest element,
  We obtain the conclusion \(Iₐ(ρ(e)) = 0\).

  The monotonicity can be derived from the monotonicity of \(Iₐ\).
  When \(e ≥ e'\), we have \(Iₐ(e) ≥ Iₐ(e')\). 
  Recall that the ordering on matrices are component order,
  since \(δₐ(e)\) and \(ρₐ(e)\) are 
  either component of \(Iₐ(e)\) or the sum of components of \(Iₐ(e)\),
  therefore \(δₐ(e) ≥ δₐ(e')\) and \(ρₐ(e) ≥ ρₐ(e')\).
\end{proof}

\begin{theorem}[Galois Connection]\label{the: Galois connection for derivative}
  Given a commutative set \(X\), for all expression \(e, e' ∈ \KA(X)\) and primitive \(a ∈ X\),
  \[a e ≤ e' ⟺ e ≤ δₐ(e').\]
\end{theorem}

\begin{proof}
  We first show \(a e ≤ e' ⟺ e ≤ δₐ(e')\):
  \(⟹\) direction can be proved by applying \(δₐ\) to both sides:
  \[a e ≤ e' ⟹ δₐ(a e) ≤ δₐ(e') ⟹ e ≤ δₐ(e').\]
  \(⟸\) direction proven by multiplying \(a\) on both sides:
  \[e ≤ δₐ(e') ⟹ a e ≤ a δₐ(e') ⟹ a e ≤ a δₐ(e') ≤ e'.\]
\end{proof}

\subsection{Decidability And Completeness}

In this section, we prove the completeness and decidability of the word inhabitance problem,
by explicitly define an algorithm to check for word inhabitance.

\begin{theorem}[Decidability and Completeness]\label{the: decidability and completeness of word inhabitant}
  Given a word \(w ∈ \Word(X)\) and an expression \(e ∈ \KA(X)\),
  we can define the following algorithm to test for inhabitants:
  \begin{align*}
    i & : \Word(X) × \KA(X) → 2 \\  
    i & (ϵ, e) ≜ E(e) \\  
    i & (a ⋅ w, e) ≜ i(w, δₐ(e))
  \end{align*}
  Such an algorithm will always terminate,
  and it is sound:
  \[w ∈ L(e) ⟺ i(w, e) = 1 ⟺ w ≤ e.\]
\end{theorem}

\begin{proof}
  The algorithm \(i\) will terminate because both \(δₐ\) and \(E\)
  can be computed by computing the interpretation \(Iₐ\) and \(I_E\).

  We first show \(w ∈ L(e) ⟺ i(w, e)\) by induction on \(w\).
  \begin{itemize}
    \item If \(w = ϵ\), then \(ϵ ∈ L(e) ⟺ E(e) = 1\) by soundness of \(E\).
    \item If \(w = a ⋅ w'\) then:
    \begin{align*}
      a ⋅ w' ∈ L(e) 
      & ⟺ w' ∈ δₐ(L(e)) 
        & \text{definition}\\  
      & ⟺ w' ∈ L(δₐ(e)) 
        & \text{soundness of \(δₐ\)} \\  
      & ⟺ i(w, δₐ(e))
        & \text{induction hypothesis}
    \end{align*}
  \end{itemize}

  We then show \(i(w, e) = 1 ⟺ w ≤ e\) by induction on \(w\).
  \begin{itemize}
    \item If \(w = ϵ\), then \(i(ϵ, e) = E(e)\).
      When \(E(e) = 1\), then \(1 = E(e) ≤ e\);  
      When \(E(e) = 0\), then \(1 ≰ e\), 
      because \(1 ≤ e\) is not true in the language interpretation.
    \item If \(w = a ⋅ w'\) then:
    \begin{align*}
      a ⋅ w' ≤ e 
      & ⟺ w' ≤ δₐ(e) 
        & \text{\Cref{the: Galois connection for derivative}} \\  
      & ⟺ i(w, δₐ(e))
        & \text{induction hypothesis}
    \end{align*}
  \end{itemize}
\end{proof}


\subsection{Fundamental Theorem}

Fundamental theorem is an important soundness condition 
for the definition of derivative and empty word predicate,
it also exhibits a strong connection between KA and automata~\cite{Silva_2010,Kozen_Silva_2020}.
Because of the significance of the fundamental theorem,
we decide to prove it for KA with atomic commutativity,
despite it is not used in the rest of the paper.

In order to show the fundamental theorem for KA with atomic commutativity,
we will establish the relation between derivative and empty word predicate in KA 
with their counterparts in KA with commutativity,
and with this relation, we can show that fundamental theorem KA implies 
the fundamental theorem in KA with commutativity

Recall that for all commutable set \(X\), 
we can construct a discrete commutable set \(X_≁\) 
by replacing the commuting relation in \(X\) with the identity relation.
Notice that \(\KA(X_≁)\) is a free KA, and 
by \Cref{the: uniqueness of E and delta in free KA}, 
derivative and empty word predicate is unique on free KAs.
Since our definition of \(E\) and \(δₐ\) are sound, 
therefore they are exactly the conventional \(E\) and \(δₐ\) when applied to a term 
in the free KA.
Therefore, the fundamental theorem holds for \(\KA(X_≁)\):
\[∀ e_≁ ∈ \KA(X_≁), e_≁ = E(e_≁) + ∑_{a ∈ X} a ⋅ δₐ(e_≁).\]

Finally, there is a canonical KA homomorphism from \(\KA(X_≁)\) to \(\KA(X)\),
by lifting the following action on the primitives:
\[[a]_∼ ≜ a.\]
This KA homomorphism imposes the commutativity of \(X\) to the input expression, 
and this homomorphism is surjective.

\begin{lemma}
  Consider a Kleene Algebra \(𝒦\) and an element \(t ∈ 𝒦\),
  there is a homomorphism:
  \begin{align*}
    h & : Dₜ(𝒦) → M_2(𝒦) \\
    h & (\begin{bmatrix}
      p & q & r \\
      0 & s & 0 \\
      0 & 0 & s
    \end{bmatrix}) = 
    \begin{bmatrix}
      p & r \\
      0 & s
    \end{bmatrix}
  \end{align*}
\end{lemma}

\begin{proof}
  Perseverance of identities and addition is trivial,
  we will only check for perseverance of multiplication and star

  The multiplication case:
  \begin{align*}
    & h(\begin{bmatrix}
      p₁ & q₁ & r₁ \\
      0 & s₁ & 0 \\  
      0 & 0 & s₁
    \end{bmatrix}
    \begin{bmatrix}
      p₂ & q₂ & r₂ \\
      0 & s₂ & 0 \\  
      0 & 0 & s₂
    \end{bmatrix}) \\
    ={} & h(\begin{bmatrix}
      p₁ p₂ & p₁q₂ + q₁s₂ & p₁ r₂ + r₁ s₂ \\
      0 & s₁ s₂ & 0 \\  
      0 & 0 & s₁ s₂
    \end{bmatrix}) \\
    = {}& \begin{bmatrix}
      p₁ p₂& p₁ r₂ + r₁ s₂ \\
      0 & s₁ s₂ 
    \end{bmatrix} \\
    = {}& h(\begin{bmatrix}
      p₁ & q₁ & r₁ \\
      0 & s₁ & 0 \\  
      0 & 0 & s₁
    \end{bmatrix}) ⋅
    h(\begin{bmatrix}
      p₂ & q₂ & r₂ \\
      0 & s₂ & 0 \\  
      0 & 0 & s₂
    \end{bmatrix}).
  \end{align*}

  The star case:
  \begin{align*}
    h(\begin{bmatrix}
      p & q & r \\
      0 & s & 0 \\  
      0 & 0 & s
    \end{bmatrix}^*)
    & = 
    h(\begin{bmatrix}
      p^*  & p^*  q s^* & p^*  r s^* \\
      0 & s^* & 0 \\  
      0 & 0 & s^*
    \end{bmatrix})\\
    & = \begin{bmatrix}
      p^* & p^*  r s^* \\
      0 & s^* 
    \end{bmatrix}
    = h(\begin{bmatrix}
      p & q & r \\
      0 & s & 0 \\  
      0 & 0 & s
    \end{bmatrix})^*.
  \end{align*}
\end{proof}

Since the derivative \(δₐ(e)\) is defined as the \(1, 3\) component of \(Iₐ(e)\),
after we apply above homomorphism \(h\) to \(Iₐ(e)\),
the derivative \(δₐ(e)\) becomes the \(1, 2\) component of the matrix:
\[∀ e ∈ \KA(X), δₐ(e) = π_{1, 2} (h(Iₐ(e))).\]

\begin{lemma}\label{the: lemma for fundamental thoerem}
  Let \(X\) be a commutable set, for all non-commutativity expressions \(e_≁ ∈ \KA(X_≁)\):
  \[E([e_≁]_∼) = [E(e_≁)]_∼, \quad δₐ([e_≁]_∼) ≥ [δₐ(e_≁)]_∼.\]
\end{lemma}

\begin{proof}
  Consider the following interpretations 
  \[I_E ∘ [-]_∼ \text{ and } [-]_∼ ∘ I_E: \KA(X_≁) → \KA(X).\]
  Their actions coincide on the primitives:
  \[∀ a ∈ X_≁,
  I_E([a]_∼) = \begin{bmatrix}
    0 & a \\ 
    0 & a 
  \end{bmatrix} = [E(a)]_∼.\]
  Therefore, by \Cref{the: ordered of homomorphism is determined by order on the primitives},
  \[∀ e_≁ ∈ \KA(X_≁), I_E([e_≁]_∼) = [I_E(e_≁)]_∼.\]
  Since \(E\) is a component of \(I_E\),
  \[E([e_≁]_∼) = [E(e_≁)]_∼.\]

  The same can be done for derivatives. 
  We consider the following interpretations:
  \[h ∘ I_a ∘ [-]_∼ \text{ and } [-]_∼ ∘ h ∘ I_a: \KA(X_≁) → \KA(X).\]
  Given a primitive \(b\), 
  \begin{itemize}
    \item if \(b = a\), then 
      \[∀ b ∈ X_≁,
      h(Iₐ([b]_∼)) = \begin{bmatrix}
        0 & 1 \\ 
        0 & b
      \end{bmatrix} = [h(Iₐ(b))]_∼;\]
    \item if \(b ≠ a\), then 
      \[h ∘ Iₐ([b]_∼) = \begin{cases}
        \begin{bmatrix}
          0 & 0 \\ 
          0 & b 
        \end{bmatrix} & b ≁ a \text{ in \(X\)}\\
        \begin{bmatrix}
          b & 0 \\ 
          0 & b 
        \end{bmatrix} & b ∼ a \text{ in \(X\)}
      \end{cases}\]
      both of which are greater than 
      \[[h(Iₐ(b))]_∼ = \begin{bmatrix}
        0 & 0 \\ 
        0 & b 
      \end{bmatrix} \quad \text{because \(b ≁ a\) in \(X_≁\)}.\]
  \end{itemize}
  Therefore, for all primitive \(b ∈ X_{≁}\), \(h(Iₐ([b]_∼)) ≥ [h(Iₐ(b))]_∼\).
  By \Cref{the: ordered of homomorphism is determined by order on the primitives}
  \[∀ e_≁ ∈ \KA(X_≁), h ∘ Iₐ([e_≁]_∼) ≥ [h ∘ Iₐ(e_≁)]_∼.\]
  Since \(δₐ\) is a component of \(h ∘ Iₐ\), 
  \[δₐ([e_≁]_∼) ≥ [δₐ(e_≁)]_∼.\qedhere\]
\end{proof}

The above lemma state that the derivative for commutative expression in \(\KA(X)\)
is always larger than their non-commutativity counterparts in \(\KA(X_≁)\).
Since \(\KA(X_≁)\) is a free KA, 
we can easily derive the fundamental theorem for \(\KA(X)\) 
using its connection with \(\KA(X_≁)\).

\begin{theorem}[Fundamental Thoerem]
  For all \(e ∈ \KA(X)\), the following equality holds:
  \[e = E(e) + ∑_{a ∈ X} a ⋅ δₐ(e)\]
\end{theorem}

\begin{proof}
  We first show \(e ≥ E(e) + ∑_{a ∈ X} a ⋅ δₐ(e)\),
  which is a direct consequence of decompositions in 
  \cref{the: decomposition of derivative and residual,the: empty word decomposition}:
  \begin{align*}
    e & ≥ E(e),\\ 
    e & ≥ a ⋅ δₐ(e), ∀ a ∈ X.
  \end{align*}

  We then show \(e ≤ E(e) + ∑_{a ∈ X} a ⋅ δₐ(e)\).
  Since \([-]_∼\) is a surjective homomorphism, 
  we consider \(e_≁ ∈ \KA(X_≁)\) s.t. \([e_≁]_∼ = e\).
  Because \(\KA(X_≁)\) is a free KA, fundamental theorem holds for \(e_≁\):
  \[e_≁ ≤ E(e_≁) + ∑_{a ∈ X} a ⋅ δₐ(e_≁).\]
  We can apply the homomorphism \([-]_∼\) to both sides:
  \[e ≤ [E(e_≁)]_∼ + ∑_{a ∈ X} a ⋅ [δₐ(e_≁)]_∼.\]
  By \Cref{the: lemma for fundamental thoerem},
  \begin{align*}
    [E(e_≁)]_∼ & = E([e_≁]_∼) = E(e), \\  
    [δₐ(e_≁)]_∼ & ≤ δₐ([e_≁]_∼) = δₐ(e).
  \end{align*}
  Thus, obtain the desired inequality:
  \[e ≤ [E(e_≁)]_∼ + ∑_{a ∈ X} a ⋅ [δₐ(e_≁)]_∼ ≤ E(e) + ∑_{a ∈ X} a ⋅ δₐ(e).\]
\end{proof}

\section{Undecidability}

In this section, we will show the undecidability result for general 
Kleene Algebra equalities with atomic commutativity hypotheses.
The undecidability result is obtained by using a proof to simulate 
the execution of a two-counter machine. 
From there, we can encode state reachability of terminating two-counter machines 
into an KA inequality. Enabling us to carry out a diagonal argument,
similar to the proof of undecidability of halting problem.

\subsection{Encoding Two-Counter Machines}

Counter machine is a well-studied machine~\cite{Minsky_1961,Minsky_1967,Lambek_1961},
and it can simulate any Turing machine~\cite[Theorem 14.1-1]{Minsky_1967} with just two counters.
In this paper, we only consider two-counter machines.
A two-counter machine \(M ≜ (S, ŝ, ι)\) consists of a finite set of \emph{state} \(S\), 
a start state \(ŝ ∈ S\), and each state is equipped with an instruction \(ι: S → I_S\),
where instructions \(I_S\) is defined as follows:
\begin{align*}
  I_S & ≜ \{ \Inc(s, q) ∣ s ∈ \{1,2\}, q ∈ Q \} \\
      & ∪ \{ \Dec(s, q) ∣ s ∈ \{1,2\}, q ∈ Q \} \\
      & ∪ \{ \If(s, q₁, q₂) ∣ s ∈ \{1,2\}, q₁,q₂ ∈ Q \} \\
      & ∪ \{ \Halt \}.
\end{align*}
Each instruction has a semantics, we define \(S_⊥ ≜ S + \{⊥\}\):
\begin{align*}
  ⟦i⟧ & : ℕ × ℕ → S_⊥ × ℕ × ℕ \\
  ⟦\Inc(1,s)⟧&(n, m) ≜ (s,n+1,m) \\
  ⟦\Inc(2,s)⟧&(n, m) ≜ (s,n,m+1) \\
  ⟦\Dec(1,s)⟧&(n, m) ≜ (s,\max(n-1,0),m) \\
  ⟦\Dec(2,s)⟧&(n, m) ≜ (s,n,\max(m-1,0)) \\
  ⟦\If(1,q₁,q₂)⟧&(n, m) ≜ 
    \begin{cases}
      (s₁,n,m) & \text{if $n = 0$} \\
      (s₂,n,m) & \text{if $n ≠ 0$}
    \end{cases} \\
  ⟦\If(2,s₁,s₂)⟧&(n, m) ≜ 
    \begin{cases}
      (s₁,n,m) & \text{if $m = 0$} \\
      (s₂,n,m) & \text{if $m ≠ 0$}
    \end{cases} \\
  ⟦\Halt⟧&(n,m) ≜ (⊥, n, m).
\end{align*}

From the semantics of the instruction, 
we can define a transition relation for any machine \(M\):
\begin{align*}
  R_M & ∈ (S × ℕ × ℕ) × (S_⊥ × ℕ × ℕ) \\  
  R_M & ≜ \{((s, m, n), ⟦ι(s)⟧(m, n)) ∣ s ∈ S; m, n ∈ ℕ\}.
\end{align*}
Note that \(R_M\) is a functional relation, 
that is for all input \((s, m, n) ∈ (S × ℕ × ℕ)\) there exists 
a unique element in \(S_⊥ × ℕ × ℕ\) relating to it.
We call the elements in \(S_⊥ × ℕ × ℕ\) \emph{configurations} of the machine \(M\).
Let \(R_M^*\) be the reflexive transition closure for \(R_M\),
and we write \(c →^* c'\) if \((c, c') ∈ R_M^*\).
We say a state \(s ∈ S\) is \emph{reachable} from input \((n, m)\)
when there exists \(n', m' ∈ ℕ\), s.t. \((ŝ, m, n) →^* (s, n', m')\).

Finally, we can consider a machine as a partial function,
we say that \(M(m, n)\) returns \((m', n')\) when \((ŝ, m, n) →^* (⊥, n', m')\).
Since complex data structure can be encoded as a pair of numbers 
using the classical Gödel numbers,
we will abuse the notation to say \(M(i)\) returns \(o\)
for input \(i\) and output \(o\) of arbitrary type, not just pairs of numbers.

Given a two-counter machine with finite state set \(S\),
We define the set \(Σ = S + \{⊥, a, b\}\) and the following commutable set \(Σ̈\):
\begin{itemize}
  \item the carrier is \(⟨Σ| ∪ |Σ⟩\), 
    where \[⟨Σ| ≜ \{σₗ ∣ σ ∈ Σ\} \text{ and } |Σ⟩ ≜ \{σᵣ ∣ σ ∈ Σ\};\]
  \item the commuting relation is \(⟨σ| ∼ |σ'⟩\), 
    where \(σ, σ' ∈ Σ\).
\end{itemize}
We call primitives in \(⟨Σ|\) \emph{left primitives} and 
primitives in \(|Σ⟩\) \emph{right primitives}.
This definition of commutativity is similar to BiKA~\cite{Antonopoulos_Koskinen_Le_Nagasamudram_Naumann_Ngo_2022},
however instead of using an underlying KA with two homomorphisms,
we simply impose the commutativity onto the primitives.
We consider \(Σ\) as a discrete commutable set,
therefore we can define the free Kleene algebra \(\KA(Σ)\) and 
the free KA with atomic commutative \(\KA(Σ̈)\).

There are three function we can define from \(Σ\) to \(\Word(Σ̈)\):
\begin{align*}
  ⟨-|: & Σ → \Word(Σ̈) &  
  |-⟩: & Σ → \Word(Σ̈) & 
  ⟨-⟩: & Σ → \Word(Σ̈) \\  
  ⟨σ| & ≜ σₗ & 
  |σ⟩ & ≜ σᵣ & 
  ⟨σ⟩ & ≜ σₗ ⋅ σᵣ.
\end{align*}

These maps can be lifted to monoidal homomorphism on words \(\Word(Σ̈) → \Word(Σ̈)\):
\(⟨w|\) is the word \(w\) with all primitives replaced by corresponding left primitives
and \(|w⟩\) is the word \(w\) with all primitives replaced by corresponding right primitives.

By composing \(⟨-|, |-⟩, ⟨-⟩\) with the natural monoidal embedding \(\Word(Σ̈) → \KA(Σ̈)\), 
where the monoidal operation of \(\KA(Σ̈)\) is multiplication with identity \(1\),
we can obtain functions in \(Σ → \KA(Σ̈)\).
Similarly, these maps can be lifted to KA homomorphisms \(\KA(Σ) → \KA(Σ̈)\):
\begin{itemize}
  \item \(⟨e|\) and \(|e⟩\) will replace all the primitives in \(e\)
    with their respective left primitives or right primitives.
  \item \(⟨e⟩\) will produce two expression with \emph{matching} left and right primitives. 
\end{itemize}
We will abbreviate the multiplication \(⟨ e₁ | ⋅ | e₂ ⟩\) as \(⟨ e₁ | e₂ ⟩\).

\begin{example}
  Consider \(a ∈ Σ\), then \(⟨a^*⟩ ∈ \KA(Σ̈)\) and 
  \[L(⟨a^*⟩) = \{⟨a^n | a^n ⟩ ∣ n ∈ ℕ\}.\]
  More generally, given an expression \(e ∈ \KA(Σ)\), 
  then \(⟨e⟩ ∈ \KA(Σ̈)\) and 
  \[L(⟨e⟩) = \{ ⟨w | w⟩ ∣ w ∈ L(e)\}.\]
\end{example}

For a word in \(\Word(Σ̈)\) we can always canonically separate it into 
its left and right components,
this separation gives a normal form for all the words in \(\Word(Σ̈)\).

\begin{definition}
  For a word \(w\), the \emph{left component} \(⟨wₗ|\) is the word formed 
  by all the left primitives in its original order,
  and the \emph{right component} \(|wᵣ⟩\) is the word formed 
  by all the right primitives in its original order.
  The concatenation of the left component and the right component 
  is equal to the original word.
  Therefore, for all word \(w, w' ∈ \Word(Σ̈)\),
  \[w = w' ⟺ wᵣ = wᵣ' ∧ wₗ = wₗ'.\]
\end{definition}

\begin{example}
  Consider the following word \[w ≜ ⟨ s | s' ⟩ ⋅ ⟨aᵐ bⁿ | a^{m+1} bⁿ ⟩,\]
  then its left component \(⟨wₗ| = ⟨s aᵐ bⁿ|\) 
  and the right component is \(|wᵣ⟩ = |s' a^{m+1} bⁿ⟩\).
  The concatenation of right and left component is equal to the original word:
  \[⟨wₗ | wᵣ ⟩ = ⟨s aᵐ bⁿ|s' a^{m+1} bⁿ⟩ = ⟨ s | s' ⟩ ⋅ ⟨aᵐ bⁿ | a^{m+1} bⁿ ⟩.\]
\end{example}

We first show couple useful lemmas about \(\KA(Σ)\) and \(\KA(Σ̈)\):
\begin{lemma}
  We first prove several lemmas that will be useful in our derivation.
  For all \(e, e₁, e₂, e₁', e₂' ∈ \KA(Σ)\) 
  \begin{align}
    ⟨ e₁ | e₂ ⟩ & = | e₂ ⟩ ⋅ ⟨ e₁ |; \\
    ⟨ e₁ | e₂ ⟩ ≤ ⟨ e₁' | e₂' ⟩ & ⟺ e₁ ≤ e₁' ∧ e₂ ≤ e₂'; 
      \label[ineq]{the: left right order decompose}\\
    ⟨e^*⟩ & ≤ ⟨e^*|e^*⟩. \label[ineq]{the: matching star less than left right star}
  \end{align}
\end{lemma}

\begin{proof}
  \(⟨ e₁ | e₂ ⟩ = | e₂ ⟩ ⋅ ⟨ e₁ |\) can be derived by induction on 
  structure of \(e₁\), the only non-trivial case is the star case,
  which can be proven using the induction rule.

  The equivalence
  \[⟨ e₁ | e₂ ⟩ ≤ ⟨ e₁' | e₂' ⟩ ⟺ e₁ ≤ e₁' ∧ e₂ ≤ e₂'\]
  can be derived as follows:
  The \(⟸\) part can be shown by multiplication preserves order.
  The \(⟹\) part can be shown by looking at the language interpretation:
  if \(L(e₁) ≰ L(e₁')\) or \(L(e₂) ≰ L(e₂')\),
  then we can derive
  \[L(⟨ e₁ | e₂ ⟩) ≰ L(⟨ e₁' | e₂' ⟩)\]

  \(⟨e^*⟩ ≤ ⟨e^*|e^*⟩\) can be shown by induction rule,
  because
  \begin{align*}
    ⟨e^*|e^*⟩ & ≥ 1,\\  
    ⟨e^*|e^*⟩ & ≥ ⟨e e^*| e e^*⟩ = ⟨e|e⟩ ⋅ ⟨e^*|e^*⟩ = ⟨e⟩ ⋅ ⟨e^*|e^*⟩,  
  \end{align*}
  by induction rule \(⟨e^*|e^*⟩ ≥ ⟨e^*⟩\).
\end{proof}


We can encode components of the machine as Kleene Algebra terms.
\begin{definition}\label{def:instruction-interpretation}
  For simplicity, we will implicitly coerce all the configuration 
  into an expression in \(\KA(Σ)\) in the following way:
  for \(s ∈ S_⊥\),
  \begin{align*}
    (s, m, n) & ↦ s aᵐ bⁿ.
  \end{align*}

  We interpret each instruction $i ∈ I_S$ as an element $[i] ∈ \KA(Σ̈)$ as follows.
  \begin{align*}
    [\Inc(1,s)] & ≜ |s⟩ ⟨a^*⟩ |a⟩ ⟨b^*⟩ \\
    [\Inc(2,s)] & ≜ |s⟩⟨a^* b^*⟩|b⟩ \\
    [\Dec(1,s)] & ≜ |s⟩⟨a^*⟩⟨a|⟨b^*⟩ + |s⟩⟨b^*⟩ \\
    [\Dec(2,s)] & ≜ |s⟩⟨a^* b^*⟩⟨b| + |s⟩⟨a^*⟩ \\
    [\If(1,s₁,s₂)] & ≜ |s₁⟩⟨b^*⟩ + |s₂⟩⟨a⁺⟩⟨b^*⟩ \\
    [\If(2,s₁,s₂)] & ≜ |s₁⟩⟨a^*⟩ + |s₂⟩⟨a^*⟩⟨b⟩⁺ \\
    [\Halt] & ≜ |⊥⟩ ⟨a^* b^*⟩.
  \end{align*}
  The transition relation is encoded as \(R_M ∈ \KA(Σ̈)\):
  \[R_M ≜ ∑_{s ∈ S} ⟨s| ⋅ [ι(s)].\]
\end{definition}

The reason for such encoding is apparent when we look the language interpretation:
\begin{corollary}[Language Soundness]\label{the: language encoding soundness of machine}
  Given a machine \(M ≜ (S, ŝ, ι)\), \(c ∈ S × ℕ × ℕ\), and \(c' ∈ S_⊥ × ℕ × ℕ\), 
  the following equivalence holds
  \begin{align}
    ⟦i⟧(m, n) = c' & ⟺ ⟨aᵐ bⁿ | c'⟩ ∈ L([i]);\label[equiv]{the: instruction language soundness}\\
    c → c' & ⟺ ⟨c | c'⟩ ∈ L(R_M).\label[equiv]{the: transition language soundness}
  \end{align}
\end{corollary}

\begin{proof}
  The equivalence
  \[⟦i⟧(m, n) = c' ⟺ ⟨aᵐ bⁿ | c' ⟩ ∈ L([i])\]
  can be shown by looking at each case of instruction \(i\), 
  and explicitly compute the language model for each instruction.

  Therefore,
  \begin{align*}
    & (s, m, n) → (s', m', n') \\
    & ⟺ (s', m', n') = ⟦ι(s)⟧(m, n) \\
    & ⟺ ⟨aᵐ bⁿ | s' a^{m'} b^{n'} ⟩ ∈ L([ι(s)]) \\  
    & ⟺ ⟨ s aᵐ bⁿ | s' a^{m'} b^{n'} ⟩ ∈ L(⟨s| ⋅ [ι(s)]) \\  
    & ⟺ ⟨ s aᵐ bⁿ | s' a^{m'} b^{n'} ⟩ ∈ L(R_M). \qedhere
  \end{align*}
\end{proof}

\begin{corollary}\label{the: transition encoding is functional}
  Because the transition relation of any machine \(M = (S, ŝ, ι)\) is functional,
  therefore for all word \(w ∈ L(S a^* b^*)\), 
  there exists a word \(w' ∈ L(S_⊥ a^* b^*)\)
  s.t. \(⟨w | w'⟩ ∈ L(R_M)\).
\end{corollary}


\subsection{From Reachability to Undecidability}

Our undecidability result relies on an equivalence between provability of 
a KA inequality and state reachability in a certain kind of machine.
In order to obtain such equivalence,
we will start with the provability of a single transition.

We will first define two useful terms.
For a machine \(M ≜ (S, ŝ, ι)\), and a subset of states \(S' ⊆ S\), 
we can define all the configuration for \(S'\) including termination:
\begin{align*}
  C_{S'} & ∈ \KA(Σ) \\
  C_{S'} & ≜ S'_⊥ a^*b^*.
\end{align*}
And the term \(N_S\) representing left-right configuration mismatch:
\begin{align*}
  N_S & ∈ \KA(Σ̈) \\
  N_S & ≜ ∑_{s ≠ s' ∈ S} ⟨s| s'⟩ (⟨a+b|+|a+b⟩)^* & \text{state mismatch} \\
    & + ⟨S⟩ ⟨a^*⟩ (⟨a|⁺ + |a⟩⁺) ⟨b^* | b^*⟩ & \text{counter \(a\) mismatch} \\  
    & + ⟨S⟩ ⟨a^* b^*⟩ (⟨b|⁺ + |b⟩⁺) & \text{counter \(b\) mismatch} \\
\end{align*}

And we can use these terms to bound the encoding of transition \(R_M\):
\begin{lemma}
  For all instructions \(i ∈ I_S\) and transition \(R_M ∈ \KA(Σ̈)\),
  let \(S\) be the state set of \(M\):
  \begin{align}
    [i] & ≤ ⟨a^* b^* | C_S⟩ = ⟨a^* b^* | S_⊥ ⋅ a^* b^*⟩; \label[ineq]{the: upper bound encoding instruction}\\
    R_M & ≤ ⟨C_S ∣ C_S⟩ = ⟨S_⊥ ⋅ a^* b^* | S_⊥ ⋅ a^* b^*⟩. \label[ineq]{the: upper bound encoding transition}
  \end{align}
\end{lemma}

\begin{proof}
  The first inequality can be proven by looking at each case of \(i\),
  and apply \Cref{the: matching star less than left right star}
  when necessary.

  The second inequality can be proven by unfolding the definition of \(R_M\):
  \begin{align*}
    R_M & = ∑_{s ∈ S} ⟨s| ⋅ [ι(s)] \\
    & ≤ ∑_{s ∈ S} ⟨s| ⋅ ⟨a^* b^* | S_⊥ ⋅ a^* b^*⟩ 
      & \text{by \Cref{the: upper bound encoding instruction}}\\
    & ≤ ⟨S ⋅ a^* b^* | S_⊥ ⋅ a^* b^*⟩ ≤ ⟨C_S ∣ C_S⟩. 
  \end{align*}
\end{proof}

With the above tools in place, 
we will establish the connection of provability and a single transition.
\begin{theorem}[provability of single transition]\label{the: single step transition soundness}
  In any machine \(M\), \((s, m, n) → (s', m', n')\) 
  if and only if the following inequality is provable:
  \[| s aᵐ bⁿ ⟩ R_M ≤ ⟨C_S⟩ ⋅ |s' a^{m'} b^{n'}⟩ + N_S ⋅ |C_{S}⟩.\]
\end{theorem}

\begin{proof}
  To prove the \(⟹\) direction, 
  we will first unfold definition of the left-hand side:
  \begin{align*}
    & | s aᵐ bⁿ ⟩ R_M \\
    & = | s aᵐ bⁿ ⟩ ⋅ (∑_{s₁ ∈ S} ⟨s₁| ⋅ [ι(s₁)]) \\   
    & = ∑_{s₁ ∈ S} | s aᵐ bⁿ ⟩ ⋅ ⟨s₁| ⋅ [ι(s₁)] \\  
    & = ⟨s| s aᵐ bⁿ ⟩ ⋅ [ι(s)] + ∑_{s₁ ≠ s} ⟨s₁| s aᵐ bⁿ ⟩ ⋅ [ι(s₁)].
  \end{align*}
  It suffices to show the following inequalities:
  \begin{align*}
    ⟨s| s aᵐ bⁿ ⟩ ⋅ [ι(s)] & ≤ ⟨C_S⟩ |s' a^{m'} b^{n'}⟩ + N_S |C_S⟩; \\  
    ∑_{s₁ ≠ s} ⟨s₁| s aᵐ bⁿ ⟩ ⋅ [ι(s₁)] & 
      ≤ ⟨C_S⟩ |s' a^{m'} b^{n'}⟩ + N_S |C_S⟩.
  \end{align*}

  We show the second inequality first:
  \begin{align*}
    & ∑_{s₁ ≠ s} ⟨s₁| s aᵐ bⁿ ⟩ ⋅ [ι(s₁)] \\   
    & ≤ ∑_{s₁ ≠ s} ⟨s₁| s aᵐ bⁿ ⟩ ⋅ ⟨a^* b^*|C_S⟩ \\  
    & ≤ ∑_{s ≠ s' ∈ S} ⟨s|s'⟩ (⟨a+b|+|a+b⟩)^* |C_S⟩\\
    & ≤ N_S |C_S⟩ ≤ ⟨C_S⟩ |s' a^{m'} b^{n'}⟩ + N_S |C_S⟩,
  \end{align*}

  To show the first inequality:
  \[⟨s| s aᵐ bⁿ ⟩ ⋅ [ι(s)] ≤ ⟨C_S⟩ |s' a^{m'} b^{n'}⟩ + N_S |C_S⟩.\] 
  We need to look at all the cases for \(ι(s)\),
  we show the \(\If(1, s)\) case as examples:
  \begin{itemize}
    \item If \(ι(s) = \If(1, s₁', s₂')\) and \(m = 0\),
      then \((s, 0, n) → (s₁', 0, n)\):
      \begin{align*}
        & ⟨s | s bⁿ⟩ ⋅ [ι(s)] \\
        & = ⟨s |s bⁿ⟩ ⋅ (|s₁'⟩ ⟨b^*⟩ + |s₂'⟩ ⟨a⁺⟩⟨b^*⟩) \\[5px]
        & = ⟨s |s bⁿ⟩ ⋅ |s₁'⟩ ⋅ (∑_{i ≤ n} ⟨bⁱ⟩ + ⟨bⁿ⟩ ⋅ ⟨b⁺⟩) \\
        & \qquad + ⟨s | s bⁿ⟩ ⋅ |s₂'⟩ ⋅ ⟨a⁺⟩⟨b^*⟩ \\[5px]
        & = (∑_{i < n} ⟨s bⁱ|s bⁿ⟩ |s₁' bⁱ⟩) 
          + ⟨ s bⁿ | s bⁿ ⟩ ⋅ |s₁' bⁿ ⟩ \\
        & \qquad + ⟨ s bⁿ b | s bⁿ⟩ |s₁' bⁿ b⟩⟨b^*⟩ \\  
        & \qquad + ⟨s a | s bⁿ⟩ ⋅ |s₂' a⟩ ⋅ ⟨a^* b^*⟩
      \end{align*}
      Notice:
      \begin{align*}
        ∑_{i < n} ⟨s bⁱ|s bⁿ⟩ |s₁' bⁱ⟩ & ≤ ⟨S⟩ ⟨a^* b^*⟩ |b⁺⟩ ⋅ |C_S⟩ ≤ N_S ⋅ |C_S⟩; \\
        ⟨ s bⁿ | s bⁿ ⟩ ⋅ |s₁' bⁿ ⟩ & ≤ ⟨C_S⟩ ⋅ |s₁' bⁿ⟩;\\
        ⟨ s bⁿ b | s bⁿ ⟩ |s₁' bⁿ b⟩⟨b^*⟩ 
            & ≤ ⟨ s bⁿ b | s bⁿ ⟩ |s₁' bⁿ b⟩⟨b^* | b^*⟩ \\  
            & ≤ ⟨ s bⁿ b b^*| s bⁿ ⟩ |s₁' bⁿ b b^*⟩ \\  
            & ≤ ⟨S⟩ ⟨a^* b^*⟩ ⟨b⁺| ⋅ |C_S⟩ ≤ N_S ⋅ |C_S⟩; \\  
        ⟨s a | s bⁿ⟩ ⋅ |s₂' a⟩ ⋅ ⟨a^* b^*⟩ 
            & ≤ ⟨s a | s bⁿ⟩ ⋅ |s₂' a⟩ ⋅ ⟨a^* | a^*⟩⟨b^* | b^*⟩ \\  
            & ≤ ⟨s a a^* b^*| s bⁿ⟩ ⋅ |s₂' a b^* a^*⟩ \\  
            & ≤ ⟨S⟩ ⟨a^*⟩ ⟨a|⁺ ⟨b^* | b^*⟩ ⋅ |C_S⟩ \\
            & ≤ N_S ⋅ |C_S⟩.
      \end{align*}
      Thus, we obtain the inequality we desire:
      \[| s aᵐ bⁿ ⟩ ⋅ ⟨s| ⋅ [ι(s)] ≤ ⟨C_S⟩ ⋅ |s₁' bⁿ⟩ + N_S |C_S⟩.\]
    \item If \(ι(s) = \If(1, s₁', s₂')\) and \(m ≠ 0\),
      then \((s, m, n) → (s₂', m, n)\), and the proof is similar to above:
      \begin{align*}
        & ⟨s|s aᵐ bⁿ⟩ ⋅ [ι(s)] \\
        & = ⟨s|s aᵐ bⁿ⟩ ⋅ (|s₁'⟩ ⟨b^*⟩ + |s₂'⟩ ⟨a⁺b^*⟩) \\[5px]
        & = ⟨s |s aᵐ bⁿ⟩ ⋅ |s₁'⟩ ⟨b^*⟩ \\
        & \qquad + ⟨s | s aᵐ bⁿ⟩ ⋅ |s₂'⟩ ⋅ 
            (∑_{i ≤ m, j ≤ n}⟨aⁱ bʲ⟩ + ⟨aᵐ a⁺ bⁿ b⁺⟩) \\[5px]
        & = ⟨s |s aᵐ bⁿ⟩ ⋅ |s₁'⟩ ⟨b^*⟩ \\
        & \qquad + (∑_{i < m, j ≤ n} ⟨s aⁱ bʲ| s aᵐ bⁿ⟩ ⋅ |s₂' aⁱ bʲ⟩) \\  
        & \qquad + (∑_{i = m, j ≤ n} ⟨s aⁱ bʲ| s aᵐ bⁿ⟩ ⋅ |s₂'aⁱ bʲ⟩) \\  
        & \qquad + ⟨s aᵐ bⁿ| s aᵐ bⁿ⟩ ⋅ |s₂'aᵐ bⁿ⟩ \\
        & \qquad + ⟨s aᵐ a⁺ bⁿ b⁺|s aᵐ bⁿ⟩ ⋅ |s₁' aᵐ a⁺ bⁿ b⁺⟩
      \end{align*}
      Then we obtain the following inequality
      \begin{align*}
        ⟨s |s aᵐ bⁿ⟩ ⋅ |s₁'⟩ ⟨b^*⟩ 
          & ≤ ⟨s b^* |s aᵐ bⁿ⟩ ⋅ |s₁' b^*⟩ \\
          & ≤ ⟨S⟩ ⟨a^*⟩ |a⟩⁺ ⟨b^* | b^*⟩ ⋅ |C_S⟩ \\  
          & ≤ N_S |C_S⟩; \\ 
        (∑_{i < m, j ≤ n} ⟨s aⁱ bʲ| s aᵐ bⁿ⟩ ⋅ |s₂' aⁱ bʲ⟩) 
          & ≤ ⟨S⟩ ⟨a^*⟩ |a⟩⁺ ⟨b^* | b^*⟩ ⋅ |C_S⟩ \\  
          & ≤ N_S |C_S⟩; \\
        (∑_{i = m, j ≤ n} ⟨s aⁱ bʲ| s aᵐ bⁿ⟩ ⋅ |s₂'aⁱ bʲ⟩)
          & ≤ ⟨S⟩ ⟨a^* b^*⟩ |b⟩⁺ ⋅ |C_S⟩\\  
          & ≤ N_S |C_S⟩; \\  
        ⟨s aᵐ bⁿ| s aᵐ bⁿ⟩ ⋅ |s₂'aᵐ bⁿ⟩ 
          & ≤ ⟨C_S⟩ ⋅ |s₂' aᵐ bⁿ⟩;  \\
        ⟨s aᵐ a⁺ bⁿ b⁺|s aᵐ bⁿ⟩ ⋅ |s₁' aᵐ a⁺ bⁿ b⁺⟩
          & ≤ ⟨S⟩ ⟨a^*⟩ |a⟩⁺ ⟨b^* | b^*⟩ ⋅ |C_S⟩ \\  
          & ≤ N_S |C_S⟩ 
        \end{align*}
      Thus, we obtain the inequality we desire:
      \[| s aᵐ bⁿ ⟩ ⋅ ⟨s| ⋅ [ι(s)] ≤ ⟨C_S⟩ ⋅ |s₂' aᵐ bⁿ⟩ + N_S |C_S⟩.\]
  \end{itemize}

  The \(⟸\) direction can be shown by looking at the language model.
  If the inequality holds, then the language interpretation holds:
  \[| s aᵐ bⁿ ⟩ L(R_M) ⊆ L(⟨C_S⟩) ⋅ |s' a^{m'} b^{n'}⟩ + L(N_S) ⋅ L(|C_{S}⟩).\]

  By \Cref{the: transition encoding is functional},
  there exists a word \(w ∈ L(R_M)\) s.t. its left component is \(⟨ s aᵐ bⁿ |\);
  we write the right component of \(w\) as \(|wᵣ⟩\).
  By \Cref{the: decidability and completeness of word inhabitant}, \(w ≤ R_M\); and
  by \Cref{the: upper bound encoding transition,the: left right order decompose},
  \[⟨ s aᵐ bⁿ | wᵣ⟩ ≤ R_M ≤ ⟨C_S|C_S⟩ ⟹ wᵣ ≤ C_S.\]
  Therefore, \(wᵣ ∈ L(C_S)\) and by the definition of \(L(N_S) ⋅ L(|C_{S}⟩)\),
  \[| s aᵐ bⁿ ⟩ ⋅ ⟨ s aᵐ bⁿ | wᵣ ⟩ = ⟨ s aᵐ bⁿ | s aᵐ bⁿ ⟩ ⋅ | wᵣ ⟩ ∉ L(N_S) ⋅ L(|C_{S}⟩).\]

  Because of the language inclusion
  \[| s aᵐ bⁿ ⟩ L(R_M) ⊆ L(⟨C_S⟩) ⋅ |s' a^{m'} b^{n'}⟩ + L(N_S) ⋅ L(|C_{S}⟩),\]
  we have
  \[⟨ s aᵐ bⁿ | s aᵐ bⁿ ⟩ ⋅ | wᵣ ⟩ ∈ L(⟨C_S⟩) ⋅ |s' a^{m'} b^{n'}⟩\]
  therefore \(wᵣ = s' a^{m'} b^{n'}\), and by \Cref{the: transition language soundness}:
  \[w = ⟨ s aᵐ bⁿ | s' a^{m'} b^{n'}⟩ ∈ R_M  
  ⟹ (s, m, n) → (s', m', n').\]
\end{proof}

We can further extend our previous result to establish a connection 
between provability and state reachability in certain type of machines. 
This result will be the core of our diagonal argument.

\begin{theorem}
  For a machine \(M ≜ (S, ŝ, ι)\) and an input \((m, n)\), 
  if the set of reachable configurations from the input is finite,
  then a set \(S' ⊆ S\) contains all the reachable states from input \((m, n)\) 
  if and only if the following inequality is provable:
  \begin{align}
    |ŝ aⁿ bᵐ⟩ R_M^* ≤ ⟨C_S^*⟩ |C_{S'}⟩ + ⟨C_S^*⟩ N_S ⟨C_S^* | C_S^*⟩.
    \label[ineq]{ineq: reachability inequality}
  \end{align}
\end{theorem}
\begin{proof}
  The \(⟹\) direction.
  We define the following term:
  \[\KA(Σ) ∋ Cᵣ ≜ ∑ \{cᵣ ∣ \text{\(cᵣ\) is reachable}\}.\]
  \(Cᵣ\) is well-defined because there are only finitely many reachable configurations.
  By \Cref{the: single step transition soundness}, 
  assume \(cᵣ → cᵣ'\), we have the following inequality:
  \begin{align*}
    |cᵣ⟩ R_M 
    & ≤ ⟨C_S⟩ |cᵣ'⟩ + N_S |C_S⟩ \\ 
    & ≤ ⟨C_S⟩ |Cᵣ⟩ + N_S |C_S⟩ & \text{\(cᵣ'\) is reachable}
  \end{align*}
  Since the above inequality is true for every reachable configuration \(cᵣ\),
  then the following inequality is true:
  \[|Cᵣ⟩ R_M  = ∑_{cᵣ} |cᵣ⟩ R_M ≤ ⟨C_S⟩ |Cᵣ⟩ + N_S |C_S⟩.\]

  To show provability of \Cref{ineq: reachability inequality}, 
  we will prove a stronger inequality:
  \[|ŝ aⁿ bᵐ⟩ R_M^* ≤ ⟨C_S^*⟩ |Cᵣ⟩ + ⟨C_S^*⟩ N_S ⟨C_S^* | C_S^*⟩.\]
  With \Cref{the: single step transition soundness,the: upper bound encoding transition},
  we can derive the following two inequality:
  \begin{align*}
    ⟨C_S^*⟩ |Cᵣ⟩ R_M 
      & ≤ ⟨C_S^*⟩ ⟨C_S⟩ |Cᵣ⟩ + ⟨C_S^*⟩ N_S |C_S⟩ \\
      & ≤ ⟨C_S^*⟩ |Cᵣ⟩ + ⟨C_S^*⟩ N_S ⟨C_S^* | C_S^*⟩ \\
    ⟨C_S^*⟩ N_S ⟨C_S^* | C_S^*⟩ R_M
      & ≤ ⟨C_S^*⟩ N_S ⟨C_S^* | C_S^*⟩ ⟨C_S | C_S⟩ \\
      & ≤ ⟨C_S^*⟩ N_S ⟨C_S^* | C_S^*⟩ \\  
      & ≤ ⟨C_S^*⟩ |Cᵣ⟩ + ⟨C_S^*⟩ N_S ⟨C_S^* | C_S^*⟩.
  \end{align*}
  Therefore,
  \begin{align*}
    & (⟨C_S^*⟩ |Cᵣ⟩ + ⟨C_S^*⟩ N_S ⟨C_S^* | C_S^*⟩) R_M  \\
      & \qquad ≤ ⟨C_S^*⟩ |Cᵣ⟩ + ⟨C_S^*⟩ N_S ⟨C_S^* | C_S^*⟩; \\
    & |ŝ aⁿ bᵐ⟩ ≤ |Cᵣ⟩ \\  
    & \qquad ≤ ⟨C_S^*⟩ |Cᵣ⟩ + ⟨C_S^*⟩ N_S ⟨C_S^* | C_S^*⟩.
  \end{align*}
  By induction rule, we have the desired inequality:
  \begin{align*}
    |ŝ aⁿ bᵐ⟩ R_M^* 
    & ≤ ⟨C_S^*⟩ |Cᵣ⟩ + ⟨C_S^*⟩ N_S ⟨C_S^* | C_S^*⟩ \\  
    & ≤ ⟨C_S^*⟩ |C_{S'}⟩ + ⟨C_S^*⟩ N_S ⟨C_S^* | C_S^*⟩.
  \end{align*}

  The \(⟸\) direction.
  We will first prove a small lemma: 
  consider a configuration \(cᵣ\) that is reachable from input \((m, n)\),
  we will show there exists a word \(w ∈ L(⟨C_S^*⟩)\) s.t.
  \[w |cᵣ⟩ ∈ L(|ŝ aⁿ bᵐ⟩) L(R_M)^*.\]
  The theorem above is shown by induction on the number of steps to reach \(cᵣ\):
  \begin{itemize}
    \item If \(cᵣ\) is reached in 0 steps, 
      then \[cᵣ = (ŝ, m, n).\]
      In this case, \(w = ϵ\) and 
      \[|cᵣ⟩ = |ŝ aⁿ bᵐ⟩ ∈ L(|ŝ aⁿ bᵐ⟩) L(R_M)^*.\]
    \item If \(cᵣ\) is reached in \(n+1\) steps,
      we find the configuration \(cᵣ'\) that is reached in \(n\) steps: 
      \[(ŝ, m, n) →^* cᵣ' → cᵣ.\]
      By induction hypothesis, there is \(w ∈ L(⟨C_S^*⟩)\), s.t.
      \[w |cᵣ'⟩ ∈ L(|ŝ aⁿ bᵐ⟩) L(R_M)^*.\]
      By \Cref{the: language encoding soundness of machine},
      \[cᵣ' → cᵣ ⟹ ⟨cᵣ' | cᵣ⟩ ∈ L(R_M).\]
      We have obtained our desired word:
      \[w |cᵣ'⟩ ⋅ ⟨cᵣ' | cᵣ⟩ = w ⟨cᵣ'|cᵣ'⟩ | cᵣ⟩ ∈ L(|ŝ aⁿ bᵐ⟩) L(R_M)^*.\]
  \end{itemize}

  Consider \(w ∈ L(⟨C_S^*⟩)\) s.t.
  \(w |cᵣ⟩ ∈ L(|ŝ aⁿ bᵐ⟩) L(R_M)^*,\) by definition
  \[w |cᵣ⟩ ∉ L(⟨C_S^*⟩ N_S ⟨C_S^* | C_S^*⟩).\]
  Because \Cref{ineq: reachability inequality} holds, 
  we have the following language inclusion:
  \[L(|ŝ aⁿ bᵐ⟩) L(R_M)^* ⊆ L(⟨C_S^*⟩) L(|C_{S'}⟩) + L(⟨C_S^*⟩ N_S ⟨C_S^* | C_S^*⟩).\]
  Therefore,
  \[w |cᵣ⟩ ∈ L(⟨C_S^*⟩) ⋅ L(|C_{S'}⟩).\]
  Finally, by unfolding the definition of \(L(⟨C_S^*⟩) ⋅ L(|C_{S'}⟩)\),
  we obtain \(s' ∈ S\).
\end{proof}

\begin{corollary}\label{the: reachability provability equivalence for terminating machine}
  For a machine \(M ≜ (S, ŝ, ι)\) that always halt regardless of the input,
  it will have finitely many reachable states from any input.
  Therefore, for any input \((m, n)\), 
  \(S'\) contains all the reachable state from \((ŝ, m, n)\) if and only if
  \cref{ineq: reachability inequality} is provable.
\end{corollary}

We can then construct our diagonal argument.
Assume that the equational theory of KA with atomic commutativity is decidable,
there is a machine \(P(S', M, M')\) that decides whether 
\cref{ineq: reachability inequality} is provable when given machine \(M ≜ (S, ŝ, ι)\) 
with the encoding of \(M'\) as input, and a subset of states \(S' ⊆ S\).

Fix two distinct states \(s₁, s₂\),
we define the diagonal machine \(D(M)\) as follows: 
let \(M ≜ (S, ŝ, ι)\) and \(S' ≜ S ∖ \{s₁\}\),
\begin{itemize}
  \item if \(P(S', M, M)\) returns true, 
    then we will go to state \(s₁\) and returns true;  
  \item if \(P(S', M, M)\) returns false,
    then we will go to state \(s₂\) and returns false.
\end{itemize}
Hence, state \(s₁\) is reachable if and only if \(P(S', M, M)\) returns true. 

Then we employ the standard technique to feed the diagonal machine to itself.
since we assumed equalities in KA with atomic commutativity is decidable, 
therefore \(D\) always terminates; hence we can enumerate all the possible 
the output of \(D(D)\):
\begin{itemize}
  \item If \(D(D)\) returns true, then \(P(S', D, D)\) returns true. 
    By \Cref{the: reachability provability equivalence for terminating machine},
    \(S' ≜ S ∖ \{s₁\}\) contains all the reachable states of \(D(D)\).
    However, by definition of \(D\), \(s₁\) is reachable when \(P(S', D, D)\) is true,
    and \(s₁ ∉ S'\). Therefore, we obtain a contradiction.
  \item  If \(D(D)\) returns false, this means that \(P(S', D, D)\) is false. 
    By \Cref{the: reachability provability equivalence for terminating machine},
    \(S' ≜ S ∖ \{s₁\}\) do not contain all the reachable state.
    However, by definition of \(D\), \(s₁\) is not reachable in this case,
    and \(S'\) contains every state other than \(s₁\).
    Hence, \(S'\) has to contain all the reachable state of \(D(D)\).
    We got a contradiction again.
\end{itemize}
Therefore, our assumption that KA with atomic commutativity is decidable has to be false.

\begin{corollary}[Incompleteness]
  There exists some commutable set \(X\) and two expression \(e₁, e₂ ∈ \KA(X)\) 
  s.t. \(L(e₁) ⊆ L(e₂)\) but \(e₁ ≰ e₂\).
  In other words, there exists inequalities in the language interpretation
  that is not derivable using the theory.
\end{corollary}

\begin{proof}
  Assume that the language interpretation is complete,
  that is for all expression \(e₁, e₂\)
  \[L(e₁) ⊆ L(e₂) ⟺ e₁ ≤ e₂.\]

  By definition of \(\KA(X)\), 
  \(e₁ = e₂\) if and only if it can be proven using the theory of KA 
  plus the commutativity in \(X\),
  therefore deciding general equality is recursively-enumerable, by enumerating the proof.

  However, since word inhabitance is decidable,
  language inclusion is co-recursively-enumerable,
  since we can simply check whether all words in \(L(e₁)\) is in \(L(e₂)\).

  If the language inclusion is equivalent to inequalities in the theory,
  then the problem of inequalities in the theory is both recursively-enumerable
  and co-recursively-enumerable, hence decidable.
  This result contradicts our undecidability result for general inequality in 
  KA with atomic commutativity hypotheses.
  Therefore, our assumption is false, and language interpretation is incomplete.
\end{proof} 

\section{Conclusion And Open Problem}

In this paper we have shown that the word inhabitance problem 
in KA with commutativity hypotheses is decidable and complete,
yet the general equalities are neither decidable nor complete.
We believe this is the first known KA extension 
where the word inhabitance problem is decidable,
yet the general equality is not.

Our method to show the decidability of word inhabitance problem
involves using the matrix model to decompose a word into several components,
which we believe is a novel technique in defining 
the empty word predicate and derivative in extensions of Kleene Algebra.
This technique also yields straight-forward proof of soundness and the fundamental theorem.

However, there are still several important open problems:
Several theorems leading to the undecidability result requires 
introspection on each case of the instructions, 
which leads to very long and tedious proof.
We suspect some of these proofs, like the proof for \Cref{the: single step transition soundness}, 
can be simplified by establishing more 
connection between the language interpretation and the free models.
The exact complexity of KA with atomic commutativity is still unknown.
In particular, we do not know whether the problem of 
deciding general equalities are RE-complete.


\bibliographystyle{ACM-Reference-Format}
\bibliography{refs}


\cleardoublepage

\chapter{Important Details}
\label{chapter:details}
\thispagestyle{myheadings}

% set this to the location of the figures for this chapter. it may
% also want to be ../Figures/2_Body/ or something. make sure that
% it has a trailing directory separator (i.e., '/')!
\graphicspath{{3_Details/Figures/}}

The use of Type 1 fonts and font embedding into the document are both dependent on a specific Latex installation and even on operating system. There is a good chance that it will work with no problem for you. However, should your thesis PDF be returned, please consider the following remedies discovered by students over many years. 

\section{Type 1 fonts}

All Boston University thesis and dissertation submissions must use only Type 1 fonts to assure high-quality rendering. Type 3 fonts are not acceptable.

For some students adding the following two lines in ``thesis.tex'' preamble has worked:\\
%
{\tt
$\backslash$usepackage[T1]\{fontenc\}\\
$\backslash$usepackage{pslatex}
   } 


The easiest way to check if fonts are embedded well and of what type, is to use Adobe Acrobat's Preflight -- it shows exactly where the Type 3 fonts are in the thesis. You can learn more here: \url{https://community.adobe.com/t5/acrobat/figure-out-where-a-specific-font-is-used-in-a-pdf/m-p/10880057?page=1#M238035}

If you don't have Adobe Acrobat (BU students get it for free), you can quickly check which fonts have which type by looking into Files $>>$ Properties $>>$ Fonts, but it doesn't tell where the text with a specific font type is.

{\bf Linux/Unix}: If you are using LaTeX or Unix, the problem is that, by default, LaTeX uses Type 3 fonts. Since most users have a tendency to use the default settings, then Type 3 fonts will be used by default. You can try to change the first line in the preamble in ``thesis.tex'' to:\\
%
{\tt $\backslash$documentstyle[12pt,times,letterpaper]\{report\}}

\noindent
since then Times fonts will be used (which are not Type 3). If there are mathematical formulas in the text, it is better to use:\\
%
{\tt $\backslash$documentstyle[12pt,times,mathptm,letterpaper]\{report\}}


\section{Font embedding}

All fonts must be embedded into the final PDF file. If they are not, sometimes equations may look strange or may not show up at all for several pages. This is often due to unembedded font problem. Should you have a font-embedding issue, this page may prove useful:\\
%
\url{https://www.karlrupp.net/2016/01/embed-all-fonts-in-pdfs-latex-pdflatex}

For those using Overleaf, this page might help:
\url{https://www.overleaf.com/learn/latex/Questions/My_submission_was_rejected_by_the_journal_because_%22Font_XYZ_is_not_embedded%22._What_can_I_do%3F}

\cleardoublepage 
 
\chapter{Decompilation Verification With GKAT}
\label{chapter:Conclusions}
\thispagestyle{myheadings}

% set this to the location of the figures for this chapter. it may
% also want to be ../Figures/2_Body/ or something. make sure that
% it has a trailing directory separator (i.e., '/')!
\graphicspath{{4_Conclusion/Figures/}}

\section{}
\cleardoublepage

%\appendix
\begin{appendices}
\chapter{Proof of xyz}
\label{appendix}
\thispagestyle{myheadings}

This is the appendix.
\end{appendices}
%==========================================================================%
% Bibliography
\newpage
\singlespace
\bibliographystyle{plain}

% each subdirectory can have its own BiBTeX file
\bibliography{thesis}
\cleardoublepage

%==========================================================================%
% Curriculum Vitae
\phantomsection\addcontentsline{toc}{chapter}{Curriculum Vitae}
\hypersetup{ urlcolor=black,linkcolor=black }

\begin{center}
{\LARGE {\bf CURRICULUM VITAE}}\\
\vspace{0.5in}
{\large {\bf Cheng Zhang}}
\end{center}

\section*{Education}

\begin{itemize}
    \item 
    2018 --- 2024, Computer Science, Doctor of Philosophy, Boston University, Boston, MA

    Primary Interest: Kleene Algebra and its application in program verification\\
    I am broadly interested in application of mathematics in computer science,
    especially in programming languages.
    I have worked on Kleene algebra, automata thoery, semantics, type systems, and their application in program verification.

    \item 2014 --- 2018, Mathematics, Bachelor of Art, \emph{with department honor, magna cum laude}, Wheaton College, Norton, MA

    Honor Thesis in Graph Theory: \href{http://hdl.handle.net/11040/24570}{King in Generalized Tournaments}.\\
    Minors: Computer Science, Economics.

    \item
    2016 --- 2017, Economics, Study Aboard, London School Of Economics, London, United Kingdom
\end{itemize}

\section*{Preprints And Drafts}

\begin{itemize}
    \item 
    Cheng Zhang, Hang Ji, Ines Santacruz, Marco Gaboardi, 
    \emph{A Symbolic Decision Procedure For GKAT, and its Complexity}, Draft

    \item 
    Cheng Zhang, Tobias Kappé, David E. Narváez, Nico Naus,
    \emph{CF-GKAT: Control Flow Verification in Nearly Linear Time}, Draft

    \item 
    Arthur Azevedo de Amorim, Cheng Zhang, Marco Gaboardi,
    \emph{\href{https://hal.science/hal-04534715/}{Kleene algebra with commutativity conditions is undecidable}}, Preprint
\end{itemize}

\section*{Publications}

\begin{itemize}
    \item 
    Cheng Zhang, Arthur Azevedo de Amorim, Marco Gaboardi,
    \emph{\href{https://arxiv.org/abs/2404.18417}{Domain Reasoning In TopKAT}}, ICALP 2024
    
    \item 
    Mark Lemay, Qiancheng Fu, William Blair, Cheng Zhang, Hongwei Xi,
    \emph{\href{https://doi.org/10.1145/3609027.3609407}{A Dependently Typed Language with Dynamic Equality}}, TyDe 2023

    \item 
    Cheng Zhang, Arthur Azevedo de Amorim, Marco Gaboardi,
    \emph{\href{https://arxiv.org/abs/2108.07707}{On Incorrectness Logic and Kleene Algebra With Top and Tests}}, POPL 2022

    \item 
    Cheng Zhang, \emph{\href{http://hdl.handle.net/11040/24570}{King in Generalized Tournaments}}, Undergraduate Thesis 

    \item 
    Cheng Zhang, Weiqi Feng, Emma Steffens, Alvaro de Landaluce, Scott Kleinman, Mark D. LeBlanc
    \emph{\href{https://dl.acm.org/doi/10.5555/3205191.3205205}{Lexos 2017: Building Reliable Software in Python}}, JCSC 2018
\end{itemize}


\section*{Honors And Fellowships}

\begin{itemize}
    \item 2020, Meta PhD Research Fellowship Finalist in Programminag Languages
    \item 2018 --- now, Phi Beta Kappa Honor Society Member.
    \item 2018, Boston University Dean's Fellowship.
    \item 2018, Phi Beta Kappa Graduate Scholarship.
    \item 2018, Fred Kollett Prize in Mathematics \& Computer Science.
    \item 2018, Madeleine F. Clark Wallace Mathematics Prize.
    \item 2017, Weaton College Faculty-Student Research Awards.
    \item 2016, Wheaton Fellows.
    \item 2014 --- 2018, Wheaton College International Scholarship.
    \item 2014 --- 2018, Wheaton College Dean's Lists.
\end{itemize}


% \section{Publications}

% \cventry{2024}
% {Cheng Zhang, Arthur Azevedo de Amorim, Marco Gaboardi}
% {\href{https://arxiv.org/abs/2404.18417}{Domain Reasoning In TopKAT}}
% {International Colloquium on Automata, Languages and Programming (ICALP)}
% {}{}

% \cventry{2023}
% {Mark Lemay, Qiancheng Fu, William Blair, Cheng Zhang, Hongwei Xi}
% {\href{https://doi.org/10.1145/3609027.3609407}{A Dependently Typed Language with Dynamic Equality}}
% {The workshop on Type-Driven Development (TyDe)}
% {}{}

% \cventry{2022}
% {Cheng Zhang, Arthur Azevedo de Amorim, Marco Gaboardi}
% {\href{https://arxiv.org/abs/2108.07707}{On Incorrectness Logic and Kleene Algebra With Top and Tests}}
% {Principle Of Programming Language (POPL)}
% {}{}

% \cventry{2020}
% {Mark Lemay, Cheng Zhang, William Blair}
% {\href{https://icfp20.sigplan.org/details/tyde-2020-papers/7/Developing-a-Dependently-Typed-Language-with-Runtime-Proof-Search-Extended-Abstract-}
% {Developing a Dependently Typed Language with Runtime Proof Search (Extended Abstract)}}
% {Workshop on Type-Driven Development (TyDe)}
% {}{}

% \cventry{2018}
% {Cheng Zhang}
% {\href{http://hdl.handle.net/11040/24570}{King in Generalized Tournaments}}
% {Wheaton College Honor Thesis}
% {}{}

% \cventry{2018}
% {Cheng Zhang, Weiqi Feng, Emma Steffens, Alvaro de Landaluce, Scott Kleinman, Mark D. LeBlanc}
% {\href{https://dl.acm.org/doi/10.5555/3205191.3205205}{Lexos 2017: Building Reliable Software in Python}}
% {Journal of Computing Sciences in Colleges}
% {}{}


% \section{Research Talks}

% \cventry{2023}
% {Cheng Zhang}
% {GKAT with Indicator Variables, Fast Decompilation Verification}
% {BU Principles of Programming and Verification (POPV) Seminar}
% {}{}

% \cventry{2023}
% {Cheng Zhang}
% {A Practical Tutorial to KAT and its Extensions}
% {Systems Software Research Group at Virginia Tech}
% {}{}

% \cventry{2022}
% {Cheng Zhang}
% {Kleene Algebra and Its Applications in Verification}
% {Boston Computation Club}
% {}{}

% \cventry{2022}
% {Cheng Zhang}
% {On Incorrectness Logic Kleene Algebra With Test}
% {Cornell Programming Language Discussion Group (PLDG)}
% {}{}

% \cventry{2022}
% {Cheng Zhang}
% {On Incorrectness Logic Kleene Algebra With Test}
% {Principle Of Programming Languages (POPL)}
% {}{}

% \cventry{2018}
% {Cheng Zhang, Mark D. LeBlanc}
% {Lexos 2017: Building Reliable Software in Python}
% {Journal of Computing Sciences in Colleges}
% {}{}

% \cventry{2018}
% {Cheng Zhang}
% {Kings in Quasi-transitive Oriented Graph}
% {Wheaton Summit For Woman In STEM}
% {}{}



% \section{Honors And Fellowships}
% \cvitem{2022}{Meta PhD Research Fellowship Finalist in Programminag Languages}
% \cvitem{2018 --- Now} {Phi Beta Kappa Honor Society Member.}
% \cvitem{2018}{Boston University Dean's Fellowship.}
% \cvitem{2018}{Phi Beta Kappa Graduate Scholarship.}
% \cvitem{2018} {Madeleine F. Clark Wallace Mathematics Prize.}
% \cvitem{2018}{Fred Kollett Prize in Mathematics \& Computer Science.}
% \cvitem{2017}{Weaton College Faculty-Student Research Awards.}
% \cvitem{2016}{Wheaton Fellows.}
% \cvitem{2014 --- 2018}{Wheaton College International Scholarship.}
% \cvitem{2014 --- 2018}{Wheaton College Dean's Lists.}
 

%==========================================================================%
\end{document}
